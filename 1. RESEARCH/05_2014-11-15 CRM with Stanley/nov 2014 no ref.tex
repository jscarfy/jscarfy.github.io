\documentclass[12pt,amsfont]{amsart}
%\usepackage{amsaddr}
\usepackage{amssymb, fullpage, amsfonts, euscript, times, hyperref}
\usepackage{ulem}
\usepackage{color}\definecolor{Blue}{rgb}{0,0,1}
\usepackage{color}\definecolor{Red}{rgb}{1,0,0}
\usepackage{algorithm}
\usepackage{algpseudocode}
\usepackage{mathtools}
\usepackage{enumerate}
\usepackage{graphicx}
\usepackage{bbm}


\DeclarePairedDelimiter{\ceil}{\lceil}{\rceil}
\DeclarePairedDelimiter{\floor}{\lfloor}{\rfloor}
\usepackage[OT2,T1]{fontenc}

\DeclareSymbolFont{cyrletters}{OT2}{wncyr}{m}{n}
\DeclareMathSymbol{\Sha}{\mathalpha}{cyrletters}{"58}
\vfuzz=2pt
\begin{document}
\thispagestyle{empty}
\parindent=0pt
\parskip=4pt
\renewcommand{\labelenumi}{(\roman{enumi})}
\renewcommand{\theenumi}{(\roman{enumi})}


\newcommand{\legendre}[2]{\genfrac{(}{)}{}{}{#1}{#2}}
\newcommand{\li}{\mathop{\rm li}}
\newcommand{\lcm}{\operatorname{lcm}}
\renewcommand{\pmod}[1]{{\ifmmode\text{\rm\ (mod~$#1$)}\else\discretionary{}{}{\hbox{ }}\rm(mod~$#1$)\fi}}
\newcommand{\bS}{{\mathbb S}}
\newcommand{\bC}{{\mathbb C}}
\newcommand{\bN}{{\mathbb N}}
\newcommand{\bR}{{\mathbb R}}
\newcommand{\bP}{\mathbb{P}}
\newcommand{\bZ}{{\mathbb Z}}
\newcommand{\bF}{{\mathbb F}}
\newcommand{\bQ}{{\mathbb Q}}
\newcommand{\Ha}{{\mathbb H}}
\newcommand{\bA}{\mathbb{A}}
\newcommand{\bG}{\mathbb{G}}
\newcommand{\bB}{\mathbb{B}}
\renewcommand{\mod}[1]{{\ifmmode\text{\rm\ (mod~$#1$)}\else\discretionary{}{}{\hbox{ }}\rm(mod~$#1$)\fi}}
\newcommand{\Tor}{{\operatorname{Tor}}}
\newcommand{\Ann}{\operatorname{Ann}}
\newcommand{\Sel}{\operatorname{Sel}}
\newcommand{\Hom}{\operatorname{Hom}}
\newcommand{\End}{\operatorname{End}}
\newcommand{\GL}{\operatorname{GL}}
\newcommand{\SL}{\operatorname{SL}}
\newcommand{\Avg}{\operatorname{Avg}}
\newcommand{\Sym}{\operatorname{Sym}}
\newcommand{\PGL}{\operatorname{PGL}}
\newcommand{\Jac}{\operatorname{Jac}}
\newcommand{\Aut}{\operatorname{Aut}}
\newcommand{\Cl}{\operatorname{Cl}}
\newcommand{\Div}{\operatorname{Div}}
\newcommand{\fp}{\mathfrak{p}}
\newcommand{\Pic}{\operatorname{Pic}}
\newcommand{\sgn}{\operatorname{sgn}}
\newcommand{\Alex}{\operatorname{Alex}}
\newcommand{\Stab}{\operatorname{Stab}}
\newcommand{\Gal}{\operatorname{Gal}}
\newcommand{\rank}{\operatorname{rank}}
\newcommand{\ord}{\operatorname{ord}}
\newcommand{\chr}{\operatorname{char}}
\newcommand{\Reg}{\operatorname{Reg}}
\newcommand{\Res}{\operatorname{Res}}
\newcommand{\Vol}{\operatorname{Vol}}
\newcommand{\OO}{\mathcal{O}}
\newcommand{\ol}{\overline}
\newcommand{\ul}{\underline}
\newcommand{\F}{\mathcal{F}}
\newcommand{\E}{\mathcal{E}}
\newcommand{\cL}{\mathcal{L}}
\newcommand{\fP}{\mathfrak{P}}
\newcommand{\fB}{\mathfrak{B}}
\newcommand{\A}{\mathcal{A}}
\newcommand{\R}{\mathcal{R}}



\numberwithin{equation}{section}%{subsection} %sets equation numbers <chapter>.<section>.<index>
\newtheorem{theorem}{Theorem}[section]
\newtheorem{lem}[theorem]{Lemma}
\newtheorem{pbm}[theorem]{Problem}
\newtheorem{pro}[theorem]{Proposition}
\newtheorem{cor}[theorem]{Corollary}
\newtheorem{cnj}[theorem]{Conjecture}
\newtheorem{dfn}[theorem]{Definition}
\newtheorem{thm}[theorem]{Theorem}
\newtheorem{rmk}[theorem]{Remark}
\newtheorem{xmp}[theorem]{Example}
\newtheorem{exe}[theorem]{Exercise}
\newenvironment{solution}
               {\let\oldqedsymbol=\qedsymbol
                \renewcommand{\qedsymbol}{$\blacktriangle$}
                \begin{proof}[\bf Solution]} 
               {\end{proof}
                \renewcommand{\qedsymbol}{\oldqedsymbol}}
                

\numberwithin{theorem}{section} %sets equation numbers <chapter>.<section>.<index>
\title{\bf Lecture notes for\\CRM Workshop on \\``Counting arithmetic objects (Ranks of elliptic curves)''\\ November 10-14, 2014}
\author{Stanley Yao Xiao \\ \href{mailto:stanley.xiao@uwaterloo.ca}{
{\texttt{\lowercase{stanley.xiao@uwaterloo.ca}}} }\\  {\textnormal{\textit{D{\lowercase{epartment of }}P{\lowercase{ure}} M{\lowercase{athematics}}\\
U{\lowercase{niversity of }}W{\lowercase{aterloo}}\\
W{\lowercase{aterloo}}, O{\lowercase{ntario}}, C{\lowercase{anada}}
N2L 3G1\\ }\\}\and \\ \\ Justin Scarfy\\
\href{mailto:scarfy@ugrad.math.ubc.ca}{
{\texttt{\lowercase{scarfy@ugrad.math.ubc.ca}}} }}
\\
{\textnormal{\textit{D{\lowercase{epartment of }}M{\lowercase{athematics}}\\ T\lowercase{he} U\lowercase{niversity of} B\lowercase{ritish} C\lowercase{olumbia}\\ R\lowercase{oom} 121, 1984 M\lowercase{athematics} R\lowercase{oad}\\V\lowercase{ancouver}, B\lowercase{ritish} C\lowercase{olumbia}, C\lowercase{anada} V6T 1Z2}}}}
\begin{abstract}
This compilation contains all the lecture notes taken at the conference ``Counting arithmetic objects (Ranks of elliptic curves)'' held at the CRM between November 10 and 14, 2014.  We would like to thank the organizers for putting together this wonderful conference with such diverse areas, for the speakers presenting their newest results, and for the CRM for its hospitality.  The lecture notes were mostly typed up live by the first of us, and later corrected and polished by the second of us. 
\end{abstract}

\maketitle

\setcounter{tocdepth}{1}
\tableofcontents

\newpage
%%%1.
\section{Arithmetic invariants of elliptic curves \emph{on average} (1/2) \\by Manjul Bhargava}\label{1}

Recall: Any elliptic curve $E$ over $\mathbb{Q}$ can be expressed uniquely in the form
\[
\displaystyle E_{A, B}: y^2 = x^3 + Ax + B, \quad A,B \in \mathbb{Z},\text{and if }p^4 \mid A\text{ then }p^6 \nmid B\text{ for all primes }p.
\]
Define the (naive) height 
\[
H(E_{A,B}):=\max\{4 |A|^3, 27 B^2\}. 
\]
Note: the discriminant $\Delta$ of the elliptic curve $E_{A,B}$ is given by
\[
 \Delta = -4A^3 - 27B^2.
 \]
Consider the number of curves $E_{A,B}$ of height at most $X$. This count is equal to
$$\displaystyle \# \{E_{A,B}: H(E_{A,B}) < X\} = cX^{5/6} +o(X^{5/6}).$$
If we order all $E_{A,B}/\mathbb{Q}$ by height, we ask: 
\begin{itemize}
\item[Q1:] the average rank of $E_{A,B}$?
\item
Conjecture 1: the average rank of $E_{A,B}$ ordered by height is equal to $1/2$ (Goldfeld, Katz-Sarnak ). 
\item[Q2:] the average size of the $n$-Selmer group of $E_{A,B}$? 
\item 
Conjecture 2: the average size of the $n$-Selmer group is equal to $\sigma(n)$ (NEW). 
\item[Q3:] the average size of the root number of $E_{A,B}$? 
\item
Conjecture 3: the average should be zero (equidistribution of root number). 
\item[Q4:] the proportion of $E_{A,B}$ satisfying BSD? 
\item Conjecture 4: it should be $100\%$. In fact, BSD says \emph{all} elliptic curves satisfy BSD. 
\end{itemize}
\begin{thm} Conjectures 2 and 3 imply Conjecture 1 and the parity conjecture.

\end{thm}

\begin{proof} Exercise. \end{proof}

\begin{thm}\label{T2} 
Conjecture 2 implies Conjecture 4. 
\end{thm}

The proof of Theorem \ref{T2} will be given in the Lectures~\ref{1} and~\ref{4}. 
\begin{rmk}
Essentially nothing known previously on Q1 to Q4 beyond trivial bounds. There were many conditional results under GRH, BSD, and other heuristics.
\end{rmk}
We begin with some latest progress made on Q2:
\begin{thm} \label{T1.4} (with Arul Shankar) 
\[
\operatorname{Avg}(\operatorname{Sel}_n(E_{A,B})) = \sigma(n)\quad \text{for } n = 1, 2, 3, 4, 5. 
\]
\end{thm}
\begin{proof} (Outline) 
\begin{enumerate}[\normalfont (a)]
\item\label{1.3a}
For each $n$, find a representation $V$ of an algebraic group $G$, defined over $\mathbb{Z}$, such that:
\begin{enumerate}[\normalfont (i)]
\item
the ring of invariants of $G(\mathbb{C})$ on $V(\mathbb{C})$ is freely generated by two elements, which we call $A,B$. \
\item
there is an injective map  
\begin{equation}\label{1.1}
\Sel_n(E_{A,B})\hookrightarrow
\left[G(\mathbb{Z}) \backslash V(\mathbb{Z}) \right]_{A,B} \tag{$\ast$}
\end{equation}
For each $n$, we require a pair $(G,V)$. 
\begin{itemize}
\item
For $n = 2$, we have $(G,V) = (\operatorname{PGL}_2, \Sym^4(2))$ ($\Sym^4(2)$ are binary quartic forms). This was due to Birch and Swinnerton-Dyer . 
\item
For $n = 3$, we have $(G,V) = (\PGL_3, \Sym^3(3))$ (ternary cubic forms) due to Cassels, Cremona-Fisher-Stoll , Aronholdt . 
\item
For $n = 4$, we have $(G,V) = ((\operatorname{GL}_2 \times \GL_4)/\sim, 2 \otimes \Sym^2 4)$ ($2 \otimes \Sym^2 4)$ refer to two quadrics in $\mathbb{P}^3$). This was due to Clebesch , Cremona-Fisher-Stoll . 
\item
For $n=5$, we have $(G,V) = ((\GL_5 \times \GL_5)/\sim, 5 \otimes \Lambda^2 5)$ ($5 \times 5$ skew-symmetric matrices in linear forms in five variables). This was due to Cayley , Sylvester , Buchsbawn-Eisenbud .
\end{itemize}

Note: An $n$-Selmer element of $E_{A,B}$ can be viewed as a map $C \rightarrow \mathbb{P}^{n-1}$ such that $\operatorname{Jac}(C) = E_{A,B}$ and $C$ is a genus one curve that has local points everywhere.
\begin{itemize}
\item
When $n=2,3,4$, it is easy to understand geometrically from this point of view.
\item
When $n = 5$, it is a genus one curve in $\mathbb{P}^4$, with five quadrics contain it. One cannot take arbitrary quadrics, as they typically intersect trivially. Thus one needs to take special consideration. 
\end{itemize}
\end{enumerate}
\end{enumerate}

\begin{rmk} If we knew a similar way to understand the map $C \rightarrow \mathbb{P}^{n-1}$ for $n > 5$ ($C$ is a genus one curve) then could likely understand $n$-Selmer for $n > 5$ (algebraic geometry problems). 
\end{rmk}
\begin{enumerate}[\normalfont (a)]
\setcounter{enumi}{1}
\item\label{1.3b}
Count $G(\mathbb{Z})$-orbits on $V(\mathbb{Z})$ having bounded $A$ and $B$ (geometry of numbers).
\item
Elements in image of~\eqref{1.1} are defined by infinitely many congruence conditions. Sieve to these elements using variant of Ekedahl sieve. 
\item
Divide by $cX^{5/6}$ gives $\sigma(n)$. 
\end{enumerate}
\end{proof}
Note: $\Avg(\Sel_5(E_{A,B})) = 6 \Rightarrow$ $5$-Selmer rank of $E_{A,B}$ is $0$ or $1$ \emph{most} of the time. \\
Open problems: 
\begin{enumerate}
\item[(0)]
Does $p$-Selmer rank zero imply analytic rank zero? We know that $p$-Selmer rank zero implies rank zero (due to Kolyvagin)  and also that analytic rank zero imply rank zero.
\item[(1)]
Does $p$-Selmer rank $1$ implies analytic rank $1$, and analytic rank $1$ implies rank $1$ (Gross-Zagier-Kolyvagin). But does $p$-Selmer rank imply rank $1$?
\item[(S)]
Special case: does $\Sel_2(E) = \bZ/2 \bZ$ (rational $2$-torsion free) imply rank $1$? It is unknown!
\end{enumerate}
\begingroup
\renewcommand{\addcontentsline}[3]{}% Remove functionality of \addcontentsline
\endgroup


%%%2
\newpage
\renewcommand{\thesubsection}{\arabic{section}.\arabic{subsection}}
\section{Iwasawa theory and ranks of elliptic curves and Selmer groups (1/2)\\ by Eric Urban}\label{2}
Let $E$ be an elliptic curve over $\bQ$. We consider the cases when $p$ a is prime of good reduction, ordinary good reduction, or multiplicative reduction.    
\[
\Sel_p(\bQ, E) \subset H^1(\bQ, E_{p^\infty}(\ol{\bQ})).
\]
We have the exact sequence
\[
0 \rightarrow E(\bQ) \otimes \bQ_p / \bZ_p \rightarrow \Sel_p(\bQ, E) \rightarrow \Sha_p(\bQ, E) \rightarrow 0.
\]
\begin{enumerate}
\item[(C0)]
If $E$ has rank $0$, then the corank $p$-$\Sel = 0$ implies $L(E, 1) \ne 0$. The converse implication is due to Gross-Zagier-Kolyvagin . (and Kato has a different method )
\item[(C1)]
If $E$ has rank $1$: then the corank $p$-$\Sel = 1$ implies $ L(E,s) = 1$ for $s = 1$, due to Gross-Zagier-Kolyvagin.
\item[(0)]
For rank $0$, use $p$ ordinary prime and it follows from the Iwasawa Main Conjecture for $E$ (by Skinner-Urban). For $p$ multiplicative reduction case, due to Skinner. For $p$ super-singular, follows from Bloch-Kato conjecture for $E$ (Skinner-Urban). 
\item[(1)]
For rank 1, $p$ ordinary done by Wei Zhang good ordinary real. Venerucci dealt with the multiplicative reduction case. In these two approaches the Iwasawa Main Conjecture for $E$ is used. 
\end{enumerate}
\subsection{Iwasawa theory, Iwasawa-Greenberg conjectures}
{\ }

Let $\mathcal{O}/\bZ_p$ be a finite extension, $k$ its residue field, $R$ be a local noetherian complete $\mathcal{O}$-algebra with residue field $k$. \\
We have a Galois representation $\rho: G_\bQ \rightarrow \GL_N(R)$ such that:
\begin{enumerate}
\item[(1)] There exists $\Sigma \subset \operatorname{Spec}(R)(\ol{\bQ}_p)$ (Zariski dense) such that for every $x \in \Sigma$, $\rho_x = G_\bQ \xrightarrow{\operatorname{ev}_x} \GL_N(R) \rightarrow \GL_N(\ol{\bQ}_p)$. $\rho_x$ is motivic, $p$-ordinay,
$$\displaystyle \rho_k |_{I_p} \sim \begin{pmatrix} \varepsilon^{m_N(x)} & \cdots & \ast \\ \vdots & \ddots & \vdots \\ 0 & \cdots & \varepsilon^{m_1(x)} \end{pmatrix}.$$

For $\operatorname{Fil}^i(V_x)$ or $\operatorname{Fil}^i/\operatorname{Fil}^{i+1}$, the action given by $\varepsilon_{\text{cyc}}^i$. 
\item[(2)]
For all $x \in \Sigma$, $\rho_x$ is critical in the sense of Deligne, i.e.
\[
\frac{L(\rho_x, 0)}{\Omega_\infty(\rho_x) \Omega_p(\rho_x)} \in \ol{\bZ}_p.
\]
\end{enumerate}
With the assumption: $\ol{\rho}^{G_\bQ} = 0$, we formulate the following conjectures:
\begin{cnj} 
{\ }
\begin{enumerate}
\item[(1)]
There exists $\mathcal{L}_\rho \in R$ such that for all $x \in \Sigma$, $\mathcal{L}_\rho(x) = \displaystyle \frac{L(\rho_x, 0)}{\Omega_\infty(\rho_x) \Omega_p(\rho_x)}$ (up to $p$-adic units). 
\item[(2)] 
If $\mathcal{L}_\rho \ne 0$ then a certain Selmer group is co-torsion over $R$.
First, one defines $\Sel(\bQ, \rho_x) \subset H^1(\bQ, \rho_x \otimes L/\mathcal{O})$, $\text{ev}_x(R) \subset \mathcal{O} \subset \ol{\bZ}_p$, $L = \operatorname{Frac}(\mathcal{O})$. An element in $\Sel(\bQ, \rho_x)$ characterizes an isomorphism class of extensions 
$$\displaystyle 0 \rightarrow \rho_x \otimes \ol{\omega}^{-m} \mathcal{O}/\mathcal{O} \rightarrow W \rightarrow \omega^{-n} \mathcal{O}/\mathcal{O} \rightarrow 0$$
such that for all $\ell \ne p$, the restriction of $I_\ell$ is split plus ordinary reduction at $p$. $W$ is free of rank $N+1$ over $\mathcal{O}/\omega^n \mathcal{O}$. $W$ is ordinary if
\[\displaystyle \rho_W |_{I_p} \sim \begin{pmatrix} 
\varepsilon^{m_N(x)} & \cdots & \cdots & \cdots & \cdots & \cdots & \cdots & \ast \\ 
\vdots & \ddots & \vdots & \cdots & \cdots & \cdots & \cdots & \vdots \\
 0 & \cdots & \varepsilon^{m_{s+1}(x)} & \cdots & \cdots & \cdots & \cdots & \vdots \\ 
0 & \cdots & 0 & 1 &  &  &  &  \\  
0 & \cdots & 0 & 0 & \varepsilon^{m_s(x)} & \cdots & \cdots & \vdots \\ 
0 & \cdots & 0 & 0 & \ddots & \cdots & \cdots & \vdots \\ 
0 & \cdots & 0 & \vdots &  0 & \ddots & \cdots & \ast \\ 
0 & \cdots & 0 & 0 & 0 & \cdots & \cdots & \varepsilon^{m_1(x)} \end{pmatrix},
\]
where $m_N(x) \geq \cdots \geq m_{s+1}(x) > 0 \geq m_s(x) \geq \cdots \geq m_1(x)$.
\end{enumerate}
\end{cnj}

Let $F_{\rho_x}^t \subset \rho_x$ the subspace generated by $N-s$ first vectors of the basics, which implies
$$\displaystyle 0 \rightarrow \rho_x/F_{\rho_x}^t \otimes \omega^{-n} \mathcal{O}/\mathcal{O} \rightarrow W/F_{\rho_x}^T \rightarrow \omega^{-n} \mathcal{O}/\mathcal{O} \rightarrow 0,$$
which then implies
$$\displaystyle [w] |_{I_p} \in \ker(H^1(\bQ, \rho_x \otimes L/\mathcal{O}) \rightarrow H^1(I_p, \rho_x/\rho^t \rho_x \otimes L/\mathcal{O})).$$

\subsection{Extra condition on $\rho$}
{\ }

There exists $F^+ \rho \subset V_\rho = R^N$ such that for all $x \in \Sigma$, $(F^+ \rho)_x = F^+(\rho_x)$. 
\begin{rmk} 
{\ }
\begin{enumerate}
\item[(1)] 
$F^+ \rho$ depends on $\Sigma$ and for a given $\rho$ there exist possibly several sets $\Sigma$ giving rise to different elements $F^t \rho$.
\item[(2)]
Different $\Sigma$'s give rise to different $p$-adic $L$-functions. 
\end{enumerate}
\end{rmk}
Selmer groups $\Sel_p$, $S$ a finite set of primes, $p \not \in S$. (Upper star below denotes the pontryagin dual)
$$\displaystyle \Sel^S(\bQ, p) = \ker(H^1(\bQ, \rho \otimes R^\ast) \rightarrow \bigoplus_{\substack{l \not \in S\\ \rho \ne 0}} H^1(I_l,\rho \otimes R^\ast) \oplus H^1(I_p, \rho/F^+ \rho \otimes R^\ast)).$$
\begin{cnj} (Iwasawa-Greenberg ): If $\mathcal{L}_\rho \ne 0$, then $X_{F^+ \rho}(\bQ, \rho) = \left(\Sel_{F^t \rho}^{\emptyset} (\bQ, \rho)\right)^\ast$ is torsion over $R$ and $\operatorname{Fil} H_R (X_{F^+ \rho} (\bQ, \rho)) = \mathcal{L}_\rho$. 
\end{cnj}
\begin{xmp} Take $\rho_0 : G_\bQ \rightarrow \GL_N(\mathcal{O})$ attached to a motive, ordinary at $p$ and critical. $\Gamma = \operatorname{Gal}(\bQ_\infty / \bQ)$, $\bQ_\infty/\bQ$ $\bZ_p$-cyclotomic extension. $\rho_0 \otimes \mathcal{O} [[\Gamma ]]$

For all $\psi$ finite order, $x_\psi : \mathcal{O}[[\Gamma]] \mapsto \ol{\bZ}_p$, $\sigma \mapsto \psi(\sigma)$, $\rho_x = \rho_0 \otimes \psi$ implies $p$-adic $L$-function interpolating $\displaystyle \frac{L(\rho_0 \otimes \psi, 0)}{\text{periods}}$. \\ \\
Special case: take $\rho_0 : G_\bQ \rightarrow \GL(T_p(E))$, $E$ ordinary at $p$. 
$$\displaystyle L(\rho_0 \otimes \psi, 0) = L(E, \psi, 1)$$
implies it is known that there exists $p$-adic $L$-function interpolating those values.  Where the Main Conjecture implies that $\operatorname{Fil}^\bullet H(\Sel(\bQ, \rho_{T_p(E)} \otimes [ \cdot ]))^\ast \sim (L_p(E, \cdot))$. 
\end{xmp}
\begin{xmp} Rankin-Selberg conditions.  Let 
$f$ be an elliptic modular form of level $N$ and weight $k$, and $g$ an elliptic modular form of level $N$ and weight $l$, both ordinary at $p$. \\
Their Galois representations $\rho_f$ is of Hodge-Tate weight $(k-1, 0)$, $\rho_f |_{I_p} \sim \begin{pmatrix} 1 & \ast \\ 0 & \varepsilon^{-k} \end{pmatrix}$ , and $\rho_g$ is of Hodge-Tate weight $(l-1, 0)$.\\
Define two Hida families $F,G$ where $F \in \mathbb{I}[[q]]$, $\mathbb{I}/\mathcal{O}[[w_1]]$
$G \in \mathbb{J}[[q]]$, $\mathbb{J}/\mathcal{O}[[w_2]]$  and two big  representations 
\begin{align*} 
\rho_F : G_\bQ &\rightarrow \GL_2(\mathbb{I}), \\
\rho_G : G_\bQ & \rightarrow \GL_2(\mathbb{J}) .\end{align*}
 
$R = \mathbb{I} \otimes_{\mathcal{O}} \mathbb{J} \otimes \mathcal{O}[[\Gamma]]$, where $\Gamma = \operatorname{Gal}(\bQ_\infty / \bQ)$ and such $\rho_{\text{R-S}} = \rho_F \otimes \rho_G \otimes [\cdot ] \subset \GL_4(R)$. Critical points inside $\operatorname{Spec}(R)(\ol{\bQ}_p)$ are given by $(f, g, \varepsilon^n \psi) = x$. 
\begin{enumerate}
\item[(I)] When $k > l$, $\rho_f \otimes \rho_g \otimes \varepsilon^n \psi$ critical if $k-1 \geq m \geq l$. \\ \\
\item[(II)] $k < l$, $\rho_f \otimes \rho_g \otimes \varepsilon^n \psi$ critical if $l-1 \geq m \geq k$.
\end{enumerate}
Then
\begin{enumerate}
\item[(I)] $F^+ \rho_x = F^+ \rho_f \otimes \rho_g$
$$\displaystyle \rho_x |_{I_p} \sim \begin{pmatrix} \varepsilon^n &  &  &  \\ 0 & \varepsilon^{1 - l + n} &  &  \\ 0 & 0 & \varepsilon^{1 - k + n} &  \\ 0 & 0 & 0 & \varepsilon^{2 - k - l + n} \end{pmatrix}$$
$F^+ \rho = F^+ \rho_F \otimes \rho_G$
\item[(II)]
$F^+ \rho_x = F^+ \rho_g \otimes \rho_f$
$F^+ \rho = F^+ \rho_G \otimes \rho_F$. 
\end{enumerate}
\end{xmp}

\begingroup
\renewcommand{\addcontentsline}[3]{}% Remove functionality of \addcontentsline
\endgroup

%%%3
\newpage
\renewcommand{\thesubsection}{\arabic{section}.\arabic{subsection}}

\section{Iwasawa theory and ranks of elliptic curves and Selmer groups (2/2)\\ by Eric Urban}\label{3}
Recall we had two Hida families $F, G$ with Galois deformation
\[
\rho=\rho_F\otimes \rho_G \otimes \mathcal{O}[[\Gamma]]
\]
and we had two types:
\begin{enumerate}
\item[(I)] $\Sigma = \{(f,g, \varepsilon_{\text{cyc}}^n \psi), k-1 \geq n \geq l\}$, $F_\rho^+ = F_{\rho_{F}}^+ \otimes \rho_G \otimes \Lambda$.
\item[(II)]
 $\Sigma = \{(f,g, \varepsilon_{\text{cyc}}^n \psi), l-1 \geq n \geq k\}$, $F_\rho^+ = \rho_F \otimes F^+ \rho_G \otimes \Lambda$.
 \end{enumerate}
Hida defined
\begin{align*}
L^{\textrm{I}}(x) &= \frac{L(f\times g,m)}{\Omega_p(g)}\\
L^{\textrm{II}}(x) &= \frac{L(f,g,m)}{\langle g, g\rangle \Omega_p(g)}. 
\end{align*}
Special case: When $g\in G$ is an CM form, $K$ an imaginary quadratic field, $p$ splits in $K$. Let $g \rightarrow \chi$ be a Hecke character of $K$ of Hodge-Tate type $(l-1, 0)$,  then
\[
\displaystyle K \subset \ol{\bQ} \rightarrow \ol{\bQ}_p \rightarrow \chi_{\mathfrak{p}} : G_K \rightarrow \ol{\bQ}_p^x.
\]

$I_p = I_{\mathfrak{p}}$, $\chi_{\mathfrak{p}} | I_{\mathfrak{p}}$ = Frobinous character $\times \varepsilon_{\text{cyc}}^{1 - l}$, $\chi_{\mathfrak{p}}^c | I_{\mathfrak{p}}$ is trivial. \\ \\
$\rho_g = \operatorname{Ind}_{G_K}^{G_\bQ} \chi_{\mathfrak{p}}$, 
$$\rho_g | I_{\mathfrak{p}} = \begin{pmatrix} 1 & \cdots & \ast \\ \vdots & \ddots & \vdots \\ 0 & \cdots & \varepsilon^{1 - \ell} \times \text{roots} \end{pmatrix}$$
then $G$ is family of CM-forms, and thus we have two $p$-adic $L$-functions:
\begin{enumerate}
\item[(I)] 
For $\sum(f, \chi, \varepsilon^n\psi)$, when $k-1 \ge n \ge l$ we have $L^{\textrm{I}}$.
\item[(II)]
For $\sum(f, \chi, \varepsilon^n\psi)$, when $l-1 \ge n \ge k$ we have $L^{\textrm{II}}$.
\end{enumerate}
\subsection{Selmer conditions}
{\ }

I) $F^+ \rho = F^+ \rho_F \otimes \operatorname{Ind}_{G_K}^{G_\bQ} \chi_\rho$, $\Sel(\bQ, \rho_x) \subset H^1 (\bQ, \rho_F \otimes \operatorname{Ind}_{G_K}^{G_\bQ} \chi_{\mathfrak{p}}) = H^1(K, \rho_F \otimes \chi_{\mathfrak{p}})$. 
$$\Sel(\bQ, l_x) = \{\ker(H^1(k, \rho_F \otimes \chi_\mathfrak{p}) \rightarrow H^1 (K, l_F / F^+ \rho_F \otimes \chi_\mathfrak{p}).$$
$\Gamma_K = \operatorname{Gal}(K_\infty/K)$, maximal $\bZ_p$ extension of $K$, isomorphic to $\Gamma_+ \times \Gamma_{-}$. \\ \\
I) $\Sel(\bQ, \rho_F \otimes \rho_G \otimes \Lambda) = \Sel^I(K_\infty, \rho_F)$, usual Selmer condition for $F$. \\ \\
II) $F^+\rho_x = \rho_F \otimes F^+ \rho_g = \rho_F \otimes \mathcal{O}$, $\rho_x/ F^+ \rho_x = \rho_F \otimes \varepsilon^{1 - l}$, 
$$\displaystyle \rho_g |_{I_\mathfrak{p}} = \begin{pmatrix} 1 & 0 \\ 0 & \varepsilon^{1 - l} \end{pmatrix}$$
implies
$$\displaystyle \Sel^{\text{II}}(\rho_x) = \ker(H^1(K, \rho_F \otimes \chi_{\mathfrak{p}} ) \rightarrow H^1 (K_\mathfrak{p}, \rho_g \otimes \chi_\mathfrak{p})$$

\begin{thm} (Skinner, Urban ) Iwasawa-Greenberg conjecture is true for $\Sel^{\textrm I}(K^\infty, \rho_F).$
\end{thm}

\begin{thm} (X. Wan ) Iwasawa-Greenberg conjecture is true for $\Sel^{\textrm{II}}(K^\infty, \rho_F).$
\end{thm}

The main idea in the proofs of these theorems is to study Eisenstein congruences. 

\begin{rmk} $f$ eigenfunction inside the Hida family ${F}$. The Iwasawa Main Conjecture implies $\Sel^I(K_\infty, f) \rightarrow L^I(K_\infty, f)$. Kato has a theorem implies that $\Sel(\bQ_\infty, f) \rightarrow L_p(f, s)$. 

\[
L_p(f,s) \times L_p(f \otimes \left(\tfrac{\cdot}{K/\bQ}\right), s) \rightarrow \Sel(F) \oplus \Sel(F \otimes\left(\tfrac{\cdot}{K/\bQ}\right)).
\]
\end{rmk}

(I) One looks at Eisenstein series for $U(2,2)$, $E(\pi, \chi)$, its Galois representation is going to be of the form $\sigma_\pi \oplus \chi_{\mathfrak{p}} \oplus \chi_{\mathfrak{p}}^{-c} \det(\sigma_\pi)$. 

\begin{rmk} $H^1(K, \rho \otimes \chi^{-1})$
$$\displaystyle 0 \rightarrow \rho \otimes \mathcal{O}/\omega^n \rightarrow W \rightarrow \mathcal{O}/\ol{\omega}^n (\chi) \rightarrow 0.$$
Ribet's idea: Find a Galois representation $r$ such that
\begin{enumerate}
\item[(1)]
$r$ is irreducible, $r: G_K \rightarrow \GL_{n+1}(\mathcal{O})$ and 
\item[(2)]
$\ol{r} = (r \pmod{\ol{\omega}}) = \ol{\rho} \oplus \ol{\chi}$. Then construct a lattice $\mathcal{L}$ in $r$ which is non-split.
\[
 0 \rightarrow \rho \otimes \mathcal{O}/ \ol{\omega}^n \rightarrow \mathcal{L}/\omega^n \mathcal{L} \rightarrow \mathcal{O}/\omega^n (\chi) \rightarrow 0.
 \]
If moreover we know that $r$ is ordinary, then $r_{\mathcal{L}} \pmod{\ol{\omega}^n}$ is ordinary as well. $(r,V)$ $\operatorname{Fil}^i V/\operatorname{Fil}^{i+1}$ action of $I_p$ is given by $\varepsilon_{\text{cyc}}^i$.  
\end{enumerate}
\end{rmk}

Look for cusp forms $E(\pi, \chi) \pmod{\ol{\omega}^n}$. Then
$$\displaystyle \operatorname{tr}(\rho_\sigma) \equiv \operatorname{tr}(\rho_H) + \chi + \chi^{-c} \det(\rho_\pi) \pmod{\ol{\omega}^n}.$$

the important point of the construction of the Eisenstein series is given by $E(\pi, \chi) \pmod{ \frac{L(\pi \otimes \chi^{-1}, 0)}{\text{period}}}$ looks like a cusp form. \\ \\
Consequences: 
\begin{enumerate}
\item[(I)]
$L(E,1) = 0$ implies $\Sel(\bQ, E)$ is of rank at least 1. Selmer rank equal zero implies $L(E,1) \ne 0$.
\item[(II)]
$L^{\textrm{II}}(E/K, 1) = 0 \Rightarrow \Sel^{\textrm{II}}(k, E)$ is of corank at least 1. 
\end{enumerate}
$$\displaystyle \ker(H^1(K, V_p E/T_p E) \rightarrow H^1(K_\mathfrak{p}, V_p E/T_p E)),$$
\begin{align*}
H_{\mathfrak{p}}^1(K, V_p E) :&= \ker(H^1(K,  V_p E) \rightarrow H^1(K_\mathfrak{p}, V_p E))\\
H_F^1(K, V_p E) :&= \ker(H^1(K, V_p E) \rightarrow H_F^1(K_\mathfrak{p}, V_p E)) \oplus H_f^1(K_{\mathfrak{p}^c}, V_p E)
\end{align*}
\begin{lem} (Skinner) if $H_f^1(K, V_p(E)) \rightarrow H_f^1(K_p, V_p(E))$ and the rank of $H_f^1(K, V_p(E)) = 1$, then 
$$\displaystyle H_{\mathfrak{p}}^1(K, V_p(E)) = 0.$$
\end{lem}

\begin{thm} (Skinner) Assume $H_f^1(K, V_p(E)) \rightarrow H_f^1(K_p, V_p(E))$ and rank $H_f^1(K, V_p(E)) = 1$, then the analytic rank is $1$.
\end{thm}
\renewcommand{\thesubsection}{\arabic{section}.R}
\begingroup
\renewcommand{\addcontentsline}[3]{}% Remove functionality of \addcontentsline
\endgroup

%%%4
\newpage
\renewcommand{\thesubsection}{\arabic{section}.\arabic{subsection}}

\section{Arithmetic invariants of elliptic curves \emph{on average} (2/2) \\
by Manjul Bhargava}\label{4}

Recall from Lecture~\ref{1}: 
\[
\Sel_n(E_{A,B}) \hookrightarrow \left[ G(\bZ) \backslash V(\bZ)\right]_{A,B}.
\]
How to count elements in the right hand side such that $H(A,B) = \max\{4|A|^3, 27B^2\} < X$ when proving Theorem~\ref{T1.4}
\begin{enumerate}
\setcounter{enumi}{1}
\item
Construct fundamental domains for $G(\bZ)$ on $V(\bR)$.
\begin{enumerate}[\normalfont (i)]
\item
Construct a fundamental domain $L$ for $G(\bR)$ on $[V(\bR)]_{H=1}$ such that $L$ is bounded in $V(\bR)$.  
\item
Construct a fundamental domain $\mathcal{F}$ for $G(\bZ)$ on $G(\bR)$ that is contained in a `Siegel set'. That is, $\F = N' A' K$ (the Iwasawa decomposition) where $N'$ is a bounded set of lower triangular matrices, $A'$ is a set of diagonal matrices, and $K$ is a compact subgroup. \\ \\
Example:
If $G = \operatorname{SL}_2$, then
\begin{align*}
 N'(t) &= \left \{ \begin{pmatrix} 1 & 0 \\ n(t) & 1 \end{pmatrix} : |n(t)| \leq 1/2 \right \},\\
A'(t) &= \left \{ \begin{pmatrix} t^{-1} & 0 \\ 0 & t \end{pmatrix} : t \geq 3^{1/4}/2^{1/2} \right\}\\
K(t)& = \operatorname{SO}_2.
\end{align*}
\end{enumerate}
\end{enumerate}

Then for any $g \in G(\bR)$, $\lambda \F g L$ is a fundamental domain for $G(\bZ)$ on $[V(\bR)]_{H=1}$. 
\begin{proof} By symbolic manipulation. 
$$\displaystyle [G(\bZ) \backslash G(\bR)] \times [G(\bR) \backslash V(\bR)] = G(\bZ) \backslash V(\bR).$$
\end{proof}
\begin{enumerate}
\setcounter{enumi}{2}
\item 
How to count points in $\Lambda \F g L$ of bounded height? (Here $\Lambda = \{\lambda: \lambda > 0\}$.) 
\end{enumerate}
\subsection{Averaging method}
{\ }

Choose $g \in G_0$ where $G_0$ is compact in $G(\bR)$. Let $N(V; X) = \#$ of \emph{generic} (corresponding to $n$-Selmer elements of order $n$) $G(\bZ)$-orbits on $V(\bZ)$ of height less than $X$:
\begin{itemize}
\item
for binary quartic forms ``generic'' means no rational root,
\item
 for ternary cubic forms ``generic'' means no rational flex. 
 \end{itemize}  
 We write
$$\displaystyle N(V; X) = \frac{\int_{g \in G_0} \# \{v \in \Lambda \F g L \cap V(\bZ)^{\text{gen}} : H(v) < X \} dg}{\int_{g \in G_0} dg}.$$
By switching order of integration, we have
$$\displaystyle N(V; X) = \frac{\int_{g \in \F \Lambda} \# \{v \in g G_0 L \cap V(\bZ)^{\text{gen}} : H(v) < X\} dg}{\int_{g \in G_0} dg}.$$
Partition $\F$ into ``cusp part'' and ``main body'' based on $A'$.  Most lattice points in the cusp are not generic. On the contrary, most points in the main body are generic. After making this observation rigorous, we see that
$$\displaystyle \frac{\int_{g \in \F \Lambda} \# \{v \in g G_0 L \cap V(\bZ)^{\text{gen}} : H(v) < X\} dg}{\int_{g \in G_0} dg} = \operatorname{Vol}(\{\Lambda \F g L : H < X\}) + o(X^{5/6}).$$
\begin{rmk} Order 2 elements in $2$-$\Sel$ are non-generic.
\end{rmk}
\begin{rmk} ``If your analytic approach is not working, there must be an algebraic reason and you should look for it and then take it out by hand.'' - Manjul Bhargava, Nov. 2014. 
\end{rmk}
\begin{lem} Suppose that $f$ is continuous on $V$. Then 
$$\displaystyle \int_{v \in V} f(v)dv = |J| \int_{g \in G} \int_{w \in L} \int_{\lambda > 0} f(gw\lambda) dA dB dg d^\times \lambda.$$
\end{lem}

$$\displaystyle \Avg(\# \text{ of } n \text{-Selmer elements of order } n) = $$ 
\begin{align*}
\int_{\substack{A,B \\ H(A,B) < X}} &\operatorname{Vol}(G(\bZ) \backslash G(\bR)) \frac{\frac{\# E_{A,B}}{n E_{A,B}(\bR)}}{\# E_{A,B}(\bR)[n]} dA dB \\
&\times\prod_p |J|_p \int_{A,B} \operatorname{Vol}(G(\bZ_p)) \cdot \frac{\# E_{A,B}(\bQ_p)/nE_{A,B}(\bQ_p)}{E_{A,B}(\bQ_p)[n]}dA dB.
\end{align*}
Many things cancel, so the final answer is
$$\displaystyle \operatorname{Vol}(G(\bZ) \backslash G(\bR)) \cdot \prod_p \operatorname{Vol}(G(\bZ_p)) = \tau(G) = n \qquad\text{(Poonen: the adelic volume)}.$$
Thus,
$$\displaystyle \Avg(\# \Sel_n(E_{A,B})) = \sigma(n).$$

\begin{cor} $\Avg \operatorname{Rank} (E_{A,B}) \leq 1.05.$
\end{cor}

\begin{proof} $\Avg(20r - 15) \leq \Avg(5^r) \leq 6$, so $\Avg(r) \leq 21/20 = 1.05.$ Equality happens if and only if $95\%$ have rank 1 and $5\%$ have rank 2. 
\end{proof}

\begin{thm} (Arul's lecture) There exists a family of congruences of $55\%$ of all $E_{A,B}$ where root number is equidistributed. \\ \\
If we add Dokchitser's theorem, then we have lots of curves have even 5-Selmer rank and also lots of curves have odd 5-Selmer rank. This improves the upper bound to $0.885$ (it comes from $0.55 \times 0.75 + 0.45 \times 1.05$). 
\end{thm}

\begin{thm} (with C. Skinner and W. Zhang )
{\  }

We have the following statistics for elliptic curves $E_{A,B}$ ordered by height.
\begin{itemize}
\item
At least $16.5\%$ of $E_{A,B}$ have rank and analytic rank $0$, 
\item
at least $20.6\%$ of $E_{A,B}$ have rank and analytic rank $1$, and most significantly 
\item[$\star$] at least $66.48\%$ of $E_{A,B}$ satisfy BSD.  
\end{itemize}
\end{thm}
\renewcommand{\thesubsection}{\arabic{section}.R}
\begingroup
\renewcommand{\addcontentsline}[3]{}% Remove functionality of \addcontentsline
\endgroup

%%%5.
\newpage
\renewcommand{\thesubsection}{\arabic{section}.\arabic{subsection}}

\section{Asymptotics and averages for families of elliptic curves with marked points
by Wei Ho}\label{5}

In Lecture~\ref{4}, Bhargava discussed a family of elliptic curves given by an equation of the form 
\[
y^2 = x^3 + Ax + B = x^3 + a_4 x + a_6, \quad\text{ where }a_4, a_6 \in \bZ,\text{ and }\Delta \ne 0.
\]
Further, We require a minimality condition, so that there is no prime $p$ for which both $p^4 | a_4$ and $p^6 | a_6$. We denote this family by $\F_0$. Recall the naive height $H(A,B) := \max\{4|A|^3, 27B^2\}$ and the theorem of Bhargava-Shankar:

\begin{thm} \label{5T1} (Bhargava-Shankar) 

The average size of $\Sel_n$ for elliptic curves in $\F_0$ is $\sigma(n)$ for $n = 2, 3, 4, 5$ when ordered by height.
\end{thm}
We have other families. Namely:
\begin{itemize}
\item 
$\F_1:$ $y^2 + a_3 y = x^3 + a_2 x^2 + a_4 x$, with one marked point. 
\item
$\F_2:$ $y^2 + a_2 xy + a_6 y = (x - a_4)(x - a_4')(x - a_4'')$, $a_4 + a_4' + a_4'' = 0$, with two marked points.
\item
$\F_0(2):$ $y^2 = x^3 + a_2 x^2 + a_4 x$, with a $2$-torsion point (over $\bQ$)
\item
$\F_0(3):$ $y^2 + a_1 xy + a_3 y = x^3$, with a $3$-torsion point (over $\bQ$).
\item
$\F_1^{\bQ(\sqrt{d})}$, $\F_{1 \in \bQ(\sqrt{d})}$, $p + \ol{p} \ne 0$. 

\end{itemize}

\subsection{Selmer results}
{\ }

For each pair $(\F, p)$, we can compute  $\Avg \Sel_p$ over that family. We have the following results

\begin{itemize}
\item
$\Avg \Sel_2 (\F_0) =  3, \Avg \Sel_3 (\F_0) = 4, \Avg \Sel_4 (\F_0) = 7, \Avg \Sel_5 (\F_0) = 6$. 
\item
$(\F_1, 2, 6), (\F_1, 3, 12).$ 
\item
$(\F_2, 2, 12)$ 
\item
$(\F_0(2), 3, 4).$
\item
$\Avg \Sel_2 (\F_0(3)) = 3$, 
\item
$\Avg \Sel_3(\F_1^{\bQ(\sqrt{d})}) = 4$. 
\end{itemize}
\begin{rmk} 
{\ }
\begin{itemize}
\item
Matches heuristics. 
\item
Points are independent.
\end{itemize}
\end{rmk}

\begin{proof} 
{\ }

(1) Given $G$ an algebraic group, $V$ a representation of $G$, we want a correspondence between
\[
V(\bQ)/G(\bQ),\text{ a coarse moduli space}, \leftrightarrow  \{
\in \F, C \text{ an } E\text{-torsor}, L \text{ degree } n \text{ line bundle on } C.
\]
\begin{rmk} 
{\ }
\begin{itemize}
\item
$V//G \cong \text{ moduli space for whatever family (affine space for these families)}$. 
\item
$V//G'$ corresponds to a weighted projective space.
\item
Stabilizers correspond to automorphism groups.
\end{itemize}
\end{rmk}

(2) Find integral representatives (for locally soluble orbits). Take orbit in $V(\bQ)/G(\bQ)$ with integer invariants. We want an element of $V(\bZ)$ with those invariants. 
\begin{rmk} Can get stuck here.
\end{rmk}

(3) Find fundamental domains. This is typically easy. \\ \\
(4) ``generic'' or ``irreducible'' elements. Typically the non-generic points lie in the cusp. \\ \\
(5) Count generic or irreducible orbits using geometry of numbers. \\ \\
(6) Apply a sieve (more complicated, especially with $p=2$). \end{proof}


\subsection{Corollaries and non-corollaries}
\begin{xmp}
In $\F_1$, there is a positive proportion of elliptic curves with rank 1 or 2. 
\end{xmp}

Why not just rank 1? Because we cannot easily find a sub-family with equidistributed root number. If we somehow have a positive portion of rank 1's, we would like to know whether we have a positive portion of curves with rank equalling analytic rank. This cannot be done for rank 2, as the current techniques seem to get stuck at rank 1. 

\begin{xmp} In $\F_0(3)$m $p$-adic methods are bad for $p = 2$. 
\end{xmp}

\begin{xmp} In $\F_0(2)$, there is a positive proportion having rank 0 or 1.
\end{xmp}

\begin{thm} (Skinner-Urban) \label{5T2} Let $E/\bQ$ be an elliptic curve with good reduction. Then subject to some $p$-adic conditions, we have
$$\displaystyle \Sel_p(E) = 0 \Rightarrow \text{rank} = \text{analytic rank} = 0.$$
\end{thm}

\begin{thm} (Skinner-Wan, Bertolini-Darmon-Prasanna ) \label{5T3} $E/\bQ$ good (ordinary) reduction, and subject to some $p$-adic conditions, we have
$$\displaystyle \Sel_p(E) \cong \bZ/p\bZ \Rightarrow \text{rank} = \text{analytic rank} = 1.$$
\end{thm}

In $\F_0(2)$, one can prove that a proportion of at least $5/8$ of curves have $\Sel_3$ rank equalling 0 or 1. Note that we get good reduction $4/9$'s of the time ($\Delta = 16a_4^2(-4a_4 + a_2^2)$) and equidistribution ($\Sel_p(E) \rightarrow E(\bQ_p)/pE(\bQ_p)$, we obtain that at least $13.89\%$ of $\F_0(2)$ satisfy conditions of Theorems \ref{5T2} and \ref{5T3}. 
\renewcommand{\thesubsection}{\arabic{section}.R}
\begingroup
\renewcommand{\addcontentsline}[3]{}% Remove functionality of \addcontentsline
\endgroup

%%%6.
\newpage
\renewcommand{\thesubsection}{\arabic{section}.\arabic{subsection}}

\section{Experiments with Arakelov class groups and ranks of elliptic curves \\
by John Voight}\label{6}

The basic theme of this Lecture is: how do archimedean considerations come into into play into heuristics for class groups and ranks of elliptic curves? 

\subsection{Basic Cohen-Lenstra heuristics}
{\ }

Recall the Cohen-Lenstra heuristics  predict the probability that the class group $\operatorname{Cl}(D)$ of an imaginary quadratic field of (fundamental) discriminant $D < 0$ has a given $p$-Sylow subgroup for $p$ odd.  To each abelian $p$-group $G$, we assign the weight 
\[
w(G) := \frac{1}{\# \operatorname{Aut}(G)}.
\] 
This weight is natural comes from many sources:
\begin{itemize}
\item if $G$ is an abelian group of order $n$ and $X$ is a set with $\# X = n$, then the number of group structures on $X$ is isomorphic to $G$ is $n!/\# \operatorname{Aut}(G) = n! w(G).$ 
\item with $w(G) = 1/\# \Aut(G)$, summing over abelian $p$-groups (Hall) , we obtain
\[
\sum_G w(G) = \prod_{n=1}^\infty (1 - p^{-n})^{-1} = \eta(p),
\]
and Cohen-Lenstra then predicted that
\[ \lim_{X \rightarrow \infty} \frac{\# \{0< - D < X: \operatorname{Cl}(D) [p^\infty] \cong G\}}{\# \{0 < -D < X\}} = w(G) \frac{1}{\eta(p)}.
 \]
\end{itemize}
Hence, for example, $\operatorname{Cl}(D)[3^\infty] \cong \bZ/9 \bZ$ occurs eight times more frequently than $\operatorname{Cl}(D)[3^\infty] \cong \bZ/3\bZ \oplus \bZ / 3\bZ$. As a consequence, the average size of $\Cl(D)[p]$ is 2 for all $p$; equivalently, on average $\Cl(D)$ has 1 element of order (exactly) $p$. 

\subsection{Cohen-Lenstra computations}
{\ }

To test these heuristics, we sampled $10,000$ fundamental discriminants $D$ with $0 < -D < 10^{10}$ at random, , and found that the average number of elements of order $p$ is:

$$
\Avg(3) = 0.97,\quad \Avg(5) = 1.03,\quad \Avg(7) = 1.02,\quad \Avg(11) = 0.97.$$ 

It is harder to confirm that $\Cl(D)[3^\infty] \cong \bZ/9\bZ$ occurs eight times more frequently than $\Cl(D)[3^\infty] \cong (\bZ/3\bZ)^2$: in a sample of $100,000$ discriminants, we find the ratio 
\[
\frac{281}{27} \sim 10. 
\]

\subsection{Cohen-Lenstra for real quadratic fields}
{\ }

The situation for real quadratic fields is slightly more complicated. Intuitively, the class group of a real quadratic field is smaller than that of an imaginary quadratic field due to the presence of the fundamental unit, and this unit gives an extra relation: so we should ``modulo out by a random element'': Specifically, first pick a random finite abelian $p$-group $G$ with weight $w(G)$, and then modulo out by a random element. (If $G$ is cyclic, one often gets a trivial group). \\ \\
\begin{rmk}
This prediction seems to give the right answer. It is made plausible by thinking about the function field analogue, specifically hyper elliptic curves: a ``real'' hyper elliptic curve ($y^2 = f(x)$ with $f(x)$ of odd degree) has a unique point at infinity, so the class group of the affine coordinate ring is the quotient of the Jacobian by a ``random'' point. Or, think in terms of lattices and Arakelov class groups. 
\end{rmk}
\subsection{Cohen-Lenstra heuristics via lattices}
{\ }

Another natural way to produce the Cohen-Lenstra weighting is by considering random lattices. If $G$ is a finite abelian group, then the number of lattices $L \subset \bZ^n$ such that $\bZ^n/L \cong G$ is asymptotic to $(\# G)^n/\#\Aut(G)$ as $n \rightarrow \infty$. \\ \\
This observation extends quite a bit. \\ \\
Friedman-Washington showed that if $M \in M_n(\bZ)$ is a random matrix with i.i.d. entries chosen according to Haar measure, then the cockerel distribution of $M$ converges to the Cohen-Lenstra measure as $n \rightarrow \infty$. \\ \\
More generally, one can show that if $M \in M_n(\bZ/N\bZ)$ is a random matrix with iid entries, then the cockerel distribution of $M$ converges to the Cohen-Lenstra measure for all finite $\bZ/N\bZ$ modules $G$. 

\subsection{Class groups as cockerels}
{\ }

It is plausible to model the class group of a number field $K$ by such cockerels for the following reason. \\ \\
The class group $\Cl(K)$ is the quotient of the group of fractional ideals modulo principal ideals. \\ \\
Let $S$ be a factor base consisting of all prime ideals $\mathfrak{p}$ satisfying $N \mathfrak{p} \leq B$ for some smoothness bound $B$. Every $\alpha \in \bZ_K$ whose norm factors into primes in $S$ gives a relation. If $L$ is the lattice spanned by the set of relations, and $B$ is big enough ($B \geq 6 \log^2 |d_K|$ suffices on the GRH), then
\[\Cl(K) \cong \bZ^{\# S} /L. \]
\subsection{Archimedean normalization for Cohen-Lenstra}
{\ }

If we model $\Cl(D)$ for $D < 0$ as the cockerel of a random matrix $M \in M_n(\bZ)$ with iid entries in $[-X,X]$, then we expect that
\[ \#\Cl(D) = \det M \sim n! X^n \]

(In this Lecture, we do not make the $\sim$ rigorous). But by the Brauer-Siegel theorem, we have
\[ \#\Cl(D) \sim \sqrt{|D|} .\]
(This estimate is pretty good on average.) Therefore
\[n! X^n \sim \sqrt{|D|}.\]
As $X, n \rightarrow \infty$, this model the usual Cohen-Lenstra heuristics for the distribution of the $p$-Sylow subgroups. 


\subsection{Arakelov class groups}
{\ }

Returning to the real quadratic case, we now keep track of units as well. \\ \\
Let $K$ be a number field. The \emph{Arakelov divisor group} of $K$ is
\[ \operatorname{Div}(K) = \bigoplus_{\mathfrak{p} < \infty} \bZ \oplus \bigoplus_{\sigma | \infty} \bR. \]
This gives a group analogous to the case where $K$ is the function field of a curve $X$ over $\bF_q$. \\ \\
A principal Arakelov divisor is a divisor of the form
\[(f) = \sum_{\mathfrak{p}} \operatorname{ord}_{\mathfrak{p}}(f) [\mathfrak{p}] + \sum_\sigma (-\log |\sigma(f)|)[\sigma] \in P(K) \]
for $f \in K^\times$. We define the \emph{degree} map
\[ \deg: \operatorname{Div}(K) \rightarrow \bR\]
\[\deg(\mathfrak{p}) = \log N \mathfrak{p} \]
\[\deg[\sigma] = 1 \text{ or } 2 \text{ according if } \sigma \text{ is real or complex}. \]
The product formula implies that $\deg(f) = 0$ for $(f) \in P(K)$. \\ \\
Let $\Div^0(K) = \ker \deg$ and $\operatorname{Pic}^0(K) = \Div^0(K)/P(K)$. Then we have an exact sequence
\[0 \rightarrow T^0(K) \rightarrow \operatorname{Pic}^0(K) \rightarrow \Cl(K) \rightarrow 0 \]
where $T^0$ is the compact topological group
\[T^0 \cong \left(\prod_{\sigma} \bR \right)^0 / \log |\bZ_K^\times|.\]
Now suppose $K = \bQ(\sqrt{D})$ is real quadratic. Then $\log |\bZ_K^\times|) = (\log |\epsilon|) \bZ$ with $\epsilon \in \bZ_K^\times$, so
\[ T^0(K) \cong \bR/(\log |\epsilon|) \bZ \]
is a circle group, and we have
\[0 \rightarrow \bR/(\log |\epsilon|) \bZ \rightarrow \operatorname{Pic}^0(K) \rightarrow \Cl(K) \rightarrow 0 \]
The size of $\operatorname{Pic}^0(K)$ is 
\[\left \lvert \operatorname{Pic}^0(K) \right \rvert = \# \Cl(K) \log |\epsilon| = hR = L(1, \chi) \frac{\sqrt{D}}{2}.\]
Typically, we expect $L(1, \chi) = O(D^\epsilon)$ for all $\epsilon > 0$, so
\[\left \lvert \operatorname{Pic}^0(K) \right \rvert \sim \sqrt{D}/2.\]
\subsection{Cohen-Lenstra heuristics, redux}
{\ }

By analogy with the imaginary quadratic case, we model the Arakelov class group of a real quadratic field $K = \sqrt{D}$ as a random homomorphism 
\[ \bZ^n \rightarrow (\bZ^{n-1} \times \bR^2)^0 \cong \bZ^{n-1} \times \bR, \]
represented by a matrix $M$ whose entries lie in $[-X,X](\cap \bZ)$ and subject to the normalization
\[ (\det M \sim) n! X^n \sim \sqrt{D}. \]
The map $\operatorname{Pic}^0(K) \rightarrow \Cl(K)$ is modelled by forgetting the last column of $M$ (having real entries); in this way, we recover the Cohen-Lenstra model of ``modelling out by an random element''. \\ \\ 
We expect to get the same answer if we instead model with a matrix with all integer entries. 
\subsection{Arakelov class groups: computations}
We consider random maps $\bZ^n = \bZ^4 \rightarrow \bZ^3 \times \bR$ represented by matrices $M$ whose entries lie in $[-X, X] = [-13, 13]$. In $100,000$ trials, we find that the average determinant of such a matrix is about $12,000$, so $\sqrt{D}/2 \in [0, 24,000]$. Therefore, this should model the Arakelov class group for discriminants $D \sim 48,000^2$. We consider $10,000$ random such discriminants. \\ \\
The average number of elements of order 3 in $\Cl(D)$ is $1/3$ (Davenport-Heilbronn ). The model gives $1.32$, and the actual count is $1.30$. \\ \\
The average regulator: our model gives 7700, the actual count is 8500. \\ \\
For instead $n = 8$ and $X = 3$, we have $\det \sim 160,000$ and regulator: Model gives $95,000$ and the actual count is $94,000$. 


\subsection{Unit signatures}
{\ }

But wait! Returning to a general number field $K$, we an only recover $f \in K^\times$ from its divisor $(f)$ up to a root of unity. (Joint work with Dummit ). \\ \\
Suppose that $K$ is totally real. We define the \emph{signature} map by
\[\sgn: K^\times \rightarrow \{\pm 1\}^n \cong (\bZ / 2\bZ)^n \]
\[f \mapsto (\sgn (\sigma(f))_{\sigma | \infty}. \]
We define the \emph{narrow Arakelov class group} $\Pic^{+0}(K)$ in the obvious way; we obtain
\[0 \rightarrow T^0(K) \rightarrow \Pic^{+0}(K) \rightarrow \Cl^+(K) \rightarrow 0 \]
where $\Cl^+(K)$ is the narrow class group of $K$ so that 
\[0 \rightarrow \bZ_{K,+}^\times / \bZ_K^{\times 2} \rightarrow \Cl^+(K) \rightarrow \Cl(K) \rightarrow 0.\]


\subsection{Signature rank}
{\ }

\newcommand{\sigrk}{\operatorname{sigrk}}
We define the \emph{signature rank} $\sigrk(\bZ_K^\times)$ of $K$ to be the rank of the signature map restricted to $\bZ_K^\times$. \\ \\
We have $\sigrk(\bZ_K^\times) \geq 1$ since $-1 \in \bZ_K^\times$, and $\sigrk(\bZ_K^\times) = 1$ if and only if $K$ possesses a fundamental system of units that are totally positive. \\ \\
So we are led to ask: what is the distribution of signature ranks over totally real fields of a fixed degree $d$?

\subsection{Armitage-Frohlich (AF)}
{\ }

Before modelling the narrow Arakelov class group, there is one important restriction: by the theorem of Armitage-Frohlich , we have
\[ \left \lceil \frac{[K : \bQ]}{2} \right \rceil - \text{rk}_2 \Cl(K)[2] \leq \sigrk(\bZ_K^\times). \]
This theorem arises from the existence of the canonical Kummer (norm residue) pairing; in the function field case, it is the Tate pairing. \\ \\
The pairing is canonical, so we do not model it separately; instead, it implies an extra compatibility and we just require that condition (AF) is satisfied fro each narrow Arakelov class group.

 
\subsection{Heuristics for signature ranks}
{\ }

Therefore, we model the narrow Arakelov class group of a totally real field of degree $d > 2$ as a random map
\[ \bZ^n \rightarrow \bZ^{n-(d-1)} \times \bR^{d-1} \times (\bZ / 2 \bZ)^d \]
represented by a matrix $M$ whose first $n$ rows belong to $[-X, X] (\cap \bZ)$ satisfying the following conditions: \\ \\
(N) The vector $(\textbf{0}, \textbf{0}, \textbf{1})$ is in the image of $M$. \\ 
(AF) Let $M_\bZ$ be the matrix keeping only the $\bZ$-columns, so cocker $M_\bZ$ models $\Cl(K)$. Let $\mathfrak{s}(\ker M_\bZ) \leq (\bZ/2\bZ)^d$ be the signed components (modelling the image of $\sgn(\bZ_K^\times)$). Then
\[ \lceil d/2 \rceil - \rank_2 \operatorname{coker} M_\bZ \leq \rank_2 \mathfrak{s}(\ker M_\bZ).\]


\subsection{Computations}
{\ }

To test this conjecture, we consider totally real cubic fields $K$ (computed by Michael Novick). To simplify, we consider a conditional probability, and we restrict to fields with odd class number $\# \Cl(K)$: vanilla Cohen-Lenstra predicts that this should happen for a large constant proportion of fields. for the 65 million cubic fields with discriminant $d_K \leq 10^9$, approximately 83$\%$ had odd class number. \\ \\
The Armitage-Frohlich (AF) condition then implies that $\sigrk(\bZ_K^\times) \ne 1$ (there cannot be a  totally positive system of fundamental units) and we can ask about the distribution of signature ranks 2, 3. \\ \\
Our heuristic implies that rank 2 should occur with probability 3/5 and rank 3 should occur with probability 2/5. Of the 54 million cubic fields, we find percentages $58.6\%$ and $41.4\%$. Also see the work of Bhargava. 

\subsection{Ranks of elliptic curves: basic heuristic}
{\ }

The archimedean normalization of Cohen-Lenstra heuristics is a warm-up for our (PPVW) heuristics for elliptic curves. \\ \\
See Bjorn Poonen's Lecture~\ref{20} for a heuristic for the rank of a random elliptic curve
\[E: y^2 = x^3 + Ax + B \text{ over } \bQ \text{ of height } H = \max(4|A|^3, 27B^2). \]
In brief: we take $n$ of moderate size with random parity; we choose $X$ such that $n! X^n \sim H^{1/2}$ and we compute the rank of the kernel of a random $n \times n$ alternating $M \in M_n(\bZ)$ with entries in $[-X, X]$. \\ \\
In the end, we predict that for each $r \geq 1$, the probability that $E$ of height $H$ has rank $\geq r$ is approximately $1/H^{(r-1)/24}.$\\ \\
The setup above says how we should model the class group and regulator together. Arguing by analogy, this gives a second way to arrive at our calibration, modelling the Shafarevich-Tate group and the elliptic regulator together.


\subsection{A few computations}
{\ }

Bektemirov-Mazur-Stein-Watkins  discuss the tension between data and conjecture for ranks of elliptic curves in some detail. \\ \\
We consider instead some statistical sampling as follows. we take elliptic curves of height $H \in [X, X+X/100]$ and compute their analytic ranks. 

\[ \begin{array}{c|cccc} 
\hline
\approx X & \text{rank } 0 & \text{rank } 1 & \text{rank } \ge 2 & \text{rank } \ge 3 \\
\hline
10^8 & 32\% & 48\% & 18\% & 2\% \\
10^{10} & 33\% & 48\% & 17\% &2 \% \\
10^{12} & 33\% & 48 \% & 16 \% &2 \% \\
\hline
\end{array}
\]
For what it's worth, $1/2 \cdot 10^{-10/24} = 19\%$ and $1/2 \cdot 10^{-20/24} = 7\%$. \\ \\
The evidence is weak, but at least the percentage of rank at least 2 appears to be going down. Further computations are in progress, using a conditional method to bound analytic ranks by Bober  (going back to work of Mestre and Fermigier ). 

\subsection{Final words}
{\ }

In this Lecture, we have tried to convince you that it is a reasonable philosophy for arithmetic objects to be modelled by kernels and cockerels of integer matrices whose size is normalized by archimedean ($L$-function) considerations. \\ \\
For more on heuristics for elliptic curves, see Lecture~\ref{20}!

\renewcommand{\thesubsection}{\arabic{section}.R}
\begingroup
\renewcommand{\addcontentsline}[3]{}% Remove functionality of \addcontentsline
\endgroup
%%7.
\newpage
\renewcommand{\thesubsection}{\arabic{section}.\arabic{subsection}}

\section{Counting simple knots via arithmetic invariant theory \\
by Allison Miller}\label{7}
Recall: $1$-knots are embeddings of the circle $\bS^1$ in $\bS^3$ which are equivalent up to some topological equivalence, whose precise form is not of relevance.  We formulate the definition of an $n$-knot as follows:
\begin{dfn}
An $n$-knot $K \subset \bS^{n+2}$ is an embedded submanifold, with $K$ homeomorphic to $\bS^n$. Similarly, this is up to topological equivalence.
\end{dfn}
Knot theory studies:
\begin{itemize}
\item
when are two knots equivalent
\item
what invariants can be used to tell knots apart?
\end{itemize}

Given an $n$-knot $K \subset \bS^{n+2}$, one obtains the \emph{knot complement} $\bS^{n+2} \setminus K$. It turns out in most situations the knot complement contains  all (or most) of the information about the knot. General $n$-knots are too complicated, as to understand them is equivalent to understanding finitely generated groups. It will therefore be prudent to study the family of \emph{simple} knots. 

\subsection{Simple $(2q-1)$-knots}

\begin{dfn} A $(2q-1)$-knot $K$ is simple if 
\[ \pi_i(\bS^{2q+1} \setminus K) = \pi_i(\bS^1) \]
for $i \leq q$, where $\pi_i$ is the $i$th homotopy group, with $\pi_1$ being the fundamental group.
\end{dfn}

\subsection{Arithmetic invariants}
{\ }

Fox and Smythe constructed a knot invariant that is an ideal class . This comes from the Alexander module for classical knots. In the case of $3$-knot $K$, the knot complement can be covered by an abelian cover $C_\infty$ leading to the covering group $\bZ$. The orientation of the knot gives us a canonical generator for the group, which we denote by $t$. The Alexander module of $K$ is then defined by $H_1(C_\infty, \bZ)$. The infinite cyclic group $\langle t \rangle$ acts on $\Alex_K$ and so $\Alex_K$ is a $\bZ[t, t^{-1}]$ module. $\Alex_K$ has the following properties:
{\ }
\begin{itemize}
\item
$\Alex_K$ is annihilated by the Alexander polynomial $\Delta(t) \ne 0$. 
\item
$\Delta(1) = 1, \Delta(t^{-1}) = t^{- 2\deg \Delta} \Delta(t)$. If $\deg \Delta = 2$ then $\Delta = mt^2 + (1-2m)t + m$ for some positive integer $m$.  
\item
$\Alex_K$ is a module over the quotient ring 
\[\mathcal{O}_\Delta = \bZ[t, t^{-1}]/\Delta(t). \]
\item
$\mathcal{O}_\Delta \otimes_\bZ \bQ$ is a finite dimensional $\bQ$-algebra. 
\item
If $\Delta = mt^2 + (1-2m)t + m$, then $\mathcal{O}_\Delta = \bZ[t, t^{-1}]/(mt^2 + (1-2m)t + m)$. 
\item
$\Alex_K$ has the property that when $\Delta$ is square-free, $\Alex_K$ is isomorphic as an $\mathcal{O}_\Delta$-module to an ideal of $\mathcal{O}_\Delta$. This gives rise to an arithmetic invariant.
\item
$\Alex_K$ satisfies ``Blanchfiled duality'' and comes with a natural hermitian pairing over $\bZ[t, t^{-1}]$. 
\end{itemize}

\subsection{Motivating questions}
{\ }
\begin{itemize}
\item
Does this arithmetic invariant fit into the context of arithmetic invariant theory?
\item
If so, can we count them?
\end{itemize}

\subsection{Alexander module of a simple $(2q-1)$-knot}
{\ }

Consider a simple $(2q-1)$-knot $K$, with an infinite cover $C_\infty$ for the knot complement $S^{2q+1} \setminus K$ generated by $t$. Then have (in terms of the homology groups)
\[
H_q(C_\infty, \bZ) = \Alex_K,
\]
this is a $\bZ[t, t^{-1}]$-module annihilated by some polynomial $\Delta(t)$, and call $\Delta(t)$ the Alexander polynomial of $K$. 

\begin{thm} (Bayer-Michel, Levine ) There are only finitely many simple $(2q-1)$-knots with a given Alexander polynomial, provided that the polynomial is square-free.
\end{thm}
\begin{thm} (Kearton, Trotter )
For $q > 1$ odd, simple knots are entirely classified by the Alexander modules along with the Blanchfield duality pairing on the Alexander modules. 
\end{thm}

\begin{thm} (Kearton, Levine, Trotter ) 
Algebraic condition for which modules and pairing are realizable. 
\end{thm}

The three theorems above enable us to ``count'' simple knots of square-free Alexander polynomials of a fixed degree with bounded height.

\subsection{Seifert hypersurfaces for knots}
\begin{dfn}
A Seifert hypersurface for a $1$-knot is a surface embedded in $\mathbb{S}^3$ with $\partial V = K$. This generalizes to simple $n$-knots. 
\end{dfn}

\begin{thm} Any simple $n$-knot can be written as $\partial V$ where $V$ is a Seifert hypersurface which is a $2q$-dimensional manifold with boundary and $V$ is $(q-1)$-connected. 
\end{thm}

This theorem is basically saying that ``all topology of $V$ comes from $H_q(V)$, the homology groups'' and thus Seifert hypersurfaces are classified by 
\[
\operatorname{rk}(H_q(V, \bZ)) = 2g,
\]
where $g$ is the genus along with a non-symmetric $\bZ$-valued pairing on $H_q(V, \bZ)$. The skew symmetric part is the intersection pairing which is a perfect pairing. \\ \\
Simple Seifert hypersurfaces are in one-to-one correspondence with $\GL_n$-equivalence class of matrices $P$ such that $\det(P - P^T) = 1$ (i.e. $(M,P) \mapsto MPM^T$). We can always change basis such that 
\[
P - P^T = J = \begin{pmatrix} 0 & -I_g \\ I_g & 0 \end{pmatrix}.
\]
This gives us the $\operatorname{Sp}_{2g}$-equivalence classes of matrices $P$ with $P - P^T = J$, and another change of variables gives $P \mapsto P + P^T = Q$, which leads to the $\operatorname{Sp}_{2g}$-equivalence classes of matrices $Q \in \Sym^2(2g)$ such that $Q \equiv J \pmod{2}$. \\
The underlying representation is 
\[
\Sym^2(2g) \rightarrow \text{ adjoint representations of } \operatorname{Sp}_{2g},
\]
with free ring of invariants generated by coefficients of 
\[ 
\det(J^{-1} P - t I_{2g}) = \det(tJ - P).
\]
The latter polynomial is equal to the Alexander polynomial after a change of variables. \\ 
Simple Seifert hyper surfaces $\rightarrow$ $\operatorname{Sp}_{2g}$-orbits on $\Sym^2(2g)$ (+ parity) $\rightarrow C$ self balanced ideal classes of $R_f = \bZ[y]/f(y)$ $\rightarrow$ characteristic polynomial $= f$. \\ \\
Simple $(2q-1)$-knots (square-free $\Delta$) $\rightarrow$ Alexander module + pairing $\rightarrow$ CSB ideal classes of $\mathcal{O}_\Delta$ (finite-to-one) Alexander polynomials $\Delta(f)$ which leads to characteristic polynomial $=f$. \\ \\
When $\Delta(t) = mt^2 + (1-2m)t + m$ we have $\operatorname{Sp}_2 = \operatorname{SL}_2$ (hypersurfaces) orbits on binary sf's of discriminant dividing $4m$. Knots correspond to binary quartic forms of discriminant $1 - 4m$ over $\bZ[1/m]$. 

\renewcommand{\thesubsection}{\arabic{section}.R}
\begingroup
\renewcommand{\addcontentsline}[3]{}% Remove functionality of \addcontentsline
\endgroup

%%%8.
\newpage
\renewcommand{\thesubsection}{\arabic{section}.\arabic{subsection}}

\section{Arithmetic statistics over global fields\\
by Jerry Xiaoheng Wang}\label{8}

Let $K$ denote a global field, which is either a number field or a function field of characteristic zero over a smooth projective variety. Let $M_\infty$ denote the set of infinite places, or a finite non-empty set of closed points on $C$ in the function field case. $\mathcal{O}$ denote its ring of integers and $K_\infty : =\prod_{v \in M_\infty} K_v$. 
\begin{rmk}
Philosophy: $\bQ$ should correspond to $K$, $\bQ_p$ should correspond to $K_{\mathfrak{p}}$, and $\bZ_p$ should correspond to $\OO_{\mathfrak{p}}$, but $\bZ$ does NOT correspond to $\OO$ in most cases. 
\end{rmk}
The first example is $\Sel_2(E)$.
\begin{itemize}
\item[Step 0]
Height of $E/K$ \\ \\
$E: y^2 = x^3 + Ax + B, A,B \in K$. $(A,B) \in \bP(4,6)(K) = \mathbb{G}_m(K) \backslash \mathbb{A}^2(K)$. For $(A,B) \in S(K)$, let $I = \{\alpha \in K | \alpha (A,B) \in S(\OO)\}$.  \\ \\
$H(A,B) = NI \prod_{v \in M_\infty} \max(|A|_v^{1/4}, |B|_v^{1/6})$ when $|M_\infty| > 1$, the set $S(K_\infty)_{< X}$ is not bounded. What is $\Avg \Sel_n(E)$? 

\item[Step 1] Orbit parametrization \\ \\
$\Sel_2(E/K)$ correspond to locally soluble orbits for the action of $G(K)$ on $V(K)$. 

\item[Step 2] Locally soluble orbits $\rightarrow$ integral orbits (not true, but close). 

\begin{lem} If $v \in V(K_{\mathfrak{p}})^{\text{sol}}$ has invariants in $\OO_{\mathfrak{p}}$, then there exists $g_p \in G(K_{\mathfrak{p}})$ such that $g_p v \in V(\OO_{\mathfrak{p}})$. \\ \\
Suppose $v \in V(K)^{\text{loc sol}}$ with invariants in $\OO$. Then there exists $g_{\mathfrak{p}} \in G(K_{\mathfrak{p}}$ such that $g_{\mathfrak{p}} v \in V(\OO_{\mathfrak{p}})$. 
\end{lem}
\[
(g_\fp) \in G(\bA_f) = \bigcup_{\beta \in \Cl(G)} \left(\prod_{\fp \not \in M_\infty} G(\OO_\fp) \right) \beta G(K),
\]
where $\Cl(G)$ is the class group of $G$ which is finite. 
$\displaystyle (g_\fp)_\fp = (g_\fp') \cdot \beta \cdot h \Rightarrow \beta h v \in V(\O_\fp)$ for all $\fp \not \in M_\infty$. 
$hv \in V_\beta = V(K) \cap \beta^{-1} \left(\prod_{\fp \not \in M_\infty} V(\OO_\fp)\right)$, with $V(\OO) = V_{\beta = 1}$. \\
$G_\beta = G(K) \cap \beta^{-1} \left(\prod_{\fp \not \in M_\infty} G(\OO_\fp)\right)\beta$. 

\begin{pro} Suppose $v \in V(K)^{\text{loc\;sol}}$ has invariants in $\OO$, then there exist $\beta \in \Cl(G)$ such that
\[G(K) v \cap V_\beta \ne \empty\]
For any subgroup $G_0 \leq G(K)$ and any subset $V_0 \subset V(K)$, any real number $X$, let 
\[N(V_0, G_0, X) =\]
\[ \# \left\{ \text{irreducible } G_0 \text{-orbit in } V_0 \text{ of height } < X \text{ where an orbit } G_0 v \text{ is weighted by } \frac{1}{\# \operatorname{Stab}_{G_0}(v)} \right \}. \]
If $m : V(K) \rightarrow [0,1]$ is $G_0$-invariant, defined by congruence conditions, then 
\[N_m(V_0, G_0, X) = \]
\[ \# \left\{ \text{irreducible } G_0 \text{-orbit in } V_0 \text{ of height } < X \text{ where an orbit } G_0 v \text{ is weighted by } \frac{m(v)}{\# \operatorname{Stab}_{G_0}(v)} \right \}. \]
\end{pro} 

\begin{thm} \[N(V(K)^{\textrm{loc sol}}, G(K), X) = \displaystyle \sum_{\beta} N_m(V_\beta, G_\beta, X)\]
where
\[m(v) = \chi_{V(K)^{\textrm{loc sol}}}(v) \frac{1}{\#\operatorname{Stab}_{G(K)}(v)} \left(\sum_\beta \sum_{V_\beta \in G_\beta \setminus V_\beta \cap G(K)v} \frac{1}{\#\Stab_{G_\beta} (v_\beta)}\right)^{-1} \]
is defined by congruence conditions $\displaystyle \prod_{\fp \not \in M_\infty} m_\fp \cdot \prod_{v \in M_\infty} m_v$. 

\end{thm}

\item[Step 3]
Count integral orbits soluble at $\infty$. \\ \\
$L(X) = G(K_\infty) \backslash V(K_\infty)_{< X}$ scaled from $L(1)$. \\ \\
$\F_\beta = G_\beta \backslash G(K_\infty)$. \\ \\
$\F_\beta \cdot L(X) \rightarrow G_\beta \backslash V(K_\infty)_{< X}$ where the fibre above $v$ has size $\displaystyle \frac{\# \Stab_{G(K_\infty)}(v)}{\# \Stab_{G_\beta}(v)}.$ \\ \\
We want
\[N_m(V_\beta, G_\beta, X) \sim \int_{F_\beta \cdot L(X)} \frac{m_\infty(v)}{\# \Stab_{G(K_\infty)} (v)} d \nu_{\infty, \beta}(v)\]
where $\nu_{\infty, \beta}$ is normalized such that $\nu_{\infty, \beta}(V_\beta \backslash V(K_\infty)) = 1.$ 

\begin{pbm}
{\ }
\begin{enumerate}
\item[(1)] Davenport's lemma over function field . ``$B \subset V(K_\infty)$ compact, $t \in K_\infty$, $\# tB \cap V_\beta = V_{\infty, \beta}(tB)$ as $|t| = \displaystyle \prod_{v \in M_\infty} |t_v|_v \rightarrow \infty$.'' \\ \\
This is proved via Poisson summation. \\
\item[(2)] $\F_\beta$ is generally not compact, and so we need to do \emph{cusp analysis}. Without loss of generality we set $V_\beta = V(0), G_\beta = G(0)$, $G$ semi-simple. 
\end{enumerate}
(i) Reduction theory, $G(0) \backslash G(K_\infty) \subset N(K_\infty) A(K_\infty) K'$ (Springer). \\ 
$A$ maximal split torus in $P$, $N$ unipotent radical of $P$, $K'$ compact subgroup of $G(K_\infty)$, and $\Delta$ is a basis of positive roots. \\
For $G = \PGL_2$, we have $A = \left \{\begin{pmatrix} t^{-1} & \\ & t \end{pmatrix} \right \}$, $N = \left \{\begin{pmatrix} 1 & 0 \\ \ast & 1 \end{pmatrix} \right \}$ and $\Delta = \{\alpha\}$, $\alpha \begin{pmatrix} t^{-1} & \\ & t \end{pmatrix} = t^2$. \\ \\
(ii) Cut off the cusp. Restrict $V$ to $A$ where $V = \displaystyle \bigoplus_{\chi \in U_0} \chi$. For example 
\[\Sym^4(2) = \chi_{x^4} \oplus \chi_{x^3 y} \oplus \chi_{x^2 y^2} \oplus \chi_{xy^3} \oplus \chi_{y^4} \]
For $v \in V$, $\chi \in U_0$, $v(\chi)$ is a $\chi$-isotypic composition. \\ \\
For $U \subset U_0$, say $v \in V(K)$ is $U$-irreducible if there exists $g \in G(K)$ such that
\[gv(\chi) = 0, \forall \chi \in U\]

Let $U_1, \cdots, U_m \subset U_0$ such that if $v$ is $U_i$-reducible for some $i$, then $v$ is reducible.  \\ \\
Cusp: $V(K_\infty)^{\text{cusp}} \subset V(K_\infty)$ consists of $v \in V(K_\infty)$ such that $|v(\chi)| < c_1$ for some $\chi \in U_0$ where $c_1$ is chosen so that if $v \in V(\OO)$, $|v(\chi)| < c_1$ implies that $v(\chi) = 0$. \\ \\
There is a combinatorial condition on the characters of $a$ that implies: the number of irreducible points in cusp is small, the volume of the cusp is small, \\ \\
(iii) The number of reducible points in the main body is small. Usually proved by $p$-adic analysis. 

\end{pbm}
\item[Step 4]
Impose soluble conditions at $\fp \not \in M_\infty$ via $m_\fp \rightarrow$ upper bound. 
\item[Step 5]
Uniformity estimate which gives a lower bound. This is done for $\Sel_n(E)$ for $n = 2, 3, 4, 5$. 
\item[Step 6]
Local volume computation. 

\end{itemize}

\begin{thm} (Bhargava-Shankar-Wang ) When elliptic curves over $K$ are ordered by height, 
\[\Avg \Sel_n(E) = \sigma(n) \]
for $n = 2, 3, 4, 5$. 
\end{thm}
\renewcommand{\thesubsection}{\arabic{section}.R}
\begingroup
\renewcommand{\addcontentsline}[3]{}% Remove functionality of \addcontentsline
\endgroup


%%%9.
\newpage
\section{Singular exponential sums associated to prehomogeneous vector spaces over finite fields \\
by Frank Thorne}\label{9}
\renewcommand{\thesubsection}{\arabic{section}.\arabic{subsection}}


Joint work with T. Taniguchi . 

\begin{xmp} Let $V$ be the space of binary cubic forms, and take $\mu_p: V(\bF_p) \rightarrow \bZ$ to be the number of roots in $\bP^1(\bF_p)$, whose values lie in $\{0,1,2,3, p+1\}$, and let $[ \cdot, \cdot] : \operatorname{SL}_2$ be an invariant bilinear form identifying $V$ with its dual. Define
\[
[g \circ x, g \circ y] := [x,y]:= x_1 y_4 - \frac{1}{3} x_2 y_3 + \frac{1}{3} x_3 y_2 - x_4 y_1.
\] 
$x := x(u,v) = x_1 u^3 + x_2 u^2 v + x_3 uv^2 + x_4 v^3$, $p \ne 3$. \end{xmp}
\[
\widehat{\mu}_p(x) := \frac{1}{p^4} \sum_{y \in V(\bF_p)} \mu_p(y) \exp\left(\frac{2\pi i}{p} [x,y]\right).
\]
\begin{pro}
\[
\widehat{\mu}_p(x) = \begin{cases} 1 + p^{-1} & \text{ if } x = 0 \\ p^{-1} & \text{ if } x \text{ has a triple root } \\ 0 & \text{ otherwise.} \end{cases} \]
\end{pro}

The idea is to count, for example, $\GL_2(\chi)$ orbits of (irreducible) binary cubic forms with discriminant in $\pm(1, X)$. Call this count $N^{\pm}(X)$. \\ \\
Let $\Phi_p$ describe some $\GL_2(\chi)$-invariant condition ``at $p$''. 

(1) $v \in V_\chi$ is singular as a binary cubic form $\bF_p$. \\
(2) $v$ has a triple root as a binary cubic form over $\bF_p$. \\
(3) (a) $v$ is a multiple of $p$ and (b) There is a $\GL_2(\chi)$-transformation of $v$ such that $p^2 | v_1, p | v_2$. \\ \\
Write $N^{\pm}(X, \Phi_p)$ or $N^{\pm}(X, p)$ for number of orbits satisfying condition described by $\Phi_p$. If $q$ is square-free, we can write $N^{\pm}(X, \Phi_q)$ or $N^{\pm}(X, q)$ for the number of orbits satisfying condition for all $p | q$. 
\subsection{Sieve axiom}
{\ }

We have 
\[
N^\pm(X,q) = C w(q) + O(X^\alpha q^\beta)
\]
 where $C$ is a constant, $w$ is a multiplicative function, $\alpha < 1$ and $\beta$ are constants. This is often the starting part for analytic number theory problems. 

\begin{xmp} Counting fields of degree $\leq 5$.
\end{xmp}

\begin{xmp} (Belabas-Fouvry) Almost prime discriminants of cubic fields. They did it without using any power-saving error terms in Davenport-Heilbronn, for example.
\end{xmp}

\begin{xmp} (Yang, Cho-Kim) Low lying zeroes of Artin $L$-functions. 
\end{xmp}

\begin{xmp} (Martin-Pollack) Average prime not to split completely 
\end{xmp}

\begin{xmp} (Lemke-Oliver-Thorne) Erd\H{o}s-Kac for number field discs. 
\end{xmp}

Compute the Fourier transform of $\Phi_p$: characteristic function of those binary cubic forms over $\bF_p$ with a triple root. 
\begin{align*} p^4 \Phi_p(y) & = \frac{1}{p^2 - p} \sum_{g \in \SL_2(\bF_p)} \sum_{m \in \F_p^\times} \exp\left(\frac{2 \pi i}{p} \left[ g \circ (m, 0, 0, 0), y \right]\right) \\
& = \frac{1}{p^2 - p} \sum_g \sum_m \exp\left(\frac{2 \pi i}{p} \left[(m, 0, 0, 0), gy \right] \right) \\
& = \frac{1}{p^2 - p} \sum_g \begin{cases} p-1 & \text{ if } [1:0] \text{ is a root of } g \circ y \\ -1 & \text{ otherwise } \end{cases}. 
\end{align*}

\subsection{What has been done so far}

\[ \begin{array}{|c|c|c|} 
\hline
\text{Space} & \text{Group} & \text{Dimension} \\
\hline
2 \otimes 2 & \GL(2) \times \GL(3) & 4 \\
3 \otimes 3 & \GL(2) \times \GL(3) & 9\\
\Sym^2(2) & \GL(1) \times \GL(2) & 3\\
\Sym^3(2) & \GL(1) \times \GL(2) & 4\\
\Sym^2(3) & \GL(1) \times \GL(3) & 6\\
\Sym^2(2) \otimes 2 & \GL(2) \times \GL(2) & 6 \\
\Sym^2(3) \otimes 2 & \GL(2) \times \GL(3) & 2\\
\Sym^4(2) & \GL(1) \times \GL(2) & 5 \\
\hline
\end{array}
\]
\renewcommand{\thesubsection}{\arabic{section}.R}
\begingroup
\renewcommand{\addcontentsline}[3]{}% Remove functionality of \addcontentsline
\endgroup

%%%%%%10
\newpage
\section{Euler systems and Jochnowitz congruences \\
by Massimo Bertolini}\label{10}
\renewcommand{\thesubsection}{\arabic{section}.\arabic{subsection}}


Theme: anti-cyclotomic Iwasawa theory. See Lecture~\ref{12} by Wei Zhang and Lecture~\ref{13} by Rodelfo Venerucci. (See applications to converse of Gross-Zagier-Kolyvagin) \\ \\
Let $E$ be an elliptic curve of conductor $N$, $f \in S_2(N)$ the associated form: $A = A_f$. $p$ an ordinary prime of $E$. $K_\infty/K$ the anti-cyclotomic $\bZ_p$-extension associated to $K = \bQ(\sqrt{-D})$. \\ \\
We make the simplifying assumptions: Let $N$ be square-free, $(D, N_p) = 1$, $p \nmid N$ (if $p \mid N$, things are ok: use work of Skinner-Zhang ; see Verevucci's Lecture~\ref{13})
\[N  = N^+ N^{-}, \qquad N^+ := \prod_{\substack{q | N \\ q \text{ split in } K}} q,\qquad N^{-} := \prod_{\substack{q | N \\ q \text{ inert}}} q.\]

Definite case: $\# \{q | N^{-}\}$ is odd. This implies that $\sgn L(f/K, \chi, s) = +1$ for all $\chi : G_\infty = \Gal(K_\infty/K) \rightarrow \ol{\bQ}^\times$. Thus, we can define $L_p(f) = \mathfrak{L}_p(f) \cdot \mathcal{L}(f)^2 = (\text{unit}) \cdot \mathfrak{L}_p(f)^2 \in \Lambda = \bZ_p[[G_\infty]]$ by interpolating $L(f/K, \chi, 1)$, which are described in theorems. \\ \\
$\widehat{R}^\times \backslash \widehat{B}^x / B^\times$, where $B$ is the definite quaternion algebra of discriminant $N^{-} \infty$, $R$ Eisenstein of level $N^+$ ($\widehat{\iota} = (\cdot) \otimes \widehat{\bZ}$). 

\begin{thm} (Definite Main Conjecture) (Bertolini-Darmon , Skinner-Urban~, Bertolini-Verevucci ) For almost all good ordinary primes $p$, 
\[\Lambda \cdot L_p(f) = \operatorname{char}_{\Lambda} \Sel_{p^\infty}(A / K_\infty)^\vee \]
\end{thm}

\begin{rmk} We know that $L_p(f) \ne 0$ by Cornut-Vatsal . \end{rmk}

Indefinite case: $\# \{q | N^{-}\}$ is even. This implies that $L(f/K, \chi, 1) = 0$. BSD implies that $\Sel_{p^\infty} (A/K_\infty)$ should not be a $\Lambda$-cotorsion. \\ \\
Heegner points on $(\widehat{R}^\times \backslash \widehat{B}^\times \times \mathcal{h}^{\pm}/B^\times$, $B$ is an indefinite quaternion algebra of discriminant $N^{-1}$ gives a class $K(1) \in \widehat{H}_\phi^1(K_\infty, T_f) = \text{Tap}(A)$, where if $S$ is a finite set of primes, $\displaystyle \widehat{H}_S^1(K_\infty, T_f) = \varprojlim\limits_{\text{cores}} H_S^1(K_m, T_f)$, $K \subset K_m \subset K_\infty$. The latter corresponds to the situation where the Selmer group with the conditions at $\rho \in S$ relaxed. 

\begin{dfn} (Indefinite Main Conjecture) (X. Wan if $p$ is split , Bertolini-Verevucci in general ) For almost all good ordinary primes $p$, 
\[L_p(f) \Lambda = \operatorname{char}_\Lambda\left(\Sel_{p^\infty}(A/K_\infty)_{\operatorname{tors}}^\vee\right)\]
\end{dfn}

\begin{rmk} Work of B. Howard. 
\end{rmk}

We describe the ingredients contained in the proof of the Indefinite Main Conjecture:  
\begin{enumerate}
\item 
Explicit reciprocity laws (Bertolini-Darmon , Skinner-Zhang)
\item 
Prove the full definite MC, adapting the induction of Bertolini-Darmon using Skinner-Verevucci over $\bQ$.
\item 
Reduce the indefinite Main Conjecture to the definite Main Conjecture.  
\end{enumerate}

(i) Assume that $f$ is in the definite case. 
\begin{dfn} $\ell \nmid Np$ is \emph{$n$-admissible ($n \geq 1$) with respect to $(f, K, p)$} if: \\
(a) $\ell$ is inert in $K$, \\
(b) $p | (\ell+1) - \epsilon a_\ell(f)$, and $p^n \nmid \ell^2 - 1$ (here $\epsilon = \pm 1$ is a choice of sign).
\end{dfn}

If $\ell$ is $n$-admissible, then there exists $f_\ell \in s_2(N\ell)$ arising on $B_\ell =$ indefinite quaternion algebra of discriminant $N^{-} \ell$ such that $f_\ell \equiv f \pmod{p^n}$. Write $X_\ell$ for the Shimura curve associated to $B_l$, with $N^+$-level structure. Set $T_{f,n} = T_f / p^n = T_{f_\ell}$, $\Lambda_n = \Lambda / p^n$. Heegner points on $X_\ell$ give a class $K_n(\ell) \in \widehat{H}_\ell(K_\infty, T_{f,n}).$ Since $\ell$ is inert in $K$, that it is not split $K_\infty / K$. This implies that
\[\widehat{H}(K_\infty, \ell, T_{f,n}) \cong H^1(K_\ell, T_{f,n}) \otimes \Lambda_n \cong \Lambda_m \oplus \Lambda_n. \]
\[\widehat{H}_{\text{fin}}^1 (K_\infty, \ell, T_{f,n}) \cong \Lambda_m, \widehat{H}_{\text{sing}}^1 (K_ \ell, T_{f,n}) \cong \Lambda_m.\]
\[\Rightarrow \partial_ \ell: \widehat{H}_ \ell(K_\infty, T_{f,n}) \rightarrow H_{\text{sing}}^1(K_\infty, l, T_{f,n}) = \Lambda_m.\]

First reciprocity law: $\partial_\ell K_n(l) = \mathcal{L}_{p,n}(f) = \mathcal{L}_p(f) \pmod{p^n}$. \\ \\
Let $\ell' \ne  \ell$ be another $n$-admissible prime. $\upsilon_{\ell'} : \widehat{H}_{\ell}^1 (K_\infty, T_{f,n}) \rightarrow \widehat{H}_{\text{fin}}^1 (K_\infty, \ell', T_{f,n})$. \\ \\
Second reciprocity law: $\upsilon_{\ell'} K_n(\ell) = \mathcal{L}_{p,n}(f_{\ell,\ell'}).$ 

(ii) The induction: $L_p(f) \cdot \Lambda = \operatorname{char}_{\Lambda} \Sel_{p^\infty}(A/K_\infty)^\vee$. Look at the image of this relation under $\chi : \Lambda \rightarrow \OO,$ which is a discrete valuation ring. Enough to check the definite MC for all $n \gg 0$ and for enough $\chi$. Induction on the order of vanishing of $\chi L_{p,n}(f) \ne 0$ by Cornut-Vatsal . Can assume that $\Sel_{p^\infty}(A/K_\infty) \otimes_{\chi} \OO \ne (0)$ (Skinner-Urban ). 

(iii) 
\begin{enumerate}
\item[(a)]
Use the second reciprocity law to relate the indefinite $L_p(f)$ to $L_p(f_\ell)$.
\item[(b)]
Compare $\Sel_{p^n}(f)$ with $\Sel_{p^n}(f_\ell)$.
\end{enumerate} 
\renewcommand{\thesubsection}{\arabic{section}.R}
\begingroup
\renewcommand{\addcontentsline}[3]{}% Remove functionality of \addcontentsline
\endgroup
%%%11.
\newpage
\section{Special values of Rankin-Selberg type $p$-adic $L$-functions \\
by Ernest Hunter Brooks}\label{11}
\renewcommand{\thesubsection}{\arabic{section}.\arabic{subsection}}

\subsection{$p$-adic Waldspurger formulas}
{\ }

Recall in Urban's Lectures~\ref{2}-\ref{3}, we saw Hida's constructions of two $p$-adic $L$-functions $L^{\text{I}}, L^{\text{II}}$ interpolating special values of classical $L$-functions $L(f,g,n)$. \\ \\
The Bertolini-Darmon-Prasanna formula  implies that, when $f$ comes from an elliptic curve $E$ and $g$ comes from a Hecke character, one has
\[L^{\text{II}}(\mathbbm{1}) = (\log (Q_{\mathbbm{1}}))^2,\]
where $\mathbbm{1}$ is the trivial character, $Q_{\mathbbm{1}}$ is a Heegner point on $E(K)$. \\
What does this mean? Why is it relevant to counting problems? Why is it true? Under what conditions? 

\subsection{$p$-adic logarithms}
{\ }

When  $A/\bQ_p$ is an abelian variety, there is a differential $\omega \in H^0(A, \Omega)$ and there is also a unique locally analytic homomorphism
\[
\log_\omega : A(\bQ_p) \rightarrow \bQ_p,
\]
such that $d \log_\omega = \omega.$   Torsion points are obviously in the kernel of $\log_\omega$ for any $\omega$. \\ \\
Conversely, if a point is in the kernel of $\log_\omega$ for all $\omega$ (or all $\omega$ in a basis) then it is torsion. The pairing 
\[(P, \omega) \mapsto \log_\omega(P) \]
gives an isomorphism of $p$-adic Lie groups from the kernel of the reduction map to $H^0(A, \Omega)^\vee$. 

\subsection{Logarithms on curves}
{\ }

Let $C/\bQ_p$ be a curve with a fixed base point $P \in C(\bQ_p)$, $J$ its Jacobian, $AJ: C \rightarrow J$ (algebraic, depends on $P$). We get an embedding $C \rightarrow J$ using $P$. We get an identification $H^0(C, \Omega_C) = H^0(J, \Omega_J)$. This gives for each $\omega \in H^0(C, \Omega_C)$ a logarithm on $C$. 
The logarithm on $C$ depends on base point, but the induced map on $\Div^0$ does not. \\ \\
If $f: C_1 \rightarrow C_2$ os a morphism of curves, $D$ is a degree zero divisor on $C_1$, and $\omega$ is a $1$-form on $C_2$, then 
\[ \log_{f^\ast \omega}(D) = \log_\omega(f(D))\]


\subsection{Elliptic curves}
{\ }

Let $E$ be an elliptic curve over $\bQ$ with square-free conductor $N$, with associated weight 2 new form $f$. Fix an imaginary quadratic field $K$ of discriminant prime to $N$ and factor $N = N^+ N^{-}$, where primes dividing $N^+$ are split in $K$ and primes dividing $N^{-}$ are inert. Now assume an even number of primes divide $N^{-}$. 


\subsection{Rankin-Selberg $L$-functions}
{\ }


Define parameters $\alpha_p$ and $\beta_p$ by
\[\mathcal{E}_p(E,s) = (1 - \alpha_p p^{-s})^{-1} (1 - \beta_p p^{-s})^{-1}.\]
where $\mathcal{E}_p$ is the Euler factor for the $L$-function of $E/\bQ$ at $p$. For $\fp$ a prime of $K$ not dividing the discriminant of $K$ or $N$, and $\chi$ a character of $\Cl(K)$, set
\[\E_\fp^{\text{R-S}} (E, \chi, s) = (1 - \chi(\fp) \alpha_{\textbf{N}\fp}(\textbf{N}\fp)^{-s})^{-1} (1 - \chi(\fp) \beta_{\textbf{N}\fp} (\textbf{N} \fp)^{-s})^{-1} \]

The global Rankin-Selberg $L$-function is (up to finitely many Euler factors)
\[L(E, \chi, s) = \prod_{\fp \nmid Nd_k} \epsilon_\fp^{\text{R-S}} \]
It admits analytic continuation, and there is a completed $L$-function $\Lambda$ which satisfies a functional equation with centre $s=1$ and sign $\pm 1$.

\subsection{The Heegner hypothesis and the $L$-function} 
{\ }

Heegner hypothesis: $N = N^{-} N^+$ as before and assume for a moment that $N^{-} = 1$. \\ \\
Analytic consequence: forces sign in functional equation to be $-1$, 
\[L(E,\chi,s) = -L(E, \chi, 2-s)\]
implying $L(E, \chi, 1) = 0$. \\ \\
Geometric consequence: There is an ideal $\mathcal{N}$ of $K$ of norm $K$, so a Heegner point $P = [\bC / \mathfrak{N}^{-1} \rightarrow \bC/\OO_K] \in X_0(N)(h).$


\subsection{Heegner points}
{\ }

Thinking of $\chi$ as a character of $\Gal(H/K)$, set
\[P_\chi = \sum_{\sigma \in \Gal(H/K)} \chi^{-1}(\sigma) P^\sigma \in \Div(X_0(N))(H) \otimes \bQ(\chi).\]
Also set
\[Q_\chi = \phi(P_\chi)\]
where $\phi$ is the map coming from the modular parametrization. In particular, $Q_{\mathbbm{1}} \in E(K).$ 


\subsection{Heights}
{\ }

The height map $\widehat{h}$ extends to $\Div(E)(H) \otimes \bQ(\chi)$ in $\bC$. The height of a point in $E(K)$ is zero if and only if it is a torsion point. 

\begin{thm} One has $L'(E, \chi, 1) = \widehat{h}(Q_\chi).$
\end{thm}

Gross-Zagier 1987 : $N^{-} = 1,$ Skinner-Zhang 2001 : $N^{-} > 1$ square-free, Yuan-S. Zhang, W. Zhang 2013 : no assumptions.


\subsection{Applications to the conjecture of Birch and Swinnerton-Dyer} 
{\ }

BSD over $K$: Comparing Euler factors, one sees
\[L(E/K, s) = L(E, \textbf{1}, s).\]
So if $L'(E/K, s) \ne 0$, then the Heegner point is non-torsion and consequently ``analytic rank one implies algebraic rank at least one.'' \\ \\
BSD over $\bQ$: Gross and Zagier apply a result of Waldspurger to show one can choose $K$ such that the above argument descends to $\bQ$. 

\subsection{Shimura curves}
{\ }

Let $B/\bQ$ be the indefinite quaternion algebra with discriminant $N^{-}$, and fix a maximal order $\OO_B$ in $B$. \\ \\
There is an Eichler order of level $N^+$, called $\OO_{B,N^+}$ in $\OO_B$. Write $\Gamma_{N^+, N^{-}}$ for its group of norm one elements. \\ \\
The group $\Gamma_{N^+, N^{-}}$ acts on the upper half plane via
\[ B \otimes \bR \rightarrow M_2(\bR).\]
The quotient $X_\bC = \mathcal{H}/\Gamma_{N^+, N^{-}}$ is a \emph{Shimura curve}. If $N^{-} \ne 1$, the Shimura curve is compact. There are no cusps. There are modular forms for $\Gamma_{N^+, N^{-}}$, but they have no $q$-expansions.

\subsection{Moduli-theoretic interpretation}
{\ }


If $N^{-} \ne 1$, the Shimura curve is compact. There are no cusps. \\ \\
To $\tau \in \mathcal{H}$ we attach the 2-dimensional complex torus
\[A_\tau = \frac{\bC^2}{\OO_B \binom{\tau}{1}} \]
There is an obvious embedding $\OO_B \hookrightarrow \operatorname{End}(A_\tau)$ and, moreover, $A_\tau$ admits a principal polarization. This motivates: a \emph{false elliptic curve} $\mathcal{A}$ over a base scheme $S$ is a p.p. relative abelian surface over $S$ together with an embedding $\OO_B \hookrightarrow \operatorname{End}(\mathcal{A})$. 


\subsection{Heegner points (again)}
{\ }


Because the pair $(K, N^+)$ satisfies the Heegner hypothesis, there is an embedding $\iota: K \hookrightarrow B$ with
\[\iota(\OO_K) \subset \OO_{B,N^+}.\]
Carayol shows that $X_\bC$ admits a canonical model $\mathcal{X}$ over $\bZ\left[\frac{1}{N}\right]$. Work of Shimura shows that the image $P$ of any $\tau \in \mathcal{H}$ fixed by $\iota(K^\times)$ satisfies $P \in \mathcal{X}(H)$.


\subsection{Modularity}
{\ }

Write $X = X_{N+, N^{-}}$ for $\mathcal{X}_\bQ$. The usual Eichler-Shimura construction gives $X_{N^+, N^{-}} \rightarrow E$, but this depends on a choice of a base point. For $N^{-} = 1$ it is usual to send the cusp at infinity to the origin. \\ \\
For $\chi \ne \mathbbm{1}$ it doesn't matter which base point we pick. On Shimura cures, replace $P_{\mathbbm{1}}$ with $\epsilon_f P_{\mathbbm{1}}$ where $\epsilon_f \in \bQ[\mathbb{T}] \subset \bQ(\operatorname{End}[X_{N^+, N^{-}}])$ is the projector $\epsilon_f H^\ast(X_{N^+, N^{-}}) = H^1(X_{N^+, N^{-}})[\omega_f]$. Then $\epsilon_f P_{\mathbbm{1}}$ is degree zero and its image on $E_f$ does not depend on any choices. 


\subsection{Gross-Zagier without the Heegner hypothesis}
{\ }

One has a divisor $P_\chi$ on $X(H)$ as before. There is an Eichler-Shimura parametrization $\phi: X \rightarrow E$ coming from $f$ as above; write $Q_\chi = \phi(P_\chi)$. 
\begin{thm} (Zhang, 2001 ), (Yuan-Zhang-Zhang, 2013 ) One has
\[L'(f, \chi, 1) \doteq \langle Q_\chi, Q_\chi \rangle.\]
\end{thm} 


\subsection{First step toward $p$-adic $L$-functions}
{\ }

By analogy with the definition of the Kubota-Leopoldt $p$-adic $L$-function, one wants to define a $p$-adic $L$-function by interpolation of special values of $L(f, \chi, n)$. \\ \\
Normalization: if we let $\chi$ vary over a family of Hecke characters, we may assume $n = 0$. A Hecke character is a character of $\mathbb{A}^K/K^\times$. The restriction of such a character to $\bC^\times$ is called its ``infinity type''. It is of the form
\[z \mapsto z^{-l_1} \ol{z}^{-l_2} \]
and we call the pair $(l_1, l_2)$ the infinity type of the character.
\subsection{Hecke characters of $K$, (2/2)} 
{\ }

The \emph{central line} $l_1 + l_2 = 2$ corresponds to Hecke characters such that is the centre of the functional equation for $L(f,\chi)$. \\ \\
{\begin{center}  \includegraphics[height=200pt]{Diagram_1.jpg}
\end{center}}

In the blue region consisting of characters of type $(1,1)$, the sign of the functional equation is negative. In the green region, the sign of the functional equation is positive. Thus, the classical $L$-function does not vanish at the center, even under the Heegner hypothesis! The sign of the functional equation for characters in the orange region is positive, but the orange region will not be used today. At each unshaded lattice point, some $\Gamma$-factor in the functional equation has a pole at $s = 0$. We won't use these today. 

\subsection{The $p$-adic $L$-function $L^{\text{I}}$}
{\ }

Fix a prime $p > 2$ which splits in $K$, with $(p,N) = 1$. 

\begin{thm} (Hida, 1988 ) There is a $p$-adic $L$-function $L^{\text{I}}(E, \chi)$, where $\chi$ ranges over central critical Hecke characters. It satisfies an interpolation law of the form
\[L^{\text{I}}(E, \chi) \doteq L(E, \chi^{-1}, 0) \]
for $\chi$ central critical in the blue region. 
\end{thm}

The real and $p$-adic periods hidden in the $\doteq$ depend on $E$ and not on $\chi$. 

\subsection{Known results on $L^{\text{I}}$}
{\ }

By the interpolation law, $L^{\text{I}}(\mathbbm{1}) = 0$. The interesting invariant is $L^{\text{I}'}(\mathbbm{1})$. 
{\ }
\begin{itemize}
\item 
Perrin-Riou (1987) : $L^{\text{I}'}(\mathbbm{1}) \doteq \langle Q_{\mathbbm{1}}, Q_{\mathbbm{1}} \rangle_P$ (under Heegner hypothesis) 
\item
Nekov\'{a}r (1995) : higher weight version
\[Q_{\textbf{1}} \rightarrow \text{ Heegner cycle on } \E \rightarrow X_0(N)\]
\item
Disegni (2013) : drops Heegner hypothesis from Perrin-Riou/Nekov\'{a}r.
\item
Shnidman (2014) : allows twists by infinite-order Hecke characters.
\end{itemize}

\subsection{The $p$-adic $L$-function $L^{\text{II}}$}

\begin{thm} (Hida, 1988 ) There is a $p$-adic $L$-function $L^{\text{II}}(E, \chi)$, where $\chi$ ranges over central critical Hecke characters. It satisfies an interpolation law of the form
\[L^{\text{II}}(E, \chi) \doteq L(E, \chi^{-1}, 0)\]
for $\chi$ central critical in the green region. 
\end{thm}

\begin{rmk} This is $L^{\text{II}}$ from Lecture~\ref{3}. The periods which occur are periods of Hecke characters (= periods of CM abelian varieties). 
\end{rmk}

\subsection{Known results on $L^{\text{II}}$}
{\ }

No reason $L^{\text{II}}(\mathbbm{1})$ has to be zero. Instead, as mentioned above, one has
\[L_p(\mathbbm{1}) \doteq (\log(\bQ_{\mathbbm{1}}))^2\]

\subsection{Assumptions needed for Bertolini-Darmon-Prasanna formula}
{\ }

\begin{itemize}
\item
Bertolini-Darmon-Prasanna 2009 : Heegner hypothesis, $p \nmid N$. Also higher weights formula (generalized Heegner cycles). 
\item
Masdeu 2011 : drops Heegner hypothesis, $(p || N^{-})$ different $p$-adic $L$=function ($p$ inert). Higher even weight. 
\item
Castella 2011 : $p | N$, higher weight.
\item
Brooks 2013 : drops Heegner hypothesis, $p \nmid N$. Higher even weight (generalized Heegner cycles). 
\item
Liu, S. Zhang, W. Zhang 2013 : no assumptions except $p$ split! Formulas for Shimura curves over totally real fields.

\end{itemize}

\subsection{Applications to analytic rank}

\begin{thm} (Skinner, Theorem B) Suppose:
\begin{itemize}
\item
(as above) $p$ is split in $K$, $N = N^+ N^{-}$ is square-free, prime to $d(K)$, satisfies the generalized Heegner hypothesis with respect to $K$. 
\item
$p \geq 5$, $E$ is $p$-ordinary, 2 splits in $K$. 
\item
The mod $p$ Galois representation $E[p]$ ramifies at some odd prime which is inert or ramified in $K$.
\item
$\dim \Sel^{\text{I}}(K,V) = 1$ and $H_f^1(K,V) \hookrightarrow H_f^1(K_\fp, V) \oplus H_f^1(K_{\ol{\fp}}, V)$
\end{itemize}
Then
\[\ord_{s=1} L(E,K,s) = 1.\]
\end{thm}

\begin{proof} (sketch) 
\begin{itemize}
\item 
The cohomological hypotheses imply
\[\ker(H^1(K,V) \rightarrow \bigoplus_{w \ne \fp} H^1(K_w, V)) = 0.\]
\item
X. Wan's divisibility in the main conjecture plus Galois cohomology (Skinner's lemma, Monday) implies that $L^{\text{II}}(\mathbbm{1}) \ne 0$. 
\item
Bertolini-Darmon-Prasanna  implies that $Q_{\mathbbm{1}}$ is not torsion. Thus we may conclude rank 1 from Gross-Zagier.
\end{itemize}
\end{proof}

\subsection{Sketch of BDP proof}
{\ }

Archimedean Waldspurger Formula: For $\chi$ of infinity type $(2+j, -j)$ with $j \geq 0$, one has (with $\doteq $ denote the meaning of equal up to a $p$-adic unit),
\[L(f, \chi^{-1}, 0) \doteq \left \lvert \sum_{\Cl(K)} \chi^{-1} (\mathfrak{a}) \textbf{N}\mathfrak{a}^j \delta_{M-S}^j f(\mathfrak{a}^{-1}, 2\pi i dz)\right \rvert^2\]
Here, 
\begin{itemize}
\item
$\delta_{M-S} = \frac{1}{2 \pi i} \left(\frac{d}{d\tau} + \frac{k}{2i \operatorname{Im} \tau} \right)$
\item
$\delta_{M-S}^j f(\mathfrak{a}^{-1})$ does not make sense, but $\delta_{M-S}^j f(\mathfrak{a}^{-1}, 2\pi i dz)$ ``does''. 
\end{itemize}

Atkin-Lehner argument: Remove absolute values (at the cost of more hidden in $\doteq$). \\ \\
Algebraize: Replace $2 \pi i dz$ with a differential $\omega_H$ on $E_\mathfrak{a}$ defined over $H$.
\[\frac{L(f, \chi^{-1}, 0)}{\Omega_\bC^{2(2+j)}} \doteq \left(\sum_{\Cl(K)} \chi^{-1} (\mathfrak{a}) \textbf{N}\mathfrak{a}^j \delta_{M-S}^j f(\mathfrak{a}^{-1}, \omega_H)\right)^2\]

Katz theorems on differential operators: both sides are algebraic. Can replace $\delta_{M-S}$ with $\theta = q \frac{d}{dq}$ (operator on $p$-adic modular forms). \\ \\
$p$-adic interpolation: For $p$-adic interpolation of these algebraic numbers, replace $f$ with $p$-depletion, $f^\flat$, and replace $\omega_H$ by a $p$-adic transcendental form $\widehat{\omega}$. The first drops an Euler factor at $\ol{\fp}$; the second produces a $p$-adic period $\Omega_p$. \\ \\
$p$-adic formula: Get a $p$-adic analytic function $L_p$ (this is $L^{\text{II}}$) by 
\begin{align*} \displaystyle \frac{L_p(\chi)}{\Omega_p^{2(2+j)}}  :&= \E_\fp(f, \chi^{-1}, 0) \frac{L(f, \chi^{-1}, 0)}{\Omega^{2(2+j)}} \\ 
& \doteq \left(\sum_{\Cl(K)} \chi^{-1} (\mathfrak{a}) \textbf{N} \mathfrak{a}^j \delta_{M-S}^j f^\flat(\mathfrak{a}^{-1}, \widehat{\omega})\right)^2
\end{align*}

Letting $j \rightarrow -1$ $p$-adically (i.e. in weight space) and $\chi_n \rightarrow \mathbbm{1}$, we get that
\[L_p(\mathbbm{1}) \doteq \E_\fp(f, \chi^{-1}, 0) \left(\sum (\theta^{-1} f^\flat(\mathfrak{a}, \widehat{\omega}))\right)^2 \]
Relate $\theta^{-1}$ to $\log$: both coincide with the Coleman primitive. Use BDP to compute with $p$-adic modular forms as uniform limits (in the $p$-adic topology) of $q$-expansions of modular forms with integral Fourier coefficients. This definition does not make sense for Shimura curves, which is the major obstruction to dropping the Heegner hypothesis. 

\subsection{$p$-adic modular forms in the proof of BDP}
{\ }

As above, the theory of Coleman integration gives the following formula:
\[\log_{\omega_f} = \theta^{-1} (f^\flat) = \theta^{-1} (f |_{1 - UV}).\]
$p$-adic Hecke operators (Serre ): 
\[f |_U (q) = \sum a_{np} q^n, \text{ } f |_V(q) = \sum a_n q^{pn}.\]
$p$-adic differential operator (Ramanujan-Atkin-Serre):
\[\theta = q \frac{d}{dq}, \text{ and } \theta^{-1} f = \lim_{i \rightarrow \infty} \theta^{p^i(p-1) -1} f. \]

\subsection{The case of Shimura curves}
{\ }

With definitions on the previous slide, the formula
\[\log_{\omega_f} = \theta^{-1} (f |_{1 - UV})\]
is meaningless for $f$ on a Shimura curve. However, Katz's geometric interpretation of $p$-adic modular forms in the classical case generalizes to Shimura curves (Kassaei, Ph. D thesis ). 

\subsection{The $p$-adic geometry of Shimura curves}
{\ }

Write $k = \ol{\bF_p}$, $W = \operatorname{Witt}(k) = \widehat{\OO_{\bQ_p^{\text{ur}}}}, L = \operatorname{Frac}(W)$. For the remainder of the lecture, $X$ can be a Shimura curve or modular curve over $L$. There is no canonical model $\mathcal{X}$ for $X/W$: 

{\begin{center}  \includegraphics[height=120pt]{Diagram_2.jpg}
\end{center}}

\subsection{The reduction map}
{\ }

The model $\mathcal{X}$ is proper over $W$ and thus one has a map of \emph{sets}:
\[X(L) = \mathcal{X}(W) \mapsto X(k).\]

{\begin{center}  \includegraphics[height=90pt]{Diagram_3.jpg}
\end{center}}

\subsection{Residue disks}
{\ }

For a fixed $P \in X(k)$, the finer in $X(L)$ above $P$ is called a \emph{residue disk}. It has a natural structure of rigid analytic space, conformal to the open unit disk in $L$.

{\begin{center}  \includegraphics[height=120pt]{Diagram_4.jpg}
\end{center}}

\subsection{The ordinary locus}
{\ }

The \emph{ordinary locus} $X^{\text{odd}}$ is the affinoid obtained from $X$ by deleting the (finitely many) residue disks above points corresponding to super singular [false] elliptic curves. 

{\begin{center}  \includegraphics[height=150pt]{Diagram_5.jpg}
\end{center}}

\subsection{Geometric interpretation of $p$-adic modular forms}
{\ }

Recall that a classical modular form of weight $k$ for $\Gamma_1(N)$ is a global section of $\ul{\omega}^{\otimes k}$ on the modular curve, where $\ul{\omega}$ is the push forward of the relative differential bundle on the universal elliptic curve. One has
\[\ul{\omega}^{\otimes 2} = \Omega_X\]

There is a similar bundle for Shimura curves (one needs to choose a projector as the obvious analogue is a rank 2 vector bundle). \\ \\
Katz showed that a $p$-adic modular form of weight $k$ gives rise to a section of $\ul{\omega}^{\otimes k}$ over the ordinary locus. 

\subsection{Geometric interpretation of $\theta$}
{\ }

There is a rank two vector bundle $\mathcal{V}$ on $X$ which comes with a flat connection $\nabla$ and an inclusion $\ul{\omega} \subset \mathcal{V}$. In the modular case, $\mathcal{V}$ is the first relative cohomolgy bundle of the universal elliptic curve (the connection has singularities at the cusps). In the Shimura curve case, the relative cohomology bundle of the universal abelian surface is too big. Get a sub-bundle from same projector as before. \\ \\
By a theorem of Dwork and Katz, over the ordinary locus, the inclusion $\ul{\omega} \rightarrow \mathcal{V}$ \emph{splits}:
\[\Psi: \mathcal{V} \rightarrow \ul{\omega}\]
In the modular curve case, Serre's operator $\theta$ coincides with the composition
\[\ul{\omega}^{\otimes k} \rightarrow \mathcal{V}^{\otimes k} \rightarrow_\nabla \mathcal{V}^{\otimes k} \otimes \Omega_X \rightarrow_{\Psi^{\otimes k}} \ul{\omega}^k \otimes \Omega \rightarrow \ul{\omega}^{k+2}.\]
We take this as the definition of $\theta$ in the Shimura curve case.

\subsection{Geometric interpretation of $p$-adic Hecke operators}
{\ }

Recall that Hecke operators $T_l$ on the space of modular forms are pullbacks induced by \emph{correspondence} on modular or Shimura curves. Recall that an elliptic curve with good ordinary reduction over $\bQ_p$ has a \emph{canonical subgroup} of order $p$ - namely the unique $C$ such that
\[E \rightarrow E/C \]
lifts the Frobenius map on the reduction. Similarly a false elliptic curve has a unique sub $\OO_B$-module lifting the kernel of Frobenius. \\ \\
Katz showed that $V$ is induced by the correspondence
\[A \mapsto A/C.\]
Similarly, $U$ is induced by the correspondence 
\[A \mapsto \frac{1}{p} \sum_{C_i \ne C} (A/C_i).\]
Take these as definitions in Shimura curve case. 

\subsection{Uniformization of ordinary residue disks}
{\ }

Start with $\E$ an elliptic curve over $L$ with good ordinary reduction; call the reduction $E$. Pick a generator $P \in T_p(E)(k)$. Lift to $\tilde{P} \in T_p(\E)(\ol{L})$. For $\sigma \in \Gal(\ol{L}/L)$ map $\sigma \mapsto (\tilde{P}^\sigma, \tilde{P}) \in \bZ_p(1).$ This is a cocycle (!), call it $\xi_\E$. 

\begin{thm} (Serre-Tate) The association $\E \rightarrow \xi_E$ gives an embedding 
\[D = \{\text{Lifts of } E \text{ to } L\} \rightarrow H^1(L, T_p(1)) = L^\times.\]
The image is the set $1 + pW$ of norm one elements. So $D$ has a natural group structure! Similarly for false elliptic curves.
\end{thm}

\subsection{Another way to say the same thing}
{\ }

For a [false] elliptic curve $\E$ in the fixed residue disk $D$, we have a tautological sequence of $p$-divisible groups:
\[0 \rightarrow \widehat{\E}(L)[p^\infty] \rightarrow \E(L)[p^\infty] \rightarrow E(K)[p^\infty] \rightarrow 0 \]
The left hand group depends only on $E$ and not $\E$, because it's $\Hom(E^\vee(k)[p^\infty], \bZ_p(1)).$

\begin{thm} (Serre-Tate) The curve $\E$ is determined by the class of this extension, and (picking a generator of $T_p(E)$)
\[D = \operatorname{Ext}^1(\bQ_p/\bZ_p, \mu_p^\infty) = 1 + pW.\]
\end{thm}

\subsection{Serre-Tate coordinates}
{\ }

Serre-Tate theory identifies the ring of rigid functions on $D$ with
\[W[[T]] \left [\frac{1}{p} \right ]\]
where $T: D \rightarrow 1 + pW \overset{x \mapsto x-1}{\rightarrow} pW.$ The bundle $\ul{\omega}$ trivializes on the disk $D$. Write $\widehat{\omega} \in \ul{\omega}(D)$ for a non-vanishing section obtained by choosing an isomorphism $\widehat{A} \rightarrow \widehat{\mathbb{G}_m}$ and pulling back $dT/T$. We can express a modular form of weight $k$ on $D$ by an expression of the form $F(T)\widehat{\omega}^{\otimes k}$. 

\subsection{Formulas in Serre-Tate coordinates}
{\ }

Differential operator: Using computations of Brakocevic  (2012) and Mori  (2011), one can show
\[\theta(F(T)\widehat{\omega}^{\otimes k}) = (1 + T)F'(T)\widehat{\omega}^{\otimes k + 2}\]
$p$-adic Hecke operators: One has
\[(F(T) \widehat{\omega}^{\otimes k})|_{UV} = \frac{1}{p} \sum_{i=0}^{p-1} F(\zeta^i(1 + T) - 1)\widehat{\omega}^{\otimes k}\]
where $\zeta \in \ol{\bQ}_p$ is a fixed non-trivial choice of $p$-th root of 1. \\ \\
Making sense of the antiderivative $\theta^{-1}$ and establishing the BDP formula is then a matter of analyzing these power series operators, which is straightforward.  
\renewcommand{\thesubsection}{\arabic{section}.R}
\begingroup
\renewcommand{\addcontentsline}[3]{}% Remove functionality of \addcontentsline
\endgroup

%%%12.
\newpage
\section{Kolyvagin's conjecture on Heegner points \\
by Wei Zhang}\label{12}
\renewcommand{\thesubsection}{\arabic{section}.\arabic{subsection}}

\subsection{Theorems}

Let $E/\bQ$ be an elliptic curve, ($f \in S_2(N), \Gamma_0(n)$-level). $E(\bQ)$ corresponds to a $L$-function, say $L(E/\bQ, s)$. For $n \geq 1$, we obtain the exact sequences
\[0 \rightarrow E(\bQ \otimes \bZ/n\bZ \rightarrow \Sel_n(E/\bQ) \rightarrow \Sha(E/\bQ)_n \rightarrow 0 \]
\[0 \rightarrow E(\bQ) \otimes \bQ_p/\bZ_p \rightarrow \Sel_{p^\infty}(E/\bQ) \rightarrow \Sha(E/\bQ)_{p^\infty} \rightarrow 0\]

Define
\[r_p(E/\bQ) := \rank_{\bZ_p} \Hom(\Sel_{p^\infty} (E/\bQ), \bQ/\bZ)\]

Note that $0 \leq r_{\text{alg}} \leq r_p$. Equality condition is is equivalent to the assertion that $\Sha_{p^\infty}(E/\bQ)$ is finite. 

\begin{thm} (Gross-Zagier , Kolyvagin) $\rank(E/\bQ) \leq 1$ implies that $r_{\text{alg}} (E/\bQ) = r_{\text{an}}$ and $\# \Sha(E/\bQ) < \infty.$ 
\end{thm}

\begin{thm} (Kato , Skinner-Urban ) For $p$ good ordinary (plus extra conditions), we have
\[r_p = 0\quad \text{is equivalent to}\quad r_{\text{an}} = 0. \]
(and now the refined BSD),
\[\left \vert \frac{L(E/\bQ, 1)}{\Omega}\right \rvert_p = \left \lvert \#\Sha(E/\bQ) \prod_{\ell | N} c_\ell \right \rvert_p, \]
where $c_\ell$ are the local Tamagawa number. 
\end{thm}
What about when $r_p = 1$?

\begin{thm} (W. Zhang) \\
(1) For $p \geq 5$, good ordinary, and \\
(2) $\ol{\rho}_{\in p} : \Gal_\bQ \rightarrow \Aut(E_p) \cong \GL_2(\bF_p)$ surjective, ramified at least at two $\ell || N$, and ramified at all $\ell || N$ such that $\ell \equiv \pm 1 \pmod{p}$. 

Then
\[r_p = 1 \Leftrightarrow r_{\text{an}} = 1 \Leftrightarrow r_{\text{alg}} = 1 \text{ and } \#\Sha < \infty.\]
\end{thm}

\begin{rmk} 
{\ }
\begin{enumerate}
\item[(1)] 
Joint work with Skinner $p || N$ + extra condition .
\item[(2)]
Skinner with cohomological condition .
\item[(3)]
For refined $\text{BSD}(p)$, $\displaystyle \operatorname{Reg}(E/\bQ) = \frac{\langle y, y \rangle_{\text{NT}}}{[E(\bQ) : \bZ y]^2}$, where $\langle \cdot, \cdot\rangle_{\text{NT}}$ is the N\'{e}ron-Tate pairing. 
\item[(4)]
$\Sel_p(E/\bQ) = \begin{cases} 0, \\ \bZ/p\bZ, \end{cases}$ implies $ r_{\text{an}} = \begin{cases} 0, \\ 1, \end{cases}$ respectively. 
\end{enumerate}
\end{rmk}

\subsection{Heegner points on Shimura curves} 
{\ }

Let $K = \bQ(\sqrt{-D})$. $N = N^+ N^{-}$, where
\[N^+ = \displaystyle \prod_{\substack{\ell | N \\ \ell \text{ split }}} \ell,\quad  \text{ and }\quad \displaystyle N^{-} = \prod_{\substack{\ell | N \\ \ell \text{ inert}}} \ell.\]
Recall the Heegner hypothesis : $N^{-}$ is square-free, and $\nu(N^{-}): = \# \{\ell : \ell | N^{-}\}$ is even. Let $K_\infty \supset K \supset \bQ$, where $K_n/K$ are the ray class fields, and  $\Gal(K_n/K) = \Pic(\OO_{K, n})$ under the Artin map, with $\OO_{k, n}:= \bZ + n \OO_K$, hence $K_1$ is the Hilbert class field.  Let $X_{N+ N^{-}} $ be the Shimura curve attached to the quaternion ramified at $N^{-}$ with $\Gamma_0(N^+)$-level, and $\phi: X_{N+ N^{-}}  \to E$ and continuous deformation to the ``Heegner points'', $P_n \in E(K_n)$, $P_1 \in E(K_1)$, $y_K = \operatorname{tr}_K^{K_1} P_1 \in E(K)$. 

Then the Gross-Zagier formula is: 
\[
\displaystyle \frac{\langle y_K, y_K \rangle}{\deg(\phi)} = \frac{L'(E/K, 1)}{\frac{1}{\sqrt{|D|}} \langle f, f \rangle_{p_m}}.
\]
\begin{dfn} $\ell \nmid NpD$ are called \emph{Kolyvagin primes} if $\ell$ is inert in $K$, $p | \gcd(\ell+1, a_\ell)$ i.e.,  $\dim_{\bF_p} E(\bF_{\ell^2})/p = 2$. Define
\begin{align*}
\Lambda :&= \{ n = \prod_{\ell} \ell : \text{ square-free product of Kolyvagin primes} \},\\
\kappa :&= \{c(n) \in H^1(K, E_p): n \in \Lambda\}.
\end{align*}
For $M \geq 1$, 
\begin{align*}
\kappa_M :&= \{c_M(n) \in H^1(K, E_{p^M}): n \in \Lambda_M\}\\
\kappa_\infty:& = \bigcup_{M > 1} \kappa_M,
\end{align*}
where $\kappa_\infty$ is called a ``Kolyvagin system''.
$\Sel_{p^M}(E/K)$. For $n = 1$, $c_M(1)$ satisfies
\begin{align*}
E(K)/p^m E(K) &\rightarrow H^1(K, E_{p^m})\\
y_K \in E(K) &\mapsto C_M(1).
\end{align*}
\[
K_\infty: = \bigcup_{M \geq 1} \kappa_M \ne 0 \text{ if } y_K \text{ is non-torsion}\]
\end{dfn}

\begin{dfn} (Vanishing order) 
\[
\ord K_\infty: = \min_{c_M(n) \ne 0} \nu(n)
\]
\end{dfn}

\begin{xmp} $\ord \kappa_\infty = 0 \Leftrightarrow c_M(1) \ne 0 \Leftrightarrow y_K \text{ is non-torsion}$ $\Leftrightarrow r_{\text{an}} = 1$. 
\end{xmp}

\subsection{Kolyvagin conjecture}

\begin{cnj} $K_\infty \ne \{0\}$ at least $c_M(n) \ne 0$ for some $M \geq 1$, $n \in \Lambda$ ($\ord K_\infty < \infty$). 
\end{cnj}

\begin{thm} (Kolyvagin) Assuming the conjecture, we have

\[\kappa_\infty = \max \{ r_p(E/K)^+, r_p(E/K)^{-}\} - 1.\]
\begin{itemize}
\item For $M \geq 1$, 
\[
\ord \kappa_M = \max \{\rank \Sel_{p^M}(E/K)^+, \rank \Sel_{p^M}(E/K)^{-}\} - 1.
\] 
\item
For $M = 1$,
\[
\ord \kappa_1 = \max \{\dim_{\bF_p} \Sel_p(E/K)^+, \dim_{\bF_p} \Sel_p(E/K)^{-}\} - 1
\]
\end{itemize}
\end{thm}

\begin{thm} Under earlier assumptions, $\kappa_\infty \ne 0$. Indeed, $\kappa_1 \ne 0$. 
\end{thm}

\begin{rmk} For $\rank = 1$, this is new. 
\end{rmk}

\begin{proof} $r_p(E/K) = 1 \Leftarrow (GZK) r_{\text{an}} = 1 \Leftrightarrow \mathcal{J}_\kappa \text{ non torsion} \Leftrightarrow \ord \kappa_\infty = 0$. Further, $r_p(E/K) = 1 \Leftrightarrow \max\{r_p(E/K)^+, r_p(E/K)^{-}\} = 1 \Leftrightarrow \ord \kappa_\infty = 0$. 
\end{proof}

Admissible BD , $m = q_1 q_2 \cdots$
\[f \text{ mod } p = f_m \text{ mod } p\]
\[\kappa(1) = \{c_1(n) \in H^1(K, E_p): n \in \Lambda\}\]
\[\kappa(m) = \{c_1(n,m) \in H^1(K, E_p): n \in \Lambda\}\]

``$\kappa(1) \equiv \kappa(m)$'' known as Jochnowitz congruence  or Bertolini-Darmon congruence . 

\renewcommand{\thesubsection}{\arabic{section}.R}
\begingroup
\renewcommand{\addcontentsline}[3]{}% Remove functionality of \addcontentsline
\endgroup
%%%13.
\newpage
\renewcommand{\thesubsection}{\arabic{section}.\arabic{subsection}}

\section{On the $p$-converse of the Kolyvagin-Gross-Zagier theorem \\
by Rodolfo Venerucci}\label{13}

Let $A/\bQ$ an elliptic curve of conductor $N_A$, $p$ an odd prime such that $p || N_A$ and  $p$ is of multiplicative reduction. 
\begin{thm} Under some technical assumptions , 
\[
\ord_{s=1} L(A,s) = 1 \Longleftrightarrow \rank_\bZ A(\bQ) = 1 \text{ and } \# \Sha(A/\bQ)_{p^\infty} < \infty
\]
\end{thm}

Credits: 
\begin{itemize}
\item 
When $p$ splits multiplicatively, Venerucci  using Bertolini-Darmon and Skinner-Urban, Skinner-Zhang  using Bertolini-Darmon and Skinner-Urban.
\item
When $p$ even split multiplicatively: Bertolini-Darmon and Skinner-Urban ($+ \varepsilon$).
\end{itemize}
From now on: $\rank_\bZ A(\bQ) = 1$ and $\Sha(A/\bQ)_{p^\infty}$ is finite.

\subsection{The $p$-converse for split multiplicative primes}
{\ }

$A/\bQ_p$ has split multiplicative reduction, that is, $G_{\bQ_p} \circlearrowright A(\ol{\bQ}_p) \cong \ol{\bQ}_p^\ast / q_A{\bZ},$ $q_A \in p \bZ_p$, Tate's power of $A/\bQ_p$. $K/\bQ$ imaginary quadratic such that 
\begin{enumerate}[(i)]
\item
$p$ splits in $K$, 
\item
$\ord_{s=1} L(A, \chi_K, s) = 1$ (here $\chi_K$ is the quadratic character of $K$). 
\end{enumerate}
\begin{rmk} For $K$ to exist we have to assume that there exists $q \ne p$ with $q || N_A$. 
\end{rmk}

\[\rho = \rho_{A,p} : G_\bQ \rightarrow \Aut(\operatorname{T}_p(A)) \cong \GL_2(\bZ_p)\]
\[\rho_\infty : G_\bQ \rightarrow \GL_2(\mathcal{R})\]
Pride's ``central non-trivial'' $p$-ordinary deformation of $\operatorname{Top}(A)$. \\ \\
$\mathcal{R}$ is a regular, finite, flat over $\bZ_p[[x]] \hookrightarrow \mathcal{A}(u)$, which is a $\bQ_p$-valued locally analytic functions over a $p$-adic disc, and $2 \in U$. 
Define
\[
U^d: = \displaystyle U \cap \bZ_{\text{even}}^{\geq 2}.
\]
For every $k \in U^d$, $\rho_k : g_\bQ \rightarrow \GL_2(\mathcal{R}) \rightarrow \bQ_p$. \\ \\
$\rho_2 \cong \rho$; $\rho_k = \begin{cases} p\text{-adic Deligne representation} \\ \text{of an eigenform } f_k \in S_k(\Gamma_0(N_A), \bZ_p) \end{cases}(\kappa/2)$ as in Urban's Lectures~\ref{2}-\ref{3}, we can attack to these slate. \\ \\
1) Mazur-Vritopound  $\mathcal{L}_p(\rho_\infty / K) = \mathcal{L}_p(\rho_\infty) \cdot \mathcal{L}_p(\mathcal{L}_\chi^{\chi_k}) \in \mathcal{R}$ such that $u \in U^d$.
\[\mathcal{L}_p(\rho_\infty) = (1 - p^{n/2 - 1}) \cdot \mathcal{L}(f_n, n/2)^{\text{alg}} \in \ol{\bQ}_p\]

The fact that $\rho(f_\infty, 2) = 0$ forces both $\mathcal{L}_p(\rho_\infty)$ to vanish to order at least 2. \\ \\
2) ``Big'' Selmer group: $\Sel(\rho_\infty) \subset H^1(\bQ, \rho_\infty \otimes \mathcal{R}^\vee)$ such that $X(\rho_\infty) = \Sel(\rho_\infty)^\vee$. Then: $X(\rho_\infty)_\kappa = X(\rho_\infty) \otimes_{\mathcal{R}, ev_\kappa} \bQ_p \sim H_f^1(\bQ, \rho_\kappa),$ for all but finitely many $\kappa \in U^d$. Moreover, $X(\rho_\infty)_2^\ast \cong \tilde{A}(\bQ) \otimes \bQ_p = A(\bQ) \otimes \bQ_p \oplus q_A \bQ_p$. \\ \\
We want to prove: $\ord_{s=1} L(A/K, s) = 2$ (if and only if $\ord_{s=1} L(A,s) = 1$). This will follow by these main steps (``Essence of Iwasawa theory''). 
\begin{enumerate}
\item[Step A]
(Skinner-Urban) Write $\chr(\rho_\infty) = \chr_{\mathcal{R}} (X(\rho_\infty)) \mathcal{R}/\mathcal{R}^\ast.$ $\chr(\rho_\infty/K) = \chr(\rho_\infty) = \chr(\rho_\infty^{\lambda_K})$.

\[\ord_{\kappa = 2} \mathcal{L}_p(\rho_\infty.\bQ) = (L_p^{\pm}, \cdots ) \leq \ord_{\kappa = 2} \chr(\rho_\infty / K)\]

\begin{rmk} The result of Skinner-Urban is over $K$, for $p$ split. 
\end{rmk}
\item[Step B]
(Bertolini-Darmon) $\displaystyle \frac{d^2}{dx^2} \mathcal{L}_p(\rho_\infty) = \log_{A/\bQ_p}^2 (\bP_{?}),$ where $\bP_{?} \in A^{?}(\bQ) \otimes \bQ$ is a Heegner point coming from a Shimura curve Pix of $A^{?}$. \\ \\
By Gross-Zagier-Zhang formula, $\bP_{?} \ne 0$ if and only if $\ord_{s=1} L(A^{?}, s) = 1$, so
\[\ord_{s=1} L(A/K,s) = 2 \text{ if and only if } \ord_{\kappa = 2} (\rho_\infty/K) = 2.\] 
\item[Step C] Prove that $\ord_{\kappa = 2} \chr(\rho_\infty^{?}) = 2$ ($\geq 2$ is easy). Use algebraic BSD formula
\[\frac{d^2}{d\kappa^2} \chr(\rho_\infty^{?})_{\kappa = 2} = \det(\langle \cdot, \cdot \rangle_{\rho_\infty, 2}^{\text{Nek}})\]
where $\langle \cdot, \cdot \rangle_{\rho_\infty, 2}^{\text{Nek}}$; $\tilde{A}^{?}(\bQ) \times \tilde{A}^{?}(\bQ) \rightarrow \bQ_p$, is alternating and ``arithmetically'' defined by N\'{e}ron's Poitou-Tate for Selmer complexes. \\ \\
We prove that for all $p \in A(\bQ)$, $\langle p, q_A\rangle = \log_{A/\bQ_p} (P)$. This implies 
\[\frac{d^2}{dn^2} \chr(\rho_\infty^{?}) = \log_{A/\bQ_p}^2 (\ol{P}),\]
$\ol{P} \bZ = A^{?}(\bQ)/\text{tors}$. Then, 
\[2 \leq \ord_{\kappa = 2} \mathcal{L}_p(\rho_0/\kappa) \leq \ord_{\kappa = 2} \chr(\rho_\infty/K) = h,\]
so BD follows. 
\end{enumerate}
\subsection{Non-split case}
{\ }

Let $K/\bQ$ be imaginary quadratic, $p$ inert in $K$. $E/K_p$ has split multiplicative reduction. \\
(i) $L(A, \chi_K, 1) \ne 0 \Rightarrow \Sha(A/K)_{p^\infty} < \infty$ \\
(ii) $\rank_\bZ A(K) = 1$.

\begin{rmk} We are in the definite case. 
\[\rho_\infty: G_K \rightarrow \GL_2(\Lambda), \Lambda = \bZ_p[[\Gal(K_\infty/K)]] \circlearrowright \mathcal{A}(\bZ_p)\]
\[\mathcal{L}_p(\rho_\infty/K) = L_p(f_A); \chr(K_\infty/K) = \chr_\Lambda \Sel_{p^\infty}(A/K_\infty)^\vee\]
\end{rmk} 

\begin{rmk} Since $A/K_p$ has split multiplicative reduction we have $\ord_{s=1} \mathcal{L}-p(\rho_\infty/K) \geq 2$, $\ord_{s=1} \chr(\rho_\infty/K) \geq 2$.
\end{rmk}

We want to prove that $\ord_{s=1} L(A/K, s) = 1$ We thus have:
\begin{enumerate}
\item[Step A]
(Bertolini's Lecture~\ref{10})
\item[Step B]
(Bertolini-Darmon ) 
\[
\displaystyle \frac{d^2}{ds^2} \mathcal{L}_p(\rho_\infty/K) = \log_{A/\bQ_p}^2 (\bP_K - a_p(A) \ol{\bP}_K).
\]
with the assumption $a_p(A) = 1$, where $\bP_K$ is ``the'' Heegner point coming from $X_{N_A^+, N_A^{-}} \rightarrow A$. This implies
\[\ord_{s=1} L(A/K, s), \ord_{s=1} \mathcal{L}_p(\rho_\infty/K) = 2 \] 
\item[Step C]
\[
\frac{d^2}{ds^2} \chr(\rho_\infty/K) = \# \Sha(A/K)_{p^\infty}, \log_{A/\bQ_p}^2(\ol{P}).
\]
\end{enumerate}
\renewcommand{\thesubsection}{\arabic{section}.R}
\begingroup
\renewcommand{\addcontentsline}[3]{}% Remove functionality of \addcontentsline
\endgroup

%%%14.
\newpage
\section{Iwasawa main conjecture for Rankin-Selberg $p$-adic $L$-functions\\ by Xin Wan }\label{14}
\renewcommand{\thesubsection}{\arabic{section}.\arabic{subsection}}


\subsection{Iwasawa-Greenberg main conjecture }
{\ }

Suppose $T$ is a $p$-adic Galois representation for $G_\bQ$ and $\dim T = d$. Let $d^\pm$ be the dimension of the eigenspace over $\bC$ corresponding to $\pm 1$, respectively. Suppose $T$ is geometric, so that
\[V \otimes \bC_p \cong \bigoplus_i \bC_p(i)^{h_i},\]
where $\bC_p$ is the algebraic closure of $\bQ_p$, and $\bC_p(i)$ refers to a Tate twist with $i$. Further, assume $ \sum_{i\geq 1} h_i = d^+$. 

\subsection{Panchishkin condition }
{\ }

$V$ contains a $\bQ_p$-subspace $W_p$ which is invariant under $G_p$ such that
\[W \otimes \bC_p \cong \bigoplus_{i \geq 1} \bC_p(i)^{h_i}.\]
If $W_p$ exists, denote it by $F^+ V$. Define $F^+(T \otimes \bQ_p/\bZ_p)$, the image of $F^+ V$. 

Greenberg defined
\begin{align*}
H_f^1(G_p, T \otimes \bQ_p/\bZ_p) &:= \ker \left \{H^1(G_p, T \otimes \bQ_p/\bQ_p) \rightarrow H^1(I_p, \frac{T \otimes \bQ_p/\bZ_p}{F^+ (T \otimes \bQ_p/\bZ_p)} \right \}\\
H_f^1(G_\bQ, T \otimes \bQ_p/\bZ_p) &:= \ker \left \{ H^1(G_\bQ, T \otimes \bQ_p/\bZ_p) \rightarrow \prod_{l \ne p} H^1(I_l, T \otimes \bQ_p/\bZ_p) \times \frac{H^1(G_p, T \otimes \bQ_p/\bZ_p)}{H_f^1(G_\bQ, T \otimes \bQ_p/\bZ_p)} \right \}
\end{align*}

\begin{xmp} Let $K/\bQ$ a quadratic imaginary field extension and suppose that $p$ splits as $v_0 \ol{v_0}$. Let $g_\xi$ be a CM form for a Hecke character $\xi$ of $K$. Let $f$ be a cuspidal eigenform. \\ \\
Case 1: If $\operatorname{weight} g_\xi < \operatorname{weight} f,$ then the Panchishkin condition is true if and only if ordering at $p$ 
\[f = \sum_{n=1}^\infty a_n q^n, \text{ } p \nmid a_p\]
Case 2: if $\operatorname{weight} g_\xi > \operatorname{weight} f$, then the Panchishkin condition is always true.
\end{xmp}

\subsection{Iwasawa theory}
{\ }

Let $K/\bQ$ be a quadratic imaginary extension of $\bQ$, $K_\infty/K$ the unique $\bZ_p^2$-extension unramified outside. $\Lambda = \bZ_p[[\Gamma]]$, $\Gamma = \Gal(K_\infty/K)$. 

\begin{dfn} 
\begin{align*}
\Sel_{K_\infty}(T \otimes \bQ_p/\bZ_p) &:= \lim_{K \subset K' \subset K_\infty} \Sel_{K'}(T \otimes \bQ_p/\bZ_p) \\
X_{K_\infty}(T) &:= (\Sel_{K_\infty}(T \otimes \bQ_p/\bZ_p))^\ast\end{align*}
finite $\Lambda$-modules.
\end{dfn}

\subsection{Analytic side}
{\ }

Conjecturally: $p$-adic $L$-functions $\mathcal{L}_{p, K_\infty}(T) \in \Lambda$ parametrizes the ``algebraic'' part of special $L$-values of $L(T \otimes \chi, 0)$ for $\chi$ finite order characters of $\Gamma$.
\begin{cnj}
Iwasawa-Greenberg main conjecture: $X_{K_\infty}(T)$ is a $\Lambda$-torsion and $\chr_\Lambda(X_{K_\infty}(T)) = \left(\cL_{p, K_\infty}(T)\right)$ as ideals of $\Lambda$. $\chr_A(M) = \{x \in A | \ord_P x \geq \log_{A_P}(M_P)\}$ for any height on prime $P$ of $A$. 
\end{cnj}
\begin{thm} (X. Wan ) Let $f$ be a weight 2, trivial character cuspidal eigenform. Suppose $2$ splits in $K$, $p \geq 5$, conductor $N$ of $f$ is square-free and divisible by at least one prime non-split in $K$. Suppose $k > 6$, $k \equiv 0 \pmod{p-1}$. If the $p$-adic avatar of $\xi \lvert \cdot \rvert^{k/2} (\omega^{-1} \cdot N_m)$ factors through $\Gamma_K$, then
\[\cL_{f, K_\infty}(\rho_f \otimes \rho_{g_\xi}) \supset \chr_{\Lambda \otimes_{\bZ_p} \bQ_p} (X_{f, K_\infty} (\rho_f \otimes \rho_{g_\xi}). \]
\end{thm}

Families of Klingen Eisenstein series on $U(3,1)$ are congruent modulo $\cL_{p, K_\infty}$ to a cusp form on $U(3,1)$. Further, it is a reducible Galois representation and thus congruent to ``more irreducible'' representations. The congruence can be established via a lattice construction to elements in Selmer group. \\ \\
How to construct family of Klingen Eisenstein series?

\[U(3,1), \begin{pmatrix} & & 1 \\ & \zeta & \\ -1 & & \end{pmatrix}, \quad \zeta \in M_2, \quad \Gamma \zeta \text{ is diagonal.} \]

$P$ is upper triangular, Klingen parabolic. Using the doubling method , $U(3,1) \times U(0,2) \hookrightarrow U(3,3)$. Siegel Eisenstein series in $U(3,3)$ (induced from Siegel parabolic $Q = \begin{pmatrix} \ast & \ast \\ 0 & \ast \end{pmatrix} \subset U(3,3)$. $\tau$ is a Hecke character of $K^\times$. Then
\[E_{\text{Kling}}(\tau, f, g_1) = \int_{U(0,2)Q \setminus U(0,2)(A)} E_{\text{Sieg}}(\tau, (g_1, g_2)) \ol{\tau}(\det g_2) f(g_2) dg_2\]
Hard part: make choices of primes above $p$. 
\begin{itemize}
\item
it is more about doing things $p$-adic analytically
\item
pulls back to (semi)-ordering Klingen Eisenstein series 
\item
the Fourier-Jacobi coefficients (not too difficult to calculate)
\end{itemize}

Turns out that Siegel-Weil sections whose Fourier coefficients for 
\[S = \begin{pmatrix} a & b & c \\ d & e & f \\ g & h & j \end{pmatrix}\]
is non-zero if and only if $S$ has $\bZ_p$ entries and $b \in \bZ_p^\times$, $\det \left(\begin{matrix} b & c \\ e & f \end{matrix}\right) \in \bZ_p^\times$. \\ \\
In this case Fourier coefficients (local) of $S$ to be $\xi_1(b) \xi_2\left(\frac{\det \begin{pmatrix} b & c \\ e & f \end{pmatrix}}{\det b} \right)$, $\xi_1, \xi_2$ depend on $\tau$. \\ \\
Choice motivated by differential operators. 

\subsection{Study the Fourier-Jacobi coefficients}
{\ }

Need: Fourier-Jacobi coefficients are co-prime to $p$-adic $L$-function. 
\[U(3,1), N = \begin{pmatrix} 1 & \ast & \ast & \ast \\ & 1 & \ast & \ast \\ & & 1 & \ast \\ & & & 1 \end{pmatrix}, \text{ } U = \begin{pmatrix} 1 & & & \\ & \ast & \ast & \\ & \ast & \ast & \\ & & & 1 \end{pmatrix}, \beta \in Q_+, F \text{ any form}\]

\[F J_\beta(F) = \int F\left( \begin{pmatrix} 1 & & & t \\ & 1 & & \\ & & 1 & \\ & & & 1 \end{pmatrix} g \right) e(-\beta t) dt \]
considered as a function on $NU$. \\ \\
We have algebraic definitions for Fourier-Jacobi coefficients involving global sections of line bundles $\cL(\beta)$ on $2$-dimensional CM abelian variety (theta function CM). Can construct another $\theta$, on $NU$ $(U(1) \leftrightarrow u(2))$. 
\[L_{\theta_1} : M(NU(\bQ) \backslash NU(\bA) \rightarrow M(U(\bQ) \backslash U(\bA)) \]
\[L_{\theta_1}(G)(u) = \int_{N(\bQ) \backslash N(\bA)} G(nu) \theta_1(nu) du \]
(can be made algebraic). \\ \\
We construct auxiliary Hida family $h$ on $U(2)$ and study
\[\langle L_{\theta_1} FJ_\beta(F), h \rangle \in Q_l[[\Gamma_K ]]\]

Idea: doubling method. 

\[FJ_\beta(E_{\text{Sieg}}(g) = \int E\left(\tau, \begin{pmatrix} 1 & & & t \\ & 1 & & \\ & & 1 & \\ & & & 1 \end{pmatrix} g \right)e(-\beta t) dt \]
considered as function on $U(2,2) \cdot N$. 
\[N \subset N' \subset U(3,3) \]
restrict to $P \times U(0,2)$. \\ \\
A calculation shows
\[FJ_\beta(E_{\text{Sieg}}) = E' \cdot \Theta\]
$E'$ is the Siegel Eisenstein series on $U(2,2)$, $\Theta$ theta function on $U(2,2) \cdot N'$, and another calculation shows
\[\Theta |_{P \times U(2,2)} = \theta_2 \boxtimes \theta_3\]

$\theta_2, \theta_3$ are theta functions on $P$ and $U(0,2)$. Easy to see $L_{\theta_1}(\theta_2$ is a constant function on $U(2,0)$. Set 
\begin{align*} A & = \langle L_{\theta_1}, FJ_1(E_{\text{Kling}}), h \rangle = B \int_{U(2) \times U(2)} E'(g_1, g_2) \cdot h(g_1) \cdot \theta_3(g_2) f(g_2) dg_1 dg_2 \\
& = B \cdot \cL_{X_n} \int_{U(2)} h(g_2) \theta_3(g_2) f(g_2) d g_2 \end{align*}

triple product $U(2) \times U(2) \hookrightarrow U(2,2)$ take $h$ family of CM forms. Triple product $\cL_1 \cdot \cL_2$, Hecke on non-vanishing modulo $p$ of special $L$-values. 

\renewcommand{\thesubsection}{\arabic{section}.R}
\begingroup
\renewcommand{\addcontentsline}[3]{}% Remove functionality of \addcontentsline
\endgroup

%%%15.
\newpage
\section{Level raising mod $2$ and arbitrary $2$-Selmer ranks\\ by Li Chao}\label{15}
\renewcommand{\thesubsection}{\arabic{section}.\arabic{subsection}}

\subsection{Motivation}
{\ }

Let $E/\bQ$ be an elliptic curve. The celebrated B-SD conjecture says that 
\[\rank(E(\bQ)) = \ord_{s=1} L(E,s).\]
The B-SD conjecture actually asserts something more refined. Indeed, they conjectured that in fact we have the following formula
\[ \frac{L^{(r)}(E,1)}{r! \Omega(E) R(E)} = \frac{ \prod_p c_p \cdot \# \Sha(E) }{(\# E(\bQ)_{\text{tor}})^2 }\]

B-SD$(\ell)$: $r = 0, \ell \geq 3$, Skinner-Urban , Kato . \\
(with additional conditions): $r = 1, \ell \geq 5$, W. Zhang . \\ \\
Question: What about $\ell = 2$? Why care about it?

\begin{rmk} For the B-SD formula itself, the prime $2$ is the most important to examine because it appears as a factor the most often.
\end{rmk}

$r = 1$, $E$ corresponds to some $f$. 
 
\[f \in S_2(N), f \equiv g \in S_2(Ng) \pmod{\ell}\]

where $g$ corresponds to some elliptic curve of rank 0 (level raising). This is done via something called the Jochnowitz congruence, due to Bertolini-Darmon . \\ \\
We want a pseudo-congruence of the form
\[L(f,1) \text{``} \equiv\text{''} L(g, 1) \pmod{\ell}\]
which makes no sense, as both sides are transcendental numbers. To get an expression that makes sense, we choose an auxiliary imaginary quadratic field $K$ corresponding to $y_K \in E(K)$ with
\[\ell \nmid y_K \Leftrightarrow \Sel_l(A/K) = 0 \]

\subsection{Level raising}
{\ }

Let $\ell$ be a prime number and 
\[\ol{\rho}_\ell = \ol{\rho}_{E,\ell} : G_\bQ \rightarrow \Aut(E[\ell]) = \GL_2(\bF_\ell)\]
Write the corresponding modular form as
\[f = \sum_{n \geq 1} a_n q^n\]

\begin{dfn} A prime $q \nmid N\ell$ is \emph{level raising} $\pmod{\ell}$ for $E$ if
\[\ol{\rho}(\operatorname{Frob}_q) = \pm \begin{pmatrix} 1 & \ast \\ & 1 \end{pmatrix} \Leftrightarrow a_q \equiv \pm (q + 1) \pmod{\ell} \]
\end{dfn}

\begin{thm} (Ribet) If $\ol{\rho}$ is absolutely irreducible and $q$ is a level raising prime, then there exists $g \in S_2(Nq)$ new at $q$ such that $f \equiv g \pmod{\ell}$.
\end{thm}

\begin{thm} (Diamond-Taylor) If $\ol{\rho}$ ($\ell \geq 3$) is absolutely irreducible and $q_1, \cdots, q_m$ are level raising, then there exists $g \in S_2(Nq_1 \cdots q_m)$ new at each $q_i$ such that $f \equiv g \pmod{l}$.
\end{thm}

\begin{rmk} $g = \displaystyle \sum_{n \geq 1} b_n q^n$, $F = \bQ(\{b_n\})$, there exists $\lambda | \ell$ of $F$ such that $a_p \equiv b_p \pmod{\lambda}$ for all $p \ne q_1, \cdots, q_m$.
\end{rmk}

\begin{xmp} $E = X_0(11)$, $l = 3, q = 7$. $a_7 = 2 \equiv -8 \equiv -(7+1) \pmod{3}$. 

See table (tex later).

\end{xmp}

\begin{rmk} At $p || Nq_1 \cdots q_m$, $b_p \in \{\pm 1\}$. 
{\ }
\begin{enumerate}
\item
$p || N$, $b_p = a_p$ ($b_p \equiv a_p \pmod{\ell}$)
\item
$p = q_i \not \equiv -1 \pmod{\ell}$, the $b_p$ is determined.
\item
$p = q_i \equiv -1 \pmod{\ell}$, Ribert proved both $b_p = \pm 1$ can occur.

\end{enumerate}
For $l = 2$, can $b_p = a_p$? The answer is no. Can both signs occur? The answer is yes.
\end{rmk}

Assumptions ($\star$):
{\ }
\begin{enumerate}
\item[(1)]
$E$ is good at 2.
\item[(2)]
$\ol{\rho} : G_\bQ \rightarrow \GL_2(\bF_2) = S_3$ is surjective. 
\item[(3)]
$N(\ol{\rho}) = N$.
\item[(4)]
$\ol{\rho} |_{G_{\bQ_2}}$ is non-trivial (i.e. $2$ does not split in $\bQ(E[2])$).
\end{enumerate}

\begin{thm} (C. Li) Assume ($\star$) and $q_1, \cdots, q_m$ are level raising. Further, suppose $p || Nq_1 \cdots q_m$ and let $\varepsilon_p \in \{\pm 1\}$ prescribe sign. Then there exists $g \in S_2(Nq_1 \cdots q_m)^{\text{New}}$ such that $f \equiv g \pmod{2}$ and $b_p = \varepsilon_p$ for all but possibly one $p || N$.  
\end{thm}

\begin{xmp} $f = 11a, q_1 = 7, q_2 = 13$. See table, tex later.
\end{xmp}

\begin{xmp} $f = 35a$, $q = 19$. See table, tex later.
\end{xmp}

Strategy: 
{\ }
\begin{enumerate}
\item[(1)]
$D_\tau T$ breaks down for $\ell=2$ because of Fontaine-Laffaille theory . This can be salvaged for $E$ super singular at $2$. 
\item[(2)]
$E$ ordinary at $2$. Key ingredient: P. Allen , $\operatorname{big}(R) = \operatorname{big}(T)$ for nearly ordinary $2$-adic representations with dihedral image. 
\item[(3)]
In both cases, a level raised form with prescribed signs everywhere but possibly ramified at one auxiliary prime $q_0$. Quadratic twist back get rid of $q_0$ at the cost of not prescribing one $b_p$, $p || N$. 
\end{enumerate} 

\subsection{$2$-Selmer ranks}
{\ }

Recall our $f,g$ corresponding to elliptic curves $E,A$ of rank $1, 0$ respectively, such that
\[f \equiv g \pmod{2}.\]
($A \circlearrowleft \OO = \OO_F$, $k = \OO/\lambda$). 

\[E[2] \otimes k \cong A[\lambda]\]
\[\Sel_2(E) \otimes k \hookrightarrow H^1(\bQ, E[2] \otimes k) = H^1(\bQ, A[\lambda]) \]
Note that $\Sel_\lambda(A) \hookrightarrow H^1(\bQ, A[\lambda])$. 

\begin{thm} Assume conditions $(\star)$ and $E$ has negative discriminant. Then for all $n \geq 0$, there exist infinitely many $A$ in the level raising family such that
\[\dim_k \Sel_\Lambda(A) = n. \]
\end{thm}

Compare with

\begin{thm} (Mazur-Rubin) Assume conditions $(\star)$ and $E$ has negative discriminant. Then for all $n \geq 0$, there exist infinitely many $E^{(d)}$ in the quadratic twist family such that
\[\dim_{\bF_2} \Sel_2(E) = n. \]
\end{thm}

\subsection{Bad news}

\begin{thm} Under certain conditions, 
\[\rank \Sel_2(E/K) = 1 \Rightarrow \rank \Sel_\Lambda(A/k) = 2. \]
\end{thm}

$E = X_0(11)$.

\[ \begin{array}{|c|c|c|c|c|c|} 
\hline
p & A & d_K & \rank(A(K)) & \dim(\Sha(A/K)[2]) & \dim \Sel_2(A/K)   \\
\hline
7 & 77a & -8 & 2 & 0 & 2\\
7 & 77b & -8 & 0 & 2 & 2\\
13 & 143a & -7 & 2 & 0 & 2\\
13 & 143a & -8 & 2 & 0 & 2\\
17 & 187a & -7 & 2 & 0 & 2\\
17 & 187a & -24 & 0 & 2 & 2\\
19 & 209a & -7 & 2 & 0 & 2\\
19 & 209a & -19 & 2 & 0 & 2\\ 
29 & 319a & -8 & 2 & 0 & 2\\
29 & 319a & -19 & 0 & 2 & 2\\
\hline

\end{array}
\]
\renewcommand{\thesubsection}{\arabic{section}.R}
\begingroup
\renewcommand{\addcontentsline}[3]{}% Remove functionality of \addcontentsline
\endgroup

%%%16.
\newpage
\renewcommand{\thesubsection}{\arabic{section}.\arabic{subsection}}

\section{$p$-adic Waldspurger formula and Heegner points
\\ by Yifeng Liu }\label{16}

Let $p$ be a prime number and $\bC_p$ the algebraic closure of $\bQ_p$. Let $E \subset \bC_p$ be a CM number field, and $F \subset E$ a maximal totally real subfield. $\mathfrak{p}$ will denote a place of $F$, $\mathfrak{P}$ a place of $E$, over $p$. $\bA$ will denote the ring of adeles of $F$, $\bA^\infty$ is the ring of finite adeles of $F$. $\mathbb{B}$ is a totally definite incoherent quaternion algebra. This means that $\mathbb{B}_i$ is definite for any $i < \infty$, and incoherent means $\displaystyle \prod_v \epsilon(\mathbb{B}_v) = -1$. \\
Let $\mathbb{B} \rightarrow (X_v)_v$ be projective system of Shimura curves over $F$. $X$ is the projective limit of $X_U$, $U$ is an open compact subset $\bB^{\infty \times} = (\bB \otimes \bA^\infty)^\times$. 

\begin{dfn} A function $X(\bC_p) \rightarrow \bC_p$ is a $p$-adic Maa{\ss}  function if it is a pullback from a locally analytic function on $X_U$ for some $U$. 
\end{dfn}

This is joint work with Shouwu Zhang and Wei Zhang . 

\begin{xmp} $f: X \rightarrow A$ where $A$ is an abelian variety over $F$. $\omega \in H^0(A, \Omega_A^1)$, $\log_\omega : A(\bC_p) \rightarrow \bC_p$, with $f^\ast \log_\omega: X(\bC_p) \rightarrow \bC_p$.
\end{xmp}

Denote by $\mathcal{A}_{\bC_p}(\bB^\times) \circlearrowright \bB^{\infty \times}$. 

\subsection{The space of all $p$-adic Maa{\ss}  functions}
{\ }

An irreducible sub-representation $\pi \subset \mathcal{A}_{\bC_p}(\bB^\times)$ of $\bB^{\infty \times}$ is (cuspidal) classical if there exists a non-zero function in $\pi$ that is of the form $f^\ast \log_\omega$. 

\begin{rmk} Assume $\pi$ is classical. There exists a unique classical sub-representation $\pi^\vee \subset \A_{\bC_p}(\bB^\times)$ such that $\pi^\vee$ is isomorphic to the centrag of $\pi$. There's no canonical pairing between $\pi$ and $\pi^\vee$.
\end{rmk}

We are given an embedding

\[e: \bA_E^\infty \hookrightarrow \bB^\infty \text{ of } \bA^\infty \text{-algebras},\]
so that
\[E^\times \subset \bA_E^{\infty \times} \subset \bB^{\infty \times} \]
$Y = X^{E^\times}$, $Y = Y^+ \sqcup Y^{-}$ such that $E^\times$ acts on the tangent space of any point in $Y^{\pm}(\bC_p)$ via the character $\left(\frac{t}{t^c}\right)^{\pm 1}, t \in E^\times$. 

\begin{dfn} For $\phi \in \A_{\bC_p}(\bB^\times)$, $\varphi_{\pm}$ locally constant functions on $Y^{\pm}(\bC_p)$, define
\[\mathcal{P}_{Y^\pm}(\phi, \varphi_{\pm}) = \int_{Y^\pm(G_p)} \phi(t) \varphi_{\pm}(t) dt \]
where $dt$ is a Haar measure on $Y^{\pm}(\bC_p)$ with total weight 1, which can be expressed as a finite sum. 
\end{dfn}

From now on: $p$ splits in $E$, $P\OO_E = \mathfrak{P} \mathfrak{P}^c$.

\begin{dfn} A character 
\[ \chi : E^\times \backslash \bA_E^{\infty \times} \rightarrow \bC_p^\times \] is a $p$-adic character of weight $k \in \bZ$ if there exists $V \subset \bA_E^{\infty \times}$ open compact such that 
\[\chi(t) = \left(\frac{t_{\fP}}{t_{\fP^c}}\right)^k \]
for all $t \in V$.
\end{dfn}

$\chi$ us $\pi$-related if 
\[\epsilon(1/2, \pi_v, \chi_v) = \chi_v(-1) \eta_v(-1) \in (\bB_v) \]
for all $p \ne v < \infty$, where 
\[ \eta = \otimes \eta_v : F^\times \backslash \bA \rightarrow \{\pm 1\} \]
quadratic character associated to $E/F$. Let $\Xi(\pi)_k$ to be the set of all $\pi$-related $p$-adic characters of weight $k$. $\mathfrak{D}(\pi)$ to be the coordinate ring of the above curve, which is a complete $\bC_p$-algebra. 

\begin{rmk} When $F = \bQ$, $\mathcal{D}(\pi) \rightarrow \OO_{\bC_p}[[\Gamma_\infty]] \left[ \frac{1}{p}\right]$, $\Gamma_\infty$ the Galois group of the anti-cyclotomic $\bZ_p$-extension of $E$ at $p$.
\end{rmk}

$L: \bC_p \rightarrow \bC$, $\chi$ $p$-adic character of weight $k$. 
\[\begin{cases}\chi_v^{(1)} = 1, & v | \infty, v \ne \iota |_F \\ 
\chi_v^{(\iota)} = \left(\frac{z}{z^c}\right)^k, & v = \iota |_F, z \in E \otimes_{F, \iota} \bR \rightarrow \bC \\ 
\chi_v^{(\iota)} = \iota \circ \chi_v, & v < \infty, v \ne p \\
\chi_p^{(\iota)} = \iota (\chi_q(t))\left(\frac{t}{t^c}\right)^k), & t \in E_p^\times \end{cases} \] 

$\chi^{\iota} = \displaystyle \bigotimes_v \chi_v^{(\iota)} : E^\times \backslash \bA_E^\times \rightarrow \bC^\times$. Denote by $\pi^+ = \pi, \pi^{-} = \pi^\vee$. We choose:
{\ }
\begin{enumerate}
\item[(1)]
$(\cdot, \cdot)_\pi$: $\pi^+ \times \pi^{-} \rightarrow \bC_p$. 
\item[(2)]
$\bC: Y^+(\bC_p) \rightarrow Y^{-}(\bC_p)$ that is $\bA_E^{\infty \times}$-equivariant.
\item[(3)]
$\psi: F_p \rightarrow \bC_p^\times$ of level 0.
\end{enumerate}

The above defines some ``period ratios'' $\Omega_\iota(\chi)$ for any $\iota : \bC_p \rightarrow \bC$, $\chi \in \Xi(\pi)_k$ with $k \geq 1$. 

\begin{thm} (Liu, Zhang, Zhang ) There is a unique element $\cL(\pi) \in \mathcal{D}(\pi)$ such that for $\chi \in \Xi(\pi)_k$ with $k \geq 1$ and $\iota: \bC_p \rightarrow \bC$ we have
\[ \iota \left(\cL(\pi)(\chi) \right) = L(1/2, \pi^{(\iota)}, \chi^{(\iota)}) \frac{S_p(2) \cdot \Omega_\iota(\chi)}{L(1, \eta) L(1, \pi^{(\iota)}, \operatorname{Ad}} \frac{\epsilon(1/2, \psi, \pi_p^{(\iota)} \otimes \chi_{\beta}^{(\iota)})}{L(1/2, \pi_p^{(\iota)} \otimes \chi_{\beta^c}^{(\iota)})^2}\]
\end{thm}

$\phi_{\pm} \in \pi^\pm,$ $\varphi_\pm \in \sigma_\chi^\pm$. $\chi \in \Xi(\pi)_0$, $\sigma_\chi^\pm \subset \Gamma(Y^\pm)$ such that $\bA^{\infty \times}$ acts via $\chi^{\pm 1}$. \\ \\
\[\mathcal{P}_{Y^\pm}(\phi_\pm, \varphi_\pm) \in \operatorname{Hom}_{\bA_E^{\infty \times}} (\pi^\pm \otimes \sigma_\chi^\pm, \bC_p)\]
By Sato-Trennell , the latter space has dimension 1. \\ \\
There is a natural basis of $\Hom_{\bA_E^{\infty \times}} (\pi^+ \otimes \sigma_{\chi}^+, \bC_p) \otimes \Hom(\cdots) (\ast)$ denoted by
\[\alpha^\natural (\phi_+, \phi_{-}, \varphi_+, \varphi_{-} )\text{``} = \text{''} \int_{\bA^{\infty \times} \backslash \bA_E^{\infty \times}} (t \phi_+, \phi_{-})_\pi (t \varphi_+, \varphi_{-}) dt \]
with $\alpha^\natural \ne 0$ as a functional. 

\begin{thm} ($p$-adic Waldspurger) 
\[P_{Y^+}(\phi_+, \varphi_+) = \cL(\pi)(\chi) \frac{L(1/2, \pi_p \otimes \chi_{\beta^c})^2}{(1/2,\psi,\pi_p \otimes \chi_{\beta^c})} \alpha^\natural(\phi_+, \phi_{-}, \varphi_+, \varphi_{-})\]
where
\[P_{Y^+}(\phi_+, \varphi_+) = \int_{Y^+(\bC_p)} (f^\ast \log_\omega) \varphi_+(t)dt = \log_\omega(H_E)\]
$H_E$ is a Heegner cycle on $A$.
\end{thm}

\renewcommand{\thesubsection}{\arabic{section}.R}
\begingroup
\renewcommand{\addcontentsline}[3]{}% Remove functionality of \addcontentsline
\endgroup
%%%17.
\newpage
\section{Colloquium - Recent advances in the arithmetic of elliptic curves\\ by Kartik Prasanna}\label{17}
\renewcommand{\thesubsection}{\arabic{section}.\arabic{subsection}}

\subsection{Prelude}
{\ }

Classical identities: 
\[\frac{1}{1^2} + \frac{1}{2^2} + \frac{1}{3^2} + \cdots = \frac{\pi^2}{6}\]
\[\frac{1}{1^4} + \frac{1}{2^4} + \frac{1}{3^4} + \cdots = \frac{\pi^4}{90}\]
In general,
\[\frac{1}{1^{2n}} + \frac{1}{2^{2n}} + \frac{1}{3^{2n}} + \cdots =  \left \lvert \frac{-1}{2} (2 \pi i)^{2n} \frac{B_{2n}}{(2n)!} \right \rvert \]
where the rational numbers $B_n$ are the \emph{Bernoulli} numbers, which are the coefficients corresponding to
\[\frac{z}{e^z - 1} = \sum_{n=0}^\infty \frac{B_n \cdot z^n}{n!}. \]
\begin{rmk}
$2\pi i$ comes from geometry, and $B_{2n}$ comes from arithmetic. 
\end{rmk}
\subsection{Products over primes}
{\ }

We first consider an \emph{absolute value} defined over $\bQ$. A function $\lvert \cdot \rvert: \bQ \rightarrow \bR^{\geq 0}$ is called an absolute value if
{\ }
\begin{itemize}
\item
$\lvert x \rvert = 0$ if and only if $x = 0$ 
\item
$\lvert x + y \rvert \leq \lvert x \rvert + \lvert y \rvert$
\item
$\lvert xy \rvert = \lvert x \rvert \cdot \lvert y \rvert$.
\end{itemize}

The usual absolute value on $\bR$ gives an immediate example. However, other, not-so-obvious examples exist. Namely, for any prime number $p$ there is an absolute value $|\cdot|_p$ which is given by 
\[ |\alpha|_p = \frac{1}{\text{power of } p \text{ dividing }\alpha} \]

\begin{xmp} $p = 5$, $|10|_5 = 1/5$, $|1/5|_5 = 5$. 
\end{xmp}

A most remarkable fact, proved by Ostrowski, is that up to equivalence these are exactly all absolute values on $\bQ$. 

\begin{thm} Take $\alpha \in \bQ, \alpha \ne 0$, we have
\[\prod_{p \leq \infty} |\alpha|_{p} = 1.\]
\end{thm}

\begin{xmp} Take $\alpha = 17/21$. If $p \ne 3, 7, 17$, then $|\alpha|_p = 1$. On the other hand, $|\alpha|_3 = 3, |\alpha|_7 = 7, |\alpha|_17 = 1/17$. Further, $|\alpha| = \alpha = 17/21$.
\end{xmp}

A more interesting example is given by, for example, the \emph{Riemann zeta function}. Indeed, we have
\[\zeta(s) = \sum_{n=1}^\infty \frac{1}{n^s} = \prod_p \left(1 + p^{-s} + p^{-2s} + \cdots \right) = \prod_p \frac{1}{1 - p^{-s}}.\]
Riemann proved the remarkable fact that $\zeta$ can in fact be extended analytically to a meromorphic function on the complex plane, with a simple pole at $s = 1$. \\ \\
Recall our earlier example that the sum of the inverse of squares is equal to $\pi^2/6$. This can be expressed in terms of the $\zeta$ function
\[\frac{1}{1^2} + \frac{1}{2^2} + \frac{1}{3^2} + \cdots = \zeta(2) = \prod_p \frac{1}{1 - p^{-2}}.\]
What about the prime at infinity? How did we account for it in the product formula above? Riemann actually proved a \emph{functional equation} for the zeta function, which is given as follows. First define
\[\Lambda(s) = \pi^{-s/2} \Gamma(s/2) \zeta(s).\]
Then $\Lambda$ satisfies the functional equation
\[\Lambda(s) = \Lambda(1-s)\]
which is valid for all $s \in \bC$. One can view the factor
\[\pi^{-s/2} \Gamma(s/2)\]
as the factor corresponding to the prime $\infty$. \\ \\
The functional equation implies, by the known locations of the poles of the gamma function $\Gamma$, that $\zeta$ has simple zeroes at $s = -2, -4, -6, \cdots$. \\ \\
The $\zeta$ function is the simplest example of a class of functions called $L$-functions. These are functions which have the following properties
{\ }
\begin{enumerate}
\item
Defined as a product over primes
\item
Analytic continuation
\item
Attached to geometry
\item
Contain arithmetic information
\end{enumerate}

Geometry: Think about $\bA^1 \setminus \{0\}$. 
\[\int_\gamma \frac{dz}{z} = 2 \pi i\]
We think about things with two loops. One such object is a torus, which we can think of as $\bC/\Lambda$ where $\Lambda$ is a lattice. A torus is an object with genus 1; which we can naturally associate to an \emph{elliptic curve}, which is given by an equation of the form
\[y^2 = x^3 + Ax + B. \]
We assume that $A,B \in \bZ$, since we want the corresponding $L$-function to have arithmetic properties. Recall that
\[\zeta(s) = \prod_p \frac{1}{1 - p^{-s}}.\]
Note that the denominator is linear in $p^{-s}$. This is because the underlying geometry contains only one loop. In the case of a torus (which naturally has two loops on its surface) we would expect something that is \emph{quadratic} in $p^{-s}$ in the denominator, say
\[L(E,s) = \prod_p \frac{1}{(1 - \alpha_p p^{-s})(1 - \beta_p p^{-s})} \]
where $\alpha_p \beta_p = p$. Further, the quantity $(1 - \alpha_p)(1 - \beta_p) = \#E / \bF_p$ which is the elliptic curve $E$ over the finite field $\bF_p$. \\ \\
What about the prime at $\infty$, and the factor corresponding to it? In this case, we have the factor
\[(2\pi)^{-s} \Gamma(s) L(E,s) \]
If we define
\[\Lambda(E,s) = (2\pi)^{-s} \Gamma(s) L(E,s), \]
then we have the functional equation
\[\Lambda(E,s) = \pm \Lambda(E,2-s).\]
We now have the following remarkable theorem, which is essentially a consequence of the ingredients that went into proving Fermat's Last Theorem.
\begin{thm} $\Lambda(E,s)$ admits analytic continuation (Wiles /Taylor-Wiles ).
\end{thm}


\subsection{What is the Birch and Swinnerton-Dyer (BSD) Conjecture}
{\ }

If $E/\bQ$ is an elliptic curve, we can interpret $E(\bQ)$ as follows:
\[E(\bQ) = \{(x,y) : x,y \in \bQ, y^2 = x^3 + Ax + B\}.\]
There is a remarkable fact that $E(\bQ)$ forms a \emph{finitely generated abelian group}. In fact, there is a finite abelian group $G$ and a non-negative integer $r$ such that
\[E(\bQ) \cong \bZ^r \oplus G.\]
This integer $r$ is called the \emph{rank} of $E$. \\ \\
Now, recall that we can associate to $E$ the $L$-function $L(E,s)$ which has a zero at $s = 1$. The Birch-Swinnerton-Dyer conjecture then asserts
\begin{cnj} \[r = \ord_{s=1} L(E,s)\]
\end{cnj}

This conjecture is an example of a ``local-global'' principle, which is a fundamental phenomenon in the theory of numbers.

\begin{rmk} The left-hand-side of the above equation is called the \emph{algebraic rank} while the right hand side is called the \emph{analytic rank}. 
\end{rmk}

\subsection{Recent progress}

\begin{thm} (Gross-Zagier/Kolyvagin) If $r_{\text{an}}(E) = 0 \text{ or } 1,$ then $r_{\text{alg}}(E) = r_{\text{an}}(E)$.
\end{thm}

Not much was known about the converse until recently. The situation change with a recent theorem of C. Skinner and W. Zhang.

\begin{thm} (C. Skinner/W. Zhang) If $r_{\text{alg}}(E) = 0 \text{ or } 1$, then $r_{\text{an}}(E) = r_{\text{alg}}(E)$, under some additional hypotheses.
\end{thm}

Hypothesis: (W. Zhang) we need that $|\Sel_p(E)| = 1 \text{ or } p$. 

\begin{thm} (M. Bhargava-A. Shankar) For $p = 2, 3, 5$, $\Avg\Sel_p(E) = p+1$.
\end{thm}

This implies that the previous two theorems apply frequently, in fact with at least $66\%$ of curves. 

\subsection{Epilogue}

Remember that $\bQ$ can be embedded into $\bR$, via the completion with respect to the usual absolute value. It can also be embedded into fields $\bQ_p$ via completion by the $p$-adic absolute value $|\cdot|_p$. The field $\bQ_p$ contains a ring which naturally corresponds to the integers, which we denote by $\bZ_p$. \\ \\
If $m \equiv 0 \pmod{p-1}$ with $n \not \equiv 0 \pmod{p-1}$, then
\[\frac{B_m}{m} \equiv \frac{B_n}{n} \pmod{p}.\]
For a power series $f(T) \in \bZ_p[[T]]$, we have
\[f(1+p)^k - 1) = \zeta(1-k) = \cdots \]
\[Q(C_p^2) \rightarrow Q(C_p) \rightarrow Q\]
$C$ is the inverse limit of $C_p, C_{p^2}, \cdots$, and we have
\[C = \frac{\bZ_p[[T]]}{(g(T))}\]

\begin{cnj} (Iwasawa Main Conjecture) $f(T) = g(T)$.
\end{cnj}

This was proved by Mazur and Wiles  in the 1980's. There are two types of $L$-functions, $L^{\text{I}}$ and $L^{\text{II}}$, both with their versions of Iwasawa main conjecture. The $L^{\text{I}}$ case was settled by Skinner-Urban  and Kato  while the $L^{\text{II}}$ case was done by Xin Wan . \\ \\
We have three types of objects, corresponding respectively to geometry, arithmetic, and analysis. For example, given an elliptic curve $E$, we can associate an $L$-function $L(E,s)$, which is an analytic object. We can also associate to it a Selmer group, which is an arithmetic object. We believe the underlying principle is controlled by something called \emph{motivic cohomology} . Unfortunately, we do not know much about it! 
\renewcommand{\thesubsection}{\arabic{section}.R}
\begingroup
\renewcommand{\addcontentsline}[3]{}% Remove functionality of \addcontentsline
\endgroup

%%%18.
\newpage
\section{Parity of ranks of elliptic curves\\ by Vladimir Dokchitser}\label{18}
\renewcommand{\thesubsection}{\arabic{section}.\arabic{subsection}}


Let $E/K: y^2 = x^3 + Ax + B$ be an elliptic curve, where $K$ is a number field. 

\begin{cnj} (Parity conjecture - ``explicit form'') 
\[\rank E/K \equiv \sum_{v | \infty \Lambda_E} \lambda_v\pmod{2}\]
where $\lambda_v \in \{0,1\}$. Further, 
\begin{itemize}
\item
$\lambda_v = 0$ if $E/K_v$ is a good reduction 
\item
$\lambda_v = 1$ if $v | \infty$ or if $E/K_v$ has split multiplicative reduction
\item
$\lambda_v = 0$ if $E/K_v$ non-split multiplicative reduction 
\item
$\displaystyle \lambda_v \equiv \frac{\$ \bF_v - 1}{2}$ if $v \nmid 2$, $E/K_v$ odd potentially multiplicative reduction 
\item
$\lambda_v = \left \lvert \frac{ \delta_E \cdot \# \bF_v}{12} \right \rvert$ if $v \nmid 2,3$, $E/K_v$ odd potentially good reduction (here $\delta_E$ is the value minimal discriminant of $E$).
\item
there exist formulae for $v | 2,3$.
\end{itemize}
\end{cnj}

\subsection{(Conjectural) consequences}

{\ }
\begin{enumerate}
\item[(1)]
If $E$ is semistable, then 
\[\rank E/K \equiv \pmod \# v | \infty + \# \text{ split multiplicative primes } \pmod{2}\]
\item[(2)]
$E: y^2 = x^3 - x$, $d \in \bZ \setminus \{0,1\}$ square-free. Then
\[ \rank E_d/\bQ = \rank E/\bQ(\sqrt{d}) - \rank E/\bQ \equiv \sum_{v | 2 \infty} \lambda(E/\bQ(\sqrt{d})_v) \pmod{2}. \]
Here we remark that $\rank E/\bQ = 0$, so we obtain the consequence
\[\rank E_d/\bQ \equiv \sum_{v | 2 \infty} \lambda(E/\bQ(\sqrt{d})_v) \equiv \begin{cases} 0 & d \equiv 1,2,3 \pmod{8} \\ 1 & d \equiv 5,6,7 \pmod{8} \end{cases} \]
\item[(3)]
Heegner hypothesis: $E/\bQ$, all $p | N_E$ in $\bQ(\sqrt{-D})$ split $(D > 0)$ implies $\rank E/\bQ(\sqrt{-D})$ is odd.
\item[(4)]
All $E/\bQ$ have even rank over $\bQ(i, \sqrt{17})$. 
\item[(5)]
$E: y^2 = x^3 + x^2 - 12x - 67/4$ (1369E1) has even rank over all extensions of $\bQ((-32)^{1/4})$. 
\item[(6)]
All $E/\bQ$ with split multiplicative reduction at 2 have odd rank over $\bQ(\zeta_8)$. 
\item[(7)]
$E: y^2 + y = x^3 + x^2 + x$ has positive rank over $\bQ(m^{1/3})$ for all $m$ while having rank 0 over $\bQ$. 
\item[(8)]
All quadratic twists of $E/\bQ(i)$ by $d \in \bQ(i)^\times$
\[E: y^2 + xy = x^3 - x^2 - 2x - 1 \text{(49A1)} \]
have positive rank.  
\end{enumerate}

\subsection{Established cases of the parity conjecture}
{\ }

Virtually nothing!

\subsection{Parity of analytic rank}
{\ }

Conjecturally, $L(E/K,s)$ is analytic on $\bC$ and satisfies 
\[L(E/K, s) = \pm(\Gamma\text{'s}, \exp\text{'s}) L(E/K, 2-s)\]
The $\pm 1$ is significant, and it is known as $w(E/K)$, the global root number. It has the form
\[w(E/K) = \prod_v w(E/K_v)\]
where $w(E/K_v)$ are local root numbers, each equal to $\pm 1$. 

\[\ord_{s=1} L(E/K,s) \equiv \begin{cases} \text{even} & \text{if } w = 1 \\ \text{odd} & \text{if } w = -1 \end{cases} \equiv \sum_v \lambda_v^{\text{an}} \pmod{2} \]
where $\lambda_v^{\text{an}} = \log_{-1} w(E/K_v).$

\begin{thm} (Rohrlich , Kobayashi , Dokchitser , Dokchitser, Whitehouse ) 
\[\lambda_v = \lambda_v^{\text{an}}\]
(so the parity conjecture is equivalent to $\rank E/K = \text{``} \rank_{\text{an}} E/K \text{''}$)
\end{thm}

\subsection{Parity of Selmer ranks}
{\ }

$p$ is a prime and let $\rank_p E/K = \bZ_p$-corank of $\Sel_p(E/K)^{\vee}$. Note that $\rank E/K$ is equal to $\rank_p E/K$ if $\# \Sha$ is finite. What about $\rank_p E/K$ modulo 2? \\ \\
Quasi-theorem (GZK): $\rank_p E/\bQ = \rank E/\bQ = \rank_{\text{an}} E/\bQ$ for $\rank_{\text{an}} E/\bQ \leq 1$. 

\begin{thm} (Cassels , Fisher , Dokchitser-Dokchitser , \v{C}esnavi\v{c}ius ) If $E/K$ is such that it admits a $p$-isogeny then 
\[\rank_p E/K \equiv \sum_{v} \lambda_v \pmod{2} \]
\end{thm}

\begin{proof} (assuming that $\# \Sha < \infty$) Cassels' work implies that
\[\frac{\Omega_E \operatorname{Reg}_E \# \Sha \prod_v c_v(E)}{\# E_{\text{tors}}^2} = \frac{\Omega_{E'} \operatorname{Reg}_{E'} \# \Sha_{E'} \cdot \prod_v c_v(E')}{\# E_{\text{tors}}^{'2}} \]
which implies (where $\square$ denotes a square number),
\[ p^{\rank E/K} \square = \frac{\operatorname{Reg}_E}{\Reg_{E'}} = \frac{\Omega_{E'}}{\Omega_E} \frac{\# \Sha_{E'}}{\# \Sha_E} \cdot \frac{\prod c_v(E')}{\prod c_v(E)} \cdot \frac{\#E_{\text{tors}}^2}{\# E_{\text{tors}}^{'2}}\]
This further implies 
\[\rank E/K \pmod{2} \equiv \left(\ord_p \frac{\square}{\square}\right) + \sum \lambda_v' \]
(check $\sum \lambda_v = \sum \lambda_v'$). 
\end{proof}

\begin{thm} (Kramer , Tunnell , Dokchitser-Dokchitser ) If $F/K$ is quadratic then
\[\rank_2 E/F = \sum_v \lambda(E/F_v) \pmod{2} (\equiv \rank_{\text{an}} E/F)\]
\end{thm}

\begin{proof} Use isogeny $E \times E^F \rightarrow \Res_{F/K}(E/F)$. 
\end{proof}

\begin{thm} (Dokchitser-Dokchitser , Mazur-Rubin , de La Rochefoucauld) If $\Gal(F/K) \cong D_{2p^n}$ (or $D_{p^n}$, depending on convention) with $p$ odd, then
\[\rank_p E/K + \rank_p E^M/K + \langle \chi, \Sel_{p^\infty} (E/F)^\vee \rangle \equiv \sum_v \mu_v \] 
\[ \equiv \rank_{\text{an}} E/K + \rank_{\text{an}} E^M/K + \ord_{s=1} L(E/M, \chi, s) \pmod{2} \]
\end{thm}

\begin{proof} Uses an isogeny between some combination of Weil restriction of scalars of $E$ from different fields. 
\end{proof}

\begin{thm} (Monsley , Nekov\'{a}r, Kim , Dokchitser-Dokchitser) For $E/\bQ$, 
\[\rank_p E/\bQ \equiv \sum_v \lambda_v \equiv \rank_{\text{an}} E/\bQ \pmod{2}\]
\end{thm}

\begin{proof} Use previous result with $F$ high in the $p$-anticyclotomic tower of $M$. This implies 
\[ \langle \chi, \Sel^\vee \rangle = \ord_{s=1} L(E/M, \chi, 1) \]
Choose $M$ so that
\[\rank E^M/\bQ = \rank_p E^M /\bQ = \rank_{\text{an}} E^M /\bQ \in \{0,1\}\]
This implies that
\[\rank_p E/\bQ \equiv \rank_{\text{an}} E/\bQ \pmod{2} \]
\end{proof}

\renewcommand{\thesubsection}{\arabic{section}.R}
\begingroup
\renewcommand{\addcontentsline}[3]{}% Remove functionality of \addcontentsline
\endgroup
%%%19.
\newpage
\renewcommand{\thesubsection}{\arabic{section}.\arabic{subsection}}

\section{The average size of the $5$-Selmer group of elliptic curves\\ by Arul Shankar}\label{19}

Joint work with M. Bhargava. \\
Let $E/\bQ$ be an elliptic curve. We will discuss the $5$-Selmer group, $\Sel_5(E)$. Recall we have the following exact sequence
\[0 \rightarrow E(\bQ)/5 E(\bQ) \rightarrow \Sel_5(E) \rightarrow \Sha_E[5] \rightarrow 0, \]
where the main point is that 
\[
\# \Sel_5(E) \geq 5^{\rank(E)}.
\]
We may ask, what is $\Avg \# \Sel_5(E)$ when $E/\bQ$'s are ordered by height? (Recall that the height of $E = E_{A,B}$ is defined by 
\[
H(E)= H(A,B):= \max\{4|A|^3, 27B^2\},
\] cf. Section~\ref{1}. Further, $A,B \in \bZ$ and $p^4 | A \Rightarrow p^6 \nmid B$ for all primes $p$.)\\ \\
We consider the co-regular representation $V = 5 \otimes \Lambda^2(5)$. The group $\GL_5 \times \GL_5$ acts on this representation via the action
\[(g_1, g_2)(A,B,C,D,E) = (g_2 A g_2^t, g_2Bg_2^t, \cdots, g_2 E g_2^t)(g_1),\]
and we take the subgroup $G$ of $\GL_5 \times \GL_5$ defined by
\[G: = \{(g_1, g_2) : (\det g_1)(\det g_2)^2 = 1\}/\left\{\begin{pmatrix} \lambda^2 & & \\ & \ddots & \\ & & \lambda^2 \end{pmatrix}, \begin{pmatrix} \lambda^{-1} & & \\ & \ddots & \\ & & \lambda^{-1} \end{pmatrix} \right \}\]

\begin{rmk} \[(A,B,C,D,E) \mapsto (aX + bY + Cz + Ds + Et) \]
get $Q_1, \cdots, Q_5$, 5 $4 \times 4$ Pfaffians. This defines a curve $C$ of genus 1. $V$ is locally soluble if $C(\bQ_\nu) \ne \emptyset$ for all $\nu$, while it is soluble if $C(\bQ) \ne \emptyset$. 
\end{rmk}

\begin{thm} (Buchsbaumm-Eisenbud , Fisher , Bhargava-Ho )
\[\Sel_5(E_{A,B}) \leftrightarrow G(\bQ) \backslash V(\bQ)_{A,B}^{\text{loc\,sol}} \leftrightarrow G(\bQ) \backslash V(\bZ)_{A,B}^{\text{loc\,sol}} \]
\end{thm} 

We can define an analogous height, which by abuse of notation we call $H$, on $V(\bR)$ by $H(v) = \max\{4|A(v)|^3, 27B(v)^2\}$. \\ \\
Question: what is $\# G(\bZ)\backslash V(\bZ)_{H < X}^{\text{ire}}$. We have $v \in V(\bQ)$ is reducible if $\Delta(v) = 0$ or if $V$ corresponds to $1 \in \Sel_5(E)$. \\ \\
Let $\F$ be a fundamental domain on $G(\bZ) \backslash G(\bR)$, $\mathcal{R}$ a fundamental domain for $G(\bR) \backslash V(\bR)$, so that $\F \cdot \R$ is a $5$-fold cover of a fundamental domain $G(\bZ) \backslash V(\bR)$. Hence
\begin{align*} \displaystyle 5 \cdot \# G(\bZ) \backslash V(\bZ)_{H < X}^{\text{irr}} & = \# \{ \F \cdot \R_{H < X} \cap V(\bZ)^{\text{irr}} \} \\
& = \int_{g \in G_0} \# \{\F g \R_{H < X} \cap V(\bZ)^{\text{irr}} \} dg \\
& = \int_{g \in \F} \# \{g G_0 \R_{H < X} \cap V(\bZ)^{\text{irr}} \} dg
\end{align*}
Here $G_0$ is some open bounded set in $G(\bR)$ having volume $1$. We can decompose $\F$ as follows
\[\F \subset N' A' K,\]
Here $N'$ is bounded, $K$ is compact, and
\[A' \rightarrow \left( \begin{pmatrix} t_1^{-4} t_2^{-3} t_3^{-2} t_4^{-1} & & & \\ & t_1 t_2^{-3} t_3^{-2} t_4^{-1} & & \\ &  & \ddots & \\ & & & t_1 t_2^2 t_3^3 t_4^4 \end{pmatrix}, \begin{pmatrix} s_1^{-4} s_2^{-3} s_3^{-2} s_4^{-1} & & & \\ & s_1 s_2^{-3} s_3^{-2} s_4^{-1} & & \\ &  & \ddots & \\ & & & s_1 s_2^2 s_3^3 s_4^4 \end{pmatrix} \right),\] 
where $s_i, t_i \gg C$. This gives us the equality to
\[\Vol(\F \cdot \R_{H < X}) + o(\Vol(\F \cdot \R_{H < X})).\]
This implies that
\[\Avg(\# \Sel_5(E) - 1) = \tau(G) = 5.\]
Hence, by our earlier comment, since $\Avg \#\Sel_5 \leq 6$, it follows that
\[\Avg(5^r) \leq 6.\]
\begin{itemize}
\item
By noticing that $20r - 15 \leq 5^r$ for all $r \geq 1$, it follows that
\[\Avg(r) \leq \frac{21}{20} = 1.05\]
This is achieved when $95\%$ have rank 1 and $5\%$ have rank 2. 
\item
Proportion of curves that have rank 0 or 1 is at least $19/24.$
\[x + 25(1-x) \leq 6 \Rightarrow x \geq \frac{19}{24}.\]
\end{itemize}

Recall that
\[w_p(E) = \begin{cases} 1 & \text{if good reduction at } p \\ -1 & \text{if split multiplicative reduction at } p \\ 1 & \text{if non-split multiplicative reduction at } p \end{cases} .\]
Write $w(E) = -\prod_p w_p(E)$. Define $d(E)$ as follows
\[d(E) = w(E) \cdot w(E_{-1}), \text{ } d(E) = \prod_p d_p(E),\]
where
\[d_p(E) = w_p(E) \cdot w_p(E_{-1}).\]

\begin{lem} If $p > 3$, then $d_p(E) = -1$ if and only if $E$ has multiplicative reduction at $p$ and $p \equiv 3 \pmod{4}$. 
\[d_{1/6}(E) = \prod_{p > 3} d_p(E), \text{ } \Delta_{6'}(E) = \frac{\Delta(E)}{2^{\Delta(E)/2} 3^{\Delta(E)/3}} \]
$d_{1/6}(E) \equiv |\Delta_{6'}(E)| \pmod{4} $ for curves having density at least $96.69\%$. Further, $d_{1/6}(E) \not \equiv |\Delta_{6'}(E)| \pmod{4}$ for curves having density at least $3.25\%$.  
\end{lem} 

\begin{thm} There exists a family $F$ of elliptic curves defined by congruence conditions on the coefficients of $E$ 
\begin{itemize}
\item
$F$ is closed under twists by $-1$, $d(E) = -1$ which implies that $w(E) = 1$ exactly half the time. 
\item
Density of $F$ is at least $55\%$. 
\end{itemize}
\end{thm}

\begin{thm} in a family where $w(E)$ is equidistributed and $\Avg(\#\Sel_5) = 6$, then
\[\Avg(r) \leq 0.75.\]
Further, the density of curves with rank 0 is at least $3/8$.
\end{thm}

\begin{proof} Use Dokchister-Dokchitser  on the $5$-Selmer rank. For even $n$, use $12n+1 \leq 5^n$ and for odd $n$, use $60n-55 \leq 5^n$. Then

\[\frac{12 \Avg_{\text{even}}(r_5) + 1}{2} + \frac{60 \Avg_{\text{odd}}(r_5) - 55}{2} \leq 5 \]
implies
\[\Avg(r_5) \leq 0.75.\]

\end{proof}
\renewcommand{\thesubsection}{\arabic{section}.R}
\begingroup
\renewcommand{\addcontentsline}[3]{}% Remove functionality of \addcontentsline
\endgroup

%%%20.
\newpage
\section{Heuristics for boundedness of ranks of elliptic curves
\\by Bjorn Poonen}\label{20}
\renewcommand{\thesubsection}{\arabic{section}.\arabic{subsection}}

(joint with Jennifer Park, John Voight, Melanie Matchett Wood)
\subsection{Introduction}
{\ }

Question (Poincare 1901): Does there exist $B$ such that for every $E/\bQ$, we have $\rank(E/\bQ) \leq B$? 

\[ 
\begin{array}{|c|c|} 
\hline
\text{History of guesses} & \text{Records}   \\
\hline
1950 \text{ N\'{e}ron: } &  \\ 
\text{probably bounded} &  \\
\hline
& 1954 \text{ N\'{e}ron: }\\
& \text{there exists} E/\bQ \text{ of rank} \geq 1\\
\hline
1966 \text{ Cassels: } & 1967   \\
\text{``Implausible'' to be bounded} &  \\
\hline
& 1967 \text{ Shafarvich and Tate :}  \\
& \text{unbounded over } \bF_q(t) \\
\hline
1982 \text{ Mestre :} & 1982 \text{ Mestre :}  \\
\text{unbounded} & \text{rank } \geq 12 \\
\hline
1986 \text{ Silverman : } &  \\
\text{folklore conjecture; unbounded} &  \\
\hline
2006 \text{ Granville }: & 2006 \text{ Elkies }  \\
\text{bounded; with heuristic} 
& \begin{cases} \exists E/\bQ \text{ of rank } \geq 28 \\ \exists E/\bQ(t) \text{ of rank } \geq 18 \\ \exists \text{ infinitely many } E/\bQ \text{ of rank } \geq 19 \end{cases} \\
\hline
\text{ Almost everyone else: unbounded} &\\
\end{array}
\]

Associated with $E/\bQ$ are the following arithmetic objects: $r = \rank E(\bQ)$, $\Sel_n E$, and $\Sha$ connected by the exact sequence
\begin{equation}\label{20.1}\tag{$\ast\ast$}
0 \rightarrow E(\bQ) \otimes \frac{\bQ_p}{\bZ_p} \rightarrow \Sel_{p^\infty} E \rightarrow \Sha[p^\infty] \rightarrow 0. 
\end{equation}

\begin{cnj} $\Sha$ is finite. 
\end{cnj}

We shall assume this conjecture from now on. \\ \\
Then there exists a non-degenerate alternating pairing $\Sha \times \Sha \rightarrow \bQ/\bZ$ so that $\#\Sha$ is a square. 

\subsection{Distribution of $\Sha$}
{\ }

Given $r$, what is the distribution of $\Sha[p^\infty]$ as $E$ ranges over rank $r$ elliptic curves? There are three conjectural answers.

{\ }
\begin{enumerate}
\item[(1)]
Delaunay (2001, 2007, 2013): For any finite abelian $p$-group $G$ with an alternating pairing, what is the probability
\[ \operatorname{Prob}(\Sha[p^\infty] \cong G) = \frac{\# G^{1-r}}{\# \Aut(G, [\cdot, \cdot])} \cdot \prod_{i \geq r+1} (1 - p^{1 - 2i}).\]
This is in analogy with Cohen-Lenstra for class groups. 
\item[(2)]
Poonen-Rains (2012), Bhargava-Kane-Lenstra-Poonen-Rains (preprint) : conjectural distribution for~\eqref{20.1} led to conjectural structure in the arithmetic of $E$, some of which was subsequently proven. 
\item[(3)]
Bhargava-Kane-Lenstra-Poonen-Rains (again) : For large $n$ with $n \equiv r \pmod{2}$, choose $A \in M_{n\times n}(\bZ_p)$ subject to $A^T = -A$ with $\rank_{\bZ_p} (\ker A) = r$. Take the limit of the distribution of
\[ \operatorname{coker} \left(\bZ_p^n \overset{A}{\rightarrow} \bZ_p^n\right)_{\text{tors}} \]
as $n \rightarrow \infty$. In the Cohen-Lenstra case, this is analogous to a suggestion of Friedman-Washington. 
\end{enumerate}

\begin{thm} (Bhargava-Kane-Lenstra-Poonen-Rains) The above three distributions coincide!
\end{thm}

\subsection{Model for rank}

To model $E/\bQ$ of height $H$: choose large $n$ of random parity, now choose $A \in M_{n \times n}(\bZ)$ subject to $A^T = -A$ and the entries of $A$ are bounded by $X$. Here $n, X$ depend on $H$. Then
{\ }
\begin{itemize}
\item
$(\operatorname{coker} A)_{\text{tors}}$ models $\Sha(E)$
\item
$\rank_\bZ(\ker A)$ models $\rank E(\bQ)$. 

\end{itemize}

\subsection{Consequences of the model}

{\ }
\begin{itemize}
\item
If $n$ is even, $\rank A = n$ with probability approaching 1 as $H \rightarrow \infty$ (so that $X \rightarrow \infty$ as well). Indeed, this is saying that it should be increasingly difficult to land on the Pfaffian hypersurface.  
\item
If $n$ is odd, $\rank A = n-1$ with probability approaching 1 as $H \rightarrow \infty$.
\end{itemize}

This suggests that, at least asymptotically, that $50\%$ of elliptic curves $E/\bQ$ have rank 0 and $50\%$ of elliptic curves have rank $1$. Further, all elliptic curves of rank at least $2$ have to land on some special hypersurface. 

\begin{thm} If $\rho < n$ and $\rho$ is even, then approximately $X^{n \rho/2}$ of the roughly $X^{n(n-1)/2}$ possible $A$'s, have rank $\leq \rho$ (as $X \rightarrow \infty$). 
\end{thm}

Most of this work was done by Eskin and Katznelson  for symmetric matrices. \\ \\
This suggests that for each $r \geq 1$, 
\begin{align*}\operatorname{Prob}(\rank E \geq r) & \overset{\text{model}}{=} \operatorname{Prob}(\rank A \leq n-r) \\ 
& \overset{\text{Thm}}{\sim} \frac{X^{n(n-r)/2}}{X^{n(n-1)/2}} \\
& \sim \frac{1}{(X^{n/2})^{r-1}}
\end{align*}

\subsection{Calibration (Watkins)}
{\ }

Consider $E$ with even rank. Let 
\[\# \Sha_0 : =\begin{cases} \#\Sha & \text{if } \rank E = 0 \\ 0 & \text{otherwise.} \end{cases} \]
Part of BSD: $L(E,1) = \displaystyle \frac{\# \Sha_0 \Omega \prod c_p}{\# E(\bQ)_{\text{tors}}^2}$
Hence
\begin{align*} \sqrt{\# \Sha_0} & \sim O(\Omega^{-1/2}) \\
& \sim O(H^{1/24}) \end{align*}

\[\operatorname{Prob}(\rank E \geq 2) = \operatorname{Prob}(\sqrt{\# \Sha_0} = 0) \sim \frac{1}{H^{1/24}} \]

Hence, we can see that
\[\frac{1}{(X^{n/2})^{r-1}} \sim \frac{1}{H^{(r-1)/24}} \]

\subsection{Conclusion}
{\ }

There are $\sim H^{5/6}$ elliptic curves of height $\leq H$. If $r - 1 > 20$, then 
\[\sum_{E/\bQ} \frac{1}{(\text{height } E)^{(r-1)/24}} \]
converges. So we expect only finitely many elliptic curves of rank $\geq r$. \\ \\
Prediction: $\rank E \leq 21$ with finitely many exceptions. 

\renewcommand{\thesubsection}{\arabic{section}.R}
\begingroup
\renewcommand{\addcontentsline}[3]{}% Remove functionality of \addcontentsline
\endgroup

\end{document}