\documentclass[12pt,amsfont]{amsart}
%\usepackage{amsaddr}
\usepackage{amssymb, fullpage, amsfonts, euscript, times, hyperref}
\usepackage{ulem}
\usepackage{color}\definecolor{Blue}{rgb}{0,0,1}
\usepackage{color}\definecolor{Red}{rgb}{1,0,0}
\usepackage{algorithm}
\usepackage{algpseudocode}
\usepackage{mathtools}
\usepackage{enumerate}
\usepackage{graphicx}
\usepackage{bbm}


\DeclarePairedDelimiter{\ceil}{\lceil}{\rceil}
\DeclarePairedDelimiter{\floor}{\lfloor}{\rfloor}
\usepackage[OT2,T1]{fontenc}

\DeclareSymbolFont{cyrletters}{OT2}{wncyr}{m}{n}
\DeclareMathSymbol{\Sha}{\mathalpha}{cyrletters}{"58}
\vfuzz=2pt
\begin{document}
\thispagestyle{empty}
\parindent=0pt
\parskip=4pt
\renewcommand{\labelenumi}{(\roman{enumi})}
\renewcommand{\theenumi}{(\roman{enumi})}


\newcommand{\legendre}[2]{\genfrac{(}{)}{}{}{#1}{#2}}
\newcommand{\li}{\mathop{\rm li}}
\newcommand{\lcm}{\operatorname{lcm}}
\renewcommand{\pmod}[1]{{\ifmmode\text{\rm\ (mod~$#1$)}\else\discretionary{}{}{\hbox{ }}\rm(mod~$#1$)\fi}}
\newcommand{\bS}{{\mathbb S}}
\newcommand{\bC}{{\mathbb C}}
\newcommand{\bN}{{\mathbb N}}
\newcommand{\bR}{{\mathbb R}}
\newcommand{\bP}{\mathbb{P}}
\newcommand{\bZ}{{\mathbb Z}}
\newcommand{\bF}{{\mathbb F}}
\newcommand{\bQ}{{\mathbb Q}}
\newcommand{\Ha}{{\mathbb H}}
\newcommand{\bA}{\mathbb{A}}
\newcommand{\bG}{\mathbb{G}}
\newcommand{\bB}{\mathbb{B}}
\renewcommand{\mod}[1]{{\ifmmode\text{\rm\ (mod~$#1$)}\else\discretionary{}{}{\hbox{ }}\rm(mod~$#1$)\fi}}
\newcommand{\Tor}{{\operatorname{Tor}}}
\newcommand{\Ann}{\operatorname{Ann}}
\newcommand{\Sel}{\operatorname{Sel}}
\newcommand{\Hom}{\operatorname{Hom}}
\newcommand{\End}{\operatorname{End}}
\newcommand{\GL}{\operatorname{GL}}
\newcommand{\SL}{\operatorname{SL}}
\newcommand{\Avg}{\operatorname{Avg}}
\newcommand{\Sym}{\operatorname{Sym}}
\newcommand{\PGL}{\operatorname{PGL}}
\newcommand{\Jac}{\operatorname{Jac}}
\newcommand{\Aut}{\operatorname{Aut}}
\newcommand{\Cl}{\operatorname{Cl}}
\newcommand{\Div}{\operatorname{Div}}
\newcommand{\fp}{\mathfrak{p}}
\newcommand{\Pic}{\operatorname{Pic}}
\newcommand{\sgn}{\operatorname{sgn}}
\newcommand{\Alex}{\operatorname{Alex}}
\newcommand{\Stab}{\operatorname{Stab}}
\newcommand{\Gal}{\operatorname{Gal}}
\newcommand{\rank}{\operatorname{rank}}
\newcommand{\ord}{\operatorname{ord}}
\newcommand{\chr}{\operatorname{char}}
\newcommand{\Reg}{\operatorname{Reg}}
\newcommand{\Res}{\operatorname{Res}}
\newcommand{\Vol}{\operatorname{Vol}}
\newcommand{\OO}{\mathcal{O}}
\newcommand{\ol}{\overline}
\newcommand{\ul}{\underline}
\newcommand{\F}{\mathcal{F}}
\newcommand{\E}{\mathcal{E}}
\newcommand{\cL}{\mathcal{L}}
\newcommand{\fP}{\mathfrak{P}}
\newcommand{\fB}{\mathfrak{B}}
\newcommand{\A}{\mathcal{A}}
\newcommand{\R}{\mathcal{R}}



\numberwithin{equation}{section}%{subsection} %sets equation numbers <chapter>.<section>.<index>
\newtheorem{theorem}{Theorem}[section]
\newtheorem{lem}[theorem]{Lemma}
\newtheorem{pbm}[theorem]{Problem}
\newtheorem{pro}[theorem]{Proposition}
\newtheorem{cor}[theorem]{Corollary}
\newtheorem{cnj}[theorem]{Conjecture}
\newtheorem{dfn}[theorem]{Definition}
\newtheorem{thm}[theorem]{Theorem}
\newtheorem{rmk}[theorem]{Remark}
\newtheorem{xmp}[theorem]{Example}
\newtheorem{exe}[theorem]{Exercise}
\newenvironment{solution}
               {\let\oldqedsymbol=\qedsymbol
                \renewcommand{\qedsymbol}{$\blacktriangle$}
                \begin{proof}[\bf Solution]} 
               {\end{proof}
                \renewcommand{\qedsymbol}{\oldqedsymbol}}
\numberwithin{theorem}{section} %sets equation numbers <chapter>.<section>.<index>
\title{\bf Lecture notes for\\LMS-CMI Research School on  \\``Bounded Gaps Between Primes''\\ September 22--26, 2014}
\author{Justin Scarfy\\
\href{mailto:scarfy@ugrad.math.ubc.ca}{
{\texttt{\lowercase{scarfy@ugrad.math.ubc.ca}}} }
\\
{\textnormal{\textit{D{\lowercase{epartment of }}M{\lowercase{athematics}}\\ T\lowercase{he} U\lowercase{niversity of} B\lowercase{ritish} C\lowercase{olumbia}\\ R\lowercase{oom} 121, 1984 M\lowercase{athematics} R\lowercase{oad}\\V\lowercase{ancouver}, B\lowercase{ritish} C\lowercase{olumbia}, C\lowercase{anada} V6T 1Z2}}}\\ \\ \and \\ \\Stanley Yao Xiao \\ \href{mailto:stanley.xiao@uwaterloo.ca}{
{\texttt{\lowercase{stanley.xiao@uwaterloo.ca}}} }\\ {\textnormal{\textit{D{\lowercase{epartment of }}P{\lowercase{ure}} M{\lowercase{athematics}}\\
U{\lowercase{niversity of }}W{\lowercase{aterloo}}\\
W{\lowercase{aterloo}}, O{\lowercase{ntario}}, C{\lowercase{anada}}
N2L 3G1}}}}
\begin{abstract}

\end{abstract}
\maketitle
\setcounter{tocdepth}{1}
\tableofcontents

\newpage
%%%1.
\section{Introduction to prime number theory. \\ $\zeta$- and $L$-functions,  the prime number theorem (1/4) \\by Andrew Granville}\label{1}



%%%2
\newpage
\renewcommand{\thesubsection}{\arabic{section}.\arabic{subsection}}
\section{Introduction to prime number theory.\\ $\zeta$- and $L$-functions, the prime number theorem (2/4) \\by Andrew Granville}\label{2}

%%%3
\newpage
\renewcommand{\thesubsection}{\arabic{section}.\arabic{subsection}}
\section{The Bombieri-Vinogradov theorem about distribution of primes in progressions. \\Introduction to sieve theory (1/4)\\by Kannan Soundararajan}\label{3}

%%%4
\newpage
\renewcommand{\thesubsection}{\arabic{section}.\arabic{subsection}}
\section{The Bombieri-Vinogradov theorem about distribution of primes in progressions. \\Introduction to sieve theory (2/4)\\by Kannan Soundararajan}\label{4}

%%%5
\newpage
\renewcommand{\thesubsection}{\arabic{section}.\arabic{subsection}}
\section{Introduction to prime number theory.\\ $\zeta$- and $L$-functions, the prime number theorem (3/4) \\by Andrew Granville}\label{5}

%%%6
\newpage
\renewcommand{\thesubsection}{\arabic{section}.\arabic{subsection}}
\section{Introduction to prime number theory.\\ $\zeta$- and $L$-functions,  the prime number theorem (4/4) \\by Andrew Granville}\label{6}

%%%7
\newpage
\renewcommand{\thesubsection}{\arabic{section}.\arabic{subsection}}
\section{The Bombieri-Vinogradov theorem about distribution of primes in progressions. \\ Introduction to sieve theory (3/4)\\by Kannan Soundararajan}\label{7}

%%%8
\newpage
\renewcommand{\thesubsection}{\arabic{section}.\arabic{subsection}}
\section{The Bombieri-Vinogradov theorem about distribution of primes in progressions. \\ Introduction to sieve theory (4/4)\\by Kannan Soundararajan}\label{8}

%%%9
\newpage
\renewcommand{\thesubsection}{\arabic{section}.\arabic{subsection}}
\section{Public Lecture\\ by Terry Tao}\label{9}


%%%10
\newpage
\renewcommand{\thesubsection}{\arabic{section}.\arabic{subsection}}
\section{Inputs from algebraic geometry (1/4)\\ by Emmanuel Kowalski}\label{10}

%%%11
\newpage
\renewcommand{\thesubsection}{\arabic{section}.\arabic{subsection}}
\section{Tutorial for Granville's Lecture Series\\ by Adam Harper}\label{7}

%%%12
\newpage
\renewcommand{\thesubsection}{\arabic{section}.\arabic{subsection}}
\section{Tutorial for Soundararajan's Lecture Series\\ by James Maynard}\label{12}



%%%13
\newpage
\renewcommand{\thesubsection}{\arabic{section}.\arabic{subsection}}
\section{Inputs from algebraic geometry (2/4)\\ by Emmanuel Kowalski}\label{13}

%%%14
\newpage
\renewcommand{\thesubsection}{\arabic{section}.\arabic{subsection}}
\section{Inputs from algebraic geometry (3/4)\\ by Emmanuel Kowalski}\label{14}

%%%15
\newpage
\renewcommand{\thesubsection}{\arabic{section}.\arabic{subsection}}
\section{The methods of Goldston, Pintz and Y{\i}ld{\i}r{\i}m and Maynard-Tao (1/3)\\by James Maynard}\label{15}

%%%16
\newpage
\renewcommand{\thesubsection}{\arabic{section}.\arabic{subsection}}
\section{Polymath discussion \\led by Ben Green}\label{16}

%%%17
\newpage
\renewcommand{\thesubsection}{\arabic{section}.\arabic{subsection}}
\section{Public Lecture \\by Yitang Zhang}\label{17}

%%%18
\newpage
\renewcommand{\thesubsection}{\arabic{section}.\arabic{subsection}}
\section{Inputs from algebraic geometry (4/4)\\ by Emmanuel Kowalski}\label{18}

%%%19
\newpage
\renewcommand{\thesubsection}{\arabic{section}.\arabic{subsection}}
\section{The methods of Goldston, Pintz and Y{\i}ld{\i}r{\i}m and Maynard-Tao (2/3)\\by James Maynard}\label{19}

%%%20
\newpage
\renewcommand{\thesubsection}{\arabic{section}.\arabic{subsection}}
\section{Poster \\by S\'{a}vio Ribas}\label{20}


%%%21
\newpage
\renewcommand{\thesubsection}{\arabic{section}.\arabic{subsection}}
\section{The methods of Goldston, Pintz and Y{\i}ld{\i}r{\i}m and Maynard-Tao (3/3)\\by James Maynard}\label{21}

%%%22
\newpage
\renewcommand{\thesubsection}{\arabic{section}.\arabic{subsection}}
\section{Tutorial for Maynard's Lecture Series\\ by James Maynard}\label{22}

\end{document}