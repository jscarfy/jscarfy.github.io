\documentclass[12pt,amsfont]{amsart}
%\usepackage{amsaddr}
\usepackage{amssymb, fullpage, amsfonts, euscript, times, hyperref}
\usepackage{ulem}
\usepackage{color}\definecolor{Blue}{rgb}{0,0,1}
\usepackage{color}\definecolor{Red}{rgb}{1,0,0}
\usepackage{algorithm}
\usepackage{algpseudocode}
\usepackage{mathtools}
\usepackage{enumerate}
\usepackage{graphicx}
\usepackage{bbm}


\DeclarePairedDelimiter{\ceil}{\lceil}{\rceil}
\DeclarePairedDelimiter{\floor}{\lfloor}{\rfloor}
\usepackage[OT2,T1]{fontenc}

\DeclareSymbolFont{cyrletters}{OT2}{wncyr}{m}{n}
\DeclareMathSymbol{\Sha}{\mathalpha}{cyrletters}{"58}
\vfuzz=2pt
\begin{document}
\thispagestyle{empty}
\parindent=0pt
\parskip=4pt
\renewcommand{\labelenumi}{(\roman{enumi})}
\renewcommand{\theenumi}{(\roman{enumi})}


\newcommand{\legendre}[2]{\genfrac{(}{)}{}{}{#1}{#2}}
\newcommand{\li}{\mathop{\operatorname li}}
\newcommand{\lcm}{\operatorname{lcm}}
\renewcommand{\mod}[1]{{\ifmmode\text{\operatorname\ (mod~$#1$)}\else\discretionary{}{}{\hbox{ }}\operatorname(mod~$#1$)\fi}}
\newcommand{\SSS}{{\mathbb S}}
\newcommand{\C}{{\mathbb C}}
\newcommand{\N}{{\mathbb N}}
\newcommand{\R}{{\mathbb R}}
\newcommand{\PP}{\mathbb{P}}
\newcommand{\Z}{{\mathbb Z}}
\newcommand{\F}{{\mathbb F}}
\newcommand{\Q}{{\mathbb Q}}
\newcommand{\Ha}{{\mathbb H}}
\newcommand{\A}{\mathbb{A}}
\newcommand{\G}{\mathbb{G}}
\newcommand{\B}{\mathbb{B}}
\renewcommand{\mod}[1]{{\ifmmode\text{\operatorname\ (mod~$#1$)}\else\discretionary{}{}{\hbox{ }}\operatorname(mod~$#1$)\fi}}
\newcommand{\Tor}{{\operatorname{Tor}}}
\newcommand{\Ann}{\operatorname{Ann}}
\newcommand{\Sel}{\operatorname{Sel}}
\newcommand{\Hom}{\operatorname{Hom}}
\newcommand{\End}{\operatorname{End}}
\newcommand{\GL}{\operatorname{GL}}
\newcommand{\SL}{\operatorname{SL}}
\newcommand{\Avg}{\operatorname{Avg}}
\newcommand{\Sym}{\operatorname{Sym}}
\newcommand{\PGL}{\operatorname{PGL}}
\newcommand{\Jac}{\operatorname{Jac}}
\newcommand{\Aut}{\operatorname{Aut}}
\newcommand{\Cl}{\operatorname{Cl}}
\newcommand{\Div}{\operatorname{Div}}
\newcommand{\fp}{\mathfrak{p}}
\newcommand{\Pic}{\operatorname{Pic}}
\newcommand{\Alex}{\operatorname{Alex}}
\newcommand{\Stab}{\operatorname{Stab}}
\newcommand{\rank}{\operatorname{rank}}
\newcommand{\ord}{\operatorname{ord}}
\newcommand{\chr}{\operatorname{char}}
\newcommand{\Reg}{\operatorname{Reg}}
\newcommand{\Res}{\operatorname{Res}}
\newcommand{\Vol}{\operatorname{Vol}}
\newcommand{\OO}{\mathcal{O}}
\newcommand{\ol}{\overline}
\newcommand{\ul}{\underline}
\newcommand{\cF}{\mathcal{F}}
\newcommand{\cE}{\mathcal{E}}
\newcommand{\cL}{\mathcal{L}}
\newcommand{\fP}{\mathfrak{P}}
\newcommand{\fB}{\mathfrak{B}}
\newcommand{\cA}{\mathcal{A}}
\newcommand{\cR}{\mathcal{R}}
\renewcommand{\sf}{{\text {\rm sf}}}
\newcommand{\textmod}{{ {\operatorname mod}}}
\newcommand{\Det}{{ {\operatorname Det}}}
\newcommand{\textMod}{{ {\operatorname Mod}}}
\newcommand{\sgn}{\operatorname{sgn}}
\newcommand{\calP}{\mathcal{P}}
\newcommand{\PSL}{{ {\operatorname PSL}}}
\newcommand{\Gal}{{ {\operatorname Gal}}}
\newcommand{\add}{{ {\operatorname add}}}
\newcommand{\sub}{{ {\operatorname sub}}}
\newcommand{\Frob}{{ {\operatorname Frob}}}
\newcommand{\Disc}{{ {\operatorname Disc}}}
\newcommand{\SO}{{ {\operatorname SO}}}
\newcommand{\Tr}{{ {\operatorname Tr}}}
\newcommand{\calF}{\mathcal{F}}
\newcommand{\mfH}{\mathfrak{H}}
\newcommand{\vt}{\vartheta}
\newcommand{\hatphi}{\widehat{\phi}}


\numberwithin{equation}{section}%{subsection} %sets equation numbers <chapter>.<section>.<index>
\newtheorem{theorem}{Theorem}[section]
\newtheorem{lem}[theorem]{Lemma}
\newtheorem{pbm}[theorem]{Problem}
\newtheorem{pro}[theorem]{Proposition}
\newtheorem{cor}[theorem]{Corollary}
\newtheorem{cnj}[theorem]{Conjecture}
\newtheorem{dfn}[theorem]{Definition}
\newtheorem{thm}[theorem]{Theorem}
\newtheorem{rmk}[theorem]{Remark}
\newtheorem{xmp}[theorem]{Example}
\newtheorem{exe}[theorem]{Exercise}
\newenvironment{solution}
               {\let\oldqedsymbol=\qedsymbol
                \renewcommand{\qedsymbol}{$\blacktriangle$}
                \begin{proof}[\bf Solution]} 
               {\end{proof}
                \renewcommand{\qedsymbol}{\oldqedsymbol}}
                

\numberwithin{theorem}{section} %sets equation numbers <chapter>.<section>.<index>
\title{\bf Lecture notes for\\SMS Summer School\\``Counting arithmetic objects''\\ June 23--July 04, 2014}
\author{Stanley Yao Xiao \\ \href{mailto:stanley.xiao@uwaterloo.ca}{
{\texttt{\lowercase{stanley.xiao@uwaterloo.ca}}} }\\ {\textnormal{\textit{D{\lowercase{epartment of }}P{\lowercase{ure}} M{\lowercase{athematics}}\\
U{\lowercase{niversity of }}W{\lowercase{aterloo}}\\
W{\lowercase{aterloo}}, O{\lowercase{ntario}}, C{\lowercase{anada}}
N2L 3G1\\ }\\}\and \\ \\ Justin Scarfy\\
\href{mailto:scarfy@ugrad.math.ubc.ca}{
{\texttt{\lowercase{scarfy@ugrad.math.ubc.ca}}} }}
\\
{\textnormal{\textit{D{\lowercase{epartment of }}M{\lowercase{athematics}}\\ T\lowercase{he} U\lowercase{niversity of} B\lowercase{ritish} C\lowercase{olumbia}\\ R\lowercase{oom} 121, 1984 M\lowercase{athematics} R\lowercase{oad}\\V\lowercase{ancouver}, B\lowercase{ritish} C\lowercase{olumbia}, C\lowercase{anada} V6T 1Z2}}}}
\begin{abstract}
This lecture notes contains all the SMS summer school material on ``Counting arithmetic objects'' held at the CRM in Montr\'eal between June 23 and July 04, 2014.  We would like to thank the organizers for putting together this summer school and inviting the experts to speak, and for the CRM for its hospitality.
\end{abstract}

\maketitle

\setcounter{tocdepth}{1}
\tableofcontents

%%%1.
\newpage
\section{Introduction and Perspectives\\by Manjul Bhargava}\label{1}

\begingroup
\renewcommand{\addcontentsline}[3]{}% Remove functionality of \addcontentsline
\endgroup


%%%2
\newpage
\renewcommand{\thesubsection}{\arabic{section}.\arabic{subsection}}
\section{Algebraic groups, representation theory, and invariant theory
\\ by Eyal Goren}\label{2}


%%%3
\newpage
\renewcommand{\thesubsection}{\arabic{section}.\arabic{subsection}}
\section{Basic algebraic number theory (number fields, class groups) \\by Eknath Ghate}\label{3}

%%%4
\newpage
\renewcommand{\thesubsection}{\arabic{section}.\arabic{subsection}}
\section{Basics of binary quadratic forms and Gauss composition
\\ by Andrew Granville}\label{4}


%%%5
\newpage
\section{Curves, geometric aspects
\\ by Henri Darmon}\label{5}

\begingroup
\renewcommand{\addcontentsline}[3]{}% Remove functionality of \addcontentsline
\endgroup


%%%6
\newpage
\renewcommand{\thesubsection}{\arabic{section}.\arabic{subsection}}
\section{Basic analytic number theory\\ by Andrew Granville}\label{6}


%%%7
\newpage
\renewcommand{\thesubsection}{\arabic{section}.\arabic{subsection}}
\section{Curves, diophantines aspects\\ by Henri Darmon
}\label{7}

%%%8
\newpage
\renewcommand{\thesubsection}{\arabic{section}.\arabic{subsection}}
\section{More algebraic groups, representation theory and invariant theory\\ by  Eyal Goren}\label{8}


%%%9
\newpage
\renewcommand{\thesubsection}{\arabic{section}.\arabic{subsection}}
\section{Cubic rings\\ by  Melanie Matchett-Wood}\label{9}

%%%10
\newpage
\renewcommand{\thesubsection}{\arabic{section}.\arabic{subsection}}
\section{Quartic and quintic rings\\ by  Melanie Matchett-Wood}\label{10}

%%%11.
\newpage
\section{Problem Set (1/7)}\label{11}
\subsection{{\bf Directly from Wood's cubic rings lecture.}}
  \begin{enumerate}
  \item
  Prove that the inverse maps $R \rightarrow \Disc(R)$ and $D \rightarrow \Z[\tau] / (\tau^2 - D\tau + \frac{D^2 - D}{4})$ induce a bijection
  between the set of quadratic rings (up to isomorphism) and
%  \[
%  \{ D \in \Z : D \equiv 0, 1 \mod{4}\}.
%  \]
  \item
  In the Delone Faddeev equations
  \[
  \omega \theta = n,  \omega^2 = m - b \omega + a \theta,  \theta^2 = \ell - d \omega + c \theta,
  \]
  prove that associativity is equivalent to the equations
  \[
  n = -ad,  \ell = -bd,  m = - ac.
  \]
  \item 
  Wood mentioned that if you write $+b$ and $+d$ in place of $-b$ and $-d$ above, the correspondence comes out slightly wrong. Try it and see what happens.
  \item
  Orders in cubic number fields correspond to irreducible cubic forms $f(x, y)$, and the number field can be recovered as $\Q[\theta] / (f (\theta, 1)).$
  What happens if you substitute $f( 1, \theta)$ for $f(\theta, 1)$? (What {\itshape must} happen?)
  \item
  For a cubic form $f$, prove that the functions on its vanishing set $V_f$ determine a cubic ring, which is the same ring obtained by the Delone-Faddeev
  correspondence. (Describe any special conditions, e.g., $f \neq 0$, which are necessary in your proof.)
  \end{enumerate} 
  
  
\subsection{{\bf Other Exercises Concerning Cubic Rings.}}
  We give more exercises for cubic than for quartic or quintic rings. Note that most or all of these exercises are interesting for all three parameterizations
  being discussed. You are {\itshape strongly encouraged} to extrapolate problems from one section to another! What is the same, and what is
  different?
\begin{enumerate}

\item A good way to get started is to compute lots of examples of the Delone-Faddeev correspondence.
(If you don't do any of the other exercises, you should probably do at least this, and the quartic and quintic analogues!)
What binary cubic form $f$ corresponds
           to the cubic ring $\Z^3$?  To $\Z[\sqrt[3]{n}]$?
           Conversely, what cubic ring corresponds to the cubic form $u^3 - uv^2 + v^3$? To $u (u - v) (u + v)$?
           To $u^3$? To $0$? Work out these, as well as other examples of your own invention, and compute all of their
           discriminants.
\item Another good way to get started is to work out the details of the Delone-Faddeev and Davenport-Heilbronn
correspondences. The exposition given at \url{http://arxiv.org/pdf/1005.0672.pdf} on pp. 4-7 leaves many small details
to be checked by the reader. Pick your favorite lemma or proposition and work out the proof in more detail than given in the paper.
\item The Delone-Faddeev correspondence is very interesting over $\mathbb{F}_p$. Assuming for simplicity
that $p \neq 2, 3$, determine all of the cubic rings over $\mathbb{F}_p$ as well as the $\GL_2(\F_p)$-equivalence
classes of cubic forms over $\mathbb{F}_p$. How many equivalence classes are there? On the cubic forms side, how large
is each $\GL_2(\F_p)$-equivalence class, and how big is each of the corresponding stabilizer groups? If you reduce an integral binary
cubic form modulo $p$, what is the relationship between the cubic ring over $\Z$ and the cubic ring over $\F_p$?
\item Work out what the Delone-Faddeev correspondence says over $\mathbb{C}$:

The fact that $\GL_2(\C)$ acts prehomogeneously on binary cubic forms over $\C$ can be restated by saying that 
all nonsingular
binary cubic forms form a single $\GL_2(\C)$-orbit, and therefore (by Delone-Faddeev) that there is exactly one nondegenerate 
cubic ring over $\C$
up to isomorphism. Work out this case of the Delone-Faddeev correspondence, describe this cubic ring, and prove its uniqueness directly.

\item Classify the set of those $\GL_2(\Z_p)$-orbits on $V(\Z_p)$
  whose discriminants are not divisible by $p$. What about those whose
  discriminants are exactly divisible by $p$ or $p^2$? Which of these
  extensions are maximal?

\item Let $f$ be an element of $V(\Z_p)$ and let $\bar{f}$ denote its
  reduction modulo $p$. Let $R_f$ and $R_{\bar{f}}$ denote the
  corresponding cubic extensions of $\Z_p$ and $\F_p$,
  respectively. Describe $R_{\bar{f}}$ in terms of $R_f$.

\item Let $f$ be an integral cubic form, and let $R$ denote the
  corresponding cubic ring over $\Z$. Show that the cubic extension of
  $\Z_p$ corresponding to the $\GL_2(\Z_p)$-orbit of $f$ is $R\otimes\Z_p$.

\item The {\it content} of a ring $R$ of a
ring of rank $n$ is the largest integer $c$ such that $R=\Z+cR'$ for some
ring $R'$ of rank $n$.  The {\it content} of an integral binary cubic form
$f$ is the gcd of the coefficients of $f$.  Show that the content of an
integral binary cubic form $f$ is equal to the content of the
corresponding cubic ring $R$.

         \item Consider the form $\operatorname{Tr}(x^2)$ on the cubic
           ring $R=R(f)$.  Now restrict this form to the sublattice of $R$
            consisting of elements of trace 0.  What is the
            interpretation of this quadratic form in terms of the corresponding binary cubic $f$?
         \item Write down some examples of cubic rings inside Galois
           cubic fields.  Do they all have three automorphisms?  What
           are the associated binary cubics? What can you say about the
           $\operatorname{Tr}(x^2)$ form for a cubic ring
           having three automorphisms?  Can you use this to give an
           explicit parametrization of such ``$C_3$-cubic rings''?
         \item Show that the cubic ring given by a binary cubic form
           lies in the field generated by the coordinates of the points
           cut out in $\PP^1$ by the form.  What if the field is
           quadratic?
         \item What can you say about the integers
           represented by a binary cubic form $f$, making use of the
           relationship with the corresponding cubic ring $R(f)$?
           
\item For all of these parameterizations, it is possible (and not terribly difficult) to write down the discriminant as an explicit
$n \times n$ determinant in $n$ variables, for $n = 4, 12, 40$ respectively. 

We will outline the cubic case here (where the discriminant polynomial is otherwise easy to write down, so that it is easy 
to check your work!) First, use the $NAK\Lambda$ decomposition
of $\GL_2(\C)$ to write down a basis for the Lie algebra $\mathfrak{gl}_2(\C)$, i.e., the tangent space of the identity of
$\GL_2(\C)$. For $g \in \mathfrak{gl}_2(\C)$ and a general $v \in V_{\C}$, compute the infinitesimal action
\[
\lim_{h \rightarrow \infty} \frac{ (I + hg) \circ v - v }{h},
\]
which will be a tangent vector in $V_{\C}$ (and a function of the four coordinates $v_1$, $v_2$, $v_3$, and $v_4$). 
The matrix of all of these, as $g$ ranges over your basis for $\mathfrak{gl}_2(\C)$,
will be singular if and only if there is a local homeomorphism between a neighborhood of $I \in \GL_2(\C)$ and a neighborhood
of $v \in V_{\C}$, which will happen if and only if $\Disc(v) \neq 0$. Compute this matrix, and observe that its determinant
is a degree $4$ polynomial in the coordinates $v_1$, $v_2$, $v_3$, and $v_4$. Since the ring of invariants of $V_{\C}$ is generated
by the discriminant, it follows that the determinant of your matrix is equal to $\Disc(v)$ times a scalar. By comparing with some
$v \in V_{\C}$ for which you otherwise compute its discriminant, and probably using PARI/GP, Sage, or some other such software,
determine this scalar and therefore the discriminant polynomial on $V_{\C}$.

\end{enumerate}

\subsection{ {\bf Quartic Rings}}
\begin{enumerate}
         \item Warm up: What pair of ternary quadratic forms
           corresponds to $\Z^4$?
To $\Z[\sqrt[4]{n}]$? To $\Z[\sqrt{a},\sqrt{b}]$? Or your favorite quartic 
ring?
\item Find pairs of ternary quadratic forms corresponding to quartic
   rings that have some special kind of structure -- for example, those
   lying inside $K \oplus \Q$ where $K$ is a cubic field.  How does
   this relate to cubic rings and binary cubic forms? What about other
   types of special structure, such as the various possible Galois
   groups?  Can you find nice representatives for the pairs of ternary
   quadratic forms corresponding to these?  Can you find
   a parametrization space for quartic rings with one of these
   structures?
\item Show that the quartic ring given by a pair of ternary quadratic
   forms lies in the field generated by the coordinates of the points
   cut out in $\PP^2$ by these forms.  In particular,
   what happens in some of the special cases considered in part (b)?
\item What can you say about the pairs of integers represented by a
   pair of ternary quadratic forms in terms of the corresponding
   quartic ring?
   
\item Oh, here's a question I like: Show that every maximal quartic ring is
represented exactly once as a pair of integral ternary quadratic forms (up
to equivalence).

\item As a generalization of the above, show that the number of cubic resolvent rings of a
quartic ring $Q$ is the number of index $c$ sublattices of $\Z^2$, where
$c$ is the content of $Q$.  [It follows, in particular, that every maximal
quartic ring is represented exactly once as a pair of integral ternary
quadratic forms (up to equivalence).]

\item For quartic (and, later, quintic) rings: similarly describe the content of a quartic ring $Q$
in terms of the corresponding pair of integral ternary quadratic forms.

\item Classify the set of maximal quartic extensions of $\Z_p$.

\item Given a maximal quartic extension of $\Z_p$, determine its cubic resolvent ring.


\end{enumerate}

\subsection{{\bf Quartic rings and their cubic resolvents.}}  A nondegenerate
  binary cubic form $f$ over a field $k$ determines three points in
  $\PP^1_{\bar{k}}$, namely the three roots of $f$. This {\it set of
    three points} is defined over $k$, and thus consists of a union of
  Galois-orbits. All three points might be defined over $k$, or the
  three points could be a union of a Galois-orbits containing two
  points and a third point defined over $k$, or the three points could
  consist of a single Galois-orbit.  (Here, $\bar{k}$ denotes the
  algebraic closure of $k$.)


Similarly, a pair $(A,B)$ of ternary quadratic forms yields four
points in $\PP^2_{\bar{k}}$ given by the intersection of the two
quadrics cut out by $A$ and $B$. This {\it set of four points} is
defined over $k$, and thus consists of a union of Galois-orbits.
Given these four points, say $P_1,P_2,P_3,P_4$, we may consider the
following set of pairs of lines:
$\{(P_1P_2,P_3P_4),(P_1P_3,P_2P_4),(P_1P_4,P_2P_3)\}$.

Recall that given a pair $(A,B)$ of integral ternary quadratic forms
(represented a symmetric $3\times 3$ matrices), the cubic resolvent
form of $(A,B)$ is defined to be $4\Det(Ax-By)$.

\begin{enumerate}
\item Let $f$ be a nondegenerate cubic form over $\Q_p$. Prove that
  the field of definitions of the corresponding three points
  determines the degree three extension of $\Q_p$ corresponding to
  $f$, and therefore its splitting type.
\item Prove the analogous problem with $f$ replaced with a pair of
  ternary quadratic forms.
\item Given a pair of ternary quadratic forms over $\Q_p$, prove that
  the corresponding set of pairs of lines is also defined over $\Q_p$.
\item Prove that the Galois action on this set of pairs of lines is
  the same as the Galois action on the three points determined by the
  cubic resolvent.
\item Use the above problems to determine the cubic resolvent of
  nondegenerate quartic extensions of $\Q_p$.
\item A nondegenerate quartic extension of $\Q$ could be a field (what
  are the possible Galois groups of the normal closure of these
  fields?), or a direct product of fields. In all these cases,
  determine the cubic resolvent of the quartic extension.
\item Give a convenient way of distinguishing a $D_4$-quartic field
  from an $S_4$-quartic field.

\end{enumerate}


\subsection{ {\bf Quintic Rings}}
\begin{enumerate}
         \item Give examples of forms corresponding to some examples of 
quintic rings.
         \item Can you find a parametrization space for quintic rings
           with some special structure (as in 2(b))?
\item Let $(A,B,C,D,E)\in V(\Q)$ correspond to the quintic extension
  $K$. Prove that if $A$ has rank $\leq 2$, then the five associated
  quadrics have a common zero defined over $\Q$ and therefore $K$ is
  not a field.
\item Let $(A,B,C,D,E)\in V(\Q)$ correspond to the quintic extension
  $K$, and let $Q_1,\ldots,Q_5$ be the five associated quadrics. Prove
  that if $Q_1$ factors over $\Q$, then $K$ is not a field.

\end{enumerate}

\subsection{ {\bf Noncommutative Rings}}
\begin{enumerate}
         \item Consider some of the analogous questions for quaternion rings!
         \item By taking the $\operatorname{Tr}(x^2)$ form on a quartic
           ring (restricted again to the trace 0 part of the ring), one
           obtains a ternary quadratic form, which corresponds to a
           quaternion ring!  What is the relation between this
           quaternion ring and the original quartic ring?
\end{enumerate}


\begingroup
\renewcommand{\addcontentsline}[3]{}% Remove functionality of \addcontentsline
\endgroup


%%%12
\newpage
\renewcommand{\thesubsection}{\arabic{section}.\arabic{subsection}}
\section{How to count rings and fields (1/2)\\ by Manjul Bhargava}\label{12}


%%%13
\newpage
\renewcommand{\thesubsection}{\arabic{section}.\arabic{subsection}}
\section{Rings associated to binary $n$-ic forms, composition of $2 \times n \times n$ boxes and class groups\\ by Melanie Matchett-Wood}\label{3}

%%%14
\newpage
\renewcommand{\thesubsection}{\arabic{section}.\arabic{subsection}}
\section{The zeta functions attached to prehomogeneous vector spaces
\\ by Takashi Taniguchi}\label{14}

%%%15
\newpage
\renewcommand{\thesubsection}{\arabic{section}.\arabic{subsection}}
\section{Problem Set (2/7)}\label{15}
\subsection{
\textbf{An easy version of Davenport's lemma and some generalizations}}


For a bounded open set $B\subset \R^n$, let $MP(B)$ denote the
greatest $d$-dimensional volume of any projection of $B$ onto a
coordinate subspace obtained by equating $n-d$ coordinates to zero,
where $d$ takes all values from $1$ to $n-1$.

\begin{enumerate}
\item
{\bf (Davenport's lemma, easy version:)}
Let $B\subset\R^n$ be a fixed open bounded set. 
Assume that $B$ is defined by finitely many polynomial inequalities.
Prove that we have
\begin{equation}\label{eqdav}
\#\{g\cdot B \cap \Z^n\}=\Vol(g\cdot B)+O(MP(g\cdot B)),
\end{equation}
where $g\in\GL_n(\R)$ is any diagonal matrix with positive entries,
and the volume of sets in $\R^n$ is normalized so that $\Z^n$ has
covolume $1$.

Prove the same estimate for $g=nt\in\GL_n(\R)$, where $n$ is a lower
triangular matrix, and $t$ is a diagonal matrix with increasing
positive diagonal entries. Hint: use the fact that only smaller
coordinates are being added to larger coordinates.

\item Modify the necessary arguments to obtain an estimate
analogous to \eqref{eqdav} when $\Z^n$ is replaced with an arbitrary
lattice. In particular, when $L$ is a lattice defined by congruence
conditions modulo finitely many prime powers
$p_1^{k_1},\ldots,p_m^{k_m}$, prove that we have
\begin{equation}\label{eqdavcong}
\#\{g\cdot B \cap \Z^n\}=\Vol(g\cdot B)\prod_{i=1}^m\Vol(L_p)+O(MP(g\cdot B)),
\end{equation}
where $L_p$ is the $p$-adic closure of $L$ in $\Z_p^n$ and the measure
on $\Z_p^n$ is normalized so that $Z_p^n$ has volume $1$.

\item Modify the necessary definitions and arguments to obtain
estimates analogous to \eqref{eqdav} and \eqref{eqdavcong} when $B$ is
an open bounded multiset.

\end{enumerate}


\subsection{
 {\bf Counting number fields using the geometry of numbers.}}

\begin{enumerate}
\item
Granville explained how to count $\sum_{n < X} d(n)$.

Defining $d_k(n)$ to be the number of ways to write $n$ as a product of $k$ positive integers, prove asymptotics
for $\sum_{n < X} d_k(n)$. (Bonus: prove them with lower order terms and power saving error terms.)

\item
In Granville's solution of the circle problem, suppose that one is interested in counting only those pairs $(x, y)$ with
$x^2 + y^2 \leq X$ and $x^2 + y^2 \equiv a \pmod{q}$, for some $a$ and $q$. Obtain an asymptotic formula in this
case, with an error term depending explicitly on $a$, $q$, and $X$. Does the added condition increase or decrease the error term?

Suppose instead that you require $x \equiv a \pmod q$. How does this change things? (Try out some other conditions instead.)

\item
Compute an asymptotic formula for
\[
\sum_{0 < -D < X} h(-D).
\]
You will use Gauss's reduction theory of binary quadratic forms. Can you incorporate 
Bhargava's averaging method?

\item
Without doing any involved computations, and presuming that asymptotic formulas can indeed be proved, explain why the following formulas
are correct. ($C$ stands for a different constant at each appearance.)
\begin{enumerate}
\item We have
$\sum_{0 < -D < X} h(-D) \sim C X^{3/2}$.
\item The number of cubic, quartic, and quintic rings (together with cubic or sextic resolvents in the latter two cases) $R$ with
$0 < |\Disc(R)| < X$ is $\sim C X$ in each case (with different constants).
\item Give a rough heuristic argument for why the proportion of such rings nonmaximal at $p$ is roughly $\frac{1}{p^2}$, and accordingly
give a rough argument for why then the number of cubic, quartic, and quintic {\itshape fields} $K$ with $0 < |\Disc(K)| < X$ is 
$\sim CX$.
\end{enumerate}

\item
Asymptotically there are $3$ times as many cubic fields $K$ with $|\Disc(K)| < X$ with mixed signature than which are totally
real. Why $3$, and not $1$ or $\frac{\pi^6}{945}$? Trace the discrepancy in Bhargava, Shankar, and Tsimerman's paper, and explain where it comes from.

\end{enumerate}

\subsection{
\textbf{Index-three subgroups in the class groups of quadratic fields}}
Let $K_2$ be a fixed quadratic field. Recall that Class Field Theory
implies that the maximal unramified extension of $K_2$ is Galois over
$K_2$ with Galois group isomorphic to the class group of $K_2$.  Thus,
index-$3$ subgroups of the class group of $K_2$ are in bijection with
degree $3$-unramified extensions of $K_2$.
\begin{enumerate}
\item Let $K_{2p}$ be a degree-$p$ unramified extension of $K_2$. By
  studying the action of $\sigma$, the non trivial automorphism of
  $K_2$, prove that $K_{2p}$ is Galois over $\Q$.
\item Prove that the Galois group of $K_{2p}$ is not cyclic.
\item We now restrict to the case when $p=3$. We know that $K_6$ is a
  Galois $S_3$ field. Let $K_3$ denote one of its cubic subfields. By
  reading this wonderful one page write up of Wood's:
  \url{http://www.math.wisc.edu/~mmwood/Splitting.pdf}, determine how primes
  split in $K_3$ in terms of how they split in $K_2$.
\item From the above problem, classify the possible cubic fields that can arise as $K_3$'s. We call these {\it nowhere totally ramified} cubic fields.
\item Let $K_3$ be a nowhere totally ramified cubic $S_3$ field, let
  $K_6$ be its Galois closure, and let $K_2$ be the quadratic subfield
  of $K_6$. Prove that $K_6$ is unramified over $K_2$.
\item Prove that the discriminant of $K_3$ is the same as the
  discriminant of $K_2$.
\item Use the above problems, and the results counting cubic fields to
  compute the average size of the $3$-torsion in the class groups of
  quadratic fields.
\end{enumerate}

\subsection{
\textbf{Binary $n$-ic forms and the ring associated to them}}
\begin{enumerate}
\item Repeat Problem 1e of the previous problem set for rings associated to binary $n$-ic forms.
\item Given a integral binary $n$-ic form $f$, construct an integral
  $2\times n\times n$ box whose resolvent is $f$.
\item Let $n$ be odd. Given a integral binary $n$-ic form $f$,
  construct a $\Z^2\times \Sym^2(\Z^n)$ box whose resolvent is $f$.
\item Why does the above solution fail when $n$ is even? Can you
  construct a binary quartic form $f$ such that no $\Z^2\times
  \Sym^2(\Z^4)$ box has $f$ as a resolvent?
\end{enumerate}

\subsection{
\textbf{Poisson Summation}}

Another tool that can be used to count lattice points is Poisson
Summation.  Let $f:\R^n\to\C$ be an $L^2$ function. Let
$\hat{f}:\R^n\to\C$ denote the Fourier transform of $f$. Then we have
the Poisson summation formula:
\begin{equation}\label{ps}
\sum_{v\in\Z^n}f(v)=\sum_{v\in\Z^n}\hat{f}(v).
\end{equation}
\begin{enumerate}
\item
\textbf{Counting smoothly} Let $\chi_B$ denote the characteristic
function of $B\subset\R^n$. We can write 
\begin{equation}\label{eqdavnew}
\#\{g\cdot B \cap \Z^n\}=\sum_{v\in\Z^n} g\cdot\chi_B(v),
\end{equation}
where $\GL_n(\R)$ acts on the space of functions $f:\R^n\to\C$ via
$g\cdot f(v):=f(g^{-1}v)$. (Check that this is a right action of
$\GL_n(\R)$ and that under this action $g\cdot\chi_B=\chi_{g\cdot B}$.)

To count ``smoothly'', we replace the characteristic function of $B$
in~\eqref{eqdavnew} by a smooth function~$f$ with compact support.

\item Prove smooth versions of the easy version of Davenport's Lemma and its
generalizations by using the Poisson summation formula. Note that in
many cases the error term is substantially improved. For which
$g\in\GL_n(\R)$ does the error term stay the same?
\end{enumerate}

\subsection{
{ {\bf Zeta functions associated to prehomogeneous vector spaces}}}

\begin{enumerate}
\item ({\it The Riemann zeta function}) The Riemann zeta function
\[
\zeta(s) = \sum_{n = 1}^{\infty} n^{-s}
\]
is the simplest example of a zeta function associated to a prehomogeneous vector space. We will review
the proof of its functional equation.
\begin{enumerate}
\item
Prove that
\[
Z(s) := \pi^{-s/2} \Gamma(s/2) \zeta(s) = \frac{1}{2}
\int_0^{\infty} \bigg( \sum_{n \in \Z - \{ 0 \} } e^{- \pi n^2 y} \bigg) y^{s/2} \frac{dy}{y}.
\]
State explicitly any conditions on $s$ which you assume in the course of your proof.

(Recall the definition of the {\itshape gamma function}: it is the Mellin transform of the function $e^{-t}$, namely
\[
\Gamma(s) = \int_0^{\infty} e^{-t} t^s \frac{dt}{t}.
\]
\item
Use the {\itshape Poisson summation formula} to prove that
\[
\sum_{n \in \Z} e^{- \pi n^2 y} = 
y^{1/2} \cdot \sum_{n \in \Z} e^{- \pi n^2/y}.
\]
This {\itshape modularity relation} illustrates the importance of the function $e^{- \pi n^2 y}$,
and indirectly explains why the gamma function is needed to `complete' the Riemann zeta
function. (For the best explanation of this phenomenon, read {\itshape Tate's thesis}.
\item
By splitting the integral into intervals $(0, 1)$ and $(1, \infty)$, and using the Poisson sum formula on the
former, prove that
\[
Z(s) = 
\frac{1}{2} \int_0^{\infty} \bigg( \sum_{n \in \Z - \{ 0 \} } e^{- \pi n^2 y} \bigg) y^{s/2} \frac{dy}{y}
- \int_0^1 y^{s/2} \frac{dy}{y}
+
\frac{1}{2} \sqrt{y} \int_0^{\infty} \bigg( \sum_{n \in \Z } e^{- \pi n^2 y} \bigg) y^{1/2 - s/2} \frac{dy}{y}.
\]
\item
Explain why the zeta function would have been much nicer (and, in particular, entire) if
it only included the term $\frac{1}{0^s}$. (The question is quite vague, but pondering it illustrates
some of the technical difficulties inherent in Shintani's work.)
\end{enumerate}
\item
Let $f : V_{\R} \rightarrow R$ be a nice test function defined on the space of binary cubic forms,
and recall that the completed Shintani zeta function is defined by the formula
\[
Z(f, s) := \int_{\GL_2^+(\R) / \SL_2(\Z)} (\det(g))^{2s} \sum_{x \in V_{\Z} - S} f(gx) dg,
\]
where $S$ is the set of points $x \in V_{\Z}$ with $|\Disc(x)| = 0$.
\begin{enumerate}
\item
Verify that
$\Disc(gx) = (\det(g))^2 \Disc(x)$.
\item
Suppose that $f$ is supported in $V^+_{\R}$, i.e., on binary cubic forms with positive discriminant.
Carrying out a Jacobian change of variables formula, prove (Shintani, p. 153, Prop. 2.4) that
\[
\int_{\GL_2^+(\R)} f(gy) dg = \frac{1}{4 \pi} \int_{V^+(\R)} |\Disc(x)|^{-1} f(x) dx.
\]
If $f$ is instead supported in $V^-(\R)$, prove the analogous formula , only with $\frac{1}{4 \pi}$ replaced
with $\frac{1}{12 \pi}$.
\item
Using the above, prove the formula
\[
Z(f, s) = \frac{1}{4 \pi} \xi^+(s) \int_{V^+} |\Disc(x)|^{s - 1} f(x) dx + 
\frac{1}{12 \pi} \xi^+(s) \int_{V^+} |\Disc(x)|^{s - 1} f(x) dx.
\]
\item
Redo all of the above, only for the simplest prehomogeneous vector space:
replace $V_{\R}$ with $\R$, $V_{\Z}$ with $\Z$, and $\Disc(x)$ with $x$. What formula do you obtain?
If you choose the test function $f(x) = e^{-x^2}$, you should recognize something especially familiar.
What?

\end{enumerate}
\end{enumerate}

\subsection{\textbf{Counting squarefree integers less than $X$}}

{\itshape This exercise will be highly relevant to the Friday and Saturday lectures. You might
wait until after these lectures to do these problems, or attempt this to get a preview.}

Let $[X]$ denote the set of positive integers $n\leq X$, and let
$[X]^\sf$ denote the subset of integers $n\in[X]$ that are
squarefree. Our aim is to prove that
$\displaystyle\lim_{X\to\infty}\frac{\#[X]^\sf}{\#[X]}=\frac{1}{\zeta(2)}$.
\begin{enumerate}
\item Let $Y$ be a fixed positive integer, and let $[X]^{\sf,Y}$ denote the
set of elements in $[X]$ that are not divisible by $p^2$ for primes
$p\leq Y$. Prove that
$$\displaystyle\lim_{X\to\infty}\frac{\#[X]^{\sf,Y}}{\#[X]}=\prod_{p\leq
    Y}(1-1/p^2).$$
\item By taking limit $Y\to\infty$ conclude that $\displaystyle\lim_{X\to\infty}\frac{\#[X]^\sf}{\#[X]}\leq\frac{1}{\zeta(2)}$.
\item ({\bf Tail estimate}) Prove that the number of positive integers
  less than $X$ that are divisible by $p^2$ is bounded by $O(X/p^2)$,
  where the implied constant is independent of $X$ and $p$.
\item Use the tail estimate to prove that
$$
\#[X]^\sf=\#[X]^{\sf,Y}+O(\sum_{p>Y}X/p^2).
$$ Divide by $X$ and take limits (first $X\to\infty$ then
$Y\to\infty$) to conclude the desired result.
\end{enumerate}

Note that the above list of problems proves that
$\#[X]^\sf=\frac{\#[X]}{\zeta(2)}+o(X)$. We can use the
inclusion-exclusion formula to improve the error term $o(X)$ to a
power saving of $X$. To this end, let $[X]_{a\,(b)}$ denote the subset
of integers $n\in[X]$ such that $n\equiv a\pmod{b}$.
\begin{enumerate}
\item Prove the inclusion-exclusion formula
$$
\#[X]^\sf=\sum_{n=1}^\infty \mu(n)\#[X]_{0\,(n^2)}
$$
and use it to prove that
$\#[X]^\sf=\frac{\#[X]}{\zeta(2)}+O(X^{1/2})$.
\item Replace the ``sharp'' count $\#[X]$ and $\#[X]^\sf$ by
  ``smooth'' counts. (Replace the characteristic function of the unit
  interval $[0,1]$ with a smooth approximation of it.) Estimate the
  smooth analogue of $\#[X]^\sf$ using inclusion exclusion. Note that
  the gains of counting smoothly are lost in the sieve.
\item Count cubefree integers both sharply and smoothly.
\end{enumerate}

%%%16
\newpage
\section{How to count rings and fields (2/2)\\ by Manjul Bhargava}\label{16}

\begingroup
\renewcommand{\addcontentsline}[3]{}% Remove functionality of \addcontentsline
\endgroup


%%%17
\newpage
\renewcommand{\thesubsection}{\arabic{section}.\arabic{subsection}}
\section{Heuristics for number field counts and applications to curves over finite fields\\ by Melanie Matchett-Wood}\label{17}


%%%18
\newpage
\renewcommand{\thesubsection}{\arabic{section}.\arabic{subsection}}
\section{Moduli space of rings\\ by Bjorn Poonen}\label{18}

%%%19
\newpage
\renewcommand{\thesubsection}{\arabic{section}.\arabic{subsection}}
\section{Problem Set (3/7)}\label{19}

%%%20
\newpage
\renewcommand{\thesubsection}{\arabic{section}.\arabic{subsection}}
\section{Zeta Function methods\\by Frank Throne }\label{20}

%%%21.
\newpage
\section{Counting Artin representations and modular forms of eight one
 \\by  Eknath Ghate}\label{21}

\begingroup
\renewcommand{\addcontentsline}[3]{}% Remove functionality of \addcontentsline
\endgroup


%%%22
\newpage
\renewcommand{\thesubsection}{\arabic{section}.\arabic{subsection}}
\section{Binary quartic forms; bounded average rank of elliptic curves
\\ by Arul Shankar (1/2)}\label{22}


%%%33
\newpage
\renewcommand{\thesubsection}{\arabic{section}.\arabic{subsection}}
\section{Selmer groups and heuristics (1/2)\\ by Bjorn Poonen}\label{23}

%%%24
\newpage
\renewcommand{\thesubsection}{\arabic{section}.\arabic{subsection}}
\section{Rational points on curves\\by Michael Stoll}\label{24}

%%%25
\newpage
\renewcommand{\thesubsection}{\arabic{section}.\arabic{subsection}}
\section{Problem Set (4/7)}\label{25}

%%%26
\newpage
\section{Binary quartic forms; bounded average rank of elliptic curves
\\ by Arul Shankar (2/2)}\label{26}

\begingroup
\renewcommand{\addcontentsline}[3]{}% Remove functionality of \addcontentsline
\endgroup


%%%27
\newpage
\renewcommand{\thesubsection}{\arabic{section}.\arabic{subsection}}
\section{Coregular spaces and genus one curves \\by Wei Ho}\label{27}


%%%28
\newpage
\renewcommand{\thesubsection}{\arabic{section}.\arabic{subsection}}
\section{Arithmetic invariant theory and hyperelliptic curves (1/2)\\ by Benedict Gross}\label{28}

%%%29
\newpage
\renewcommand{\thesubsection}{\arabic{section}.\arabic{subsection}}
\section{Problem Set (5/7)}\label{29}

%%%30
\newpage
\renewcommand{\thesubsection}{\arabic{section}.\arabic{subsection}}
\section{Applications to the Birch and Swinnerton-Dyer conjecture\\ by  Manjul Bhargava}\label{30}

%%%31
\newpage
\section{Selmer groups and heuristics (2/2)
\\by Bjorn Poonen}\label{31}

\begingroup
\renewcommand{\addcontentsline}[3]{}% Remove functionality of \addcontentsline
\endgroup


%%%32
\newpage
\renewcommand{\thesubsection}{\arabic{section}.\arabic{subsection}}
\section{Arithmetic invariant theory and hyperelliptic curves (2/2)\\ by Benedict Gross}\label{32}


%%%33
\newpage
\renewcommand{\thesubsection}{\arabic{section}.\arabic{subsection}}
\section{Problem Set (6/7)}\label{33}

%%%34
\newpage
\renewcommand{\thesubsection}{\arabic{section}.\arabic{subsection}}
\section{Chabauty methods and hyperelliptic curves\\ by  Bjorn Poonen}\label{34}

%%%35
\newpage
\renewcommand{\thesubsection}{\arabic{section}.\arabic{subsection}}
\section{Topological and algebraic geometry method over function fields (1/2) \\ by Jordan Ellenberg}\label{35}

%%%36
\newpage
\section{Counting methods over global fields\\ by Jerry Wang}\label{36}

\begingroup
\renewcommand{\addcontentsline}[3]{}% Remove functionality of \addcontentsline
\endgroup


%%%37
\newpage
\renewcommand{\thesubsection}{\arabic{section}.\arabic{subsection}}
\section{Problem Set (7/7)
}\label{37}


%%%38
\newpage
\renewcommand{\thesubsection}{\arabic{section}.\arabic{subsection}}
\section{The Chabauty method and symmetric powers of curves
\\ by Jennifer Park}\label{38}

%%%39
\newpage
\renewcommand{\thesubsection}{\arabic{section}.\arabic{subsection}}
\section{Topological and algebraic geometry methods over function fields (2/2)\\ by Jordan Ellenberg}\label{39}

%%%40
\newpage
\renewcommand{\thesubsection}{\arabic{section}.\arabic{subsection}}
\section{Future perspectives\\ by  Manjul Bhargava}\label{40}

\end{document}