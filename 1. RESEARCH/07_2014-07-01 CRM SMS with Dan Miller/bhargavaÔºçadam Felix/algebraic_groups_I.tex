\chapter{Algebraic groups, representation theory, and invariant theory \\ Eyal Goren}\label{ch:2}

A \textbf{linear algebraic group} is a Zariski-closed subgroup of $\GL_{N}(\overline{\Bbbk})$ for some $N$, where $\Bbbk$ is a field of characteristic zero.  We will denote the \textbf{algebraic closure} of $\Bbbk$ by $\overline{\Bbbk}$.  We let
\begin{align*}
B &= \left\{\begin{pmatrix} \ast & \ast \\ 0 & \ast\end{pmatrix}: \ast \in \overline{\Bbbk}\right\} \\
U &= \left\{\begin{pmatrix} 1 & \ast \\ 0 & 1\end{pmatrix}: \ast \in \overline{\Bbbk}\right\} \\
T &= \left\{\begin{pmatrix} \ast & 0 \\ 0 & \ast\end{pmatrix}: \ast \in \overline{\Bbbk}\right\}.
\end{align*}
$B$ is known as the Borel subgroup, $U$ is the unipotent matrices and $T$ is the (maximal) torus.\\
$q$ will denote a symmetric bilinear form $(q_{ij})_{1 \le i,j \le N}$ and $q(x)=\frac{1}{2}q(x,x)$.  We have
\begin{equation*}
\SO=\SO(q)=\{M \in \GL_{N}: Mq^{t}M=q, \det(M)=1\}.
\end{equation*}
For example, $q=offdiag(1,1,\dotsc,1)$.  Then $q(x)=\frac{1}{2}(x_{1}x_{2n+1}+x_{2}x_{2n}+\dotsb+x_{n+1}^{2})$, where $N=2n+1$.  Then, $T=\{ diag(t_{1},t_{2},\dotsc, t_{n},1,t_{n}^{-1},t_{n-1}^{-2},\dotsc,t_{1}^{-1})\}$ and $B=\{\begin{pmatrix} \ast & \ast \\ 0 & \ast\end{pmatrix} \cap \SO(q)\}$.