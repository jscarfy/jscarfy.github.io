\chapter{Basic algebraic number theory \\ (number fields, class groups, why they are useful and interesting) \\ Eknath Ghate}\label{ch:3}

\section{Algebraic number theory and Class groups via CM (Complex Multiplication)}\label{sec:3.1}

\begin{definition}
A \textbf{number field} $K$ is a finite field extension of $\mathbb{Q}$.
\end{definition}
One important invariant in algebraic number theory is the class group attached to $K$:
\begin{equation*}
\cl_{K} = \frac{I_{K}}{P_{K}},
\end{equation*}
where $I_{K}$ is the free Abelian group generated by the prime ideals of $\mathscr{O}_{K}$ (fractional ideals) and $P_{K}$ is the set of principal fractional ideals.
\begin{theorem}
$\cl_{K}$ is a finite group.
\end{theorem}
We let $h_{K}=|\cl_{K}|$ is called the \textbf{class number} of $K$.  Note that $h_{K}=1$ if and only if $\mathscr{O}_{K}$is a UFD.  We can also associate $h_{K}$ with the number of quadratic forms.  Counting the number of cubic fields $K$ that are nowhere totally ramified of $|\disc(K)| < X$ is equivalent to summing $\#h_{3}(F)$ for $|\disc F| < X$.  \textcolor{red}{(diagram of field extensions)}  We get that the average of
\begin{equation*}
h_{3}(F) = \begin{cases}
\frac{4}{3} & \text{if } F \text{ is real quadratic}, \\
2 & \text{if } F \text{ is imaginary quadratic}.
\end{cases}
\end{equation*}
Starting point of class field theory:  Let $H/K$ be the maximal unramified Abelian extension of $K$.  We call $H$ the \textbf{Hilbert class field}.  Then the Artin map
\begin{equation*}
I_{K} \to \gal(H/K)
\end{equation*}
is induced by
\begin{equation*}
\mathfrak{p} \mapsto \sigma_{\mathfrak{p}}:=\frob_{\mathfrak{p}}:=(\mathfrak{p},H/K).
\end{equation*}
That is, $\sigma_{\mathfrak{p}}(X) \equiv X^{N_{K/\mathbb{Q}}(\mathfrak{p})} \bmod{\mathfrak{P}}$, when $\mathfrak{P}|\mathfrak{p}$ in $H$.  This map induces an isomorphism $\cl_{K} \to{\sim} \gal(H/K)$.
\begin{remarks}
\begin{enumerate}[label=(\arabic{enumi})]
\item (Easy) $\mathfrak{p}$ splits completely in $H$ if and only if $\mathfrak{p}$ is principal.
\item (Harder) Principal ideal theorem.
\begin{theorem}
Every ideal of $K$ becomes principal in $H$.
\end{theorem}
\textcolor{red}{(diagram 2)}
Since $G{1} \leftrightarrow H$, $\Trace=0$.\\
\indent Let $K=\cl$ and $H-\mathbb{Q}$.  Assume from now on that $K=\mathbb{Q}(\sqrt{d})$ is an imaginary field.  There are nine $d$'s for which $h_{\cl(\mathbb{Q})(\sqrt{d})}=1$.  They are $d=-1,-2,-3,-7,-11,-19,-43,-67$ and $-163$.  Also $h_{K} \to \infty$ as $d \to \infty$.\\
\indent For example, $K=\mathbb{Q}(\sqrt{-23})$.  The theory of CM describes how to generate $H$ explicitly:
\begin{theorem}
$H=K(j(E))$.
\end{theorem}
Let $E$ be an elliptic curve with CM by $\mathscr{O}_{K}$.  Then, $\mathbb{Z} \subsetneq \End(E) \cong \mathscr{O}_{K}$.  Then, $j(E)$ is the $j$-invariant of $E$.  For $E:y^{2}=4x^{3}-g_{2}x-g_{3}$.  Then,
\begin{equation*}
j(E)=1728\frac{g_{2}^{3}}{g_{2}^{3}-g_{3}^{2}}.
\end{equation*}
Alternatively, $j(E)= \frac{1}{q}+744+196884q^{2}+\dotsb$, where $q=e^{2\pi i\tau}$ and $E=\mathbb{C}/\langle 1,\tau \rangle$.
\end{enumerate}
\end{remarks}
\indent For the rest of the talk, we will sketch a proof of this.\\
\indent Let $\mathscr{E}_{\mathbb{C}}(K)=\{\mathbb{C}-\text{isomorphism classes of elliptic curves } E \text{ with  CM by } \mathscr{O}_{K}\}$.  Then $\cl_{K} \leftrightarrow^{1:1} \mathscr{E}_{\mathbb{C}}(K)$ given by $[\mathfrak{a}] \to \mathbb{C}/\mathfrak{a}$, where $\mathfrak{a} \subset I_{K}$ and $\mathfrak{a} \subset \mathscr{O}_{K} \subset K \subset \mathbb{C}$.\\
\indent We can obtain a simply transitive action of $\cl_{K}$ on $\mathscr{E}_{\mathbb{C}}(K)$.  Then, $[\mathfrak{a}]\cdot \mathbb{C}/\mathfrak{b}=\mathbb{C}/\mathfrak{a}^{-1}\mathfrak{b}$.\\
\indent Elliptic curves with CM have rational models.
\begin{theorem}
$\mathscr{E}_{\overline{\mathbb{Q}}}(K) \to^{1:1} \mathscr{E}_{\mathbb{C}}(K)$ given by $E \mapsto E$.
\end{theorem}
\begin{proof}
E is always defined over $\mathbb{Q}(j(E))$.  $E$ has CM by $\mathscr{O}_{K} \Rightarrow j(E) \in \overline{\mathbb{Q}}$.  Indeed, if $\sigma \in \aut(\mathbb{C})$, then $j(E^{\sigma})=j(E)^{\sigma}$.  Therefore, $E^{\sigma}$ has CM since $\End(E^{\sigma}) \cong \End(E) \cong \mathscr{O}_{K}$.  Also, $E$ has CM by $\mathscr{O}_{K}$.  By finiteness, $j(E) \in \overline{\mathbb{Q}}$.\\
\indent Towards $H$:  Fix $E \in \mathscr{E}_{\overline{\mathbb{Q}}}(K)=:\mathscr{E}(K)$.  For each $\sigma \in \gal(\overline{\mathbb{Q}}/K)$, there exists a unique $[\mathfrak{a}] \in \cl_{K}$ such that $E^{\sigma}=[\mathfrak{a}]\cdot E$.
\end{proof}
\begin{definition}
$F \cdot \gal(\overline{\mathbb{Q}}/K) \to \cl_{K}$ defined by $\sigma \mapsto F(\sigma) = [\mathfrak{a}]$.
\end{definition}
\begin{proposition}
\begin{enumerate}[label=(\arabic{enumi})]
\item $F$ does not depend on $E$.
\item $F$ is homomorphism.
\end{enumerate}
\end{proposition}
\begin{proof}
(1) is subtle. (2) follows from (1).:  $F(\sigma z) E \cong E^{\sigma z}=(E^{\sigma})^{z}=(F(\sigma) \cdot E)^{z}$ (by (1)) $=F(z) (F(\sigma)\cdot E)=(F(\sigma)F(z))E$ as $\cl_{K}$ is Abelian.\\
\indent Let $L$ be the fixed field of $\ker F$.
\end{proof}
\begin{claim}
$L=K(j(E))$.
\end{claim}
\begin{proof}
$\gal(\overline{\mathbb{Q}}/L)=\{\sigma \in \gal(\overline{\mathbb{Q}}/K):E^{\sigma} \cong F(\sigma) \cdot E \cong E\}$.  Then the claim is true if and only if $j(E)^{\sigma}=j(E)$.
\end{proof}
Note that $F$ maps $\gal(L/K)$ injects into $\cl_{K}$.  Thus, $L/K$ is Abelian.  Let $m$ be the conductor of $L/K$.  This is the greatest common divisor of all $m \subset \mathscr{O}_{K}$ such that $i(K_{m,1})$ is a subset of the kernel of the Artin map on $L/K$.  This is because $K_{m,1}=\langle \alpha \rangle$ where $\alpha \equiv 1 \bmod{m}$.\\
\indent One checks that $I_{K}^{m} \to^{Art_{L/K} \gal(L/K} \to^{injection and F} \cl_{K}$.  This induces a map which sends $[\mathfrak{a}] \mapsto [\mathfrak{a}]$.  Therefore $F$ is surjective.  So $F$ is an isomorphism from $\gal(L/K) \to \cl_{K}$.  So, $L=K(j(E))$.  Also note that $F((\langle \alpha \rangle, L/K))=1$ implies $\gcd(\langle \alpha \rangle,L/K)=1$ by injectivity of $F$ on $\gal(L/K)$.  Here $\langle \alpha \rangle \in I_{K}^{m}$ is principal.  Hence, $m=1$ and $L/K$ is unramified.  So, $K \subset L \subset H$.  However, $[L:K]=h_{K}$ and $[H:K]=h_{K}$.  Thus, $H=L=K(j(E))$.
\begin{theorem}
$j(E) \in \overline{\mathbb{Z}}$.  ($e^{\pi\sqrt{163}} \textquotedblleft\in\textquotedblright\mathbb{Z}$.)  Also, $R(F(m))=K(j(E))$, $h(E[m])$, where $h$ is Weber function and $m \subset \mathscr{O}_{K}$.
\end{theorem}