\chapter{Basic analytic number theory \\ Andrew Granville}\label{ch:6}

\section{$\sum\limits_{n \le x} a_{n}$ for various natural arithmetic $a_{n}$}

Consider
\begin{equation*}
[x]:=\sum_{1 \le n \le x} 1=x+\text{a bounded error}
\end{equation*}
\textcolor{red}{(insert diagram 1)}\\
Consider
\begin{equation*}
\#\{(x,y) \in \mathbb{Z}^{2}:x^{2}+y^{2} \le T\} \approx \text{Area of } \{(x,y) \in \mathbb{R}^{2}: x^{2}+y^{2} \le T\}+\text{a bounded that's a multiple of the radius }(=\sqrt{T})
\end{equation*}
\indent \textcolor{red}{(insert diagram 2)}\\
\indent \textcolor{red}{(insert diagram 3)}\\
We want to consider
\begin{equation*}
\#\{\text{lattice point inside the triangle } (x,y) \in \mathbb{Z}^{2}: x,y > 0, y+\alpha x \le T\}=\text{Area of Triangle} \left(\frac{1}{2} \frac{T^{2}}{\alpha}\right)+\text{an error term that is bounded by a multiple of } T.
\end{equation*}
Suppose $\alpha=-1$.  Then, consider $N \in \mathbb{Z}$.  We want $x+y \le T$, where $T=N-\varepsilon$ and $T=N+\varepsilon$.  Here the error term changes by size $N$.  This phenomenon occurs also on as $\alpha \in \mathbb{Q}$.\\
\indent Suppose $\alpha \in \mathbb{R} \setminus \mathbb{Q}$.  Then the problem becomes much more difficult.  For example, $\alpha=1+1/N_{1}+1/N_{2}+1/N_{3}+\dotsb$, where $N_{1}=10^{100}$, $N_{2}=2^{N_{1}}$, $N_{3}=2^{N_{2}}$.  Then, the error term becomes very large.\\
\indent Let $\tau(n)=\#\{d|n\}$.  Consider
\begin{align*}
\sum_{n \le T} \tau(n) &= \sum_{n \le T} \sum_{\substack{x,y \ge 1 \\ xy=n}} 1 = \sum_{\substack{x,y \ge 1 \\ xy \le T}} 1\\
&\approx \text{area of } \{xy \le T: x,y \ge 0\}+\text{an error term that depends on boundary}
\end{align*}
\textcolor{red}{(draw the picture with lattice lines)}.  In the above , we can change $x,y \ge 0$ to $x,y \ge 1/2$.  Now the area is
\begin{equation*}
\int_{1/2}^{2T} \frac{T}{x} \,\, dx = T\log (4T)=T\log T+\text{an error bounded by } T.
\end{equation*}
We consider
\begin{align*}
\sum_{\substack{a,b \ge 1 \\ ab \le T}} 1 &= \sum_{1 \le a \le T} \sum_{1 \le b \le \frac{T}{a}} 1 \\
&= \sum_{1 \le a \le T} \left(\frac{T}{a}+O(1)\right) \\
&= T\sum_{a \le T} \frac{1}{a}+O(T) \\
&= T\log T+O(T).
\end{align*}
Here, we used
\begin{equation*}
\sum_{n=1}^{N} \frac{1}{n} = \log N+\gamma+O\left(\frac{1}{N}\right),
\end{equation*}
where $\gamma=\lim\limits_{N \to \infty} \sum_{n \le N} \frac{1}{n}-\log N$.\\
\indent Alternatively, Dirichlet approached this summation as follows:  let $m=\min\{a,b\}$ and $n=\max\{a,b\}$.
\begin{align*}
\sum_{\substack{a,b \le T \\ a,b \ge 1}}1 &= 2 \sum_{1 \le m \le \sqrt{T}} \sum_{m < n \le \frac{T}{m}}1+O(\sqrt{T}) \\
&= 2\sum_{1 \le m \le \sqrt{T}} \left(\frac{T}{m}-m+O(1)\right)+O(\sqrt{T}) \\
&= 2T\sum_{m \le \sqrt{T}} \frac{1}{m}-T+O(\sqrt{T})=T\log T+(2\gamma-1)T+O(\sqrt{T}).
\end{align*}

\section{Complex Analysis and Number Theory}

Consider
\begin{equation*}
\int_{0}^{1} e^{2\pi i nt} \,\, dt = \begin{cases}
1 \quad &\text{if } n=0, \\
0 &\text{if } n \ne 0.
\end{cases}
\end{equation*}
Consider Goldbach's conjecture:  $2N=p+q$ if and only if $2N-p-q=0$.  Then,
\begin{equation*}
\sum_{p,q \text{ prime}} \begin{cases}
1 \quad &\text{if } p+q-2N=0 \\
0 & \text{otherwise}
\end{cases}=
\sum_{p,q \text{ prime}} \int_{0}^{1} e^{2\pi i (p+q-2N)t} \,\, dt = \int_{0}^{1} e^{-4\pi iNt}\left(\sum_{p} e^{2\pi ipt}\right)^{2} \,\, dt.
\end{equation*}
We would like an indicator function for inequality:  for example, let $y \in \mathbb{R}$
\begin{equation*}
\frac{1}{2\pi i} \int_{c-i\infty}^{c+i\infty} \frac{e^{sy}}{s} \,\, ds = \begin{cases}
1 \quad &\text{if } y > 0 \\
\frac{1}{2} &\text{if } y=0 \\
0 &\text{if } y < 0
\end{cases}.
\end{equation*}
Here $c > 0$.  This is known as Perron's formula\\
\indent Let $w=e^{y}$.  Then, we are interested in whether $w > 1$ or $w < 1$.  Then,
\begin{equation*}
\sum_{n \le x} a_{n} = \sum_{n \ge 1} a_{n} \begin{cases}
1 \quad &\text{if } \frac{x}{n} > 1 \\
0 &\text{if } \frac{x}{n} < 1 = \sum_{n \ge 1} a_{n}\frac{1}{2\pi i} \int_{c-i\infty}^{c+i\infty} \left(\frac{x}{n}\right)^{s} \frac{ds}{s} = \frac{1}{2\pi i} \int_{c-i\infty}^{c+i\infty} A(s)\frac{x^{s}}{s} \,\, ds,
\end{cases}
\end{equation*}
where $A(s)=\sum_{n \ge 1} \frac{a_{n}}{n^{s}}$.\\
\indent For example,
\begin{equation*}
[x] = \sum_{n \le x} 1 = \frac{1}{2\pi i} \int_{2-\infty}^{2+i\infty} \zeta(s)\frac{x^{s}}{s} \,\, ds
\end{equation*}
Note that $|x^{s}|=x^{\Re(s)}$ and $\zeta(s)$ is analytic except at $s=1$ with a pole of order $1$ with residue $1$.  So our above summation becomes
\begin{equation*}
[x] = 1\frac{x^{1}}{1}+\zeta(0)x^{0}+\text{error}=x-\frac{1}{2}+\text{error}.
\end{equation*}
For the divisor function $\tau(n)$, we first need to compute the divisor Dirichlet series.  We have
\begin{equation*}
D(s):= \sum_{n \ge 1} \frac{\tau(n)}{n^{s}}=\sum_{n \ge 1} \frac{1}{n^{2}} \sum_{\substack{ab=n \\ a,b \ge 1}} 1 = \sum_{a,b \ge 1} \frac{1}{(ab)^{s}}=\zeta(s)^{2}.
\end{equation*}
So,
\begin{equation*}
\sum_{n \le x} \tau(n)= \frac{1}{2\pi i} \int_{2-i\infty}^{2+i\infty} \zeta(s)^{2}\frac{x^{s}}{s} \,\, ds.
\end{equation*}
Near $s=1$, we have
\begin{equation*}
\zeta(s)=\frac{1}{s-1}+\gamma+c(s-1)+\dotsb,
\end{equation*}
and so,
\begin{equation*}
\frac{\zeta(s)^{2}x^{s}}{s}=\left(\frac{1}{s-1}+\gamma+c(s-1)+\dotsb\right)^{2}x(1+(s-1)\log x+\dotsb)(1-(s-1)+(s-1)^{2}+\dotsb) = x\left(\frac{1}{(s-1)^{2}}+\frac{1}{s-1}(\log x+2\gamma-1)+\dotsb\right).
\end{equation*}
Therefore,
\begin{equation*}
\sum_{n \le x} \tau(n) = x\log x+(2\gamma-1)x+\dotsb.
\end{equation*}
Let's consider Riemann's memoir.  We have
\begin{equation*}
\zeta(s)=\prod_{p}\left(1-\frac{1}{p^{s}}\right)^{-1}
\end{equation*}
for $\Re(s) > 1$.  Then, the logarithmic derivative of $\zeta(s)$ is
\begin{equation*}
-\frac{\zeta^{\prime}}{\zeta(s)}=\sum_{\substack{p \text{ prime} \\ m \ge 1}} \frac{\log p}{p^{ms}}.
\end{equation*}
So
\begin{equation*}
\sum_{p^{m} \le x} \log p=\frac{1}{2\pi i} \int_{2-i\infty}^{2+i\infty} -\frac{\zeta^{\prime}(s)}{\zeta(s)} \frac{x^{s}}{s} \,\, ds=x-\frac{\zeta^{\prime}(0)}{\zeta(0)}-\sum_{\varrho:\zeta(\varrho)=0} \frac{x^{\varrho}}{\varrho}.
\end{equation*}
Let's consider the circle problem:
\begin{equation*}
\sum_{a^{2}+b^{2} \le T}=\sum_{n \le T} R(n),
\end{equation*}
where $R(n)=\#\{(a,b):n=a^{2}+b^{2}\}=4r(n)$.  The associated Dirichlet series is
\begin{equation*}
\left(1-\frac{1}{2^{s}}\right)^{-1}\prod_{p \equiv 1 \bmod{4}}\left(1-\frac{1}{p^{s}}\right)^{-2}\prod_{p \equiv 3 \bmod{4}}\left(1-\frac{1}{p^{2s}}\right)^{-1}.
\end{equation*}
Note that
\begin{equation*}
L\left(s,\left(\frac{-4}{\cdot}\right)\right)=\prod_{p \equiv 1 \bmod{4}}\left(1-\frac{1}{p^{s}}\right)^{-1}\prod_{p \equiv 1 \bmod{4}}\left(1+\frac{1}{p^{s}}\right)^{-1}.
\end{equation*}
Thus, our Dirichlet series is $\zeta(s)L\left(s,\left(\frac{-4}{\cdot}\right)\right)$.  Therefore,
\begin{equation*}
\sum_{a^{2}+b^{2} \le T} 4 \int \zeta L(s \\
=4 L(1,-4)+error
\end{equation*}

\subsection{Sieving}

Consider
\begin{align*}
\sum_{\substack{n \le x \\ n \text{ is squarefree}}} 1 &= [x]-\sum_{p} \#\{n \le x:p^{2}|n\}+\sum_{p,q} \#\{n \le x:p^{2}q^{2}|n\}-\dotsb \\
&=[x]-\sum_{p} \left[\frac{x}{p^{2}}\right] +\sum_{p,q} \left[\frac{x}{p^{2}q^{2}}\right]-\dotsb \\
&= x\prod_{p}\left(1-\frac{1}{p^{2}}\right)+\text{error term}=\frac{6}{\pi^{2}}x+\text{error term}
\end{align*}
Alternatively,
\begin{equation*}
\sum_{\substack{n \le x \\ n \text{ is squarefree}}} 1 = \sum_{\substack{n \le x \\ p^{2} \nmid n \\ \forall p \le y}}1+\text{error}_{1}.
\end{equation*}
Here, we'll take $y=\log x$ and $\text{error}_{1} \le \sum\limits_{y < p < \sqrt{x}} \#\{n \le x:p^{2}|n\}$.\\
\indent A more difficult problem is the following:
\begin{align*}
\#\{n \le x: n^{2}+1 \text{ is squarefree}\} &= \sum_{\substack{n \le x \\ p^{2} \nmid n^{2}+1 \\ \forall p \le y}} 1+O\left(\sum_{y < p < x} \#\{n \le x: p^{2}|n^{2}+1\}\right).
\end{align*}
The main term is of the form
\begin{equation*}
x\prod_{p \le y} \frac{\#\{n \bmod{p^{2}}:p^{2} \nmid n^{2}+1\}}{p^{2}}.
\end{equation*}
The error term is much more difficult as the set in the summation is bounded by $2\left(\frac{x}{p^{2}}+1\right)$.

\section{$abc$-conjecture}

Suppose $a+b=c$ with $\gcd(a,b)=1$ and $a,b > 0$.  Then, the $abc$-conjecture states,
\begin{equation*}
\prod_{p|ab(a+b)} p > \kappa_{\varepsilon} \max\{|a|,|b|\}^{1-\varepsilon}.
\end{equation*}
Let $H(a,b)=\max\{|a|,|b|\}$.\\
\indent Elkies used a map $X \to \mathbb{P}^{1}$, where $t=a/c$ to show that, for any $F(x.y) \in \mathbb{Z}[x,y]$ homogeneous with no repeated roots.  Then,
\begin{equation*}
\prod_{p|F(a,b)} p > \kappa_{\varepsilon} H(a,b)^{\deg F-2-\varepsilon}.
\end{equation*}
Let $f(x)$ be a polynomial of degree $d$.  Let $F(x,y)=y^{d+1}f(x/y)$.  Then,
\begin{equation*}
\prod_{p|f(n)} p > \kappa_{\varepsilon} |n|^{\deg f-1-\varepsilon}.
\end{equation*}
Let $f(n)=q^{2}m \approx n^{\deg f}$.  Also, $f(n)/q=qm=\prod_{p|f(n)}p > \kappa^{\deg f-1-\varepsilon}$.  So, $q < n^{1+\varepsilon}$.\\
\indent Let's consider $p^{2}|4a^{3}+27b^{2}=\res(f,f^{\prime})$