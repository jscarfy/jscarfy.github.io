\documentclass[12pt, twoside,letterpaper,smaller]{amsbook}
\usepackage{latexsym,anysize,enumerate,soul,fancyhdr,xcolor,pst-node,pstricks,,array,url,verbatim,mathrsfs,graphicx,txfonts}
\usepackage[colorlinks=true,urlcolor=red,citecolor=red,linkcolor=OliveGreen]{hyperref}
\usepackage[mathscr]{eucal}
\usepackage{microtype}
\DisableLigatures{encoding=*,family=*}
\usepackage{palatino}
\linespread{1.05}
\usepackage[OT2,T1]{fontenc}
\DeclareSymbolFont{cyrletters}{OT2}{wncyr}{m}{n}
\DeclareMathSymbol{\Sha}{\mathalpha}{cyrletters}{"58}
\usepackage[all,cmtip]{xy}

\newcounter{result}

\theoremstyle{plain}
\newtheorem{theorem}{Theorem}[chapter]
\newtheorem{corollary}[theorem]{Corollary}
\newtheorem{lemma}[theorem]{Lemma}
\newtheorem{proposition}[theorem]{Proposition}
\newtheorem*{claim}{Claim}
\newtheorem*{aoc}{The Axiom of Choice}
\newtheorem*{zl}{Zorn's Lemma}
\newtheorem*{wop}{The Well Ordering Principle}
\newtheorem*{lagrange}{Lagrange's Theorem}

\theoremstyle{definition}
\newtheorem*{definition}{Definition}
\newtheorem*{notation}{Notation}
\newtheorem*{remark}{Remark}
\newtheorem*{remarks}{Remarks}
\newtheorem*{conjecture}{Conjecture}
\newtheorem*{observation}{Observation}
\newtheorem*{solution}{Solution}
\newtheorem*{example}{Example}
\newtheorem*{examples}{Examples}
\newtheorem{exercise}{Exercise}[section]

\newcommand{\gal}{\textnormal{\text{Gal}}}
\newcommand{\trace}{\textnormal{\text{tr}}}
\newcommand{\Trace}{\textnormal{\text{Tr}}}
\newcommand{\rk}{\textnormal{\text{rank}}}
\newcommand{\disc}{\textnormal{\text{disc}}}
\newcommand{\rad}{\textnormal{\text{rad}}}
\newcommand{\lcm}{\textnormal{\text{lcm}}}
\newcommand{\diag}{\textnormal{\text{diag}}}
\newcommand{\kernel}{\textnormal{\text{ker}}}
\newcommand{\ann}{\textnormal{\text{ann}}}
\newcommand{\spec}{\textnormal{\text{spec}}}
\newcommand{\mspec}{\textnormal{\text{mspec}}}
\newcommand{\image}{\textnormal{\text{im}}}
\newcommand{\coker}{\textnormal{\text{coker}}}
\newcommand{\GL}{\textnormal{\text{GL}}}
\newcommand{\SL}{\textnormal{\text{SL}}}
\newcommand{\PSL}{\textnormal{\text{PSL}}}
\newcommand{\orth}{\textnormal{\text{O}}}
\newcommand{\transpose}{\textnormal{\text{T}}}
\newcommand{\SO}{\textnormal{\text{SO}}}
\newcommand{\Unipotent}{\textnormal{\text{U}}}
\newcommand{\SP}{\textnormal{\text{SP}}}
\newcommand{\sylow}{\textnormal{\text{Syl}}}
\newcommand{\aut}{\textnormal{\text{Aut}}}
\newcommand{\inn}{\textnormal{\text{Inn}}}
\newcommand{\content}{\textnormal{\text{cont}}}
\newcommand{\jac}{\textnormal{\text{jac}}}
\newcommand{\functions}{\textnormal{\text{functions}}}
\newcommand{\op}{\textnormal{\text{op}}}
\newcommand{\End}{\textnormal{\text{end}}}
\newcommand{\support}{\textnormal{\text{supp}}}
\newcommand{\tor}{\textnormal{\text{tor}}}
\newcommand{\sel}{\textnormal{\text{Sel}}}
\newcommand{\sym}{\textnormal{\text{Sym}}}
\newcommand{\cl}{\textnormal{\text{Cl}}}
\newcommand{\frob}{\textnormal{\text{Frob}}}
\newcommand{\ord}{\textnormal{\text{ord}}}
\newcommand{\res}{\textnormal{\text{res}}}

\def\thechapter{\Roman{chapter}}
\def\thepart{\Roman{part}}
\pagenumbering{roman}
\makeindex
\numberwithin{equation}{chapter}
\marginsize{2.25cm}{2.25cm}{2.25cm}{2.25cm}
\setul{.25ex}{.1ex}
\begin{document}
\title{Counting Arithmetic Objects \\ Centre de recherches math\'{e}matiques}
\maketitle
\tableofcontents

%\pagestyle{fancy}
%\renewcommand{\chaptermark}[1]{\markboth{#1}{}}
%\renewcommand{\sectionmark}[1]{\markright{#1}{}}
%\renewcommand{\headrulewidth}{1pt}
% 
%\fancyhf{}
%\fancyhead[LE]{\thepage \qquad \small{\MakeUppercase{\leftmark}}}
%\fancyhead[RO]{\small{\MakeUppercase{\rightmark}} \qquad \thepage}
% 
%\fancypagestyle{plain}{ %
%\fancyhf{} % remove everything
%\renewcommand{\headrulewidth}{1pt}
%\renewcommand{\footrulewidth}{0pt}}
%
%\setcounter{page}{1}
%\pagenumbering{arabic}



\part{Background}

\chapter{Introduction and perspective:  Counting Arithmetic Objects \\ Manjul Bhargava}\label{ch:1}

\noindent \underline{\textbf{Question}}:  Given a class $\mathscr{C}$ of objects of arithmetic interest, how many objects are there in $\mathscr{C}$, up to $\sim$ (isomorphism), have bounded invariants?
\begin{examples}
{\ }
\begin{enumerate}[\upshape (\alph{enumi})]
\item $\mathscr{C}$:  Number fields of a given degree.  \textbf{Invariants}:  Discriminant of the number field.
\item $\mathscr{C}$:  Class group elements of number fields of a given degree.  \textbf{Invariants}:  Discriminant of the number field.
\item $\mathscr{C}$:  Rational points on some family of curves.  \textbf{Invariants}:  Coefficients of the equation defining the curve.
\item $\mathscr{C}$:  Elliptic curves in some family, where each elliptic curve is weighted by rank.  \textbf{Invariants}:  Coefficients of the equation defining the curve.
\item $\mathscr{C}$:  Fix $n \in \mathbb{N}$:  $n$-Selmer group elements of Jacobians of curves in some family.  \textbf{Invariants}:  Coefficients of the equation defining the curve.
\end{enumerate}
\end{examples}
Given such a class of objects of arithmetic interest, how many are there with respect to their basic invariants?  How are they distributed with respect to their basic invariants?  Beyond the case of degree $2$ number fields and genus $0$ curves, little was known at beginning of the 20\textsuperscript{th} century.\\
\indent \underline{\textbf{Strategy}}:
\begin{equation*}
\{\text{Objects in } \mathscr{C}\}/\sim \quad \xrightarrow{injection} \quad G(\mathbb{Z})\setminus V(\mathbb{Z}),
\end{equation*}
where $G$ is an algebraic group and $V$ is a representation over $\mathbb{Z}$.  We would like this map to correspond to
\begin{equation*}
\text{invariants of objects} \to \text{fundamental polynomial invariants of action of } G \text{ on } V.
\end{equation*}
(Good choices of such maps often come from algebraic geometry but must work out the theory over $\mathbb{Z}$.)
\begin{examples}
{\ }

\begin{enumerate}[\upshape (\alph{enumi})]
\item (Gau\ss{}, 1801)
\begin{equation*}
\{\text{ideal classes of (orders in) quadratic fields}\}/\sim \quad \xrightarrow{injection} \quad \SL_{2}(\mathbb{Z})\setminus \{\text{integer binary quadratic forms} ax^{2}+bxy+cy^{2}\}.
\end{equation*}
Here the discriminant maps to $b^{2}-4ac$, which is called the \textbf{discriminant} of the binary quadratic field or the \textbf{unique polynomial invariant}.  Note that $I=\langle \alpha,\beta\rangle$ maps to $\frac{N(\alpha x+\beta y)}{N(I)}$, where $N$ is the norm associated to the field.  The above map is one-to-one if we restrict the map to non-square discriminants.  \textcolor{red}{(Put this in diagram form.)}
\item (Levi 1920's, Delone-Faddeev 1960's)
\begin{equation*}
\{\text{cubic rings}\}/\sim \quad \xrightarrow{\text{injection (actually isomorphism)}} \quad \GL_{2}(\mathbb{Z})\setminus\{\text{integer binary cubic forms} ax^{3}+bx^{2}y+cxy^{2}+dy^{3}\}.
\end{equation*}
Here the discriminant (of a cubic ring \textcolor{red}{(to be defined later)}) maps to the discriminant of the right-hand side.  This is also known as \textbf{unique polynomial invariant}.  In the case of the integer binary cubic form $ax^{3}+bx^{2}y+cxy^{2}+dy^{3}$, the discriminant is $b^{2}c^{2}-4ac^{3}-4b^{3}d-27a^{2}d^{2}+18abcd$.  Note that $R=\langle 1,\alpha,\beta\rangle$ maps to $\sqrt{\frac{\disc(\alpha x+\beta y)}{\disc{R}}}=[R:\mathbb{Z}[\alpha x+\beta y]]$, where $\disc(\alpha)=\disc(\mathbb{Z}[\alpha])$.
\item $\{\text{quartic fields}\}/\sim \quad \xrightarrow{}\qquad \disc \to \disc=$unique polynomial invariant.
\item $\{\text{quintic fields}\}/\sim \quad \xrightarrow{}\qquad \disc \to \disc=$unique polynomial invariant.
\item (BSD)
\begin{align*}
\{\sigma \in E(\mathbb{Q}):2E(\mathbb{Q}): E \text{ is an elliptic curve of the form} E_{A,B}:y^{2}=x^{3}+Ax+B, A,B \in \mathbb{Z}\} \quad &\xrightarrow{injection} \sel_{2}(E) \\
&\xrightarrow{injection} \quad \GL_{2}(\mathbb{Z}) \setminus \{\text{integer binary quartic forms}\}.
\end{align*}
Here the invariants $A$ and $B$ map to $I$ and $J$, some fundamental invariants for binary quartic of degrees $2$ and $3$.
\item (Cremona-Fisher-Stoll):  Let $n \in \{3,4,5\}$.  Then, consider
\begin{equation*}
\{\sigma \in E(\mathbb{Q})/nE(\mathbb{Q}): E=E_{A,B}\} \quad \xrightarrow{injection} \dots
\end{equation*}
Here the invariants $A$ and $B$ map to a similar $I$ and $J$ as discussed above.
\item
\begin{equation*}
\{\text{rational points on odd hyperelliptic curves of the form} y^{2}=x^{2g+1}+a_{1}x^{2g}+\dots+a_{2g+1}\} \quad \xrightarrow{}\quad \SO_{2g+1}(\mathbb{Z})\setminus \sym^{2}(\mathbb{Z}^{2g+1}).
\end{equation*}
Here $a_{i} \in \mathbb{Z}$ and the invariants exactly correspond.
\item
\begin{equation*}
\{\text{rational points on general even degree hyperelliptic curves of the form} z^{2}=a_{0}x^{2g+2}+a_{1}x^{2g+1}+\dots+a_{2g+1}\} \quad \xrightarrow{} \quad \SO_{2g+1}(\mathbb{Z})\setminus \sym^{2}(\mathbb{Z}^{2g+1}).
\end{equation*}
Here $a_{i} \in \mathbb{Z}$ and the invariants exactly correspond.
\end{enumerate}
\end{examples}
There are many more examples.\\
\indent \underline{\textbf{Question}}:  How many orbits of $G(\mathbb{Z})$ on $V(\mathbb{Z})$ are there having bounded invariants?  Gau\ss{} worked out Case 1.  He showed:  let $h(D)$ be the number of $\SL_{2}(\mathbb{Z})$-orbits of integer binary quadratic forms of discriminant $D$.  Then,
\begin{theorem}[Gau\ss{}, Lipschitz, Mertens]
\begin{equation*}
\sum_{0 < -D < X} h(D) \sim \frac{\pi}{18}X^{\frac{3}{2}}.
\end{equation*}
\end{theorem}
\begin{proof}
\begin{enumerate}[\upshape (\arabic{enumi})]
\item Gau\ss shows that every integer binary quadratic form $ax^{2}+bxy+cy^{2}$ with $D=b^{2}-4ac < 0$ has a unique $\gal_{2}(\mathbb{Z})$-equivalent form satisfying $|b| < a  \le c$ or $0 < b=a \le c$.
\item By the geometry of numbers, we have
\begin{equation*}
\sum_{0 < -D < X} h(D) \sim \#\{(a,b,c): 0 < 4ac-b^{2} < X, |b| < a < c\}
\end{equation*}
Gau\ss{} conjectured that this should be the volume of this region.  Gau\ss{} found the integral of a related region.  The rest was made rigorous by Lipschitz and Mertens.  We note that this is a lot trickier.  For example, the region
\begin{equation*}
\#\{(a,b,c): 0 \le 4ac-b^{2} < X, |b| < a \le c\}
\end{equation*}
has infinitely many points.  \textcolor{red}{(See the picture)}.\\
\indent There are a few ways to show that this volume conjecture works:
\begin{enumerate}[\upshape (\arabic{enumii})]
\item Explicitly evaluate it as a triple summation $\sum_{a}\sum_{b}\sum_{c}$, approximation by integral.  Keep track of the error term (Exercise:  Lipschitz, Mertens)
\item Davenport developed some general principles for bounded regions.  Davenport used a principle to reprove count of the binary quadratics and extended the argument to a count of binary cubic forms.  This requires knowing explicit inequalities for the region.
\item Zeta function and $L$-function methods:  for binary quadratic forms, Siegel, Goldfeld-Hoffstein, Shintani, Datskovsky,... where able to use these methods.  They did not need explicit bounds.  However, this method is limited as it becomes difficult if the degree of the objects one is studying becomes complicated.
\item Hybrid methods:  average over a compact continuum of fundamental domains.  Some advantages of this method are that is does not need explicit inequalities but one can still use elementary geometry of numbers.
\end{enumerate}
Works on above mentioned 10 representations gives a count of cubic, quartic and quintic fields boundedness of average rank of elliptic curves, hyperelliptic curves with few rational points.\\
\indent \underline{\textbf{Question}}  What if $\mathbb{Q}$ is replaced by another base field?  a number field, function field?
\begin{enumerate}[\upshape (\alph{enumii})]
\item Function field:  we can use algebraic geometry and topological methods.  De Jong worked on boundedness of average rank 
\item Generic method of (4) above.  Ellenberg has many results of this type.
\end{enumerate}
\end{enumerate}
\end{proof}
\chapter{Algebraic groups, representation theory, and invariant theory \\ Eyal Goren}\label{ch:2}

A \textbf{linear algebraic group} is a Zariski-closed subgroup of $\GL_{N}(\overline{\Bbbk})$ for some $N$, where $\Bbbk$ is a field of characteristic zero.  We will denote the \textbf{algebraic closure} of $\Bbbk$ by $\overline{\Bbbk}$.  We let
\begin{align*}
B &= \left\{\begin{pmatrix} \ast & \ast \\ 0 & \ast\end{pmatrix}: \ast \in \overline{\Bbbk}\right\} \\
U &= \left\{\begin{pmatrix} 1 & \ast \\ 0 & 1\end{pmatrix}: \ast \in \overline{\Bbbk}\right\} \\
T &= \left\{\begin{pmatrix} \ast & 0 \\ 0 & \ast\end{pmatrix}: \ast \in \overline{\Bbbk}\right\}.
\end{align*}
$B$ is known as the Borel subgroup, $U$ is the unipotent matrices and $T$ is the (maximal) torus.\\
$q$ will denote a symmetric bilinear form $(q_{ij})_{1 \le i,j \le N}$ and $q(x)=\frac{1}{2}q(x,x)$.  We have
\begin{equation*}
\SO=\SO(q)=\{M \in \GL_{N}: Mq^{t}M=q, \det(M)=1\}.
\end{equation*}
For example, $q=offdiag(1,1,\dotsc,1)$.  Then $q(x)=\frac{1}{2}(x_{1}x_{2n+1}+x_{2}x_{2n}+\dotsb+x_{n+1}^{2})$, where $N=2n+1$.  Then, $T=\{ diag(t_{1},t_{2},\dotsc, t_{n},1,t_{n}^{-1},t_{n-1}^{-2},\dotsc,t_{1}^{-1})\}$ and $B=\{\begin{pmatrix} \ast & \ast \\ 0 & \ast\end{pmatrix} \cap \SO(q)\}$.
\chapter{Basic algebraic number theory \\ (number fields, class groups, why they are useful and interesting) \\ Eknath Ghate}\label{ch:3}

\section{Algebraic number theory and Class groups via CM (Complex Multiplication)}\label{sec:3.1}

\begin{definition}
A \textbf{number field} $K$ is a finite field extension of $\mathbb{Q}$.
\end{definition}
One important invariant in algebraic number theory is the class group attached to $K$:
\begin{equation*}
\cl_{K} = \frac{I_{K}}{P_{K}},
\end{equation*}
where $I_{K}$ is the free Abelian group generated by the prime ideals of $\mathscr{O}_{K}$ (fractional ideals) and $P_{K}$ is the set of principal fractional ideals.
\begin{theorem}
$\cl_{K}$ is a finite group.
\end{theorem}
We let $h_{K}=|\cl_{K}|$ is called the \textbf{class number} of $K$.  Note that $h_{K}=1$ if and only if $\mathscr{O}_{K}$is a UFD.  We can also associate $h_{K}$ with the number of quadratic forms.  Counting the number of cubic fields $K$ that are nowhere totally ramified of $|\disc(K)| < X$ is equivalent to summing $\#h_{3}(F)$ for $|\disc F| < X$.  \textcolor{red}{(diagram of field extensions)}  We get that the average of
\begin{equation*}
h_{3}(F) = \begin{cases}
\frac{4}{3} & \text{if } F \text{ is real quadratic}, \\
2 & \text{if } F \text{ is imaginary quadratic}.
\end{cases}
\end{equation*}
Starting point of class field theory:  Let $H/K$ be the maximal unramified Abelian extension of $K$.  We call $H$ the \textbf{Hilbert class field}.  Then the Artin map
\begin{equation*}
I_{K} \to \gal(H/K)
\end{equation*}
is induced by
\begin{equation*}
\mathfrak{p} \mapsto \sigma_{\mathfrak{p}}:=\frob_{\mathfrak{p}}:=(\mathfrak{p},H/K).
\end{equation*}
That is, $\sigma_{\mathfrak{p}}(X) \equiv X^{N_{K/\mathbb{Q}}(\mathfrak{p})} \bmod{\mathfrak{P}}$, when $\mathfrak{P}|\mathfrak{p}$ in $H$.  This map induces an isomorphism $\cl_{K} \to{\sim} \gal(H/K)$.
\begin{remarks}
\begin{enumerate}[label=(\arabic{enumi})]
\item (Easy) $\mathfrak{p}$ splits completely in $H$ if and only if $\mathfrak{p}$ is principal.
\item (Harder) Principal ideal theorem.
\begin{theorem}
Every ideal of $K$ becomes principal in $H$.
\end{theorem}
\textcolor{red}{(diagram 2)}
Since $G{1} \leftrightarrow H$, $\Trace=0$.\\
\indent Let $K=\cl$ and $H-\mathbb{Q}$.  Assume from now on that $K=\mathbb{Q}(\sqrt{d})$ is an imaginary field.  There are nine $d$'s for which $h_{\cl(\mathbb{Q})(\sqrt{d})}=1$.  They are $d=-1,-2,-3,-7,-11,-19,-43,-67$ and $-163$.  Also $h_{K} \to \infty$ as $d \to \infty$.\\
\indent For example, $K=\mathbb{Q}(\sqrt{-23})$.  The theory of CM describes how to generate $H$ explicitly:
\begin{theorem}
$H=K(j(E))$.
\end{theorem}
Let $E$ be an elliptic curve with CM by $\mathscr{O}_{K}$.  Then, $\mathbb{Z} \subsetneq \End(E) \cong \mathscr{O}_{K}$.  Then, $j(E)$ is the $j$-invariant of $E$.  For $E:y^{2}=4x^{3}-g_{2}x-g_{3}$.  Then,
\begin{equation*}
j(E)=1728\frac{g_{2}^{3}}{g_{2}^{3}-g_{3}^{2}}.
\end{equation*}
Alternatively, $j(E)= \frac{1}{q}+744+196884q^{2}+\dotsb$, where $q=e^{2\pi i\tau}$ and $E=\mathbb{C}/\langle 1,\tau \rangle$.
\end{enumerate}
\end{remarks}
\indent For the rest of the talk, we will sketch a proof of this.\\
\indent Let $\mathscr{E}_{\mathbb{C}}(K)=\{\mathbb{C}-\text{isomorphism classes of elliptic curves } E \text{ with  CM by } \mathscr{O}_{K}\}$.  Then $\cl_{K} \leftrightarrow^{1:1} \mathscr{E}_{\mathbb{C}}(K)$ given by $[\mathfrak{a}] \to \mathbb{C}/\mathfrak{a}$, where $\mathfrak{a} \subset I_{K}$ and $\mathfrak{a} \subset \mathscr{O}_{K} \subset K \subset \mathbb{C}$.\\
\indent We can obtain a simply transitive action of $\cl_{K}$ on $\mathscr{E}_{\mathbb{C}}(K)$.  Then, $[\mathfrak{a}]\cdot \mathbb{C}/\mathfrak{b}=\mathbb{C}/\mathfrak{a}^{-1}\mathfrak{b}$.\\
\indent Elliptic curves with CM have rational models.
\begin{theorem}
$\mathscr{E}_{\overline{\mathbb{Q}}}(K) \to^{1:1} \mathscr{E}_{\mathbb{C}}(K)$ given by $E \mapsto E$.
\end{theorem}
\begin{proof}
E is always defined over $\mathbb{Q}(j(E))$.  $E$ has CM by $\mathscr{O}_{K} \Rightarrow j(E) \in \overline{\mathbb{Q}}$.  Indeed, if $\sigma \in \aut(\mathbb{C})$, then $j(E^{\sigma})=j(E)^{\sigma}$.  Therefore, $E^{\sigma}$ has CM since $\End(E^{\sigma}) \cong \End(E) \cong \mathscr{O}_{K}$.  Also, $E$ has CM by $\mathscr{O}_{K}$.  By finiteness, $j(E) \in \overline{\mathbb{Q}}$.\\
\indent Towards $H$:  Fix $E \in \mathscr{E}_{\overline{\mathbb{Q}}}(K)=:\mathscr{E}(K)$.  For each $\sigma \in \gal(\overline{\mathbb{Q}}/K)$, there exists a unique $[\mathfrak{a}] \in \cl_{K}$ such that $E^{\sigma}=[\mathfrak{a}]\cdot E$.
\end{proof}
\begin{definition}
$F \cdot \gal(\overline{\mathbb{Q}}/K) \to \cl_{K}$ defined by $\sigma \mapsto F(\sigma) = [\mathfrak{a}]$.
\end{definition}
\begin{proposition}
\begin{enumerate}[label=(\arabic{enumi})]
\item $F$ does not depend on $E$.
\item $F$ is homomorphism.
\end{enumerate}
\end{proposition}
\begin{proof}
(1) is subtle. (2) follows from (1).:  $F(\sigma z) E \cong E^{\sigma z}=(E^{\sigma})^{z}=(F(\sigma) \cdot E)^{z}$ (by (1)) $=F(z) (F(\sigma)\cdot E)=(F(\sigma)F(z))E$ as $\cl_{K}$ is Abelian.\\
\indent Let $L$ be the fixed field of $\ker F$.
\end{proof}
\begin{claim}
$L=K(j(E))$.
\end{claim}
\begin{proof}
$\gal(\overline{\mathbb{Q}}/L)=\{\sigma \in \gal(\overline{\mathbb{Q}}/K):E^{\sigma} \cong F(\sigma) \cdot E \cong E\}$.  Then the claim is true if and only if $j(E)^{\sigma}=j(E)$.
\end{proof}
Note that $F$ maps $\gal(L/K)$ injects into $\cl_{K}$.  Thus, $L/K$ is Abelian.  Let $m$ be the conductor of $L/K$.  This is the greatest common divisor of all $m \subset \mathscr{O}_{K}$ such that $i(K_{m,1})$ is a subset of the kernel of the Artin map on $L/K$.  This is because $K_{m,1}=\langle \alpha \rangle$ where $\alpha \equiv 1 \bmod{m}$.\\
\indent One checks that $I_{K}^{m} \to^{Art_{L/K} \gal(L/K} \to^{injection and F} \cl_{K}$.  This induces a map which sends $[\mathfrak{a}] \mapsto [\mathfrak{a}]$.  Therefore $F$ is surjective.  So $F$ is an isomorphism from $\gal(L/K) \to \cl_{K}$.  So, $L=K(j(E))$.  Also note that $F((\langle \alpha \rangle, L/K))=1$ implies $\gcd(\langle \alpha \rangle,L/K)=1$ by injectivity of $F$ on $\gal(L/K)$.  Here $\langle \alpha \rangle \in I_{K}^{m}$ is principal.  Hence, $m=1$ and $L/K$ is unramified.  So, $K \subset L \subset H$.  However, $[L:K]=h_{K}$ and $[H:K]=h_{K}$.  Thus, $H=L=K(j(E))$.
\begin{theorem}
$j(E) \in \overline{\mathbb{Z}}$.  ($e^{\pi\sqrt{163}} \textquotedblleft\in\textquotedblright\mathbb{Z}$.)  Also, $R(F(m))=K(j(E))$, $h(E[m])$, where $h$ is Weber function and $m \subset \mathscr{O}_{K}$.
\end{theorem}
\chapter{Basics of binary quadratic forms and Gauss composition \\ Andrew Granville}\label{ch:4}
\chapter{Curves:  Geometric properties \\ Henri Darmon}\label{ch:5}

\begin{definition}
A \textbf{curve} over a field $\Bbbk$ is a smooth, geometrically connected variety of dimension $1$ over the field $\Bbbk$.
\end{definition}
What we should keep in mind for curves is equations:  for example, $x^{2}-Dy^{2}=1$, $y^{2}=x^{3}+ax+b$, or $x^{n}+y^{n}=z^{n}$.\\
\indent We will be assuming that the characteristic of $\Bbbk$ is $0$.
\indent One of the most important tools in this area is the Riemann-Roch Theorem.\\
\indent Let $X$ be a projective curve.  Then, $X(\mathbb{C})$ looks like \textcolor{red}{(diagram 1)}.  We say that $U$ is \textbf{Zarski open} if $U=X \setminus \{p_{1},p_{2},\dotsc,p_{s}$, where $p_{i} \in X(\overline{\Bbbk})$.  We denote by $\mathscr{O}_{X}$ by the sheaf of regular functions on $X$.  We denote by $\mathscr{O}(U)$ the regular functions.\\
\indent \underline{\textbf{Question}}:  We would like to understand $\mathscr{O}(U)$ as a ring.  Note that $\mathscr{O}(U)=\Bbbk[f_{1},f_{2},\dotsc,f_{n}]/\langle\text{relations } p_{i}(f_{1},f_{2},\dotsc,f_{n})=0 \text{ for } i=1,2,\dotsc,n\rangle$.  Here $(f_{1},f_{2},\dotsc,f_{n}):U \to^{injection} \mathbb{A}^{n}$.  Liouville proved $\mathscr{O}(X)=\Bbbk$.\\
\indent \underline{\textbf{Assumption}}:  there exists $\infty \in X(\Bbbk)$ and $U=X \setminus \{\infty\}$.\\
\indent Then, $\mathscr{O}(U,n\infty)=\{f \in \mathscr{O}(U): \ord_{\infty} f \ge -n\}$.  Note that
\begin{equation*}
\Bbbk=\mathscr{O}(X) \subset \mathscr{O}(U;\infty) \subset \mathscr{O}(U;2\infty) \subset \dotsb \subset \mathscr{O}(U;n\infty) \subset \dotsb \subset \mathscr{O}(U).
\end{equation*}
Note that the dimension of $\mathscr{O}(U,(n+1)\infty)$ in $\mathscr{O}(U,n\infty)$ is at most $1$.  Thus, $\dim_{\Bbbk} \mathscr{O}(U;n\infty) \le n+1$.
\begin{theorem}[Riemann-Roch; Crude Form]
there is an integer $g$, depending on $X$, but not on $\infty$ or $n$, such that
\begin{equation*}
\dim \mathscr{O}(U;n\infty) \ge n+1-g
\end{equation*}
with equality if $n$ is sufficiently large.
\end{theorem}
\begin{proof}[Idea of proof]
Choose a parameter $t$ at $\infty$.  Then, the principal part satisfies $PP_{\infty} \cdot \mathscr{O}(U;n\infty) \to t^{-n}\Bbbk[t]/\Bbbk[t]$.  What are the obstructions to producing $f$ with given $PP_{\infty}$?\\
\indent \underline{\textbf{Residue Theorem}}:  If $\omega$ on $X(\Bbbk)$ is a meromorphic differential, then
\begin{equation*}
\sum_{\rho \in X(\overline{\Bbbk})} \res_{\rho}(\omega)=0.
\end{equation*}
\end{proof}
For example, if $\omega=(a_{m}t^{-m}+a_{m-1}t^{-m+1}+\dotsb+a_{0}+a_{1}t+\dotsb)dt$, then $\res(\omega)=a_{-1}$.
\begin{corollary}
If $\omega \in \Omega^{1}(X)$ is a global regular differential and $f \in \mathscr{O}(U)$, then $\res_{\infty}(f\omega)=0$.
\end{corollary}
Define $\res_{\infty}: t^{-n}\Bbbk[t]/\Bbbk[t] \to \Omega^{1}(X)\check{}$ where $f \mapsto (\omega \mapsto \res_{\infty}(f\omega)$.  Then, we have the following sequence
\begin{equation*}
0 \to \Bbbk_{dim 1} \to \mathscr{O}(U;n\infty) \to^{PP_{\infty}} t^{-n}\Bbbk[t]/\Bbbk[t]_{dim n} \to^{\res_{\infty}} \Omega^{1}(X)\check{}_{dim g} \to \Omega(X;-n\infty)\check{}_{dim \to 0 as n \to \infty} \to 0.
\end{equation*}
Define $\Omega^{1}(X;-n\infty)=\{\omega \in \Omega^{1}(X) \text{ such that } \ord_{\infty}(\omega) \ge n\}$.  Note that $\Omega^{1}(X;-n\infty) \subset \Omega^{1}(X)$ and $\Omega^{1}(X)\check{} \to^{\text{surjection}} \Omega^{1}(X;-n\infty)\check{}$.  The Riemann-Roch Theorem gives us that the above sequence is exact.  Note the dimensions of the sequence are indicated above.  A more precise version of the Riemann-Roch Theorem is the following:  let $g=\dim_{\Bbbk} \Omega^{1}(X)$.  Then, $\dim \mathscr{O}(U;n\infty)-\dim \Omega^{1}(X;-n\infty)=n+1-g$.

\subsection{Vocabulary}

A \textbf{division} of $X$ is a formal finite linear combination of points in $X(\overline{\Bbbk})$ with integer coefficients: $\sum_{P \in X(\overline{\Bbbk})} n_{P}\cdot P$, $n_{P} \in \mathbb{Z}$ and $n_{P}=0$ for all but finitely many $P \in X(\overline{\Bbbk})$.  We denote the divisor of $X(\overline{\Bbbk})$ by $\div(X(\overline{\Bbbk})$.  Note that $\div(X(\Bbbk))=\div(X(\overline{\Bbbk}))^{G_{\Bbbk}}$.\\
\indent $\Bbbk(X)=\lim_{\to_{U}} \mathscr{O}(U)$, the rational functions.  For $f \in \Bbbk(X)$, $\div(f)=\sum_{P \in X(\overline{\Bbbk})} \ord_{P}(f) \cdot P \subset \div(X/\Bbbk)$ is called a \textbf{principal divisor}.  We say that $D_{1} \ge D_{2}$ if $n_{P}(D_{1}) \ge n_{P}(D_{2}$ for all $P \in X(\overline{\Bbbk})$.  We denote by $\mathscr{L}(D)$ the set $\{f \in \Bbbk(X) \text{ such that } \div(f) \ge -D\}$.  Then, $\mathscr{O}(U;n\infty)=\mathscr{L}(n\infty)$.  So $\Omega_{\text{mer}}^{1}(X)=\lim_{\to_{U}} \Omega^{1}(U)$ is a one-dimensional vector space over $\Bbbk(X)$.  choose $\omega \in \Omega_{\text{mer}}^{1}(X)$ with $\div(\omega)=K$.  Note that $\omega \ne 0$.
\begin{definition}
$K$ is called the \textbf{canonical divisor class}.
\end{definition}
\begin{remark}
$\mathscr{L}(K) \cong \Omega^{1}(X)$ given by $f \mapsto f\omega$.  Also, $\Omega^{1}(X;-n\infty)=\mathscr{L}(K-n\infty)$.  Then, the final form of the Riemann-Roch Theorem is the following:  for all (positive) divisors $D \in \div(X/\Bbbk)$, then $\dim(\mathscr{L}(D))-\dim\mathscr{L}(K-D)=\deg(D)+1-g$.
\end{remark}
\indent Some consequences:  $g=\dim \mathscr{L}(K)$
\begin{enumerate}[\upshape (a)]
\item $D=0$.  Then, the Riemann-Roch Theorem gives us $1-\dim \mathscr{L}(K)=1-g$.
\item $D=K$.  Then, $\dim \mathscr{L}(K)-1=\deg (K)+1-g$.  However, $g=\dim \mathscr{L}(K)$.  So , $g-1=\deg(K)+1-g$.  Thus, $\deg(K)=2g-2$.  Therefore, the number of zeros of $\omega \in \Omega^{1}(X)$ is $2g-2$.
\end{enumerate}
Let $g=0$.  If $X(\Bbbk) \ne \varnothing$, then $\infty \in X(\Bbbk)$.  So, $\dim \mathscr{L}(n\infty)=n+1$.  So $\mathscr{L}(\infty)=\Bbbk \oplus \Bbbk t$, $\mathscr{L}(2\infty)=\Bbbk \oplus \Bbbk t \oplus \Bbbk t^{2}$.  Continuing in this fashion $\mathscr{L}(n\infty)=\Bbbk \oplus \Bbbk t \oplus \dotsb \oplus \Bbbk t^{n}$.  Thus, $\mathscr{O}(U)=\Bbbk[t]$, $U \cong \mathbb{A}^{1}$ and $X \cong \mathbb{P}_{1}$.\\
\indent Fact:  $X$ has a rational divisor of degree $2$ ($g=0$ in $\deg K=2g-2$).  Then, $P+P^{\prime}$ with $P \in X(\overline{\Bbbk})$.  So $\deg(-K)=2$ and $-K=P+P^{\prime}$.  Hence, $\mathscr{L}(P+P^{\prime})=\Bbbk \oplus \Bbbk u \oplus \Bbbk v$, $\mathscr{L}(2P+2P^{\prime})=\Bbbk \oplus \Bbbk u \oplus \Bbbk v+\Bbbk uv+\Bbbk v^{2}+\Bbbk u^{2}$.  So there is a relation:  $a+bu+cv+duv+ev^{2}+fu^{2}=0$.\\
\indent Let $g=1$.  A curve of genus $1$ with $\infty \in X(\Bbbk)$ is an elliptic curve.  So $\dim \Omega^{1}(X)=1$.  So,  $\omega \in \Omega^{1}(X)$ is everywhere non-vanishing.  So $\Omega^{1}(X) \cong \mathscr{O}_{X}$ and $\mathscr{O}(X)=\Bbbk$.  Note that $\dim \mathscr{L}(D)-\dim(\mathscr{L}(-D))=\deg(D)$.  So $\dim \mathscr{L}(\infty)-\dim \mathscr{L}(-\infty)=1$.  Hence, $\dim \mathscr{L}(-n\infty)=0$ for all $n > 0$.  Thus, $\dim\mathscr{L}(D)=\deg(D)$.  Then, $\mathscr{L}(2\infty)=\Bbbk \oplus \Bbbk x$, $\mathscr{L}(3\infty)=\Bbbk \oplus \Bbbk x \oplus \Bbbk y$, $\mathscr{L}(4\infty)=\Bbbk \oplus \Bbbk x \oplus \Bbbk y \oplus \Bbbk x^{2}$, $\mathscr{L}(5\infty)=\Bbbk \oplus \Bbbk x \oplus \Bbbk y \oplus \Bbbk x^{2} \oplus \Bbbk xy$, $\mathscr{L}(6\infty)=\Bbbk \oplus \Bbbk x \oplus \Bbbk y \oplus \Bbbk x^{2} \oplus \Bbbk xy \oplus \Bbbk y^{2}+\Bbbk x^{3}$.  So $y^{2}-x^{3} \in \mathscr{L}(5\infty)$.  So, there exists $a_{i}$ such that $y^{2}+a_{1}xy+a_{3}y=x^{3}+a_{2}x^{2}=a_{4}x+a_{6}$, which can be transformed by $y^{2}=x^{3}=ax+b$.\\
\indent Suppose $X(\Bbbk)=\varnothing$.  The simplest case:  $X$ has a rational divisor of degree $2$.  This is true for elements of $\sel_{2}(E/\Bbbk)$.
\begin{theorem}
$X$ has an equation of the form:  $y^{2}=ax^{4}+bx^{3}+c^{2}+dx+e$.
\end{theorem}
\begin{proof}
$\mathscr{L}(P+P^{\prime})=\Bbbk \oplus \Bbbk x$, $\mathscr{L}(2P+2P^{\prime})=\Bbbk \oplus \Bbbk x \oplus \Bbbk x^{2} \oplus \Bbbk y$, $\mathscr{L}(4P+4P^{\prime})=\dotsb$.
\end{proof}
\chapter{Basic analytic number theory \\ Andrew Granville}\label{ch:6}

\section{$\sum\limits_{n \le x} a_{n}$ for various natural arithmetic $a_{n}$}

Consider
\begin{equation*}
[x]:=\sum_{1 \le n \le x} 1=x+\text{a bounded error}
\end{equation*}
\textcolor{red}{(insert diagram 1)}\\
Consider
\begin{equation*}
\#\{(x,y) \in \mathbb{Z}^{2}:x^{2}+y^{2} \le T\} \approx \text{Area of } \{(x,y) \in \mathbb{R}^{2}: x^{2}+y^{2} \le T\}+\text{a bounded that's a multiple of the radius }(=\sqrt{T})
\end{equation*}
\indent \textcolor{red}{(insert diagram 2)}\\
\indent \textcolor{red}{(insert diagram 3)}\\
We want to consider
\begin{equation*}
\#\{\text{lattice point inside the triangle } (x,y) \in \mathbb{Z}^{2}: x,y > 0, y+\alpha x \le T\}=\text{Area of Triangle} \left(\frac{1}{2} \frac{T^{2}}{\alpha}\right)+\text{an error term that is bounded by a multiple of } T.
\end{equation*}
Suppose $\alpha=-1$.  Then, consider $N \in \mathbb{Z}$.  We want $x+y \le T$, where $T=N-\varepsilon$ and $T=N+\varepsilon$.  Here the error term changes by size $N$.  This phenomenon occurs also on as $\alpha \in \mathbb{Q}$.\\
\indent Suppose $\alpha \in \mathbb{R} \setminus \mathbb{Q}$.  Then the problem becomes much more difficult.  For example, $\alpha=1+1/N_{1}+1/N_{2}+1/N_{3}+\dotsb$, where $N_{1}=10^{100}$, $N_{2}=2^{N_{1}}$, $N_{3}=2^{N_{2}}$.  Then, the error term becomes very large.\\
\indent Let $\tau(n)=\#\{d|n\}$.  Consider
\begin{align*}
\sum_{n \le T} \tau(n) &= \sum_{n \le T} \sum_{\substack{x,y \ge 1 \\ xy=n}} 1 = \sum_{\substack{x,y \ge 1 \\ xy \le T}} 1\\
&\approx \text{area of } \{xy \le T: x,y \ge 0\}+\text{an error term that depends on boundary}
\end{align*}
\textcolor{red}{(draw the picture with lattice lines)}.  In the above , we can change $x,y \ge 0$ to $x,y \ge 1/2$.  Now the area is
\begin{equation*}
\int_{1/2}^{2T} \frac{T}{x} \,\, dx = T\log (4T)=T\log T+\text{an error bounded by } T.
\end{equation*}
We consider
\begin{align*}
\sum_{\substack{a,b \ge 1 \\ ab \le T}} 1 &= \sum_{1 \le a \le T} \sum_{1 \le b \le \frac{T}{a}} 1 \\
&= \sum_{1 \le a \le T} \left(\frac{T}{a}+O(1)\right) \\
&= T\sum_{a \le T} \frac{1}{a}+O(T) \\
&= T\log T+O(T).
\end{align*}
Here, we used
\begin{equation*}
\sum_{n=1}^{N} \frac{1}{n} = \log N+\gamma+O\left(\frac{1}{N}\right),
\end{equation*}
where $\gamma=\lim\limits_{N \to \infty} \sum_{n \le N} \frac{1}{n}-\log N$.\\
\indent Alternatively, Dirichlet approached this summation as follows:  let $m=\min\{a,b\}$ and $n=\max\{a,b\}$.
\begin{align*}
\sum_{\substack{a,b \le T \\ a,b \ge 1}}1 &= 2 \sum_{1 \le m \le \sqrt{T}} \sum_{m < n \le \frac{T}{m}}1+O(\sqrt{T}) \\
&= 2\sum_{1 \le m \le \sqrt{T}} \left(\frac{T}{m}-m+O(1)\right)+O(\sqrt{T}) \\
&= 2T\sum_{m \le \sqrt{T}} \frac{1}{m}-T+O(\sqrt{T})=T\log T+(2\gamma-1)T+O(\sqrt{T}).
\end{align*}

\section{Complex Analysis and Number Theory}

Consider
\begin{equation*}
\int_{0}^{1} e^{2\pi i nt} \,\, dt = \begin{cases}
1 \quad &\text{if } n=0, \\
0 &\text{if } n \ne 0.
\end{cases}
\end{equation*}
Consider Goldbach's conjecture:  $2N=p+q$ if and only if $2N-p-q=0$.  Then,
\begin{equation*}
\sum_{p,q \text{ prime}} \begin{cases}
1 \quad &\text{if } p+q-2N=0 \\
0 & \text{otherwise}
\end{cases}=
\sum_{p,q \text{ prime}} \int_{0}^{1} e^{2\pi i (p+q-2N)t} \,\, dt = \int_{0}^{1} e^{-4\pi iNt}\left(\sum_{p} e^{2\pi ipt}\right)^{2} \,\, dt.
\end{equation*}
We would like an indicator function for inequality:  for example, let $y \in \mathbb{R}$
\begin{equation*}
\frac{1}{2\pi i} \int_{c-i\infty}^{c+i\infty} \frac{e^{sy}}{s} \,\, ds = \begin{cases}
1 \quad &\text{if } y > 0 \\
\frac{1}{2} &\text{if } y=0 \\
0 &\text{if } y < 0
\end{cases}.
\end{equation*}
Here $c > 0$.  This is known as Perron's formula\\
\indent Let $w=e^{y}$.  Then, we are interested in whether $w > 1$ or $w < 1$.  Then,
\begin{equation*}
\sum_{n \le x} a_{n} = \sum_{n \ge 1} a_{n} \begin{cases}
1 \quad &\text{if } \frac{x}{n} > 1 \\
0 &\text{if } \frac{x}{n} < 1 = \sum_{n \ge 1} a_{n}\frac{1}{2\pi i} \int_{c-i\infty}^{c+i\infty} \left(\frac{x}{n}\right)^{s} \frac{ds}{s} = \frac{1}{2\pi i} \int_{c-i\infty}^{c+i\infty} A(s)\frac{x^{s}}{s} \,\, ds,
\end{cases}
\end{equation*}
where $A(s)=\sum_{n \ge 1} \frac{a_{n}}{n^{s}}$.\\
\indent For example,
\begin{equation*}
[x] = \sum_{n \le x} 1 = \frac{1}{2\pi i} \int_{2-\infty}^{2+i\infty} \zeta(s)\frac{x^{s}}{s} \,\, ds
\end{equation*}
Note that $|x^{s}|=x^{\Re(s)}$ and $\zeta(s)$ is analytic except at $s=1$ with a pole of order $1$ with residue $1$.  So our above summation becomes
\begin{equation*}
[x] = 1\frac{x^{1}}{1}+\zeta(0)x^{0}+\text{error}=x-\frac{1}{2}+\text{error}.
\end{equation*}
For the divisor function $\tau(n)$, we first need to compute the divisor Dirichlet series.  We have
\begin{equation*}
D(s):= \sum_{n \ge 1} \frac{\tau(n)}{n^{s}}=\sum_{n \ge 1} \frac{1}{n^{2}} \sum_{\substack{ab=n \\ a,b \ge 1}} 1 = \sum_{a,b \ge 1} \frac{1}{(ab)^{s}}=\zeta(s)^{2}.
\end{equation*}
So,
\begin{equation*}
\sum_{n \le x} \tau(n)= \frac{1}{2\pi i} \int_{2-i\infty}^{2+i\infty} \zeta(s)^{2}\frac{x^{s}}{s} \,\, ds.
\end{equation*}
Near $s=1$, we have
\begin{equation*}
\zeta(s)=\frac{1}{s-1}+\gamma+c(s-1)+\dotsb,
\end{equation*}
and so,
\begin{equation*}
\frac{\zeta(s)^{2}x^{s}}{s}=\left(\frac{1}{s-1}+\gamma+c(s-1)+\dotsb\right)^{2}x(1+(s-1)\log x+\dotsb)(1-(s-1)+(s-1)^{2}+\dotsb) = x\left(\frac{1}{(s-1)^{2}}+\frac{1}{s-1}(\log x+2\gamma-1)+\dotsb\right).
\end{equation*}
Therefore,
\begin{equation*}
\sum_{n \le x} \tau(n) = x\log x+(2\gamma-1)x+\dotsb.
\end{equation*}
Let's consider Riemann's memoir.  We have
\begin{equation*}
\zeta(s)=\prod_{p}\left(1-\frac{1}{p^{s}}\right)^{-1}
\end{equation*}
for $\Re(s) > 1$.  Then, the logarithmic derivative of $\zeta(s)$ is
\begin{equation*}
-\frac{\zeta^{\prime}}{\zeta(s)}=\sum_{\substack{p \text{ prime} \\ m \ge 1}} \frac{\log p}{p^{ms}}.
\end{equation*}
So
\begin{equation*}
\sum_{p^{m} \le x} \log p=\frac{1}{2\pi i} \int_{2-i\infty}^{2+i\infty} -\frac{\zeta^{\prime}(s)}{\zeta(s)} \frac{x^{s}}{s} \,\, ds=x-\frac{\zeta^{\prime}(0)}{\zeta(0)}-\sum_{\varrho:\zeta(\varrho)=0} \frac{x^{\varrho}}{\varrho}.
\end{equation*}
Let's consider the circle problem:
\begin{equation*}
\sum_{a^{2}+b^{2} \le T}=\sum_{n \le T} R(n),
\end{equation*}
where $R(n)=\#\{(a,b):n=a^{2}+b^{2}\}=4r(n)$.  The associated Dirichlet series is
\begin{equation*}
\left(1-\frac{1}{2^{s}}\right)^{-1}\prod_{p \equiv 1 \bmod{4}}\left(1-\frac{1}{p^{s}}\right)^{-2}\prod_{p \equiv 3 \bmod{4}}\left(1-\frac{1}{p^{2s}}\right)^{-1}.
\end{equation*}
Note that
\begin{equation*}
L\left(s,\left(\frac{-4}{\cdot}\right)\right)=\prod_{p \equiv 1 \bmod{4}}\left(1-\frac{1}{p^{s}}\right)^{-1}\prod_{p \equiv 1 \bmod{4}}\left(1+\frac{1}{p^{s}}\right)^{-1}.
\end{equation*}
Thus, our Dirichlet series is $\zeta(s)L\left(s,\left(\frac{-4}{\cdot}\right)\right)$.  Therefore,
\begin{equation*}
\sum_{a^{2}+b^{2} \le T} 4 \int \zeta L(s \\
=4 L(1,-4)+error
\end{equation*}

\subsection{Sieving}

Consider
\begin{align*}
\sum_{\substack{n \le x \\ n \text{ is squarefree}}} 1 &= [x]-\sum_{p} \#\{n \le x:p^{2}|n\}+\sum_{p,q} \#\{n \le x:p^{2}q^{2}|n\}-\dotsb \\
&=[x]-\sum_{p} \left[\frac{x}{p^{2}}\right] +\sum_{p,q} \left[\frac{x}{p^{2}q^{2}}\right]-\dotsb \\
&= x\prod_{p}\left(1-\frac{1}{p^{2}}\right)+\text{error term}=\frac{6}{\pi^{2}}x+\text{error term}
\end{align*}
Alternatively,
\begin{equation*}
\sum_{\substack{n \le x \\ n \text{ is squarefree}}} 1 = \sum_{\substack{n \le x \\ p^{2} \nmid n \\ \forall p \le y}}1+\text{error}_{1}.
\end{equation*}
Here, we'll take $y=\log x$ and $\text{error}_{1} \le \sum\limits_{y < p < \sqrt{x}} \#\{n \le x:p^{2}|n\}$.\\
\indent A more difficult problem is the following:
\begin{align*}
\#\{n \le x: n^{2}+1 \text{ is squarefree}\} &= \sum_{\substack{n \le x \\ p^{2} \nmid n^{2}+1 \\ \forall p \le y}} 1+O\left(\sum_{y < p < x} \#\{n \le x: p^{2}|n^{2}+1\}\right).
\end{align*}
The main term is of the form
\begin{equation*}
x\prod_{p \le y} \frac{\#\{n \bmod{p^{2}}:p^{2} \nmid n^{2}+1\}}{p^{2}}.
\end{equation*}
The error term is much more difficult as the set in the summation is bounded by $2\left(\frac{x}{p^{2}}+1\right)$.

\section{$abc$-conjecture}

Suppose $a+b=c$ with $\gcd(a,b)=1$ and $a,b > 0$.  Then, the $abc$-conjecture states,
\begin{equation*}
\prod_{p|ab(a+b)} p > \kappa_{\varepsilon} \max\{|a|,|b|\}^{1-\varepsilon}.
\end{equation*}
Let $H(a,b)=\max\{|a|,|b|\}$.\\
\indent Elkies used a map $X \to \mathbb{P}^{1}$, where $t=a/c$ to show that, for any $F(x.y) \in \mathbb{Z}[x,y]$ homogeneous with no repeated roots.  Then,
\begin{equation*}
\prod_{p|F(a,b)} p > \kappa_{\varepsilon} H(a,b)^{\deg F-2-\varepsilon}.
\end{equation*}
Let $f(x)$ be a polynomial of degree $d$.  Let $F(x,y)=y^{d+1}f(x/y)$.  Then,
\begin{equation*}
\prod_{p|f(n)} p > \kappa_{\varepsilon} |n|^{\deg f-1-\varepsilon}.
\end{equation*}
Let $f(n)=q^{2}m \approx n^{\deg f}$.  Also, $f(n)/q=qm=\prod_{p|f(n)}p > \kappa^{\deg f-1-\varepsilon}$.  So, $q < n^{1+\varepsilon}$.\\
\indent Let's consider $p^{2}|4a^{3}+27b^{2}=\res(f,f^{\prime})$
\chapter{Curves:  Diophantine properties \\ Henri Darmon}\label{ch:7}

The majority of cases of interest in this chapter are $\Bbbk=\mathbb{Q}$ or $\Bbbk$ an algebraic number field.\\
\indent \underline{\textbf{Diophantine Questions}}:  What is $X(\Bbbk)$?  Is it finite?  What is $\#X(\Bbbk)$ for the typical $X$?\\
\indent We will somewhat ignore the integral points and focus more on rational points.  However, if $X$ is projective, $X(\mathbb{Z})=X(\mathbb{Q})$ and $X(\mathscr{O}_{K})=X(K)$.  If $X$ is affine, then $X$ injects into $\mathbb{A}^{n}$ for some $n \in \mathbb{N}$.  Also, $X(\mathbb{Z})=X(\mathbb{Q}) \cap \mathbb{A}^{n}(\mathbb{Z})$.  This will depend on the chosen equations.\\
\indent What is $X(\mathscr{O}_{K})$ and $X(\mathscr{O}_{K,s})$?  Here $\mathscr{O}_{K,s}=\mathscr{O}_{K}[s^{-1}]$.  Let $X=\tilde{X}\setminus\{P_{1},P_{2},\dotsc,P_{s}\}$.  It has invariants $g$, the genus of $\tilde{X}$ and $s$, the number of deleted points.  Let $\chi(X)=2-2g-s \in \mathbb{Z}$.  A lot of Diophantine behaviour of $X$ is governed by $\chi(X)$.\\
\indent \underline{\textbf{Fundamental Trichotomy}}:  $\chi > 0$, $\chi=0$, or $\chi < 0$.\\
\indent If $\chi > 0$, then we have the following theorem:
\begin{theorem}
$X(\mathscr{O}_{K,s})$ is either empty or infinite.
\end{theorem}
\begin{proof}
In the affine case, $g=0$ and $s=1$.  So $X=\mathbb{A}^{1}$ and $X(\mathscr{O}_{K,s})=\mathscr{O}_{K,s}$.\\
\indent In the projective case $g=0$ and $s=0$.  So $X=\mathbb{P}_{1}$ or $X$ is a conic.  Thus, $X(\mathscr{O}_{K,s})=X(\Bbbk)=\Bbbk \cup \{\infty\}$.
\end{proof}
\indent We also have the following result:  \underline{\textbf{Hasse-Minkowski}}:  For $g=0$, $X(\mathbb{Q}) \ne \varnothing$ if and only if $X(\mathbb{Q}_{p}) \ne \varnothing$ for all primes $p$ and $X(\mathbb{R}) \ne \varnothing$.\\
\indent If $\chi > 0$, then we have the following theorem:
\begin{theorem}[Siegel, Faltings]
$\#X(\mathscr{O}_{K,s}) < \infty$.
\end{theorem}
\begin{proof}
In the affine case (Siegel $\sim$ 1932), prototypical examples $g=0$ and $s=3$, which corresponds to $\mathbb{P}_{1} \setminus \{0,1,\infty\}$, or $g=1$ and $s=1$, which corresponds to $E \setminus \{\infty\}$.  let $\mathscr{O}_{\mathbb{P}_{1} \setminus \{0,1,\infty\}/\mathbb{Z}} = \mathbb{Z}[x,1/x,1/(x-1)]$.  Also, $\mathbb{P}_{1} \setminus \{0,1,\infty\}(\mathscr{O}_{K,s})=\hom(\mathbb{Z}[x,1/x,1/(x-1)],\mathscr{O}_{K,s})=\{u \in \mathscr{O}_{K,s} \text{ such that } u-1 \in \mathscr{O}_{K,s}^{\ast}\}$.  Here the $s$-unit equation makes an appearance.  This leads to $y^{2}=x^{3}+ax+b$, where $(x,y) \in \mathscr{O}_{K,s}^{2}$.\\
\indent In the projective case, we have $g > 1$ and $s=0$.  This is Faltings theorem.
\end{proof}
\indent If $\chi=0$, then we have the following theorem:
\begin{theorem}[Dirichlet, Mordell-Weil]
If $X(\mathscr{O}_{K,s})$ is non-empty, then it can be equipped with a natural group structure and is finitely generated.
\end{theorem}
\begin{proof}
In the affine case, $g=0$ and $s=2$ corresponds to $\mathbb{P}_{1} \setminus \{0,\infty\}=\mathbb{G}_{m}$.  Thus, $X(\mathscr{O}_{K,s})=\mathscr{O}_{K,s}^{\ast}$.  This comes from Dirichlet's unit theorem.\\
\indent In the projective case, $g=1$ and $s=0$ corresponds to $E$.  This is the Mordell-Weil theorem that $E(\mathbb{Q})$ and $E(\Bbbk)$ are finitely generated.
\end{proof}
\indent \underline{\textbf{Ranks}}  In the affine case, $\rk(\mathscr{O}_{K,s}^{\ast})=r_{1}+r_{2}-1+\#s$.\\
\indent In the projective case, the rank of an elliptic curves is much more subtle.\\
\indent We have $\mathbb{P}_{1} \setminus \{P,P^{\prime}\}$ corresponds to $x^{2}-Dy^{2}=1$.  The rank should be $0$ if $D < 0$ and $1$ if $D > 0$.\\
\indent \underline{\textbf{Question}}:  Is rank of $E(\mathbb{Q})$ unbounded?  It is conjectured that $\rk E(\mathbb{Q})$ is $0$ half of the time and $\rk E(\mathbb{Q})$ is $1$ half of the time.  Bhargava, Shankar and many other authors have shown that there are a positive density of elliptic curves of rank $0$ or $1$.\\
\indent \underline{\textbf{Proof of Mordell-Weil}}:  There are two ingredients:
\begin{enumerate}[\upshape (a)]
\item Heights:  $h:E(\mathbb{Q}) \to \mathbb{R}$.  It can be shown $\{h(P) < X\}$ is finite, $h(mP)=m^{2}h(P)$ and $h(P+Q)+h(P-Q)=2h(P)+2h(Q)$.
\item Weak Mordell-Weil Theorem:  $E(\mathbb{Q})/nE(\mathbb{Q})$ is finite for all $n \ge 1$.
\end{enumerate}
Then, the above to ingredients implies that Mordell-Weil is true.  The method is to use descent.  Let $\{P_{1},P_{2},\dotsc,P_{r}\}$ set of representatives for $E(\mathbb{Q})/nE(\mathbb{Q})$.  So for $X$ sufficiently larger than $h(P_{j})$.  Let $S=\{P_{1},P_{2},\dotsc,P_{r}\} \cup \{P: h(P) < X\}$.  Then, $S$ generates $E(\mathbb{Q})$.  Let $P$ be a point not in $\mathbb{Z}[S]$ and has minimum height.  Then, there exists $P-P_{j}=nQ$ and $h(Q) < h(P)$, where $Q \in \mathbb{Q}[S]$.

\section{Proof of the weak Mordell-Weil Theorem}

We will assume $n=2$ and $E[2]$ is defined over $\mathbb{Q}$.  So the elliptic curve is of the form $y^{2}=(x-a)(x-b)(x-c)$ with $a,b,c \in \mathbb{Q}$.  \textcolor{red}{(insert diagram 1)}  Given $P \in E(\mathbb{Q})$, choose $\tilde{P} \in E(\overline{\mathbb{Q}})$ such that $2\tilde{P}=P$.  Let $\zeta \in \gal(\overline{\mathbb{Q}}/\mathbb{Q})=:G_{\mathbb{Q}}$.  Let $c_{P}(\zeta)=\tilde{P}^{\zeta}-\tilde{P}$.  Some properties of $c_{P}$:
\begin{enumerate}[\upshape (1)]
\item $c_{P} \in \hom_{\text{continuous}} (G_{\mathbb{Q}}, E[2])$
\item $c_{P_{1}}=c_{P_{2}}$ if and only if $P_{1}-P_{2} \in 2E(\mathbb{Q})$ (somewhere $\delta:E(\mathbb{Q})/2E(\mathbb{Q}) \to^{injectively} \hom(G_{\mathbb{Q}}, E[2])$ defined by $P \mapsto c_{P}$)
\item Let $L=\mathbb{Q}(\sqrt{\ell})$, where $\ell|2(a-b)(b-c)(a-c)$.
\end{enumerate}
Then, $c_{P}$ factors through $\gal(L/\mathbb{Q})$.  So $P=(x,y)$ and $\tilde{P}$ is defined over $\mathbb{Q}(\sqrt{x-a},\sqrt{x-b},\sqrt{x-c}) \subset L$.  This implies $\delta: E(\mathbb{Q})/2E(\mathbb{Q}) \to^{injection} \hom(\gal(L/\mathbb{Q}),E[2])$.\\
\indent The highbrow version:  \textcolor{red}{(insert diagram 2)}\\
\indent Take $G_{\mathbb{Q}}$-invariants.  \textcolor{red}{(insert diagram 3-6)}

\section{Geometric interpretation of $\sel_{n}(E/\mathbb{Q})$}

Consider $H^{1}(G_{\mathbb{Q}},E[n])$ and $H^{1}(G_{\mathbb{Q}},\aut(X))$, which is the set of $\mathbb{Q}$-forms of $X$.  Then, $E \to^{n} E$ is the basic $n$-cover.  Hence, $\aut(E \to^{n} E)=E[n]$.
\begin{definition}
An \textbf{$n$-cover} of $E$ is a curve $C$ of genus $1$ equipped with a $\mathbb{Q}$-rational map $\tilde{n}:C \to E$ and a $\overline{\mathbb{Q}}$-isomorphism $C \cong E$ such that $C \to^{\varphi} E$ \textcolor{red}{(diagram 7)} commutes.
\end{definition}
Then $H^{1}(\mathbb{Q},E)$ is the set of curves of genus $1$ such that $\jac(C)=_{\mathbb{Q}}E$.  Then, we can map $H^{1}(\mathbb{Q},E[n]) \to H^{1}(\mathbb{Q},E)$, where $(C \to^{\tilde{n}} E) \mapsto C$.  Then, $\sel_{n}(E(\mathbb{Q}))$ is the set of $n$-covers of $E$ \quad $(C \to^{\tilde{n}} E)$ such that $C(\mathbb{Q}_{\ell}) \ne \varnothing$ for all $\ell$ and $C(\mathbb{R}) \ne \varnothing$.  Also, $\Sha(E/\mathbb{Q})$ is the set of $C$ of genus $1$ such that $\jac{C} \cong E$ with $C(\mathbb{Q}_{\ell}) \ne \varnothing$ for all $\ell$ and $C(\mathbb{R}) \ne \varnothing$.
\begin{theorem}[Swinnerton-Dyer]
If $C \to^{\tilde{2}} E$ is an element of $\sel_{2}(E/\mathbb{Q})$, then $C$ has a $\mathbb{Q}$-rational divisor of degree $2$.
\end{theorem}
\begin{proof}
Degree $2$ divisors of $C$ correspond to rational points on $\sym^{2}(C)=(C \times C)/2$.  Define a rational morphism $\sym^{2}(C) \to E$ defined by $\{P,Q\} \mapsto P+Q$.  Since $E=\jac(E)$, $P,Q \in C$ implies $P-Q \in E$.  Then, $2P=\tilde{2}P$.  Hence, $P+Q=\tilde{2}(P)+(Q-P)$.  Let $X=\varphi^{-1}(O)$.  What is $X$?  Over $\overline{\mathbb{Q}}$, $\varphi:\sym^{2}(E) \to E$ defined by $(P,Q) \mapsto P+Q$.  Then, $X(\mathbb{Q})=\{P,-P\}:P \in E(\overline{\mathbb{Q}})\}^{G_{\mathbb{Q}}}$ and $X=E/\langle P \mapsto -P\rangle=\mathbb{P}_{1}$.  The same reasoning, replacing $\overline{\mathbb{Q}}$ by $\mathbb{Q}_{\ell}$.  We obtain $C \cong_{\mathbb{Q}_{\ell}} E$.  So $X \cong_{\mathbb{Q}_{\ell}} \mathbb{P}_{1}$ for all $\ell$ and $\ell=\infty$.  so $X \cong_{\mathbb{Q}} \mathbb{P}_{1}$.  So Hasse-Minkowski is true.
\end{proof}
\begin{corollary}
If $C \to^{\tilde{2}} E$ is in $\sel_{2}(E/\mathbb{Q})$, then $C$ has an equation of the form $y^{2}=f(x)$, where $\deg f(x)=4$.
\end{corollary}
We can get a positive proportion of $E$ with $\sel_{n}(E/\mathbb{Q})=0$ or $\sel_{n}(E/\mathbb{Q}) \cong \mathbb{Z}/n\mathbb{Z}$ for $n=2,3,4,5$.
\begin{theorem}
\begin{enumerate}[\upshape (1)]
\item If $\sel_{n}(E/\mathbb{Q})=0$, ten $\rk E(\mathbb{Q})=0$.
\item If $\sel_{n}(E/\mathbb{Q}) \cong \mathbb{Z}/n\mathbb{Z}$, ten $\rk E(\mathbb{Q})=1$.
\end{enumerate}
\end{theorem}
(2) uses the connection between elliptic curves and $L$-functions (BSD).

\part{Rings and representations with one invariant}

\chapter{More algebraic groups, representation theory, and invariant theory \\ Eyal Goren}\label{ch:8}
\chapter{Cubic rings \\ Melanie Wood}\label{ch:9}
\chapter{Quartic and quintic rings \\ Melanie Wood}\label{ch:10}
\chapter{Problem session I \\ Jennifer Park, Arul Shankar and Frank Thorne}\label{ch:11}
\chapter{How to count rings and fields I \\ Manjul Bhargava}\label{ch:12}
\chapter{Rings associated to binary $n$-ic forms, composition of $2 \times n \times n$ boxes and class groups \\ Melanie Wood}\label{ch:13}
\chapter{The zeta functions attached to prehomogeneous vector spaces \\ Takashi Taniguchi}\label{ch:14}
\chapter{Problem session II \\ Arul Shankar and Frank Thorne}\label{ch:15}
\chapter{How to count rings and fields II \\ Manjul Bhargava}\label{ch:16}
\chapter{Heuristics for number field counts and applications to curves over finite fields \\ Melanie Wood}\label{ch:17}
\chapter{Moduli space of rings \\ Bjorn Poonen}\label{ch:18}
\chapter{Problem session III \\ Arul Shankar and Frank Thorne}\label{ch:19}
\chapter{Zeta function methods \\ Frank Thorne}\label{ch:20}
\chapter{Counting Artin representations and modular forms of weight one \\ Eknath Ghate}\label{ch:21}

\part{Curves and Representations with a free ring of invariants}

\chapter{Binary quartic forms:  bounded average rank of elliptic curves \\ Arul Shankar}\label{ch:22}
\chapter{Selmer groups and heuristics I \\ Bjorn Poonen}\label{ch:23}
\chapter{Rational points on curves \\ Michael Stoll}\label{ch:24}
\chapter{Problem session IV \\ Jennifer Park and Arul Shankar}\label{ch:25}
\chapter{Coregular spaces and genus one curves \\ Wei Ho}\label{ch:26}
\chapter{Arithmetic invariant theory and hyperelliptic curves I \\ Benedict Gross}\label{ch:27}
\chapter{TBA}\label{ch:28}
\chapter{Problem session V \\ Wei Ho and Jerry Wang}\label{ch:29}
\chapter{Applications to the Birch and Swinnerton-Dyer conjecture \\ Manjul Bhargava}\label{ch:30}
\chapter{Selmer groups and heuristics II \\ Bjorn Poonen}\label{ch:31}
\chapter{Arithmetic invariant theory and hyperelliptic curves II \\ Benedict Gross}\label{ch:32}
\chapter{Problem session VI \\ Jennifer Park and Jerry Wang}\label{ch:33}
\chapter{Chabauty methods and hyperelliptic curves \\ Bjorn Poonen}\label{ch:34}
\chapter{Topological and algebraic geometry methods over function fields I \\ Jordan Ellenberg}\label{ch:35}
\chapter{Counting methods over global fields \\ Jerry Wang}\label{ch:36}
\chapter{Problem session VII \\ Jordan Ellenberg, Jennifer Park and Jerry Wang}\label{ch:37}
\chapter{The Chabauty method and symmetric powers of curves \\ Jennifer Park}\label{ch:38}
\chapter{Topological and algebraic geometry methods over function fields II \\ Jordan Ellenberg}\label{ch:39}
\chapter{Future perspectives \\ Manjul Bhargava}\label{ch:39}

\clearpage
\printindex

\end{document}