\documentclass[12pt,amsfont]{amsart}
%\usepackage{amsaddr}
\usepackage{amssymb, fullpage, amsfonts, euscript, times, hyperref}
\usepackage{ulem}
\usepackage{color}\definecolor{Blue}{rgb}{0,0,1}
\usepackage{color}\definecolor{Red}{rgb}{1,0,0}
\usepackage{algorithm}
\usepackage{algpseudocode}
\usepackage{mathtools}
\usepackage{enumerate}
\usepackage{graphicx}
\usepackage{bbm}


\DeclarePairedDelimiter{\ceil}{\lceil}{\rceil}
\DeclarePairedDelimiter{\floor}{\lfloor}{\rfloor}
\usepackage[OT2,T1]{fontenc}

\DeclareSymbolFont{cyrletters}{OT2}{wncyr}{m}{n}
\DeclareMathSymbol{\Sha}{\mathalpha}{cyrletters}{"58}
\vfuzz=2pt
\begin{document}
\thispagestyle{empty}
\parindent=0pt
\parskip=4pt
\renewcommand{\labelenumi}{(\roman{enumi})}
\renewcommand{\theenumi}{(\roman{enumi})}


\newcommand{\legendre}[2]{\genfrac{(}{)}{}{}{#1}{#2}}
\newcommand{\li}{\mathop{\rm li}}
\newcommand{\lcm}{\operatorname{lcm}}
\renewcommand{\pmod}[1]{{\ifmmode\text{\rm\ (mod~$#1$)}\else\discretionary{}{}{\hbox{ }}\rm(mod~$#1$)\fi}}
\newcommand{\bS}{{\mathbb S}}
\newcommand{\C}{{\mathbb C}}
\newcommand{\N}{{\mathbb N}}
\newcommand{\R}{{\mathbb R}}
\newcommand{\PP}{\mathbb{P}}
\newcommand{\Z}{{\mathbb Z}}
\newcommand{\F}{{\mathbb F}}
\newcommand{\Q}{{\mathbb Q}}
\newcommand{\Ha}{{\mathbb H}}
\newcommand{\A}{\mathbb{A}}
\newcommand{\G}{\mathbb{G}}
\newcommand{\B}{\mathbb{B}}
\renewcommand{\mod}[1]{{\ifmmode\text{\rm\ (mod~$#1$)}\else\discretionary{}{}{\hbox{ }}\rm(mod~$#1$)\fi}}
\newcommand{\Tor}{{\operatorname{Tor}}}
\newcommand{\Ann}{\operatorname{Ann}}
\newcommand{\Sel}{\operatorname{Sel}}
\newcommand{\Hom}{\operatorname{Hom}}
\newcommand{\End}{\operatorname{End}}
\newcommand{\GL}{\operatorname{GL}}
\newcommand{\SL}{\operatorname{SL}}
\newcommand{\Avg}{\operatorname{Avg}}
\newcommand{\Sym}{\operatorname{Sym}}
\newcommand{\PGL}{\operatorname{PGL}}
\newcommand{\Jac}{\operatorname{Jac}}
\newcommand{\Aut}{\operatorname{Aut}}
\newcommand{\Cl}{\operatorname{Cl}}
\newcommand{\Div}{\operatorname{Div}}
\newcommand{\fp}{\mathfrak{p}}
\newcommand{\Pic}{\operatorname{Pic}}
\newcommand{\sgn}{\operatorname{sgn}}
\newcommand{\Alex}{\operatorname{Alex}}
\newcommand{\Stab}{\operatorname{Stab}}
\newcommand{\Gal}{\operatorname{Gal}}
\newcommand{\rank}{\operatorname{rank}}
\newcommand{\ord}{\operatorname{ord}}
\newcommand{\chr}{\operatorname{char}}
\newcommand{\Reg}{\operatorname{Reg}}
\newcommand{\Res}{\operatorname{Res}}
\newcommand{\Vol}{\operatorname{Vol}}
\newcommand{\OO}{\mathcal{O}}
\newcommand{\ol}{\overline}
\newcommand{\ul}{\underline}
\newcommand{\cF}{\mathcal{F}}
\newcommand{\cE}{\mathcal{E}}
\newcommand{\cL}{\mathcal{L}}
\newcommand{\fP}{\mathfrak{P}}
\newcommand{\fB}{\mathfrak{B}}
\newcommand{\cA}{\mathcal{A}}
\newcommand{\cR}{\mathcal{R}}



\numberwithin{equation}{section}%{subsection} %sets equation numbers <chapter>.<section>.<index>
\newtheorem{theorem}{Theorem}[section]
\newtheorem{lem}[theorem]{Lemma}
\newtheorem{pbm}[theorem]{Problem}
\newtheorem{pro}[theorem]{Proposition}
\newtheorem{cor}[theorem]{Corollary}
\newtheorem{cnj}[theorem]{Conjecture}
\newtheorem{dfn}[theorem]{Definition}
\newtheorem{thm}[theorem]{Theorem}
\newtheorem{rmk}[theorem]{Remark}
\newtheorem{xmp}[theorem]{Example}
\newtheorem{exe}[theorem]{Exercise}
\newenvironment{solution}
               {\let\oldqedsymbol=\qedsymbol
                \renewcommand{\qedsymbol}{$\blacktriangle$}
                \begin{proof}[\bf Solution]} 
               {\end{proof}
                \renewcommand{\qedsymbol}{\oldqedsymbol}}
                

\numberwithin{theorem}{section} %sets equation numbers <chapter>.<section>.<index>
\title{\bf Seminar Sujatha (I)\\``$p$-adic $L$-functions and Fontaine's Rings''\\ August--November, 2014}
\author{Vicky Siqi Wei \\ \href{mailto:vicky.wei@math.ubc.ca}{
{\texttt{\lowercase{vicky.wei@math.ubc.ca}}} \\ }\\ 
\and \\ \\ Justin Scarfy\\
\href{mailto:scarfy@ugrad.math.ubc.ca}{
{\texttt{\lowercase{scarfy@ugrad.math.ubc.ca}}} }
\\
\\
{\textnormal{\textit{D{\lowercase{epartment of }}M{\lowercase{athematics}}\\ T\lowercase{he} U\lowercase{niversity of} B\lowercase{ritish} C\lowercase{olumbia}\\ R\lowercase{oom} 121, 1984 M\lowercase{athematics} R\lowercase{oad}\\V\lowercase{ancouver}, B\lowercase{ritish} C\lowercase{olumbia}, C\lowercase{anada} V6T 1Z2}}}}
\begin{abstract}
This seminar covers the study of $p$-adic zeta functions in the case of the cyclotomic
$\Z_p$-extension of a number filed $F$. We begin with the Coleman power series associated to norm compatible elements and the Coates-Wiles homomorphism. We will then switch to $p$-adic Hodge theory and cover Fonatine's theory of $(\varphi, \Gamma)$-modules. We will then revisit the Coates-Wiles homomorphism from this set-up as well as the general machinery of $p$-adic $L$-functions.  We thank all the speakers for speaking at the seminar and for their feedback on preparing these notes, especially Sujatha for her expertise and for her encouragement for us to write this up.
\end{abstract}

\maketitle

\setcounter{tocdepth}{1}
\tableofcontents

\setcounter{section}{-1}
%%%0
\newpage
\section{Introduction and Overview\\ by Sujatha}\label{0}

\newpage
%%%1.
\section{\\ by Sujatha}\label{1}

%%%2
\newpage
\section{\\ by Sujatha}\label{2}

%%%3
\newpage
\section{\\ by Sujatha}\label{3}

%%%4
\newpage
\section{\\ by Sujatha}\label{4}

%%%5
\newpage
\section{Iwasawa algebra and p-adic measures (1/2)\\ by Zheng Li}\label{5}

%%%6
\newpage
\section{Iwasawa algebra and p-adic measures (2/2)\\ by Zheng Li}\label{6}

%%%7
\newpage
\section{\\ by Sujatha}\label{7}

%%%8
\newpage
\section{ \\ by Sujatha}\label{8}

%%%9
\newpage
\section{ \\ by Sujatha}\label{9}

%%%10
\newpage
\section{Basic theory of $(\varphi, \Gamma)$-modules (1/2)\\ by Shen-Ning Tung}\label{10}

%%%11
\newpage
\section{Basic theory of $(\varphi, \Gamma)$-modules (2/2)\\ by Shen-Ning Tung}\label{11}

%%%12
\newpage
\section{$B_{dR}$ and de Rham Galois representations (1/2)\\ by Miljan}\label{12}

%%%13
\newpage
\section{$B_{dR}$ and de Rham Galois representations (2/2)\\ by Miljan}\label{13}

%%%14
\newpage
\section{Formalisms of $p$-adic Hodge Theory
 (1/4)\\ by Zheng Li}\label{14}

%%%15
\newpage
\section{Formalisms of $p$-adic Hodge Theory
 (2/4)\\ by Zheng Li}\label{15}

%%%16
\newpage
\section{Formalisms of $p$-adic Hodge Theory
 (3/4)\\ by Zheng Li}\label{16}

%%%17
\newpage
\section{Introduction to Euler systems (1/5)
\\ by Miljan}\label{17}

%%%18
\newpage
\section{Formalisms of $p$-adic Hodge Theory
 (4/4)\\ by Zheng Li}\label{18}

%%%19
\newpage
\section{Introduction to Euler systems (2/5)
\\ by Miljan}\label{19}

%%%20
\newpage
\section{Further Developments in $p$-adic Hodge Theory (1/3)\\ by Shen-Ning Tung}\label{20}

%%%21
\newpage
\section{Introduction to Euler systems (3/5)
\\ by Miljan}\label{21}

%%%22
\newpage
\section{Further Developments in $p$-adic Hodge Theory (2/3)\\ by Shen-Ning Tung}\label{22}

%%%23
\newpage
\section{Introduction to Euler systems (4/5)
\\ by Miljan}\label{23}

%%%24
\newpage
\section{Further Developments in $p$-adic Hodge Theory (3/3)\\ by Shen-Ning Tung)}\label{24}

%%%25
\newpage
\section{Introduction to Euler systems (5/5)
\\ by Miljan}\label{25}

\end{document}