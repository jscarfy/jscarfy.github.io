\documentclass[12pt, amsfont]{amsart}
\usepackage{amsfonts, euscript, hyperref}
\usepackage{hyperref}
\usepackage{etaremune}
\usepackage[OT2,T1]{fontenc}
\DeclareSymbolFont{cyrletters}{OT2}{wncyr}{m}{n}
\DeclareMathSymbol{\Sha}{\mathalpha}{cyrletters}{"58}

\begin{document}
\begin{center}
{\bf UBC Masters in Mathematics: \\
Research Proposal}\\
{   Proposed by: Pu Justin Scarfy Yang}\\
Email: \href{mailto: scarfy@ugrad.math.ubc.ca}
{scarfy@ugrad.math.ubc.ca}\\ 
\textsc{url}: \href{http://www.ugrad.math.ubc.ca/~scarfy/}{http://www.ugrad.math.ubc.ca/~scarfy/}
\end{center}

{\ }\\
{\bf Background\\ }
\hspace{.25in}


Arithmetic enjoys a privileged position within mathematics as a fertile source of fundamental questions. Among the seven Millennium problems listed by the Clay Institute~\cite{Clay}, not fewer than three: the Birch and Swinnerton-Dyer conjecture, the Hodge conjecture, and the Riemann hypothesis, were handed down by the Queen of Mathematics. Even by the standards of a subject which has remained vibrant since the days of Fermat and Gau\ss, the last two decades have witnessed a real golden age, with landmarks too numerous to list completely: such as the striking progress on the Birch and Swinnerton-Dyer conjecture arising from the work of Gross-Zagier \cite{GZ1986}, Kolyvagin \cite{Ko1989}, and Kato \cite{Ka2004}; the proofs of the Shimura-Taniyama-Weil conjecture \cite{BCDT2001}, Serre's conjectures \cite{KW2009}, the Fontaine-Mazur conjecture for two-dimensional Galois representations \cite{Ki2009}, and the Sato-Tate conjectures \cite{CHR2008} which grew out of Wiles' epoch-making proof of Fermat's Last Theorem \cite{Wi1995},~\cite{TW1995}; the revolutionary ideas of Bourgain \cite{Bo2008} and Gowers \cite{Go2007} blending techniques in harmonic analysis and additive combinatorics, the Fields-medal winning breakthrough of Green and Tao on primes in arithmetic progressions \cite{GT2008}, and the work of Goldston, Pintz, and Y{\i}ld{\i}r{\i}m \cite{GPY2009},~\cite{GPY2010}, and its spectacular recent strengthenings by Zhang \cite{Zh2014}, and Maynard \cite{Ma2015} and Tao \cite{Poly2014}, on bounded gaps between primes. Recent innovations in arithmetic geometry by the innovation of Perfectoid spaces~\cite{Scho2012}, and subsequent topological realization of the absolute Galois group~\cite{KS2016} by Peter Scholze also shed new lights on the Langlands Programme, a web of conjectures that connect number theory, harmonic analysis, and geometry.

{\ }\\
{\bf Proposed Objective\\ }
\hspace{.25in}
A major task in mathematics today is to harmonize the continuous and the discrete, to include them in one comprehensive mathematics, and to eliminate obscurity from both.  The arithmetic properties of various interesting objects (e.g. number fields, varieties, or even a motive, i.e. discrete objects) are all encoded in their respective $L$-functions, i.e. continuous objects, of which the Riemann zeta function is the simplest example. The understanding of these $L$-functions undergoes three phases~\cite{Ka1993a}: 
\begin{enumerate}
\item
analytic properties and the rationality of values,
\item
algebraic properties illuminated by the $p$-adic properties of values,
\item
arithmetic-geometric point of view of interpreting the values. 
\end{enumerate}

To date much of these have been achieved for the Riemann zeta function~\cite{CRSS2015} (except the locations of the zeros, or the zero free region).  The deep conjectures of Deligne~\cite{De1979} and Beilinson~\cite{Be1985} were among the first to place these problems in a broad framework.  Ambiguity in the work of Beilinson for interpreting these values, as they are interpreted only up to nonzero rational number multiples.  In their article published in the Grothendieck Festschrift~\cite{BK1990}, Spencer Bloch and Kazuya Kato removed this ambiguity in their formulation of what is now called the Tamagawa number conjecture  (or the Bloch-Kato conjecture for motives).  They also proved that the conjecture is invariant under isogeny, and the book~\cite{CRSS2015} exploits this isogeny invariance condition in the optic of $K$-theory for the Riemann zeta function. I hope to investigate the isogeny invariance for specific motives and elaborates on accompanying results in cohomology and $K$-theory.  Techniques involved would require heavy analytic number theory which I had the privilege to learn much from Professor Greg Martin, algebraic number theory and algebraic geometry which I had the pleasure to absorbed much by attending many of Professor Sujatha Ramdorai's seminars and courses, as well as seeing the big picture which I enjoyed by attending various conferences and workshops \cite{cv}.

{\ }\\
\begin{thebibliography}{99} % don't worry about the 99
\bibitem[Clay]{Clay}
\href{http://www.claymath.org/millennium-problems}{The Millennium Prize Problems, by Clay Research Institute.}


\bibitem[Be1985]{Be1985}
A. Beilinson.
\newblock \href{run:bib/Be1985.pdf}{`` Higher regulators and values of $L$-functions''}.
\newblock J. Soviet Math. {\bf 30} (1985), 2036--2070.



\bibitem[BK1990]{BK1990}
S. Bloch and K. Kato.
\newblock { \href{run:bib/BK1990.pdf}{``$L$-Functions and Tamagawa Numbers of Motives''}}.
\newblock The Grothendieck Festschrift I, Modern Birkh\"{a}user Classics (1990), 333--400.

\bibitem[Bo2008]{Bo2008}
J.~Bourgain, ``Roth's theorem in progressions revisited'', {\em J. Anal. Math.} {\bf 104} (2008), 155--192.

\bibitem[BCDT2001]{BCDT2001}
C.~Breuil, B.~Conrad, F.~Diamond, and R.~Taylor, \href{run:bib/BCDT2011.pdf}{``On the Modularity of Elliptic Curves Over $\mathbb{Q}$: Wild $3$-Adic Exercises''}, {\em Journal of the American Mathematical Society} {\bf 14} (4) (2001): 843--939.

\bibitem[CHR2008]{CHR2008}
L.~Clozel, M.~Harris, and R.~Taylor, \href{run:bib/CHR2008.pdf}{``Automorphy for some $l$-adic lifts of automorphic mod l Galois representations''}, {\em Publ. Math. Inst. Hautes \'Etudes Sci.} {\bf 108} (2008), 1--181.

\bibitem[CRSS2015]{CRSS2015}
J. Coates, A. Raghuram, A. Saikia, and R. Sujatha.
\newblock {\em The Bloch-Kato Conjecture for the Riemann Zeta Function}. {\bf Cambridge University Press} (2015).

\bibitem[De1979]{De1979}
P. Deligne.
\href{run:bib/De1979.pdf}{``Valeurs de functions L et p\'{e}riods d'int\'{e}grales}.
\newblock Proc. Symp. Pure Math. {\bf 33}  {\bf AMS} (1979), 313--346.

\bibitem[GPY2009]{GPY2009}
D. A. Goldston, J. Pintz, and C. Y. Y{\i}ld{\i}r{\i}m, ``Primes in tuples. I'', {\em Ann. of
Math.} {\bf 170} (2009), 819--862.

\bibitem[GPY2010]{GPY2010}
D. A. Goldston, J. Pintz, and C. Y. Y{\i}ld{\i}r{\i}m, ``Primes in tuples. II'', {\em Acta
Math.} {\bf 204} (2010), 1--47.

\bibitem[Go2007]{Go2007}
T.~Gowers, ``Hypergraph regularity and the multidimensional Szemeredi theorem'', {\em Ann. of Math.} {\bf 166 }(3) (2007), 897--946.

\bibitem[GZ1986]{GZ1986}
B.~H.~Gross and D.~B.~Zaiger.
\newblock {`` Heegner points and derivatives of $L$-series''}.
\newblock Invent. math. {\bf 84} (1986), 225--320

\bibitem[GT2008]{GT2008}
B.~Green and T.~Tao, ``The primes contain arbitrarily long arithmetic progressions'', {\em Ann. of Math. }{\bf 167 }(2) (2008), 481--547.

\bibitem[Kat1993a]{Ka1993a}
K. Kato.
\newblock{``Lectures on the approach to Iwasawa theory for Hasse-Weil $L$- functions via $B_{dR}$, I''}. \newblock{Arithmetic Algebraic Geometry: Lectures given at the 2nd Session of the Centro Internazionale Matematico Estivo (C.I.M.E.) held in Trento, Italy, June 24--July 2, 1991, Springer Lecture Notes in Mathematics {\bf 1553} (1993), 50--163}

\bibitem[Kat1993b]{Ka1993b}
K. Kato.
\newblock{``Lectures on the approach to Iwasawa theory for Hasse-Weil $L$- functions via $B_{dR}$, II''}. \newblock{unpublished}

\bibitem[Kat2004]{Ka2004}
K.~Kato, ``$p$-adic Hodge theory and values of zeta functions of modular forms'', {\em Ast\'erisque} {\bf 295} (2004), ix, 117--290, Cohomologies $p$-adiques et applications arithm\'etiques. III. 

\bibitem[Kis2009]{Ki2009}
M.~Kisin, ``The Fontaine--Mazur conjecture for $\operatorname{GL}_2$'', {\em Journal of the American Mathematical Society} {\bf 22} (3) (2009), 641--690.

\bibitem[Kol1989]{Ko1989}
V.~A.~Kolyvagin, ``Finiteness of $E(\mathbb{Q})$ and $\Sha(E, \mathbb{Q})$ for a class of Weil curves". {\em Math. USSR Izv.} {\bf 32} (1989), 523--541.

\bibitem[KS2016]{KS2016}
R. A. Kucharczyk and P.~Scholze ``Topological realisations of absolute Galois groups'', arXiv:1609.04717.


\bibitem[KW2009]{KW2009}
C.~Khare and J.-P.~Wintenberger, ``On Serre's reciprocity conjecture for $2$-dimensional mod $p$ representations of $\operatorname{Gal}(\bar{\mathbb{Q}}/\mathbb{Q})$'', {\em Ann. of Math.} {\bf 169} (1) (2009), 229--253.

\bibitem[May2015]{Ma2015}
J.~Maynard, ``Small gaps between primes'', {\em Annals~of~Mathematics} {\bf 181} (2015), no.~1, 383--413.

\bibitem[Poly2014]{Poly2014}
DHJ~Polymath, ``Variants of the Selberg sieve, and bounded intervals containing many primes'', {\em  Research in the Mathematical Sciences} (2014), no.~1:12.


\bibitem[Scho2012]{Scho2012}
P. Scholze, ``Perfectoid spaces'', {\em  Publications math�matiques de l'IH�S} (2012), no.~1,  245--313.


\bibitem[TW1995]{TW1995}
R.~Taylor and A.~Wiles. ``Ring-theoretic properties of certain Hecke algebras''.  {\emph Annals of Mathematics} {\bf 141} (1995): 553--572.

\bibitem[Wil1995]{Wi1995}
A.~Wiles, ``Modular elliptic curves and Fermat's Last Theorem'', {\em Annals~of~Mathematics} {\bf 142 } (1995), 443--551.


\bibitem[Zha2014]{Zh2014}
Y.~Zhang, ``Bounded gaps between primes'', {\em Annals~of~Mathematics} {\bf 197} (2014), no.~3, 1121--1174.



\bibitem[CV]{cv}
\href{http://www.ugrad.math.ubc.ca/~scarfy/cv.html}{Justin Scafy's Curriculum Vitae}.  With the full list of professional activities I have involved in.
\end{thebibliography}

\end{document}