\documentclass[12pt,reqno]{amsart}
\usepackage{fullpage}
\usepackage{amsfonts}
\usepackage{amssymb}
\usepackage{times}
\usepackage{graphicx}
\vfuzz=2pt

% some "funny lines" referred to later:
\newtheorem{thm}{Theorem}[section]
\newtheorem{cor}[thm]{Corollary}
\newtheorem{lem}[thm]{Lemma}
\newtheorem{prop}[thm]{Proposition}
{ \theoremstyle{remark}\newtheorem*{remark}{Remark} }
\newcommand{\C}{\mathbb{C}}
\newcommand{\Z}{\mathbb{Z}}
\newcommand{\R}{\mathbb{R}}

\begin{document}

\title{Arithogeometic Mean}
\author{Justin Scarfy}
\email{scarfy@ugrad.math.ubc.ca}
\maketitle

Given any elliptic curve $E$ with Weistrau{\ss}  equation
\[
y^2=x(x+r)(x+s),
\] 
assume neither $r$ nor $r$ is real and nonnegative.\\

Let $L \subset \C$ be the lattice with period $\omega = \frac{dx}{2y}$, and let $\Phi: \C/L \stackrel{\cong}{\longrightarrow} E(\C)$ be isomorphism with $\Phi^{-1}(\omega)=dz$, where $z$ is parameter on $\C$.\\

Now write $L=\Z\gamma+\Z\delta$ where $\gamma =\int_0^\infty \omega \in \R$, we know that $\Phi[0, \gamma] \subseteq E(\R)$ with $\Phi(0)=\infty$, and so $\Phi(\gamma /2)=(0, 0)$.\\

Other points of order two are $(-r, 0)$ and $(-s, 0)$, so we assume $\Phi(\delta /2)=(-r, 0)$ and $\Phi(\delta /2+\gamma/2)=(-s, 0)$, \Big(OR $\Phi(\delta /2)=(-s, 0)$ and $\Phi(\delta /2+\gamma/2)=(-r, 0)$\Big).\\

Let $\Lambda$ be the lattice $\Z \gamma+\Z \delta \subseteq \C$. Invariant under complex conjugation (why?), and hence corresponds to a real elliptic curve. Find Weistrau{\ss} equation 
\[
v^2=u(u+R)(u+S)
\]
and an isomorphism $\Psi: \C/\Lambda \stackrel{\cong}{\longrightarrow} F(\C)$ (why $F$ here?). Constant $c$ in $\Psi^{-1}(\frac{du}{2v})=c\,dz$
\end{document}
