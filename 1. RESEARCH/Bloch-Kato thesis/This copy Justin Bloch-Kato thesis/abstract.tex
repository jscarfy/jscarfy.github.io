%% The following is a directive for TeXShop to indicate the main file
%%!TEX root = diss.tex

\chapter{Abstract}
\tab
A major task in mathematics today is to harmonize the continuous and the discrete, to include them in one comprehensive mathematics, and to eliminate obscurity from both.  The arithmetic properties of various interesting objects (e.g. number fields, varieties, or even a motive, i.e. discrete objects) are all encoded in their respective $L$-functions, i.e. continuous objects, of which the Riemann zeta function is the simplest example. The understanding of these $L$-functions undergoes three phases~\cite{Ka1993a}: 
\begin{enumerate}
\item
analytic properties and the rationality of values,
\item
algebraic properties illuminated by the $p$-adic properties of values,
\item
arithmetic-geometric point of view of interpreting the values. 
\end{enumerate}

To date much of these have been achieved for the Riemann zeta function~\cite{CRSS2015} (except the locations of the zeros, or the zero free region).  The deep conjectures of Deligne~\cite{De1979} and Beilinson~\cite{Be1985} were among the first to place these problems in a broad framework.  Ambiguity in the work of Beilinson for interpreting these values, as they are interpreted only up to nonzero rational number multiples.  In their article published in the Grothendieck Festschrift~\cite{BK1990}, Spencer Bloch and Kazuya Kato removed this ambiguity in their formulation of what is now called the Tamagawa number conjecture  (or the Bloch-Kato conjecture for motives).  They also proved that the conjecture is invariant under isogeny, and the book~\cite{CRSS2015} exploits this isogeny invariance condition in the optic of $K$-theory for the Riemann zeta function.

This thesis investigates the isogeny invariance for specific motives and elaborates on accompanying results in cohomology and $K$-theory.