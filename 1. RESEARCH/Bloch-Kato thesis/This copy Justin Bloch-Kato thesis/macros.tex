% This file provides examples of some useful macros for typesetting
% dissertations.  None of the macros defined here are necessary beyond
% for the template documentation, so feel free to change, remove, and add
% your own definitions.
%
% We recommend that you define macros to separate the semantics
% of the things you write from how they are presented.  For example,
% you'll see definitions below for a macro \file{}: by using
% \file{} consistently in the text, we can change how filenames
% are typeset simply by changing the definition of \file{} in
% this file.
% 
%% The following is a directive for TeXShop to indicate the main file
%%!TEX root = diss.tex

\newcommand{\NA}{\textsc{n/a}}	% for "not applicable"
\newcommand{\eg}{e.g.,\ }	% proper form of examples (\eg a, b, c)
\newcommand{\ie}{i.e.,\ }	% proper form for that is (\ie a, b, c)
\newcommand{\etal}{\emph{et al}}

% Some useful macros for typesetting terms.
\newcommand{\file}[1]{\texttt{#1}}
\newcommand{\class}[1]{\texttt{#1}}
\newcommand{\latexpackage}[1]{\href{http://www.ctan.org/macros/latex/contrib/#1}{\texttt{#1}}}
\newcommand{\latexmiscpackage}[1]{\href{http://www.ctan.org/macros/latex/contrib/misc/#1.sty}{\texttt{#1}}}
\newcommand{\env}[1]{\texttt{#1}}
\newcommand{\BibTeX}{Bib\TeX}

% Define a command \doi{} to typeset a digital object identifier (DOI).
% Note: if the following definition raise an error, then you likely
% have an ancient version of url.sty.  Either find a more recent version
% (3.1 or later work fine) and simply copy it into this directory,  or
% comment out the following two lines and uncomment the third.
\DeclareUrlCommand\DOI{}
\newcommand{\doi}[1]{\href{http://dx.doi.org/#1}{\DOI{doi:#1}}}
%\newcommand{\doi}[1]{\href{http://dx.doi.org/#1}{doi:#1}}

% Useful macro to reference an online document with a hyperlink
% as well with the URL explicitly listed in a footnote
% #1: the URL
% #2: the anchoring text
\newcommand{\webref}[2]{\href{#1}{#2}\footnote{\url{#1}}}

% epigraph is a nice environment for typesetting quotations
\makeatletter
\newenvironment{epigraph}{%
	\begin{flushright}
	\begin{minipage}{\columnwidth-0.75in}
	\begin{flushright}
	\@ifundefined{singlespacing}{}{\singlespacing}%
    }{
	\end{flushright}
	\end{minipage}
	\end{flushright}}
\makeatother

% \FIXME{} is a useful macro for noting things needing to be changed.
% The following definition will also output a warning to the console
\newcommand{\FIXME}[1]{\typeout{**FIXME** #1}\textbf{[FIXME: #1]}}

\crefname{figure}{Figure}{Figures}
%MATH ENVIRONMENTS
\newtheorem*{thm*}{Theorem}
\newtheorem{thm}{Theorem}[chapter]
\crefname{thm}{Theorem}{Theorems}
\newtheorem{cor}[thm]{Corollary}
\crefname{cor}{Corollary}{Corollarys}
\newtheorem*{cor*}{Corollary}
\crefname{cor*}{Corollary}{Corollarys}
\newtheorem{lem}[thm]{Lemma}
\crefname{lem}{Lemma}{Lemmas}
\newtheorem{prop}[thm]{Proposition}
\crefname{prop}{Proposition}{Propositions}
\newtheorem{conj}{Conjecture}
\crefname{conj}{Conjecture}{Conjectures}
\newtheorem*{conj*}{Conjecture}
\crefname{conj*}{Conjecture}{Conjectures}
\theoremstyle{definition}
\newtheorem{defn}[thm]{Definition}
\crefname{defn}{Definition}{Definitions}



\theoremstyle{remark}\newtheorem*{remark}{Remark}
\newtheorem*{rem*}{Remark}

\newcommand{\namedthm}[2]{\theoremstyle{plain}
\newtheorem*{thm#1}{#1}\begin{thm#1}#2\end{thm#1}}

%OTHER ENVIRONMENTS
\newenvironment{dedication}
    {\vspace{6ex}\begin{quotation}\begin{flushright}\begin{em}}
    {\par\end{em}\end{flushright}\end{quotation}  \pagebreak}
    
% Macro for 'List of Symbols', 'List of Notations' etc...
\newcommand{\listofsymbols}{%% The following is a directive for TeXShop to indicate the main file
%%!TEX root = diss.tex

\chapter*{List of Symbols\hfill} 
\addcontentsline{toc}{chapter}{\protect{}List of Symbols}

%mark your variables in your source code with \newnot{YOUR_SYMBOL_LABEL}
%%%%%%%%%%%%%%%%%%%%%%%
%Sample List of Symbols
%%%%%%%%%%%%%%%%%%%%%%%
\begin{tabbing}
% YOU NEED TO ADD THE FIRST ONE MANUALLY TO ADJUST THE TABBING AND SPACES

%%$\mathfrak{a}$~~~~~~~~~~~~~~\=\parbox{5.4in}{integral ideal associated to $\sigma \in \Pr^1(F)$, except in \cref{ch:PHecke}\dotfill \protect\pageref{sym:ka}}\\


%ADD THE REST OF SYMBOLS WITH THE HELP OF MACRO


%%\addsymbol A_v: {area measure of $F_v$}{sym:A_v}
%%\addsymbol \text{AQUE}: {Arithmetic Quantum Unique Ergodicity}{sym:AQUE}
%%
%%
%%\addsymbol B_v^{\pm}(s_v; y_v): {special function composed of Bessel functions}{sym:B_v}
%%\addsymbol B^{\bd\delta}(\bs; \by): {$\prod_{v \mid \infty} B_v^{\delta_v}(s_v; y_v)$ for $\bd\delta \in \{ \pm \}^m$}{sym:Bdelta}
%%
%%\addsymbol c_{\cP}: {large positive constant depending only on $\cP$}{sym:ManyPrimes}
%%\addsymbol c_{\sigma}(\phi; \alpha): {$\alpha$-Whittaker coefficient of $\phi$ at cusp $\sigma$}{sym:WhittCoeff}
%%\addsymbol c_{\sigma}^{(\xi)}(\phi; \n): {$\n$-Whittaker ideal coefficient of $\phi$ at cusp $\sigma$ for $\xi \in \O^{\times}/V$}{sym:WhittIdeal}
%%\addsymbol \C: {field of complex numbers}{sym:ComplexField}
%%
%%\addsymbol D_F: {absolute discriminant of $F$}{sym:D_F}
%%\addsymbol \cD: {fundamental domain for $\Lambda \backslash \X$}{sym:CuspStabFD}
%%\addsymbol \cD(y): {deep in a cusp, $\cD(y) = \{ (\bx, \by) \in \cD : \N \by \in (y, \infty)\}$}{sym:cD(y)}
%%\addsymbol \kD: {absolute different ideal of $F$}{sym:Different}
%%
%%\addsymbol e(x): {$e(x) = e^{2\pi i x}$}{sym:Exp}
%%
%%\addsymbol f: {mock $\cP$-Hecke multiplicative function}{sym:MockP}
%%\addsymbol f_{\cP}: {$\cP$-Hecke multiplicative function}{sym:PHecke}
%%\addsymbol F: {number field}{sym:NumberField}
%%\addsymbol F_v: {completion of $F$ at place $v$}{sym:LocalField}
%%\addsymbol F_{\infty}: {$F_{\infty} = \prod_{v \mid \infty} F_v$}{sym:F_infty}
%%\addsymbol \cF: {fundamental domain for $\Gamma \backslash \X$}{sym:GammaFD}
%%\addsymbol \cF(Y): {function associated to mock $\cP$-Hecke multiplicative function $f$}{sym:F(Y)}
%%
%%\addsymbol g_{\sigma}: {element of $\PGL(2,F)$ such that $g_{\sigma}(\infty) = \sigma$}{sym:g_sigma}
%%\addsymbol G: {$G = \prod_{v \mid \infty} G_v = \PGL(2,F_{\infty})$}{sym:G}
%%\addsymbol G_v: {$G_v = \PGL(2,F_v)$}{sym:G_v}
%%
%%\addsymbol h_F: {class number of $F$}{sym:h_F}
%%\addsymbol \kH^2: {hyperbolic $2$-space}{sym:H2}
%%\addsymbol \kH^3: {hyperbolic $3$-space}{sym:H3}
%%\addsymbol \HQ: {Hamilton's quaternions}{sym:HQ}
%%
%%
%%\addsymbol I_{\nu}(y): {$I$-Bessel function, $\nu \in \C, y \in \R_{>0}$}{sym:IBessel}
%%\addsymbol \Im: {imaginary part of $\kH^2, \kH^3$ or $\X$}{sym:Im}
%%
%%
%%\addsymbol J: {well chosen positive integer based on $\cF(Y)$}{sym:J}
%%
%%\addsymbol K: {$\prod_{v \mid \infty} K_v$, maximal compact subgroup of $G$}{sym:K}
%%\addsymbol K_v: {maximal compact subgroup of $G_v$ for $v \mid \infty$}{sym:K_v}
%%\addsymbol K_{\nu}(y): {$K$-Bessel function, $\nu \in \C, y \in \R_{>0}$}{sym:KBessel}
%%
%%
%%\addsymbol m: {number of archimedean places of $F$, i.e. $m=r_1+r_2$}{sym:m}
%%\addsymbol M: {congruence locally symmetric space, i.e. $M = \Gamma \backslash \X$}{sym:M}
%%
%%\addsymbol n: {degree of $F$, i.e. $n = [F:\Q]$}{sym:Degree}
%%\addsymbol \kN: {integral ideal of $F$ representing level}{sym:Level}
%%\addsymbol \N: {global norm, i.e. $\N = \N^F_\Q$, extended to $\X$}{sym:GlobalNorm}
%%\addsymbol \N_v: {local norm at $v \mid \infty$, i.e. $\N = \N^{F_v}_{\R}$}{sym:LocalNorm}
%%\addsymbol \cN: {$\cP$-friable ideals}{sym:PFriable}
%%\addsymbol \cN_{\kN}: {$\cP_{\kN}$-friable ideals}{sym:P_NFriable}
%%
%%\addsymbol \O: {ring of integers of $F$}{sym:RingOfIntegers}
%%\addsymbol \O^{\times}: {group of integral units of $F$}{sym:Units}
%%\addsymbol \O^{\times}_1: {roots of unity of $F$}{sym:RootsOfUnity}
%%
%%\addsymbol \p: {prime ideal of $F$, often principal}{sym:PrimeIdeal}
%%\addsymbol \cP: {subset of unramified prime ideals not dividing $\kN$}{sym:P}
%%\addsymbol \cP_{\kN}: {set of unramified \emph{principal} prime ideals not dividing $\kN$}{sym:P_N}
%%\addsymbol \cP(Y): {$\{ \p \in \cP : \N\p \in [ \sqrt{Y}/2, \sqrt{Y}] \}$}{sym:cPY}
%%\addsymbol \cP_j: {suitably chosen subset of $\cP(Y)$}{sym:P_j}
%%\addsymbol \Pr^1(F): {projective linear space of $F$}{sym:P1(F)}
%%
%%\addsymbol \Q: {field of rational numbers}{sym:Rationals}
%%\addsymbol \text{QUE}: {Quantum Unique Ergodicity}{sym:QUE}
%%
%%\addsymbol r_1: {number of real places of $F$}{sym:r_1}
%%\addsymbol r_2: {number of complex places of $F$}{sym:r_2}
%%\addsymbol  \R: {field of real numbers}{sym:RealNumbers}
%%
%%\addsymbol s_v: {$\lambda_v = s_v(1-s_v) \in \C$}{sym:s_v}
%%\addsymbol \bs: {$\bs = (s_v)_{v \mid \infty} \in \C^m$}{sym:Sbold}
%%\addsymbol \cS(y): {compact centre of $\Gamma \backslash \X$, grows as $y \rightarrow \infty$}{sym:CompactCentre}
%%
%%\addsymbol T(\n): {$\n$-Hecke operator (usually for $\n \in \cN_{\kN}$)}{sym:HeckeOp}
%%\addsymbol \cT(y): {$\by$-coordinate of $\cD(y)$, i.e. $\{ \by \in \R_{>0}^m : \widehat{\by} \in \cV, \N \by \in (y, \infty) \}$}{sym:cT(y)}
%%\addsymbol \Tr: {global trace, i.e. $\Tr = \Tr^F_{\Q}$, extended to $F_{\infty}$}{sym:GlobalTrace}
%%\addsymbol \Tr_v: {local trace for $v \mid \infty$, i.e. $\Tr_v = \Tr^{F_v}_{\R}$}{sym:LocalTrace}
%%
%%\addsymbol \cU: {fundamental domain for $(\O^{\times}_1 \cap V) \backslash F_{\infty} / \kN$}{sym:cU}
%%
%%\addsymbol v \mid \infty: {archimedean place of $F$}{sym:Place}
%%\addsymbol V: {$V = (1+\kN) \cap \O^{\times}$}{sym:V}
%%\addsymbol \cV: {fundamental domain for $V \backslash \widehat{\cY}$}{sym:cV}
%%\addsymbol \vol(\bz): {volume measure for $\bz \in \X$}{sym:vol}
%%\addsymbol V_1(\bx): {volume measure for $\bx$-coordinate in $\X$}{sym:V_1}
%%\addsymbol V_2(\by): {volume measure for $\by$-coordinate in $\X$}{sym:V_2}
%%
%%\addsymbol W(\bs; \bz): {full Whittaker function}{sym:FullWhittaker}
%%\addsymbol W_v(s_v; z_v): {local Whittaker function at $v \mid \infty$}{sym:LocalWhittaker}
%%
%%\addsymbol \bx: {$\bx = (x_v)_{v \mid \infty} \in F_{\infty}$, coordinate of $\bz = (\bx, \by) \in \X$}{sym:bx}
%%\addsymbol \X: {symmetric space of $G$, equals $(\kH^2)^{r_1} \times (\kH^3)^{r_2}$}{sym:X_F}
%%
%%\addsymbol y_0: {(small) positive constant depending only on $F$}{sym:y_0}
%%\addsymbol \by: {$\by = (y_v)_{v \mid \infty} \in \R_{>0}^m$, coordinate of $\bz = (\bx, \by) \in \X$}{sym:by}
%%\addsymbol \widehat{\by}: {$\widehat{\by} = (y_v/(\N \by)^{1/n})_{v \mid \infty} \in \widehat{\cY}$}{sym:byhat}
%%\addsymbol Y_0: {(large) positive constant depending only on $F$}{sym:Y_0}
%%\addsymbol \widehat{\cY}: { $\{ \by \in \R_{>0}^m : \N \by = 1\}$, hyperplane in $\R_{>0}^m$ }{sym:cYhat}
%%
%%\addsymbol \bz: {element of $\X$, alternate coordinates $\bz = (\bx, \by)$}{sym:bz}
%%\addsymbol \Z: {ring of rational integers}{sym:Integers}
%%
%%% GREEK SYMBOLS
%%
%%\addsymbol \Delta_v: {Laplacian at place $v \mid \infty$}{sym:Delta}
%%
%%\addsymbol \phi(\bz):  {Hecke-Maa\ss\, cusp form on $\Gamma \backslash \X$}{sym:Phi}
%%\addsymbol \phi_{\sigma}(\bz): {$\phi_{\sigma}(\bz) = \phi(g_{\sigma}\bz)$}{sym:PhiSigma}
%%\addsymbol \widehat{\phi_{\sigma}}(\by; \alpha): {$\alpha$-Fourier coefficient of $\phi$ at cusp $\sigma$}{sym:FourierCoeff}
%%
%%\addsymbol \Gamma: {congruence group of level $\kN$, often $\Gamma = \Gamma_0(\kN)$}{sym:Gamma}
%%\addsymbol \Gamma(\kN): {principal congruence subgroup of level $\kN$}{sym:Gamma(N)}
%%\addsymbol \Gamma_0(\kN): {specific congruence group of level $\kN$}{sym:Gamma_0(N)}
%%\addsymbol \Gamma^{(\sigma)}: {$\Gamma^{(\sigma)} = g_{\sigma}^{-1} \Gamma g_{\sigma}$}{sym:GammaSigma}
%%\addsymbol \Gamma_F: {$\Gamma_F = \PGL(2,\O)$}{sym:Gamma_F}
%%
%%\addsymbol \lambda_v: {$\Delta_v$-eigenvalue of Maa\ss\, form $\phi$}{sym:lambda_v}
%%\addsymbol \bd\lambda: {$\bd\lambda = (\lambda_v)_{v \mid \infty} \in \C^m$}{sym:lambda}
%%\addsymbol \lambda_{\phi}(\n): {$T(\n)$-Hecke eigenvalue of $\phi$}{sym:HeckeEig}
%%\addsymbol \Lambda: {subgroup of $\Gamma^{(\sigma)}$-stabilizer of $\infty \in \Pr^1(F)$ for any $\sigma \in \Pr^1(F)$}{sym:Lambda}
%%
%%\addsymbol \mu_{\phi}: {measure associated to a Hecke-Maa\ss\, cusp form $\phi$}{sym:mu_phi}
%%
%%\addsymbol \Omega: {set of inequivalent representatives of $\Gamma \backslash \Pr^1(F)$}{sym:Omega}
%%
%%\addsymbol \rho(\sigma; \bz): {function measuring closeness of $\bz \in \X$ to $\sigma \in \Pr^1(F)$}{sym:rho}
%%
%%\addsymbol \sigma: {element of $\Pr^1(F)$ regarded as cusp of $\Gamma$, often $\sigma = (\alpha:\beta)$}{sym:sigma}
%%
%%\addsymbolEND \langle\cdot,\cdot\rangle: {trace form billinear pairing on $F_{\infty} \times F_{\infty}$}{sym:Pairing}
% .
% .
% .
% ALWAYS KEEP THE FOLLOWING LINE
\end{tabbing}
%\pagebreak
}
\def\addsymbol #1: #2#3{$#1$ \> \parbox{5.4in}{#2 \dotfill \pageref{#3}}\\} 
\def\addsymbolEND #1: #2#3{$#1$ \> \parbox{5.4in}{#2 \dotfill \pageref{#3}}} 
\newcommand{\sym}[1]{\refstepcounter{notation} \label{#1}} 

%MATH FORMAT
\newcommand{\ds}{\displaystyle}
\newcommand{\fk}[1]{\mathfrak{#1}}
\newcommand{\red}[1]{\textcolor{red}{#1}}
\newcommand{\tab}{\vspace{-0.2in} $ $}
\newcommand{\bd}{\boldsymbol}

%MATH SYMBOLS
\newcommand{\1}{ {\bf 1}}

\newcommand{\ka}{\mathfrak{a}}
\newcommand{\A}{\mathbb{A}}

\newcommand{\B}{\mathcal{B}}
\newcommand{\Lbrace}{\left\lbrace}
\newcommand{\Rbrace}{\right\rbrace}

\newcommand{\C}{\mathbb{C}}

\newcommand{\cD}{\mathcal{D}}
\newcommand{\kD}{\mathfrak{D}}
\newcommand{\F}{\mathcal{F}}
\newcommand{\cF}{\mathcal{F}}

\newcommand{\G}{\Gamma}
\newcommand{\GL}{ {\rm GL}}

\renewcommand{\H}{\mathbb{H}}
\newcommand{\cH}{\mathcal{H}}
\newcommand{\kH}{\mathbb{H}}
\newcommand{\HQ}{\mathcal{H}}

\renewcommand{\Im}{\text{Im}}
\newcommand{\m}{\mathfrak{m}}

\newcommand{\n}{\mathfrak{n}}
\newcommand{\N}{\mathbb{N}}
\newcommand{\cN}{\mathcal{N}}
\newcommand{\kN}{\mathfrak{N}}

\newcommand{\p}{\mathfrak{p}}
\newcommand{\cP}{\mathcal{P}}
\renewcommand{\Pr}{\mathbb{P}}
\newcommand{\PGL}{ {\rm PGL}}
\newcommand{\PO}{ {\rm PO}}
\newcommand{\PSL}{ {\rm PSL}}
\newcommand{\PU}{ {\rm PU}}

\renewcommand{\O}{\mathcal{O}}

\newcommand{\q}{\mathfrak{q}}
\newcommand{\Q}{\mathbb{Q}}

\newcommand{\R}{\mathbb{R}}
\renewcommand{\Re}{\text{Re}}

\newcommand{\bs}{\mathbf{s}}
\newcommand{\stab}{{\rm stab}}
\newcommand{\cS}{\mathcal{S}}
\newcommand{\SL}{ {\rm SL}}
\newcommand{\SO}{ {\rm SO}}
\newcommand{\SU}{ {\rm SU}}

\newcommand{\bt}{\mathbf{t}}
\newcommand{\cT}{\mathcal{T}}
\newcommand{\Tr}{ \textbf{Tr}}

\newcommand{\cU}{\mathcal{U}}

\newcommand{\cV}{\mathcal{V}}
\newcommand{\vol}{\text{\rm vol}}

\newcommand{\cW}{\mathcal{W}}
\newcommand{\wkarrow}{\overset{\text{wk-}\ast}{\longrightarrow}}

\newcommand{\bx}{\mathbf{x}}
\newcommand{\X}{\mathcal{X}}

\newcommand{\by}{\mathbf{y}}
\newcommand{\cY}{\mathcal{Y}}

\newcommand{\bz}{\mathbf{z}}
\newcommand{\Z}{\mathbb{Z}}
% END
