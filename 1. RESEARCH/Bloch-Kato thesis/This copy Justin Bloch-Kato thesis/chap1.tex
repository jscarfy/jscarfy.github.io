%% The following is a directive for TeXShop to indicate the main file
%%!TEX root = diss.tex

\chapter{Bloch-Kato's Tamagawa Numbers conjecture and the isogeny invariance}
\label{chap1}

\tab

\section{From the Birch and Swinnerton-Dyer conjecture to the Bloch-Kato conjecture}
The Bloch-Kato's Tamagawa Numbers conjecture~\cite{BK1990}, can be seen as a generalization of the Birch and Swinnerton-Dyer conjecture for motives.

Need: Fontaine's topological period rings $B_{dR}$ and $B_{crys}$, where the former is a complete valued field with residue field $\C_p$

For a motivic pair $(V, D)$ with weights $\le w$ and a finite set of places $\Omega$ of $\Q$ containing $\infty$, the $L$-function $L_\Omega(V, s)$ is defined to be as the Euler product
\[
L_{\Omega}(V, s):=\prod_{p\not\in \Omega}P_p(V, p^{-s})^{-1},
\]
it is absolute convergent for $\operatorname{Re}(s)>\tfrac{w}{2}+1$.

Fixing a $\Z$-lattice $M\subset V$ 


\section{Motives, Fontaine's $p$-adic period rings, and other essential gadgets}
To define Tamagawa measures one needs groups $A(\Q_P), p \le \infty$, and $A(\Q)$. Bloch and Kato define such groups for a motivic pair $(V, D)$ as follows:
Assuming the motivic pair $(V, D)$ has weight $\le -1$


Fontaine's $p$-adic period rings~\cite{Fon1982},~\cite{FM1987},

Tamagawa number~\cite{Wei1982}

\newpage
\section{The isogeny invariance explained}

\newpage
\section{$K$-theoretic background}
Reference~\cite{HB-K1},~\cite{HB-K2},~\cite{Sri1993},~\cite{TT1990},~\cite{Wei2013},~\cite{Wal1987a} ,~\cite{Wal1987b}

We will need to use higher $K$-theory, where $K_0$ was invented by Grothendieck in proving the Riemann-Roch Theorem,

Quillen higher $K$-theory

the $+$ construction

the $Q$ construction for schemes

Waldhausen $K$-theory

$K$-theory of Thomason \& Trobaugh.
%%%%%%%%%%%%%%%%%%%%%%%%%%%%%%%%%%%%%%%%%%%%%%%%%%%%%%%%%%%%%%%%%%%%%%
\endinput

Any text after an \endinput is ignored.
You could put scraps here or things in progress.
