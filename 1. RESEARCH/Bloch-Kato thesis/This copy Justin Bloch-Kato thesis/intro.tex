%% The following is a directive for TeXShop to indicate the main file
%%!TEX root = diss.tex
\chapter*{Introduction}
\addcontentsline{toc}{chapter}{\protect{}Introduction}
\label{ch:Introduction}

\section{The leading role played by arithmetic within mathematics and recent breakthroughs}
$ $ 

Arithmetic enjoys a privileged position within mathematics as a fertile source of fundamental questions. Among the seven Millennium problems listed by the Clay Institute~\cite{Clay}, not fewer than three: the Birch and Swinnerton-Dyer conjecture, the Hodge conjecture, and the Riemann hypothesis, were handed down by the Queen of Mathematics. Even by the standards of a subject which has remained vibrant since the days of Fermat and Gau\ss, the last two decades have witnessed a real golden age, with landmarks too numerous to list completely: such as the striking progress on the Birch and Swinnerton-Dyer conjecture arising from the work of Gross-Zagier \cite{GZ1986}, Kolyvagin \cite{Ko1989}, and Kato \cite{Ka2004}; the proofs of the Shimura-Taniyama-Weil conjecture \cite{BCDT2001}, Serre's conjectures \cite{KW2009}, the Fontaine-Mazur conjecture for two-dimensional Galois representations \cite{Ki2009}, and the Sato-Tate conjectures \cite{CHR2008} which grew out of Wiles' epoch-making proof of Fermat's Last Theorem \cite{Wi1995},~\cite{TW1995}; the revolutionary ideas of Bourgain \cite{Bo2008} and Gowers \cite{Go2007} blending techniques in harmonic analysis and additive combinatorics, the Fields-medal winning breakthrough of Green and Tao on primes in arithmetic progressions \cite{GT2008}, and the work of Goldston, Pintz, and Y{\i}ld{\i}r{\i}m \cite{GPY2009},~\cite{GPY2010}, and its spectacular recent strengthenings by Zhang \cite{Zh2014}, and Maynard \cite{Ma2015} and Tao \cite{Poly2014}, on bounded gaps between primes. Recent innovations in arithmetic geometry by the innovation of Perfectoid spaces~\cite{Scho2012}, and subsequent topological realization of the absolute Galois group~\cite{Scho2016} by Peter Scholze also shed new lights on the Langlands Programme, a web of conjectures that connects number theory, harmonic analysis, and geometry.

\newpage
\section{On the Riemann Hypothesis}
$ $ 

Little essential progress has been made to the Riemann Hypothesis in the past two decades for the Riemann zeta function (or the Generalized Riemann Hypothesis for automorphic $L$-functions), which predicts that all the zeros of such $L$-functions are critical: They lie on the real line $\operatorname{Re}s=\tfrac{1}{2}$ with multiplicity one.  The sharpest result for the Riemann zeta function is due to Feng~\cite{Fe2012}, proving that at least $41.28\%$ of the zeros of the Riemann zeta function,
\[
\zeta(s):=\sum_{n\ge 1} \frac{1}{n^s}\qquad (s:=\sigma+it),
\]
lie on the critical line, by introducing a new mollifier and applying the original method of Levinson~\cite{Le1974} and its subsequent strengthening by Conrey~\cite{Co1989}, which gave at least $34.20\%$ and $40.88\%$ of the zeros of $\zeta(s)$ are critical, respectively. The best zero-free region to date for the Riemann zeta function is obtained by Ford~\cite{Fo2002} with
\[
\sigma \ge 1-\frac{1}{57.54(\log|t|)^{2/3}(\log\log|t|)^{1/3}},\qquad |t|\ge 3.
\] 

A better zero-free region would imply a stronger result in the error term of the prime number theorem, 

Other industries including computing the higher moments of these automorphic $L$-functions, 

the attempt of using random matrices to explain the spacing between the zeros  

Montgomery pair-correlation conjecture

the Siegel-Walfisz theorem and the large sieve developed by Bombieri are often used in place of the Generalized Riemann Hypothesis for Dirichlet $L$-functions to prove theorems and do estimates.

\newpage
\section{On the Hodge conjecture}
Algebraic cycles

Chow groups

K\"ahler manifold

Known cases:

Generalizations:

Main difficulties/ best result in hoped cases.


\newpage
\section{On the Birch and Swinnerton-Dyer conjecture}
$ $
Recall the Dirichlet class number formula for a number field $K$~\cite{Neu1999},
\[
\operatorname{Res}_{s=1} \zeta_K (s)=\frac{2^{r_1}(2\pi)^{r_2}}{w|d_K|^{1/2}}hR,
\]
where the left hand side is the Dedekind zeta function,
\[
\zeta_K(s):=\sum_{\mathfrak{a}\subset \mathcal{O}_K}\frac{1}{\mathfrak{N}(\mathfrak{a})},  
\]
with $\mathfrak{a}$ {ideals in}\;$\mathcal{O}_K$, \textrm{the ring of integers of} $K$, and $\mathfrak{N}$ its norm.  On the right hand side, $r_1$ and $r_2$ the numbers of real and complex embeddings of $K$, respectively.  $h$ denotes the class number of $\mathcal{O}_K$, which measures the failure of $\mathcal{O}_K$ being a unique factorization domain, $R$ the regulator, the determinant $w$ the root number, and $d_K$ the discriminate of the number field $K$
{\ }\\

The Birch and Swinnerton-Dyer conjecture (BSD), formulated by B. Birch and Swinnerton-Dyer (1965) when studying the asymptotics of
\[
\prod_{p\le x}\frac{\# E(\F_p)}{p},
\] 

has seen tremendous progress since the first breakthrough of Coates and Wiles.


The BSD conjecture 
\begin{enumerate}[\bf (a)]
\item
(Weak BSD)
\[
\operatorname{ord}_{s=1} L(E/K, s)=r_K,
\]
\item
(Strong BSD)
\[
\lim_{s\to 1}\frac{L(E/K, s)}{(s-1)^{r_p}}=\Omega_{E/K}\times \operatorname{Reg}_{\infty, K}(E) \times \frac{|\Sha_K(E)| \prod_{p\le \infty}[E(K_p): E_0(K_p)]}{\sqrt{\Delta_K}\times |E(K)|_{\operatorname{tors}}^2},
\]
can be seen as a generalization of Dirichlet class number formula, in the sense
$\Sha(E)$ measures the failure of 
\end{enumerate}

Note Iwasawa main conjecture, proved in cases
\begin{enumerate}[1.]
\item
$\Q$~\cite{MW1984}
\item
totally real number fields~\cite{Wil1990}
\item
imaginary quadratic fields~\cite{Rub1988}~\cite{Rub1991}
\item
Dirichlet characters~\cite{HK2003}
\item
CM elliptic curves at supersingular primes~\cite{PR2004}
\item
for elliptic curves over anticyclotomic $\Z_p$-extensions~\cite{BD2005}.
\item
non-commutative main conjecture for totally real $p$-adic Lie extension of a number field~\cite{Kak2013}~\cite{CSSV2013}.
\item
(automorphic) $\GL_2$~\cite{SU2014}
\end{enumerate}


One promising view

algebraic $K$-theory,

Suslin and Voevodsky,

Bloch

Motivic cohomology
{\ }\\

Euler systems~\cite{Rub2000}
{\ }\\

Classical Euler systems:

1. Siegels' cyclotomic units gives Kubota-Leopoldt $p$-adic $L$ function, but no BSD application~\cite{CS2006}.

2. Elliptic units Coates and Wiles' homomorphism

3. Heegner points gives anticyclotomic $p$-adic $L$-function of
{\ }\\

Kato's Euler systems:

1. Beilinson-Kato elements

2. Beilinson-Flach elements

3. Gross-Kudla-Schoen cycles 

Note that the monograph~\cite{Del2008} is devoted to the study of the BSD conjecture over the universal deformation rings of an elliptic curve.

Known cases of BSD
\begin{enumerate}
\item
Coates and Wiles (1977) proved that if $E$ is a curve over a number field $F$ with complex multiplication by an imaginary quadratic field $K$ of class number $1$, $F = K$ or $\Q$, and $L(E, 1)$ is not $0$ then $E(F)$ is a finite group. This was extended to the case where $F$ is any finite abelian extension of $K$ by Arthaud (1978).
\item
Gross and Zagier (1986) showed that if a modular elliptic curve has a $L'(E, 1)=0$  then it has a rational point of infinite order; see Gross-Zagier theorem.
\item
Kolyvagin (1989) showed that a modular elliptic curve $E$ for which $L(E, 1)$ is not zero has rank $0$, and a modular elliptic curve $E$ for which $L(E, 1)$ has a first-order zero at $s = 1$ has rank $1$.
\item
Rubin (1991) showed that for elliptic curves defined over an imaginary quadratic field $K$ with complex multiplication by $K$, if the $L$-series of the elliptic curve was not zero at $s = 1$, then the $p$-part of the Tate-Shafarevich group had the order predicted by the Birch and Swinnerton-Dyer conjecture, for all primes $p > 7$.
\item
Breuil et al. (2001), extending work of Wiles (1995), proved that all elliptic curves defined over the rational numbers are modular, which extends results $2$ and $3$ to all elliptic curves over the rationals, and shows that the $L$-functions of all elliptic curves over $\Q$ are defined at $s = 1$.
\item
Bhargava and Shankar (2015) proved that the average rank of the Mordell-Weil group of an elliptic curve over $\Q$ is bounded above by $7/6$. Combining this with the $p$-parity theorem of Nekov\'ar (2009) and Dokchitser \& Dokchitser (2010) and with the proof of the main conjecture of Iwasawa theory for $\GL(2)$ by Skinner \& Urban (2014), they conclude that a positive proportion of elliptic curves over $\Q$ have analytic rank zero, and hence, by Kolyvagin (1989), satisfy the Birch and Swinnerton-Dyer conjecture.
\end{enumerate}
%\begin{figure}
%\centering
%\includegraphics[ width=1\textwidth]{sl2z.png}
%\caption{Fundamental domain for $\SL(2,\Z)$}
%\label{fig:SL2Z} % label should change 
%\end{figure}



\endinput

Any text after an \endinput is ignored.
You could put scraps here or things in progress.
