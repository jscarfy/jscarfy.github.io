%%%%%%%%%%%%%%%%%%%%%%%%%%%%%%%%%%%%%%%%%%%%%%%%%%%%%%%%%%%%%%%%%%%%%%
% Template for a UBC-compliant dissertation
% At the minimum, you will need to change the information found
% after the "Document meta-data"
%
%!TEX TS-program = pdflatex
%!TEX encoding = UTF-8 Unicode

%% The ubcdiss class provides several options:
%%   fogscopy
%%       set parameters to exactly how FoGS specifies
%%         * single-sided
%%         * page-numbering starts from title page
%%         * the lists of figures and tables have each entry prefixed
%%           with 'Figure' or 'Table'
%%       This can be tested by `\iffogscopy ... \else ... \fi'
%%   10pt, 11pt, 12pt
%%       set default font size
%%   oneside, twoside
%%       whether to format for single-sided or double-sided printing
%%   balanced
%%       when double-sided, ensure page content is centred
%%       rather than slightly offset (the default)
%%   singlespacing, onehalfspacing, doublespacing
%%       set default inter-line text spacing; the ubcdiss class
%%       provides \textspacing to revert to this configured spacing
%%   draft
%%       disable more intensive processing, such as including
%%       graphics, etc.
%%

% For submission to FoGS
\documentclass[fogscopy,onehalfspacing,12pt,letterpaper]{ubcdiss}

%\documentclass[msc,oneside]{ubcthesis}
\usepackage{amssymb, amsmath, amsthm, verbatim, url}

% For your own copies (looks nicer)
% \documentclass[balanced,twoside,11pt]{ubcdiss}
\usepackage{anysize}
\marginsize{1.25in}{1in}{1in}{1in}

%%
%%%%%%%%%%%%%%%%%%%%%%%%%%%%%%%%%%%%%%%%%%%%%%%%%%%%%%%%%%%%%%%%%%%%%%
%%%%%%%%%%%%%%%%%%%%%%%%%%%%%%%%%%%%%%%%%%%%%%%%%%%%%%%%%%%%%%%%%%%%%%
%%
%% FONTS:
%% 
%% The defaults below configures Times Roman for the serif font,
%% Helvetica for the sans serif font, and Courier for the
%% typewriter-style font.  Configuring fonts can be time
%% consuming; we recommend skipping to END FONTS!
%% 
%% If you're feeling brave, have lots of time, and wish to use one
%% your platform's native fonts, see the commented out bits below for
%% XeTeX/XeLaTeX.  This is not for the faint at heart. 
%% (And shouldn't you be writing? :-)
%%

%% NFSS font specification (New Font Selection Scheme)
%\usepackage{mathptmx,courier}
%\usepackage[scaled=.92]{helvet}
\usepackage{mathrsfs}

%% Math or theory people may want to include the handy AMS macros
\usepackage{amssymb}
\usepackage{amsmath}
\usepackage{amsfonts}
\usepackage{amsthm}
\usepackage[OT2,T1]{fontenc}
\DeclareSymbolFont{cyrletters}{OT2}{wncyr}{m}{n}
\DeclareMathSymbol{\Sha}{\mathalpha}{cyrletters}{"58}

%% The pifont package provides access to the elements in the dingbat font.   
%% Use \ding{##} for a particular dingbat (see p7 of psnfss2e.pdf)
%%   Useful:
%%     51,52 different forms of a checkmark
%%     54,55,56 different forms of a cross (saltyre)
%%     172-181 are 1-10 in open circle (serif)
%%     182-191 are 1-10 black circle (serif)
%%     192-201 are 1-10 in open circle (sans serif)
%%     202-211 are 1-10 in black circle (sans serif)
%% \begin{dinglist}{##}\item... or dingautolist (which auto-increments)
%% to create a bullet list with the provided character.
\usepackage{pifont}

%%%%%%%%%%%%%%%%%%%%%%%%%%%%%%%%%%%%%%%%%%%%%%%%%%%%%%%%%%%%%%%%%%%%%%
%% Configure fonts for XeTeX / XeLaTeX using the fontspec package.
%% Be sure to check out the fontspec documentation.
%\usepackage{fontspec,xltxtra,xunicode}	% required
%\defaultfontfeatures{Mapping=tex-text}	% recommended
%% Minion Pro and Myriad Pro are shipped with some versions of
%% Adobe Reader.  Adobe representatives have commented that these
%% fonts can be used outside of Adobe Reader.
%\setromanfont[Numbers=OldStyle]{Minion Pro}
%\setsansfont[Numbers=OldStyle,Scale=MatchLowercase]{Myriad Pro}
%\setmonofont[Scale=MatchLowercase]{Andale Mono}

%% Other alternatives:
%\setromanfont[Mapping=tex-text]{Adobe Caslon}
%\setsansfont[Scale=MatchLowercase]{Gill Sans}
%\setsansfont[Scale=MatchLowercase,Mapping=tex-text]{Futura}
%\setmonofont[Scale=MatchLowercase]{Andale Mono}
%\newfontfamily{\SYM}[Scale=0.9]{Zapf Dingbats}
%% END FONTS
%%%%%%%%%%%%%%%%%%%%%%%%%%%%%%%%%%%%%%%%%%%%%%%%%%%%%%%%%%%%%%%%%%%%%%
%%%%%%%%%%%%%%%%%%%%%%%%%%%%%%%%%%%%%%%%%%%%%%%%%%%%%%%%%%%%%%%%%%%%%%



%%%%%%%%%%%%%%%%%%%%%%%%%%%%%%%%%%%%%%%%%%%%%%%%%%%%%%%%%%%%%%%%%%%%%%
%%%%%%%%%%%%%%%%%%%%%%%%%%%%%%%%%%%%%%%%%%%%%%%%%%%%%%%%%%%%%%%%%%%%%%
%%
%% Recommended packages
%%
\usepackage{checkend}	% better error messages on left-open environments
\usepackage{graphicx}	% for incorporating external images

%% booktabs: provides some special commands for typesetting tables as used
%% in excellent journals.  Ignore the examples in the Lamport book!
\usepackage{booktabs}

%% listings: useful support for including source code listings, with
%% optional special keyword formatting.  The \lstset{} causes
%% the text to be typeset in a smaller sans serif font, with
%% proportional spacing.
\usepackage{listings}
\lstset{basicstyle=\sffamily\scriptsize,showstringspaces=false,fontadjust}

%% The acronym package provides support for defining acronyms, providing
%% their expansion when first used, and building glossaries.  See the
%% example in glossary.tex and the example usage throughout the example
%% document.
%% NOTE: to use \MakeTextLowercase in the \acsfont command below,
%%   we *must* use the `nohyperlinks' option -- it causes errors with
%%   hyperref otherwise.  See Section 5.2 in the ``LaTeX 2e for Class
%%   and Package Writers Guide'' (clsguide.pdf) for details.
\usepackage[nohyperlinks]{acronym}
%% The ubcdiss.cls loads the `textcase' package which provides commands
%% for upper-casing and lower-casing text.  The following causes
%% the acronym package to typeset acronyms in small-caps
%% as recommended by Bringhurst.
\renewcommand{\acsfont}[1]{{\scshape \MakeTextLowercase{#1}}}

%% color: add support for expressing colour models.  Grey can be used
%% to great effect to emphasize other parts of a graphic or text.
%% For an excellent set of examples, see Tufte's "Visual Display of
%% Quantitative Information" or "Envisioning Information".
\usepackage{color}
\definecolor{greytext}{gray}{0.5}

%% comment: provides a new {comment} environment: all text inside the
%% environment is ignored.
%%   \begin{comment} ignored text ... \end{comment}
\usepackage{comment}

%% The natbib package provides more sophisticated citing commands
%% such as \citeauthor{} to provide the author names of a work,
%% \citet{} to produce an author-and-reference citation,
%% \citep{} to produce a parenthetical citation.
%% We use \citeeg{} to provide examples
\usepackage[numbers,sort&compress]{natbib}
\newcommand{\citeeg}[1]{\citep[e.g.,][]{#1}}

%% The titlesec package provides commands to vary how chapter and
%% section titles are typeset.  The following uses more compact
%% spacings above and below the title.  The titleformat that follow
%% ensure chapter/section titles are set in singlespace.
\usepackage[compact]{titlesec}
\titleformat*{\section}{\singlespacing\raggedright\bfseries\Large}
\titleformat*{\subsection}{\singlespacing\raggedright\bfseries\large}
\titleformat*{\subsubsection}{\singlespacing\raggedright\bfseries}
\titleformat*{\paragraph}{\singlespacing\raggedright\itshape}

%% The caption package provides support for varying how table and
%% figure captions are typeset.
\usepackage[format=hang,indention=-1cm,labelfont={bf},margin=1em]{caption}

%% url: for typesetting URLs and smart(er) hyphenation.
%% \url{http://...} 
\usepackage{url}
\urlstyle{sf}	% typeset urls in sans-serif


%%%%%%%%%%%%%%%%%%%%%%%%%%%%%%%%%%%%%%%%%%%%%%%%%%%%%%%%%%%%%%%%%%%%%%
%%%%%%%%%%%%%%%%%%%%%%%%%%%%%%%%%%%%%%%%%%%%%%%%%%%%%%%%%%%%%%%%%%%%%%
%%
%% Possibly useful packages: you may need to explicitly install
%% these from CTAN if they aren't part of your distribution;
%% teTeX seems to ship with a smaller base than MikTeX and MacTeX.
%%
%\usepackage{pdfpages}	% insert pages from other PDF files
%\usepackage{longtable}	% provide tables spanning multiple pages
%\usepackage{chngpage}	% support changing the page widths on demand
%\usepackage{tabularx}	% an enhanced tabular environment

%% enumitem: support pausing and resuming enumerate environments.
%\usepackage{enumitem}
\usepackage{enumerate}

%% rotating: provides two environments, sidewaystable and sidewaysfigure,
%% for typesetting tables and figures in landscape mode.  
%\usepackage{rotating}

%% subfig: provides for including subfigures within a figure,
%% and includes being able to separately reference the subfigures.
%\usepackage{subfig}

%% ragged2e: provides several new new commands \Centering, \RaggedLeft,
%% \RaggedRight and \justifying and new environments Center, FlushLeft,
%% FlushRight and justify, which set ragged text and are easily
%% configurable to allow hyphenation.
%\usepackage{ragged2e}

%% The ulem package provides a \sout{} for striking out text and
%% \xout for crossing out text.  The normalem and normalbf are
%% necessary as the package messes with the emphasis and bold fonts
%% otherwise.
%\usepackage[normalem,normalbf]{ulem}    % for \sout

%%%%%%%%%%%%%%%%%%%%%%%%%%%%%%%%%%%%%%%%%%%%%%%%%%%%%%%%%%%%%%%%%%%%%%
%% HYPERREF:
%% The hyperref package provides for embedding hyperlinks into your
%% document.  By default the table of contents, references, citations,
%% and footnotes are hyperlinked.
%%
%% Hyperref provides a very handy command for doing cross-references:
%% \autoref{}.  This is similar to \ref{} and \pageref{} except that
%% it automagically puts in the *type* of reference.  For example,
%% referencing a figure's label will put the text `Figure 3.4'.
%% And the text will be hyperlinked to the appropriate place in the
%% document.
%%
%% Generally hyperref should appear after most other packages

%% The following puts hyperlinks in very faint grey boxes.
%% The `pagebackref' causes the references in the bibliography to have
%% back-references to the citing page; `backref' puts the citing section
%% number.  See further below for other examples of using hyperref.
%% 2009/12/09: now use `linktocpage' (Jacek Kisynski): FoGS now prefers
%%   that the ToC, LoF, LoT place the hyperlink on the page number,
%%   rather than the entry text.
\usepackage[bookmarks,bookmarksnumbered,%
    citebordercolor={0.8 0.8 0.8},filebordercolor={0.8 0.8 0.8},%
    linkbordercolor={0.8 0.8 0.8},pagebordercolor={0.8 0.8 0.8},%
    urlbordercolor={0.8 0.8 0.8},%
    pagebackref,linktocpage%
    ]{hyperref}
%% The following change how the the back-references text is typeset in a
%% bibliography when `backref' or `pagebackref' are used
\renewcommand\backrefpagesname{\(\rightarrow\) pages}
\renewcommand\backref{\textcolor{greytext} \backrefpagesname\ }

%% The following uses most defaults, which causes hyperlinks to be
%% surrounded by colourful boxes; the colours are only visible in
%% PDFs and don't show up when printed:
\usepackage[bookmarks,bookmarksnumbered]{hyperref}

%% The following disables the colourful boxes around hyperlinks.
%\usepackage[bookmarks,bookmarksnumbered,pdfborder={0 0 0}]{hyperref}

%% The following disables all hyperlinking, but still enabled use of
%% \autoref{}
%\usepackage[draft]{hyperref}


%% The following commands causes chapter and section references to
%% uppercase the part name.
\renewcommand{\chapterautorefname}{Chapter}
\renewcommand{\sectionautorefname}{Section}
\renewcommand{\subsectionautorefname}{Section}
\renewcommand{\subsubsectionautorefname}{Section}

%% If you have long page numbers (e.g., roman numbers in the 
%% preliminary pages for page 28 = xxviii), you might need to
%% uncomment the following and tweak the \@pnumwidth length
%% (default: 1.55em).  See the tocloft documentation at
%% http://www.ctan.org/tex-archive/macros/latex/contrib/tocloft/
% \makeatletter
% \renewcommand{\@pnumwidth}{3em}
% \makeatother

% EXTRA PACKAGES
\usepackage[capitalise]{cleveref}
\usepackage{floatrow}

\newcounter{notation}

%%%%%%%%%%%%%%%%%%%%%%%%%%%%%%%%%%%%%%%%%%%%%%%%%%%%%%%%%%%%%%%%%%%%%%
%%%%%%%%%%%%%%%%%%%%%%%%%%%%%%%%%%%%%%%%%%%%%%%%%%%%%%%%%%%%%%%%%%%%%%
%%
%% Some special settings that controls how text is typeset
%%
% \raggedbottom		% pages don't have to line up nicely on the last line
% \sloppy		% be a bit more relaxed in inter-word spacing
% \clubpenalty=10000	% try harder to avoid orphans
% \widowpenalty=10000	% try harder to avoid widows
% \tolerance=1000

%% And include some of our own useful macros
% This file provides examples of some useful macros for typesetting
% dissertations.  None of the macros defined here are necessary beyond
% for the template documentation, so feel free to change, remove, and add
% your own definitions.
%
% We recommend that you define macros to separate the semantics
% of the things you write from how they are presented.  For example,
% you'll see definitions below for a macro \file{}: by using
% \file{} consistently in the text, we can change how filenames
% are typeset simply by changing the definition of \file{} in
% this file.
% 
%% The following is a directive for TeXShop to indicate the main file
%%!TEX root = diss.tex

\newcommand{\NA}{\textsc{n/a}}	% for "not applicable"
\newcommand{\eg}{e.g.,\ }	% proper form of examples (\eg a, b, c)
\newcommand{\ie}{i.e.,\ }	% proper form for that is (\ie a, b, c)
\newcommand{\etal}{\emph{et al}}

% Some useful macros for typesetting terms.
\newcommand{\file}[1]{\texttt{#1}}
\newcommand{\class}[1]{\texttt{#1}}
\newcommand{\latexpackage}[1]{\href{http://www.ctan.org/macros/latex/contrib/#1}{\texttt{#1}}}
\newcommand{\latexmiscpackage}[1]{\href{http://www.ctan.org/macros/latex/contrib/misc/#1.sty}{\texttt{#1}}}
\newcommand{\env}[1]{\texttt{#1}}
\newcommand{\BibTeX}{Bib\TeX}

% Define a command \doi{} to typeset a digital object identifier (DOI).
% Note: if the following definition raise an error, then you likely
% have an ancient version of url.sty.  Either find a more recent version
% (3.1 or later work fine) and simply copy it into this directory,  or
% comment out the following two lines and uncomment the third.
\DeclareUrlCommand\DOI{}
\newcommand{\doi}[1]{\href{http://dx.doi.org/#1}{\DOI{doi:#1}}}
%\newcommand{\doi}[1]{\href{http://dx.doi.org/#1}{doi:#1}}

% Useful macro to reference an online document with a hyperlink
% as well with the URL explicitly listed in a footnote
% #1: the URL
% #2: the anchoring text
\newcommand{\webref}[2]{\href{#1}{#2}\footnote{\url{#1}}}

% epigraph is a nice environment for typesetting quotations
\makeatletter
\newenvironment{epigraph}{%
	\begin{flushright}
	\begin{minipage}{\columnwidth-0.75in}
	\begin{flushright}
	\@ifundefined{singlespacing}{}{\singlespacing}%
    }{
	\end{flushright}
	\end{minipage}
	\end{flushright}}
\makeatother

% \FIXME{} is a useful macro for noting things needing to be changed.
% The following definition will also output a warning to the console
\newcommand{\FIXME}[1]{\typeout{**FIXME** #1}\textbf{[FIXME: #1]}}

\crefname{figure}{Figure}{Figures}
%MATH ENVIRONMENTS
\newtheorem*{thm*}{Theorem}
\newtheorem{thm}{Theorem}[chapter]
\crefname{thm}{Theorem}{Theorems}
\newtheorem{cor}[thm]{Corollary}
\crefname{cor}{Corollary}{Corollarys}
\newtheorem*{cor*}{Corollary}
\crefname{cor*}{Corollary}{Corollarys}
\newtheorem{lem}[thm]{Lemma}
\crefname{lem}{Lemma}{Lemmas}
\newtheorem{prop}[thm]{Proposition}
\crefname{prop}{Proposition}{Propositions}
\newtheorem{conj}{Conjecture}
\crefname{conj}{Conjecture}{Conjectures}
\newtheorem*{conj*}{Conjecture}
\crefname{conj*}{Conjecture}{Conjectures}
\theoremstyle{definition}
\newtheorem{defn}[thm]{Definition}
\crefname{defn}{Definition}{Definitions}



\theoremstyle{remark}\newtheorem*{remark}{Remark}
\newtheorem*{rem*}{Remark}

\newcommand{\namedthm}[2]{\theoremstyle{plain}
\newtheorem*{thm#1}{#1}\begin{thm#1}#2\end{thm#1}}

%OTHER ENVIRONMENTS
\newenvironment{dedication}
    {\vspace{6ex}\begin{quotation}\begin{flushright}\begin{em}}
    {\par\end{em}\end{flushright}\end{quotation}  \pagebreak}
    
% Macro for 'List of Symbols', 'List of Notations' etc...
\newcommand{\listofsymbols}{%% The following is a directive for TeXShop to indicate the main file
%%!TEX root = diss.tex

\chapter*{List of Symbols\hfill} 
\addcontentsline{toc}{chapter}{\protect{}List of Symbols}

%mark your variables in your source code with \newnot{YOUR_SYMBOL_LABEL}
%%%%%%%%%%%%%%%%%%%%%%%
%Sample List of Symbols
%%%%%%%%%%%%%%%%%%%%%%%
\begin{tabbing}
% YOU NEED TO ADD THE FIRST ONE MANUALLY TO ADJUST THE TABBING AND SPACES

%%$\mathfrak{a}$~~~~~~~~~~~~~~\=\parbox{5.4in}{integral ideal associated to $\sigma \in \Pr^1(F)$, except in \cref{ch:PHecke}\dotfill \protect\pageref{sym:ka}}\\


%ADD THE REST OF SYMBOLS WITH THE HELP OF MACRO


%%\addsymbol A_v: {area measure of $F_v$}{sym:A_v}
%%\addsymbol \text{AQUE}: {Arithmetic Quantum Unique Ergodicity}{sym:AQUE}
%%
%%
%%\addsymbol B_v^{\pm}(s_v; y_v): {special function composed of Bessel functions}{sym:B_v}
%%\addsymbol B^{\bd\delta}(\bs; \by): {$\prod_{v \mid \infty} B_v^{\delta_v}(s_v; y_v)$ for $\bd\delta \in \{ \pm \}^m$}{sym:Bdelta}
%%
%%\addsymbol c_{\cP}: {large positive constant depending only on $\cP$}{sym:ManyPrimes}
%%\addsymbol c_{\sigma}(\phi; \alpha): {$\alpha$-Whittaker coefficient of $\phi$ at cusp $\sigma$}{sym:WhittCoeff}
%%\addsymbol c_{\sigma}^{(\xi)}(\phi; \n): {$\n$-Whittaker ideal coefficient of $\phi$ at cusp $\sigma$ for $\xi \in \O^{\times}/V$}{sym:WhittIdeal}
%%\addsymbol \C: {field of complex numbers}{sym:ComplexField}
%%
%%\addsymbol D_F: {absolute discriminant of $F$}{sym:D_F}
%%\addsymbol \cD: {fundamental domain for $\Lambda \backslash \X$}{sym:CuspStabFD}
%%\addsymbol \cD(y): {deep in a cusp, $\cD(y) = \{ (\bx, \by) \in \cD : \N \by \in (y, \infty)\}$}{sym:cD(y)}
%%\addsymbol \kD: {absolute different ideal of $F$}{sym:Different}
%%
%%\addsymbol e(x): {$e(x) = e^{2\pi i x}$}{sym:Exp}
%%
%%\addsymbol f: {mock $\cP$-Hecke multiplicative function}{sym:MockP}
%%\addsymbol f_{\cP}: {$\cP$-Hecke multiplicative function}{sym:PHecke}
%%\addsymbol F: {number field}{sym:NumberField}
%%\addsymbol F_v: {completion of $F$ at place $v$}{sym:LocalField}
%%\addsymbol F_{\infty}: {$F_{\infty} = \prod_{v \mid \infty} F_v$}{sym:F_infty}
%%\addsymbol \cF: {fundamental domain for $\Gamma \backslash \X$}{sym:GammaFD}
%%\addsymbol \cF(Y): {function associated to mock $\cP$-Hecke multiplicative function $f$}{sym:F(Y)}
%%
%%\addsymbol g_{\sigma}: {element of $\PGL(2,F)$ such that $g_{\sigma}(\infty) = \sigma$}{sym:g_sigma}
%%\addsymbol G: {$G = \prod_{v \mid \infty} G_v = \PGL(2,F_{\infty})$}{sym:G}
%%\addsymbol G_v: {$G_v = \PGL(2,F_v)$}{sym:G_v}
%%
%%\addsymbol h_F: {class number of $F$}{sym:h_F}
%%\addsymbol \kH^2: {hyperbolic $2$-space}{sym:H2}
%%\addsymbol \kH^3: {hyperbolic $3$-space}{sym:H3}
%%\addsymbol \HQ: {Hamilton's quaternions}{sym:HQ}
%%
%%
%%\addsymbol I_{\nu}(y): {$I$-Bessel function, $\nu \in \C, y \in \R_{>0}$}{sym:IBessel}
%%\addsymbol \Im: {imaginary part of $\kH^2, \kH^3$ or $\X$}{sym:Im}
%%
%%
%%\addsymbol J: {well chosen positive integer based on $\cF(Y)$}{sym:J}
%%
%%\addsymbol K: {$\prod_{v \mid \infty} K_v$, maximal compact subgroup of $G$}{sym:K}
%%\addsymbol K_v: {maximal compact subgroup of $G_v$ for $v \mid \infty$}{sym:K_v}
%%\addsymbol K_{\nu}(y): {$K$-Bessel function, $\nu \in \C, y \in \R_{>0}$}{sym:KBessel}
%%
%%
%%\addsymbol m: {number of archimedean places of $F$, i.e. $m=r_1+r_2$}{sym:m}
%%\addsymbol M: {congruence locally symmetric space, i.e. $M = \Gamma \backslash \X$}{sym:M}
%%
%%\addsymbol n: {degree of $F$, i.e. $n = [F:\Q]$}{sym:Degree}
%%\addsymbol \kN: {integral ideal of $F$ representing level}{sym:Level}
%%\addsymbol \N: {global norm, i.e. $\N = \N^F_\Q$, extended to $\X$}{sym:GlobalNorm}
%%\addsymbol \N_v: {local norm at $v \mid \infty$, i.e. $\N = \N^{F_v}_{\R}$}{sym:LocalNorm}
%%\addsymbol \cN: {$\cP$-friable ideals}{sym:PFriable}
%%\addsymbol \cN_{\kN}: {$\cP_{\kN}$-friable ideals}{sym:P_NFriable}
%%
%%\addsymbol \O: {ring of integers of $F$}{sym:RingOfIntegers}
%%\addsymbol \O^{\times}: {group of integral units of $F$}{sym:Units}
%%\addsymbol \O^{\times}_1: {roots of unity of $F$}{sym:RootsOfUnity}
%%
%%\addsymbol \p: {prime ideal of $F$, often principal}{sym:PrimeIdeal}
%%\addsymbol \cP: {subset of unramified prime ideals not dividing $\kN$}{sym:P}
%%\addsymbol \cP_{\kN}: {set of unramified \emph{principal} prime ideals not dividing $\kN$}{sym:P_N}
%%\addsymbol \cP(Y): {$\{ \p \in \cP : \N\p \in [ \sqrt{Y}/2, \sqrt{Y}] \}$}{sym:cPY}
%%\addsymbol \cP_j: {suitably chosen subset of $\cP(Y)$}{sym:P_j}
%%\addsymbol \Pr^1(F): {projective linear space of $F$}{sym:P1(F)}
%%
%%\addsymbol \Q: {field of rational numbers}{sym:Rationals}
%%\addsymbol \text{QUE}: {Quantum Unique Ergodicity}{sym:QUE}
%%
%%\addsymbol r_1: {number of real places of $F$}{sym:r_1}
%%\addsymbol r_2: {number of complex places of $F$}{sym:r_2}
%%\addsymbol  \R: {field of real numbers}{sym:RealNumbers}
%%
%%\addsymbol s_v: {$\lambda_v = s_v(1-s_v) \in \C$}{sym:s_v}
%%\addsymbol \bs: {$\bs = (s_v)_{v \mid \infty} \in \C^m$}{sym:Sbold}
%%\addsymbol \cS(y): {compact centre of $\Gamma \backslash \X$, grows as $y \rightarrow \infty$}{sym:CompactCentre}
%%
%%\addsymbol T(\n): {$\n$-Hecke operator (usually for $\n \in \cN_{\kN}$)}{sym:HeckeOp}
%%\addsymbol \cT(y): {$\by$-coordinate of $\cD(y)$, i.e. $\{ \by \in \R_{>0}^m : \widehat{\by} \in \cV, \N \by \in (y, \infty) \}$}{sym:cT(y)}
%%\addsymbol \Tr: {global trace, i.e. $\Tr = \Tr^F_{\Q}$, extended to $F_{\infty}$}{sym:GlobalTrace}
%%\addsymbol \Tr_v: {local trace for $v \mid \infty$, i.e. $\Tr_v = \Tr^{F_v}_{\R}$}{sym:LocalTrace}
%%
%%\addsymbol \cU: {fundamental domain for $(\O^{\times}_1 \cap V) \backslash F_{\infty} / \kN$}{sym:cU}
%%
%%\addsymbol v \mid \infty: {archimedean place of $F$}{sym:Place}
%%\addsymbol V: {$V = (1+\kN) \cap \O^{\times}$}{sym:V}
%%\addsymbol \cV: {fundamental domain for $V \backslash \widehat{\cY}$}{sym:cV}
%%\addsymbol \vol(\bz): {volume measure for $\bz \in \X$}{sym:vol}
%%\addsymbol V_1(\bx): {volume measure for $\bx$-coordinate in $\X$}{sym:V_1}
%%\addsymbol V_2(\by): {volume measure for $\by$-coordinate in $\X$}{sym:V_2}
%%
%%\addsymbol W(\bs; \bz): {full Whittaker function}{sym:FullWhittaker}
%%\addsymbol W_v(s_v; z_v): {local Whittaker function at $v \mid \infty$}{sym:LocalWhittaker}
%%
%%\addsymbol \bx: {$\bx = (x_v)_{v \mid \infty} \in F_{\infty}$, coordinate of $\bz = (\bx, \by) \in \X$}{sym:bx}
%%\addsymbol \X: {symmetric space of $G$, equals $(\kH^2)^{r_1} \times (\kH^3)^{r_2}$}{sym:X_F}
%%
%%\addsymbol y_0: {(small) positive constant depending only on $F$}{sym:y_0}
%%\addsymbol \by: {$\by = (y_v)_{v \mid \infty} \in \R_{>0}^m$, coordinate of $\bz = (\bx, \by) \in \X$}{sym:by}
%%\addsymbol \widehat{\by}: {$\widehat{\by} = (y_v/(\N \by)^{1/n})_{v \mid \infty} \in \widehat{\cY}$}{sym:byhat}
%%\addsymbol Y_0: {(large) positive constant depending only on $F$}{sym:Y_0}
%%\addsymbol \widehat{\cY}: { $\{ \by \in \R_{>0}^m : \N \by = 1\}$, hyperplane in $\R_{>0}^m$ }{sym:cYhat}
%%
%%\addsymbol \bz: {element of $\X$, alternate coordinates $\bz = (\bx, \by)$}{sym:bz}
%%\addsymbol \Z: {ring of rational integers}{sym:Integers}
%%
%%% GREEK SYMBOLS
%%
%%\addsymbol \Delta_v: {Laplacian at place $v \mid \infty$}{sym:Delta}
%%
%%\addsymbol \phi(\bz):  {Hecke-Maa\ss\, cusp form on $\Gamma \backslash \X$}{sym:Phi}
%%\addsymbol \phi_{\sigma}(\bz): {$\phi_{\sigma}(\bz) = \phi(g_{\sigma}\bz)$}{sym:PhiSigma}
%%\addsymbol \widehat{\phi_{\sigma}}(\by; \alpha): {$\alpha$-Fourier coefficient of $\phi$ at cusp $\sigma$}{sym:FourierCoeff}
%%
%%\addsymbol \Gamma: {congruence group of level $\kN$, often $\Gamma = \Gamma_0(\kN)$}{sym:Gamma}
%%\addsymbol \Gamma(\kN): {principal congruence subgroup of level $\kN$}{sym:Gamma(N)}
%%\addsymbol \Gamma_0(\kN): {specific congruence group of level $\kN$}{sym:Gamma_0(N)}
%%\addsymbol \Gamma^{(\sigma)}: {$\Gamma^{(\sigma)} = g_{\sigma}^{-1} \Gamma g_{\sigma}$}{sym:GammaSigma}
%%\addsymbol \Gamma_F: {$\Gamma_F = \PGL(2,\O)$}{sym:Gamma_F}
%%
%%\addsymbol \lambda_v: {$\Delta_v$-eigenvalue of Maa\ss\, form $\phi$}{sym:lambda_v}
%%\addsymbol \bd\lambda: {$\bd\lambda = (\lambda_v)_{v \mid \infty} \in \C^m$}{sym:lambda}
%%\addsymbol \lambda_{\phi}(\n): {$T(\n)$-Hecke eigenvalue of $\phi$}{sym:HeckeEig}
%%\addsymbol \Lambda: {subgroup of $\Gamma^{(\sigma)}$-stabilizer of $\infty \in \Pr^1(F)$ for any $\sigma \in \Pr^1(F)$}{sym:Lambda}
%%
%%\addsymbol \mu_{\phi}: {measure associated to a Hecke-Maa\ss\, cusp form $\phi$}{sym:mu_phi}
%%
%%\addsymbol \Omega: {set of inequivalent representatives of $\Gamma \backslash \Pr^1(F)$}{sym:Omega}
%%
%%\addsymbol \rho(\sigma; \bz): {function measuring closeness of $\bz \in \X$ to $\sigma \in \Pr^1(F)$}{sym:rho}
%%
%%\addsymbol \sigma: {element of $\Pr^1(F)$ regarded as cusp of $\Gamma$, often $\sigma = (\alpha:\beta)$}{sym:sigma}
%%
%%\addsymbolEND \langle\cdot,\cdot\rangle: {trace form billinear pairing on $F_{\infty} \times F_{\infty}$}{sym:Pairing}
% .
% .
% .
% ALWAYS KEEP THE FOLLOWING LINE
\end{tabbing}
%\pagebreak
}
\def\addsymbol #1: #2#3{$#1$ \> \parbox{5.4in}{#2 \dotfill \pageref{#3}}\\} 
\def\addsymbolEND #1: #2#3{$#1$ \> \parbox{5.4in}{#2 \dotfill \pageref{#3}}} 
\newcommand{\sym}[1]{\refstepcounter{notation} \label{#1}} 

%MATH FORMAT
\newcommand{\ds}{\displaystyle}
\newcommand{\fk}[1]{\mathfrak{#1}}
\newcommand{\red}[1]{\textcolor{red}{#1}}
\newcommand{\tab}{\vspace{-0.2in} $ $}
\newcommand{\bd}{\boldsymbol}

%MATH SYMBOLS
\newcommand{\1}{ {\bf 1}}

\newcommand{\ka}{\mathfrak{a}}
\newcommand{\A}{\mathbb{A}}

\newcommand{\B}{\mathcal{B}}
\newcommand{\Lbrace}{\left\lbrace}
\newcommand{\Rbrace}{\right\rbrace}

\newcommand{\C}{\mathbb{C}}

\newcommand{\cD}{\mathcal{D}}
\newcommand{\kD}{\mathfrak{D}}
\newcommand{\F}{\mathcal{F}}
\newcommand{\cF}{\mathcal{F}}

\newcommand{\G}{\Gamma}
\newcommand{\GL}{ {\rm GL}}

\renewcommand{\H}{\mathbb{H}}
\newcommand{\cH}{\mathcal{H}}
\newcommand{\kH}{\mathbb{H}}
\newcommand{\HQ}{\mathcal{H}}

\renewcommand{\Im}{\text{Im}}
\newcommand{\m}{\mathfrak{m}}

\newcommand{\n}{\mathfrak{n}}
\newcommand{\N}{\mathbb{N}}
\newcommand{\cN}{\mathcal{N}}
\newcommand{\kN}{\mathfrak{N}}

\newcommand{\p}{\mathfrak{p}}
\newcommand{\cP}{\mathcal{P}}
\renewcommand{\Pr}{\mathbb{P}}
\newcommand{\PGL}{ {\rm PGL}}
\newcommand{\PO}{ {\rm PO}}
\newcommand{\PSL}{ {\rm PSL}}
\newcommand{\PU}{ {\rm PU}}

\renewcommand{\O}{\mathcal{O}}

\newcommand{\q}{\mathfrak{q}}
\newcommand{\Q}{\mathbb{Q}}

\newcommand{\R}{\mathbb{R}}
\renewcommand{\Re}{\text{Re}}

\newcommand{\bs}{\mathbf{s}}
\newcommand{\stab}{{\rm stab}}
\newcommand{\cS}{\mathcal{S}}
\newcommand{\SL}{ {\rm SL}}
\newcommand{\SO}{ {\rm SO}}
\newcommand{\SU}{ {\rm SU}}

\newcommand{\bt}{\mathbf{t}}
\newcommand{\cT}{\mathcal{T}}
\newcommand{\Tr}{ \textbf{Tr}}

\newcommand{\cU}{\mathcal{U}}

\newcommand{\cV}{\mathcal{V}}
\newcommand{\vol}{\text{\rm vol}}

\newcommand{\cW}{\mathcal{W}}
\newcommand{\wkarrow}{\overset{\text{wk-}\ast}{\longrightarrow}}

\newcommand{\bx}{\mathbf{x}}
\newcommand{\X}{\mathcal{X}}

\newcommand{\by}{\mathbf{y}}
\newcommand{\cY}{\mathcal{Y}}

\newcommand{\bz}{\mathbf{z}}
\newcommand{\Z}{\mathbb{Z}}
% END


%%%%%%%%%%%%%%%%%%%%%%%%%%%%%%%%%%%%%%%%%%%%%%%%%%%%%%%%%%%%%%%%%%%%%%
%%%%%%%%%%%%%%%%%%%%%%%%%%%%%%%%%%%%%%%%%%%%%%%%%%%%%%%%%%%%%%%%%%%%%%
%%
%% Document meta-data: be sure to also change the \hypersetup information
%%

\title{On the isogeny invariance of the Bloch-Kato's Tamagawa Numbers conjecture:{\ }\\{\ }\\ a $K$-theoretic point of view}
%\subtitle{If you want a subtitle}

\author{Justin Scarfy}
\previousdegree{B.Sc., The University Of British Columbia, 2017}

% What is this dissertation for?
\degreetitle{Master of Science}

\institution{The University Of British Columbia}
\campus{Vancouver}

\faculty{The Faculty of Graduate Studies}
\department{Mathematics}
\submissionmonth{June}
\submissionyear{2018}

%% hyperref package provides support for embedding meta-data in .PDF
%% files
\hypersetup{
  pdftitle={thesis  (DRAFT: \today)},
  pdfauthor={Justin Scarfy},
  pdfkeywords={thesis}
}

%%%%%%%%%%%%%%%%%%%%%%%%%%%%%%%%%%%%%%%%%%%%%%%%%%%%%%%%%%%%%%%%%%%%%%
%%%%%%%%%%%%%%%%%%%%%%%%%%%%%%%%%%%%%%%%%%%%%%%%%%%%%%%%%%%%%%%%%%%%%%
%% 
%% The document content
%%

%% LaTeX's \includeonly commands causes any uses of \include{} to only
%% include files that are in the list.  This is helpful to produce
%% subsets of your thesis (e.g., for committee members who want to see
%% the dissertation chapter by chapter).  It also saves time by 
%% avoiding reprocessing the entire file.
%\includeonly{intro,conclusions}
%\includeonly{discussion}

\begin{document}
%\pdfpageheight 11in 
%\pdfpagewidth 8.5in
%%%%%%%%%%%%%%%%%%%%%%%%%%%%%%%%%%%%%%%%%%%%%%%%%%
%% From Thesis Components: Tradtional Thesis
%% <http://www.grad.ubc.ca/current-students/dissertation-thesis-preparation/order-components>

% Preliminary Pages (numbered in lower case Roman numerals)
%    1. Title page (mandatory)
\maketitle
\pagestyle{plain}

%    2. Abstract (mandatory - maximum 350 words)
%% The following is a directive for TeXShop to indicate the main file
%%!TEX root = diss.tex

\chapter{Abstract}
\tab
A major task in mathematics today is to harmonize the continuous and the discrete, to include them in one comprehensive mathematics, and to eliminate obscurity from both.  The arithmetic properties of various interesting objects (e.g. number fields, varieties, or even a motive, i.e. discrete objects) are all encoded in their respective $L$-functions, i.e. continuous objects, of which the Riemann zeta function is the simplest example. The understanding of these $L$-functions undergoes three phases~\cite{Ka1993a}: 
\begin{enumerate}
\item
analytic properties and the rationality of values,
\item
algebraic properties illuminated by the $p$-adic properties of values,
\item
arithmetic-geometric point of view of interpreting the values. 
\end{enumerate}

To date much of these have been achieved for the Riemann zeta function~\cite{CRSS2015} (except the locations of the zeros, or the zero free region).  The deep conjectures of Deligne~\cite{De1979} and Beilinson~\cite{Be1985} were among the first to place these problems in a broad framework.  Ambiguity in the work of Beilinson for interpreting these values, as they are interpreted only up to nonzero rational number multiples.  In their article published in the Grothendieck Festschrift~\cite{BK1990}, Spencer Bloch and Kazuya Kato removed this ambiguity in their formulation of what is now called the Tamagawa number conjecture  (or the Bloch-Kato conjecture for motives).  They also proved that the conjecture is invariant under isogeny, and the book~\cite{CRSS2015} exploits this isogeny invariance condition in the optic of $K$-theory for the Riemann zeta function.

This thesis investigates the isogeny invariance for specific motives and elaborates on accompanying results in cohomology and $K$-theory.
\cleardoublepage

%    3. Preface
%\include{preface}
%\cleardoublepage

%    4. Table of contents (mandatory - list all items in the preliminary pages
%    starting with the abstract, followed by chapter headings and
%    subheadings, bibliographies and appendices)
\tableofcontents
\cleardoublepage	% required by tocloft package

%    5. List of tables (mandatory if thesis has tables)
%\listoftables
%\cleardoublepage	% required by tocloft package

%    6. List of figures (mandatory if thesis has figures)
\listoffigures
\cleardoublepage	% required by tocloft package

%    7. List of illustrations (mandatory if thesis has illustrations)
%    8. Lists of symbols, abbreviations or other (optional)

%    9. Glossary (optional)
%\input{glossary}	% always input, since other macros may rely on it

\listofsymbols
\cleardoublepage

\textspacing		% begin one-half or double spacing

%   10. Acknowledgements (optional)
%% The following is a directive for TeXShop to indicate the main file
%%!TEX root = diss.tex

\chapter{Acknowledgments}

\tab

First and foremost, I am deeply indebted to Professor Sujatha Ramdorai for providing me with extraordinary amounts of time and patience (since my first year undergraduate), sharing her seemingly endless bounty of mathematical knowledge, and challenging me with this exciting and multi-faceted problem. I have learned a great deal from our many conversations and feel privileged to have studied under her supervision. I am also very grateful for the funding I have received from NSERC, the department, and Professor Sujatha. 

In addition, I am extremely grateful to Professor Greg Martin for his patience and suffering since my first year undergraduate studies in guiding me through the world of analytic number theory, enabling me to understand the many recent exciting breakthroughs in the field, especially in understanding the notorious Riemann Hypothesis, (and for reminding me that mathematician is a profession, not just a seeker of knowledge) and I want to thank Professor Lior Silberman for enabling me to understand its automorphic version.  I also want to thank Professors Patrick Brosnan and Bruno Kahn on pointing me to the latest advances in the Hodge conjecture, to Professor Tai-Peng Tsai for his inspiring lectures on the latest results on the Navier-Stokes existence and smoothness problem, to Professor Izabella \L aba for her lectures with the latest results on the Kakeya conjecture, to Professor {\'A}kos Mayar on his lectures on the Green-Tao Theorem, to Professor (MORE TO BE ADDED) , and to every faculty member here at UBC whom I have either taken or sat in on a course with, and for PIMS giving me the use of an office as a luxury to do uninterrupted work.

Studying at the Department of Mathematics at UBC has been wonderful with the extremely helpful and friendly department staff, particularly Roseann Kinsey and Marlowe Dickson. Involvement in the graduate student and number theory community has been a great personal and educational experience, punctuated with many deep (non-)mathematical discussions, for which I would like to thank (COLLEAGUES) amongst many others. 

Simply put, I am only here because of my parents, who have been selflessly supportive throughout my life, let alone my education. I dedicate this thesis to them, as their hardships and struggles have always been for my future well-being. 

%   11. Dedication (optional)
\begin{dedication}
To my parents
\begin{figure}[h]
\includegraphics[width=16cm]{parents.jpg}
\end{figure}
\end{dedication}

% Body of Thesis (not all sections may apply)
\mainmatter

%    1. Introduction
\setcounter{page}{1}
%% The following is a directive for TeXShop to indicate the main file
%%!TEX root = diss.tex
\chapter*{Introduction}
\addcontentsline{toc}{chapter}{\protect{}Introduction}
\label{ch:Introduction}

\section{The leading role played by arithmetic within mathematics and recent breakthroughs}
$ $ 

Arithmetic enjoys a privileged position within mathematics as a fertile source of fundamental questions. Among the seven Millennium problems listed by the Clay Institute~\cite{Clay}, not fewer than three: the Birch and Swinnerton-Dyer conjecture, the Hodge conjecture, and the Riemann hypothesis, were handed down by the Queen of Mathematics. Even by the standards of a subject which has remained vibrant since the days of Fermat and Gau\ss, the last two decades have witnessed a real golden age, with landmarks too numerous to list completely: such as the striking progress on the Birch and Swinnerton-Dyer conjecture arising from the work of Gross-Zagier \cite{GZ1986}, Kolyvagin \cite{Ko1989}, and Kato \cite{Ka2004}; the proofs of the Shimura-Taniyama-Weil conjecture \cite{BCDT2001}, Serre's conjectures \cite{KW2009}, the Fontaine-Mazur conjecture for two-dimensional Galois representations \cite{Ki2009}, and the Sato-Tate conjectures \cite{CHR2008} which grew out of Wiles' epoch-making proof of Fermat's Last Theorem \cite{Wi1995},~\cite{TW1995}; the revolutionary ideas of Bourgain \cite{Bo2008} and Gowers \cite{Go2007} blending techniques in harmonic analysis and additive combinatorics, the Fields-medal winning breakthrough of Green and Tao on primes in arithmetic progressions \cite{GT2008}, and the work of Goldston, Pintz, and Y{\i}ld{\i}r{\i}m \cite{GPY2009},~\cite{GPY2010}, and its spectacular recent strengthenings by Zhang \cite{Zh2014}, and Maynard \cite{Ma2015} and Tao \cite{Poly2014}, on bounded gaps between primes. Recent innovations in arithmetic geometry by the innovation of Perfectoid spaces~\cite{Scho2012}, and subsequent topological realization of the absolute Galois group~\cite{Scho2016} by Peter Scholze also shed new lights on the Langlands Programme, a web of conjectures that connects number theory, harmonic analysis, and geometry.

\newpage
\section{On the Riemann Hypothesis}
$ $ 

Little essential progress has been made to the Riemann Hypothesis in the past two decades for the Riemann zeta function (or the Generalized Riemann Hypothesis for automorphic $L$-functions), which predicts that all the zeros of such $L$-functions are critical: They lie on the real line $\operatorname{Re}s=\tfrac{1}{2}$ with multiplicity one.  The sharpest result for the Riemann zeta function is due to Feng~\cite{Fe2012}, proving that at least $41.28\%$ of the zeros of the Riemann zeta function,
\[
\zeta(s):=\sum_{n\ge 1} \frac{1}{n^s}\qquad (s:=\sigma+it),
\]
lie on the critical line, by introducing a new mollifier and applying the original method of Levinson~\cite{Le1974} and its subsequent strengthening by Conrey~\cite{Co1989}, which gave at least $34.20\%$ and $40.88\%$ of the zeros of $\zeta(s)$ are critical, respectively. The best zero-free region to date for the Riemann zeta function is obtained by Ford~\cite{Fo2002} with
\[
\sigma \ge 1-\frac{1}{57.54(\log|t|)^{2/3}(\log\log|t|)^{1/3}},\qquad |t|\ge 3.
\] 

A better zero-free region would imply a stronger result in the error term of the prime number theorem, 

Other industries including computing the higher moments of these automorphic $L$-functions, 

the attempt of using random matrices to explain the spacing between the zeros  

Montgomery pair-correlation conjecture

the Siegel-Walfisz theorem and the large sieve developed by Bombieri are often used in place of the Generalized Riemann Hypothesis for Dirichlet $L$-functions to prove theorems and do estimates.

\newpage
\section{On the Hodge conjecture}
Algebraic cycles

Chow groups

K\"ahler manifold

Known cases:

Generalizations:

Main difficulties/ best result in hoped cases.


\newpage
\section{On the Birch and Swinnerton-Dyer conjecture}
$ $
Recall the Dirichlet class number formula for a number field $K$~\cite{Neu1999},
\[
\operatorname{Res}_{s=1} \zeta_K (s)=\frac{2^{r_1}(2\pi)^{r_2}}{w|d_K|^{1/2}}hR,
\]
where the left hand side is the Dedekind zeta function,
\[
\zeta_K(s):=\sum_{\mathfrak{a}\subset \mathcal{O}_K}\frac{1}{\mathfrak{N}(\mathfrak{a})},  
\]
with $\mathfrak{a}$ {ideals in}\;$\mathcal{O}_K$, \textrm{the ring of integers of} $K$, and $\mathfrak{N}$ its norm.  On the right hand side, $r_1$ and $r_2$ the numbers of real and complex embeddings of $K$, respectively.  $h$ denotes the class number of $\mathcal{O}_K$, which measures the failure of $\mathcal{O}_K$ being a unique factorization domain, $R$ the regulator, the determinant $w$ the root number, and $d_K$ the discriminate of the number field $K$
{\ }\\

The Birch and Swinnerton-Dyer conjecture (BSD), formulated by B. Birch and Swinnerton-Dyer (1965) when studying the asymptotics of
\[
\prod_{p\le x}\frac{\# E(\F_p)}{p},
\] 

has seen tremendous progress since the first breakthrough of Coates and Wiles.


The BSD conjecture 
\begin{enumerate}[\bf (a)]
\item
(Weak BSD)
\[
\operatorname{ord}_{s=1} L(E/K, s)=r_K,
\]
\item
(Strong BSD)
\[
\lim_{s\to 1}\frac{L(E/K, s)}{(s-1)^{r_p}}=\Omega_{E/K}\times \operatorname{Reg}_{\infty, K}(E) \times \frac{|\Sha_K(E)| \prod_{p\le \infty}[E(K_p): E_0(K_p)]}{\sqrt{\Delta_K}\times |E(K)|_{\operatorname{tors}}^2},
\]
can be seen as a generalization of Dirichlet class number formula, in the sense
$\Sha(E)$ measures the failure of 
\end{enumerate}

Note Iwasawa main conjecture, proved in cases
\begin{enumerate}[1.]
\item
$\Q$~\cite{MW1984}
\item
totally real number fields~\cite{Wil1990}
\item
imaginary quadratic fields~\cite{Rub1988}~\cite{Rub1991}
\item
Dirichlet characters~\cite{HK2003}
\item
CM elliptic curves at supersingular primes~\cite{PR2004}
\item
for elliptic curves over anticyclotomic $\Z_p$-extensions~\cite{BD2005}.
\item
non-commutative main conjecture for totally real $p$-adic Lie extension of a number field~\cite{Kak2013}~\cite{CSSV2013}.
\item
(automorphic) $\GL_2$~\cite{SU2014}
\end{enumerate}


One promising view

algebraic $K$-theory,

Suslin and Voevodsky,

Bloch

Motivic cohomology
{\ }\\

Euler systems~\cite{Rub2000}
{\ }\\

Classical Euler systems:

1. Siegels' cyclotomic units gives Kubota-Leopoldt $p$-adic $L$ function, but no BSD application~\cite{CS2006}.

2. Elliptic units Coates and Wiles' homomorphism

3. Heegner points gives anticyclotomic $p$-adic $L$-function of
{\ }\\

Kato's Euler systems:

1. Beilinson-Kato elements

2. Beilinson-Flach elements

3. Gross-Kudla-Schoen cycles 

Note that the monograph~\cite{Del2008} is devoted to the study of the BSD conjecture over the universal deformation rings of an elliptic curve.

Known cases of BSD
\begin{enumerate}
\item
Coates and Wiles (1977) proved that if $E$ is a curve over a number field $F$ with complex multiplication by an imaginary quadratic field $K$ of class number $1$, $F = K$ or $\Q$, and $L(E, 1)$ is not $0$ then $E(F)$ is a finite group. This was extended to the case where $F$ is any finite abelian extension of $K$ by Arthaud (1978).
\item
Gross and Zagier (1986) showed that if a modular elliptic curve has a $L'(E, 1)=0$  then it has a rational point of infinite order; see Gross-Zagier theorem.
\item
Kolyvagin (1989) showed that a modular elliptic curve $E$ for which $L(E, 1)$ is not zero has rank $0$, and a modular elliptic curve $E$ for which $L(E, 1)$ has a first-order zero at $s = 1$ has rank $1$.
\item
Rubin (1991) showed that for elliptic curves defined over an imaginary quadratic field $K$ with complex multiplication by $K$, if the $L$-series of the elliptic curve was not zero at $s = 1$, then the $p$-part of the Tate-Shafarevich group had the order predicted by the Birch and Swinnerton-Dyer conjecture, for all primes $p > 7$.
\item
Breuil et al. (2001), extending work of Wiles (1995), proved that all elliptic curves defined over the rational numbers are modular, which extends results $2$ and $3$ to all elliptic curves over the rationals, and shows that the $L$-functions of all elliptic curves over $\Q$ are defined at $s = 1$.
\item
Bhargava and Shankar (2015) proved that the average rank of the Mordell-Weil group of an elliptic curve over $\Q$ is bounded above by $7/6$. Combining this with the $p$-parity theorem of Nekov\'ar (2009) and Dokchitser \& Dokchitser (2010) and with the proof of the main conjecture of Iwasawa theory for $\GL(2)$ by Skinner \& Urban (2014), they conclude that a positive proportion of elliptic curves over $\Q$ have analytic rank zero, and hence, by Kolyvagin (1989), satisfy the Birch and Swinnerton-Dyer conjecture.
\end{enumerate}
%\begin{figure}
%\centering
%\includegraphics[ width=1\textwidth]{sl2z.png}
%\caption{Fundamental domain for $\SL(2,\Z)$}
%\label{fig:SL2Z} % label should change 
%\end{figure}



\endinput

Any text after an \endinput is ignored.
You could put scraps here or things in progress.


%    2. Main body
% Generally recommended to put each chapter into a separate file
%% The following is a directive for TeXShop to indicate the main file
%%!TEX root = diss.tex

\chapter{Bloch-Kato's Tamagawa Numbers conjecture and the isogeny invariance}
\label{chap1}

\tab

\section{From the Birch and Swinnerton-Dyer conjecture to the Bloch-Kato conjecture}
The Bloch-Kato's Tamagawa Numbers conjecture~\cite{BK1990}, can be seen as a generalization of the Birch and Swinnerton-Dyer conjecture for motives.

Need: Fontaine's topological period rings $B_{dR}$ and $B_{crys}$, where the former is a complete valued field with residue field $\C_p$

For a motivic pair $(V, D)$ with weights $\le w$ and a finite set of places $\Omega$ of $\Q$ containing $\infty$, the $L$-function $L_\Omega(V, s)$ is defined to be as the Euler product
\[
L_{\Omega}(V, s):=\prod_{p\not\in \Omega}P_p(V, p^{-s})^{-1},
\]
it is absolute convergent for $\operatorname{Re}(s)>\tfrac{w}{2}+1$.

Fixing a $\Z$-lattice $M\subset V$ 


\section{Motives, Fontaine's $p$-adic period rings, and other essential gadgets}
To define Tamagawa measures one needs groups $A(\Q_P), p \le \infty$, and $A(\Q)$. Bloch and Kato define such groups for a motivic pair $(V, D)$ as follows:
Assuming the motivic pair $(V, D)$ has weight $\le -1$


Fontaine's $p$-adic period rings~\cite{Fon1982},~\cite{FM1987},

Tamagawa number~\cite{Wei1982}

\newpage
\section{The isogeny invariance explained}

\newpage
\section{$K$-theoretic background}
Reference~\cite{HB-K1},~\cite{HB-K2},~\cite{Sri1993},~\cite{TT1990},~\cite{Wei2013},~\cite{Wal1987a} ,~\cite{Wal1987b}

We will need to use higher $K$-theory, where $K_0$ was invented by Grothendieck in proving the Riemann-Roch Theorem,

Quillen higher $K$-theory

the $+$ construction

the $Q$ construction for schemes

Waldhausen $K$-theory

$K$-theory of Thomason \& Trobaugh.
%%%%%%%%%%%%%%%%%%%%%%%%%%%%%%%%%%%%%%%%%%%%%%%%%%%%%%%%%%%%%%%%%%%%%%
\endinput

Any text after an \endinput is ignored.
You could put scraps here or things in progress.

%% The following is a directive for TeXShop to indicate the main file
%%!TEX root = diss.tex

\chapter{The isogeny invariance for elliptic curves with complex multiplication}
\label{chap1}

\tab

\section{CM elliptic curves}
Reference: \cite{Hi2013}, \cite{Si2009}, \cite{Si1994}, \cite{Ko1993}, \cite{Kn1992},~\cite{Ca1991},~\cite{Mi2006}, \cite{CVG1999}, \cite{Sh1987}

\newpage
\section{The isogeny invariance for CM elliptic curves}

\newpage
\section{The $K$-theoretic viewpoint}


%%%%%%%%%%%%%%%%%%%%%%%%%%%%%%%%%%%%%%%%%%%%%%%%%%%%%%%%%%%%%%%%%%%%%%
\endinput

Any text after an \endinput is ignored.
You could put scraps here or things in progress.

%% The following is a directive for TeXShop to indicate the main file
%%!TEX root = diss.tex

\chapter{The isogeny invariance for elliptic curves without complex multiplication}
\label{chap1}

\tab


\tab

\section{non-CM elliptic curves}


\newpage
\section{The isogeny invariance for non-CM elliptic curves}

\newpage
\section{The $K$-theoretic viewpoint}


%%%%%%%%%%%%%%%%%%%%%%%%%%%%%%%%%%%%%%%%%%%%%%%%%%%%%%%%%%%%%%%%%%%%%%
\endinput

Any text after an \endinput is ignored.
You could put scraps here or things in progress.

%% The following is a directive for TeXShop to indicate the main file
%%!TEX root = diss.tex

\chapter{The isogeny invariance for modular forms}
\label{chap1}

\tab
\section{Modular forms}
Reference~\cite{Mi1989},~\cite{DS2005},~\cite{AF1995}


\newpage
\section{The isogeny invariance for modular forms}

\newpage
\section{The $K$-theoretic viewpoint}


%%%%%%%%%%%%%%%%%%%%%%%%%%%%%%%%%%%%%%%%%%%%%%%%%%%%%%%%%%%%%%%%%%%%%%
\endinput

Any text after an \endinput is ignored.
You could put scraps here or things in progress.

%% The following is a directive for TeXShop to indicate the main file
%%!TEX root = diss.tex

\chapter{The isogeny invariance for abelian varieties with complex multiplication}
\label{chap1}

\tab


\tab

\section{CM abelian varieties}

References:~\cite{Sh1998},~\cite{Mu1974},~\cite{BL2004}

\newpage
\section{The isogeny invariance for CM abelian varieties}

\newpage
\section{The $K$-theoretic viewpoint}



%%%%%%%%%%%%%%%%%%%%%%%%%%%%%%%%%%%%%%%%%%%%%%%%%%%%%%%%%%%%%%%%%%%%%%
\endinput

Any text after an \endinput is ignored.
You could put scraps here or things in progress.

%% The following is a directive for TeXShop to indicate the main file
%%!TEX root = diss.tex

\chapter{Proposed isogeny invariance for tori and motif}
\label{chap1}

\tab
All of the above geometric objects are in the form of a pure motive: which are smooth projective varieties.

To date many problems occur in mixed motives due to standard conjectures.

The best understanding to date is due to Voevodsky's motivic cohomology~\cite{MWV2006}

%%%%%%%%%%%%%%%%%%%%%%%%%%%%%%%%%%%%%%%%%%%%%%%%%%%%%%%%%%%%%%%%%%%%%%
\endinput

Any text after an \endinput is ignored.
You could put scraps here or things in progress.

%% The following is a directive for TeXShop to indicate the main file
%%!TEX root = diss.tex
\chapter*{Conclusion}
\addcontentsline{toc}{chapter}{\protect{}Conclusion}
\label{ch:Conclusion}

Categorification?

Derived version?

\endinput

Any text after an \endinput is ignored.
You could put scraps here or things in progress.


%    3. Notes
%    4. Footnotes

%    5. Bibliography
\begin{singlespace}
\raggedright
\bibliographystyle{alpha}
\bibliography{biblio}
\end{singlespace}

\appendix
%    6. Appendices (including copies of all required UBC Research
%       Ethics Board's Certificates of Approval)
%\include{reb-coa}	% pdfpages is useful here
%\include{appendix}

\backmatter
%    7. Index
% See the makeindex package: the following page provides a quick overview
% <http://www.image.ufl.edu/help/latex/latex_indexes.shtml>


\end{document}