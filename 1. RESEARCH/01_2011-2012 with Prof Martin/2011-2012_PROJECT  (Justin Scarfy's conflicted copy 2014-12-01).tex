\documentclass[12pt, amsfont]{amsart}
\usepackage{amssymb,amsthm,geometry}
\usepackage{amsrefs}

    \usepackage[pdftex]{graphicx}
    \pdfcompresslevel=9
    \usepackage[colorlinks=true, pdfstartview=FitH, linkcolor=blue,
        citecolor=blue, urlcolor=blue]{hyperref}
    

\geometry{left=1in, right=1in, top=.75in, bottom=.75in}
\title[Comparative prime number theory: a survey]{Comparative Prime Number Theory:\\ 
    A survey}
\author{Greg Martin}
\author{Justin Scarfy}
\address{University of British Columbia\\ Department of Mathematics \\ Room 121\\ 1984 Mathematics Road\\Vancouver, BC Canada V6T 1Z2}
\email{gerg@math.ubc.ca} 
\email{scarfy@ugrad.math.ubc.ca}

\thanks{The second of us, an undergraduate student at the University of British Columbia, is very grateful for Professor Greg Martin's  introduction to this fascinating topic, and for his guidance and continuous encouragement throughout this project.}

\subjclass[2010]{11N13 (11Y35)}
\newcommand{\h}{\frac 1 2}
\renewcommand{\mod}[1]{{\ifmmode\text{\rm\ (mod~$#1$)}\else\discretionary{}{}{\hbox{ }}\rm(mod~$#1$)\fi}}
\newcommand{\Li}{\rm Li}
\newcommand{\Var}{\rm Var}
\newcommand{\li}{\rm li}
\newcommand{\ord}{\rm ord}
\newcommand{\N}{{\mathbb N}}
\newcommand{\Z}{{\mathbb Z}}
\newcommand{\Q}{{\mathbb Q}}
\newcommand{\R}{{\mathbb R}}
\newcommand{\C}{{\mathbb C}}
\newcommand{\E}{{\mathbb E}}
\newcommand{\A}{{\mathcal A}}
\newcommand{\Prob}{{\mathbb P}}
\newcommand{\bigO}[1]{{\cal O}\left(#1\right)}
\newcommand{\cA}{{\cal A}}
\newcommand{\cB}{{\cal B}}
\newcommand{\cK}{{\cal K}}
\newcommand{\p}{\mathfrak p}




\numberwithin{equation}{section}%{subsection} %sets equation numbers <chapter>.<section>.<index>
\newtheorem{thm}{Theorem}[subsection]
\newtheorem{lem}[thm]{Lemma}
\newtheorem{cor}[thm]{Corollary}
\newtheorem{cnj}[thm]{Conjecture}
\newtheorem{dfn}[thm]{Definition}
\newtheorem{rmk}[thm]{Remark}
\newtheorem{xmp}[thm]{Example}
\newtheorem{pbm}[thm]{Problem}

\numberwithin{thm}{section} %sets equation numbers <chapter>.<section>.<index>
\begin{document}
\begin{abstract}
Comparative prime number theory is the study of the {\em{discrepancies}} of distributions when we compare the number of primes in different residue classes. This work presents a list of the problems being investigated in comparative prime number theory, their generalizations, and an extensive list of references on both historical and current progresses. 
\end{abstract}
\maketitle
%1. Introduction
\section{Introduction}\label{intro section}
%\numberwithin{thm}{subsection} %sets equation numbers <chapter>.<section>.<index>
In a letter between P. Chebyshev and M. Fuss, dated 1853~\cite{1853.Chebyshev}, the former indicates (without proof):
For a positive continuous decreasing function $f$, the series
\begin{equation} \label{eq.cheb} %eq.cheb
\sum_{p \; {\rm {odd\;prime}}} (-1)^{\frac {(p+1)}{2}} f(p): = f(3)-f(5)+f(7)+f(11)-f(13)-f(17)+\hdots
\end{equation}
diverges. In particular, when $f(x)=e^{-10x}$, the series \eqref{eq.cheb} tends to infinity.
The significance of this assertion is to say that there {\em{should be}} more primes in the residue class 3 modulo 4 than in the residue class 1 modulo 4, despite the fact that Dirichlet in 1837 proved that for any $a, k$ with $(a, k)=1$ there are infinitely many primes $p$ with $p\equiv a \mod k$.
Hardy, Littlewood, and Landau in 1918 proved that Chebyshev's assertion is equivalent to the problem of whether the function 
\begin{equation*}%L
L(s):= \sum_{n=0}^{\infty} \frac{(-1)^n}{(2n+1)^s} \; \; \; \; \; (s=\sigma + it)
\end{equation*}
vanishes or not in the half-plane $\sigma > \h$. (The necessity was shown by Landau~\cite{1918.Landau.1} and the sufficiency by Hardy--Littlewood~\cite{1918.Littlewood}, with a simpler proof by Landau~\cite{1918.Landau.2}.)

However, Littlewood~\cite{1918.Littlewood} in 1914 showed that the number of primes in the residue class 3 modulo 4 and the number of primes in the residue class 1 modulo 4 ``race'', taking turns to be in the lead.  
On the other hand, the number of primes in the residue class 1 seems to take the lead in the race only a ``negligible'' amount of time, and this phenomenon is known as {\bf Chebyshev's bias}.  
To illustrate precisely what Littewood had proven and further developments on this topic, we need the aid of the following notations:

Throughout this paper, $p$ will always be an {\bf{odd} prime}. As usual, for a positive integer $k$ with $(k, l)=1$:
\begin{dfn}
\[ 
\pi(x; k, l) := \sum_{\substack {p\leq x \\ p\equiv l \mod k}} 1
\]
\end{dfn}
We note that $\pi(x)=\pi(x; 1, 1)$, the prime counting function up to $x$.
By the prime number theorem for arithmetic progressions, the functions $\pi(x; k, l)$ with fixed $k$ and $(l, k)=1$ are all asymptotically $x/(\varphi(k)\log x)$, where $\varphi(k)$ is the Euler's totient function. As Chebyshev investigated, the difference between the functions $\pi(x; k, l)$ for fixed $k$ exhibits interesting behaviours:
\begin{dfn}
\[
\delta_\pi(x; k, l_1, l_2):=\pi(x; k, l_1)-\pi(x; k, l_2)
\]
\end{dfn}
\begin{xmp}
In Chebyshev's case, $\delta_\pi(x; 4, 3, 1)$ is negative for the first time when $x=26$,$861$~\cite{1957.Leech}, and $\delta_\pi(x; 3, 2, 1)$ is negative for the first time at $x=608$,$981$,$813$,$029$~\cite{1977.Bays}.
\end{xmp}

\begin{xmp}
Littlewood ~\cite{1918.Littlewood} proved in 1914 that, in the above notations, $\delta_\pi(x; 4, 3, 1)$ and $\delta_\pi(x; 3, 2, 1)$ switch their signs infinitely many times.
\end{xmp}

\begin{xmp}
Phragm\'en ~\cite{1891.Phragmen} proved the existence of an unbounded sequence $x_1 < x_2 < x_3 < \cdots$ such that
\[
\frac{\pi(x_\nu; 4, 3)- \pi(x_\nu; 4, 1)}{{\sqrt {x_\nu}}/{\log x_\nu}} \rightarrow 1
\]
\end{xmp}

In their series of papers published between 1962 and 1972, S. Knapowski and P. Tur\'an ~\cite{1962.Knaposwki_1,1964.Turan_1} list a number of problems that generalized Littlewood's theorem and also attempted to compare $\pi(x; k, l_1)$ and $\pi(x; k, l_2)$, with the assumption that $l_1 \not\equiv l_2 \mod k$ and $(l_1, k)=(l_2, k)=1$:

%P 01
\begin{pbm}[Infinity of sign changes]\label{P:S-C}
For which $(l_1, l_2)$-pairs  does the function $\delta_\pi(x; k, l_1, l_2)$ change its sign infinitely often?
\end{pbm}

%P 02
\begin{pbm}[Big sign changes]\label{P:B-S}
Given $\epsilon >0$, do there exist two sequences 
\begin{align*}
x_1&<x_2<x_3<\cdots \rightarrow  \infty\\
y_1&<y_2<y_3<\cdots \rightarrow  \infty
\end{align*}
such that
\begin{align*}
\pi(x_\nu ; k, l_1)-\pi(x_\nu ; k,  l_2) &> x_\nu ^ {1/2 - \epsilon} \\
\pi(y_\nu ; k, l_1)-\pi(y_\nu ; k,  l_2) &< -y_\nu ^ {1/2 - \epsilon}?
\end{align*}
The use of the function $x^{1/2 -\epsilon}$ is motivated by the fact that if GRH for $k$ (see Conjecture~\ref{GRH} below) is true, then the inequality
\[
|\delta_\pi(x; k, l_1, l_2)|=O(x^{1/2}\log x)
\]
holds for $x\ge2$.
\end{pbm}

%%P 03
%\begin{pbm}[Localized Sign Changes]\label{P:L-S}
%Given $\epsilon >0$ small then for what function $h_k(T)>0$ can we be sure that for each $(l_1, l_2)$-pairs with and $T\geq 1$ both the inequalities 
%\begin{align*}
%\max_{T \leq x \leq T+ h_k(T)}\delta_\pi(x; k, l_1, l_2)  &> T^ {\frac 1 2 - \epsilon} \\
%\min_{T \leq x \leq T+ h_k(T)}\delta_\pi(x; k, l_1, l_2) &< -T^ {\frac 1 2 - \epsilon} 
%\end{align*}
%hold?
%\end{pbm}

%P 03
\begin{pbm}[Localized sign changes]\label{P:L-S}
Prove that there exists an $T>T_0(k)$ and suitable $A(T)<T$ such that the function $\delta_\pi(x; k, l_1, l_2)$ changes sign in the interval 
$$
A(T)\leq x \leq T.
$$
\end{pbm}

%P 04
\begin{pbm}[Localized big sign changes]\label{P:L-B}
Prove that for $T> T_0(k)$ and suitable $A(T)<T$, the functions $\delta_\pi(x; k, l_1, l_2)$ satisfy both the inequalities
\begin{align*}
\max_{A(T)\leq x \leq T}\delta_\pi(x; k, l_1, l_2) &>  \frac{T^{1/2}}{\Phi(T)}\\
\min_{A(T)\leq x \leq T}\delta_\pi(x; k, l_1, l_2) &< {-} \frac{T^{1/2}}{\Phi(T)},
\end{align*}
where $\Phi(x)$ is a positive function satisfying 
$$\lim_{x\to \infty}\frac{\log\Phi(x)}{\log x}= 0.$$
\end{pbm}


%P 05
\begin{pbm}[First sign change]\label{P:F-S}
For what function $a(k)$ can we assert that for each $(l_1, l_2)$-pair with $l_1 \neq l_2$, all functions in
$
\delta_\pi(x; k, l_1, l_2)$
vanish at least once in
$
1 \leq x \leq a(k)?
$
\end{pbm}

%P 06
\begin{pbm}[Asymptotic behaviour of sign changes]\label{P:A-B}
Let  $w_\pi(T; l_1, l_2)$ denote the number of sign changes of $\delta_\pi(x; k, l_1, l_2)$ in the interval $[1, T]$. What is the asymptotic behaviour of $w_\pi(T; l_1, l_2)$ as $T \rightarrow \infty$?
\end{pbm}

%P 07
%\begin{pbm}[Asymptotic estimation of the number of sign changes]\label{P:A-E}
%For a fixed $(l_1, l_2)$-pair, what is the asymptotic behaviour of $N_{l_1 l_2}(Y)$ as $Y \rightarrow \infty$, where $N_{l_1, l_2}(Y)$ denotes the number of integers $m \leq Y$ with 
%\[
%\pi(m ; k, l_1) \geq \pi(m ; k,  l_2)?
%\]
%\end{pbm}

%Problem 8 Race problem
\begin{pbm}[Race-problem of Shanks--R\'enyi]\label{P:P-R}
For each permutation $\{l_1, l_2, l_3, \hdots, l_{\varphi(k)}\}$ of the set of reduced residue classes modulo $k$, do there exist infinitely many integers $m$ with
\[
\pi(m; k, l_1) < \pi(m; k, l_2) < \pi(m; k, l_3) < \cdots  < \pi(m; k, l_{\varphi(k)})?
\]
G.\ G.\ Lorentz noticed the fact that comparison of primes of any two arithmetical progressions mod $k_1$ and $k_2$ $(k_1 \neq k_2)$ is not trivial in the case when
\begin{equation*}
\varphi(k_1)= \varphi(k_2)
\end{equation*}
and analogous problems occur for moduli $k_1, k_2, k_3, \hdots, k_r$ with 
\begin{equation*}
\varphi(k_1)= \varphi(k_2)= \cdots = \varphi(k_r).
\end{equation*}
\end{pbm}

\begin{dfn}
Define
\[
\Li(x):=\int_0^x\frac{dt}{\log t}.
\]
\end{dfn}

%P 09
\begin{pbm}[Littlewood generalizations]\label{P:L-G}
Do there exist infinitely many integers $m_\nu$ such that for $j=1, 2, 3, \hdots \varphi(k)$, we simultaneously have
$$
\pi(m_\nu, k, l_j)> \frac {\Li(m_\nu)}{\varphi(k)} ?
$$
and if the assertion is valid, what are the distribution-properties of the sequence~$m_\nu$?
\end{pbm}

%P 11
\begin{pbm}[Average preponderance problems]\label{P:A-P}
Denote by $N_\pi(x)$ the number of integers $n\leq x$ with the property $\delta_\pi(n; 4, 3, 1)>0$.
Does the relation
\begin{equation*}
\lim_{x \to \infty} \frac{N_\pi(x)}x = 0
\end{equation*}
hold? In other words, does the set of integers $n$ with the property $\delta_\pi(n; 4, 3, 1)>0$ have density~$0$?
\end{pbm}

In the previous problems, the number of \emph{all} primes $\leq x$ in a fixed progression occurred. One can imagine that one can much better locate relatively small intervals where the primes of some progression preponderate.
%P 12
\begin{pbm}[Strongly localized accumulation problems]\label{P:S-L}
When $T$ is sufficiently large, is it true that for suitable $T \leq U_1 < U_2 \leq 2T$, we have
\begin{equation*}
\sum_{\substack{U_1\leq p \leq U_2 \\ p \equiv l_1 \mod k}}1 - \sum_{\substack{U_1\leq p \leq U_2 \\ p \equiv l_2 \mod k}}1 > \frac{\sqrt T}{\Phi(T)}?
\end{equation*}
\end{pbm}
where $\Phi(x)$ shares the same property as in Problem~\ref{P:L-B}

%P 15
\begin{pbm}[Union problem]\label{P:U-P}
For a given modulus $k$, do there exist two disjoint subsets $A$ and $B$, consisting of the same number of reduced residue classes, such that 
\begin{equation*}
\sum_{\substack{p\in A \\ p\leq x}} 1 \geq \sum_{\substack{p\in B \\ p\leq x}} 1
\end{equation*}
for all sufficiently large $x$?
\end{pbm}

\begin{rmk}\label{l_1 and l_2}
One can expect that there are ``more'' primes in the residue class $l_1\mod k$ than $l_2 \mod k$ if and only if the number of incongruent solutions of the congruence
\begin{equation}\label{l_1}
x^2 \equiv l_1 \mod k
\end{equation} 
is less than that of the congruence
\begin{equation}\label{l_2}
x^2 \equiv l_2 \mod k.
\end{equation}
\end{rmk} 

Besides the functions $\pi(x; k, l)$, the distributions of primes in arithmetic progressions can be studied by some other functions that are easier to work with. Let $\Lambda(n)$ denote the von Mangoldt Lambda function, namely:
\begin{dfn}
\[
\Lambda(n):= 
\begin{cases}  
\log{p} & {\rm if }\;\; n=p^k\\
      0 & {\rm otherwise} 
\end{cases}
\]
\end{dfn}
And thus the following functions are studied:
%
\begin{dfn}
\begin{align*}
\psi(x; k, l) :=& \sum_{\substack {n\leq x \\ n\equiv l \mod k}} \Lambda(n) \\
\vartheta(x; k, l) :=& \sum_{\substack {p\leq x \\ p\equiv l \mod k}} {\log p}  \\
\Pi(x; k, l) :=& \sum_{\substack {n\leq x \\ n\equiv l \mod k}} \frac {\Lambda(n)}{ \log n}
\end{align*}
\end{dfn}
%Which are generalized versions of the classically studied functions in number theory:
%\begin{dfn}
%\begin{align*}
%\psi(x):=&\sum_{n\leq x}\Lambda(n)\\
%\vartheta(x):=&\sum_{p\leq x}\log p\\
%\Pi(x):=&\sum_{n\leq x}\frac{\Lambda(n)}{\log n}
%\end{align*}
%\end{dfn}
We therefore have the following corresponding analogues for $\delta_\pi(x; k, l_1, l_2)$, where the subscript is replaced by a different prime-counting function:
%
\begin{align*}
\delta_\psi(x; k, l_1, l_2):=&\psi(x; k, l_1)-\psi(x; k, l_2)\\
\delta_\vartheta(x; k, l_1, l_2):=&\vartheta(x; k, l_1) -\vartheta(x; k, l_2)  \\
\delta_\Pi(x; k, l_1, l_2):=&\Pi(x; k, l_1) -\Pi(x; k, l_2) 
\end{align*}
Further we define $w_f(T; k, l_1, l_2)$ to be the number of sign changes of $\delta_f(x; k, l_1, l_2)$ in the interval $[0, T]$ with fixed $k$, where $f \in \{\pi, \psi, \Pi, \theta\}$.

Since Chebyshev's original paper dealt with the case where each term in the sum contains a factor of $e^{-10x}$, we would able to form the {\em{mutatis mutadis}} definitions if we were to multiply a $e^{-nr}$ term to each term in the above sums:
\begin{align*}
\psi(x; k, l) \; \; \; \; \; {\rm to} \; \; \; &\sum_{n\equiv l \mod k} \Lambda(n)e^{-nr} \\
\Pi(x; k, l) \; \; \; \; \; {\rm to}  \; \; \; &\sum_{n\equiv l \mod k}\frac{ \Lambda(n)}{\log n}e^{-nr} \\
\vartheta(x; k, l) \; \; \; \; \; {\rm to} \; \; \; &\sum_{n\equiv l \mod k} \log{p}\, e^{-nr} \\
\pi(x; k, l) \; \; \; \; \; {\rm to} \; \; \;  &\sum_{n\equiv l \mod k} e^{-nr}\\
{\Li}(x) \; \; \; \; {\rm to} \; \; \;  &\int_0 ^\infty \frac{e^{-yr}}{\log y}\, dy 
\end{align*}

\begin{dfn}
The difference functions $\delta_f$ are replaced by $\Delta_F$'s:
\begin{align*}
\Delta_\psi(r; k, l_1, l_2):=&\sum_{n\equiv l_1 \mod k} \Lambda(n)e^{-nr} - \sum_{n\equiv l_2 \mod k} \Lambda(n)e^{-nr}\\
\Delta_\Pi(r; k, l_1, l_2):=&\sum_{n\equiv l_1 \mod k}\frac{ \Lambda(n)}{\log n}e^{-nr} - \sum_{n\equiv l_2 \mod k}\frac{ \Lambda(n)}{\log n}e^{-nr} \\
\Delta_\vartheta(r; k, l_1, l_2):=&\sum_{n\equiv l_1 \mod k} \log{p} e^{-nr}- \sum_{n\equiv l_2 \mod k} \log{p} e^{-nr} \\
\Delta_\pi(r; k, l_1, l_2):=& \sum_{n\equiv l_1 \mod k} e^{-nr} - \sum_{n\equiv l_2 \mod k} e^{-nr}  .
\end{align*}
Similarly  $w_f(T; k, l_1, l_2)$ is replaced by $W_F(T; k, l_1, l_2)$, for $F \in \{\psi, \Pi, \vartheta, \pi\}$.
\end{dfn}

\section{The Classical Tools}
The classical methods used to investigate the oscillatory properties of the functions $\delta_f(x; k, l_1, l_2)$ and $\Delta_F(x; k, l_1, l_2)$ for $F \in \{\psi, \Pi, \vartheta, \pi\}$ are inspired by the ones used to study the oscillatory term of the prime number theorem, namely $\pi(x) - \Li(x)$.
%\begin{dfn}
%\begin{align*}
%\Delta_1(x):&= \pi(x)-\Li (x)\\
%\Delta_2(x):&= \Pi(x)-\Li (x)\\
%\Delta_3(x):&= \theta(x) -x  \\
%\Delta_4(x):&= \psi(x)-x
%\end{align*}
%Further, let $V_i(Y)$ denote the number of sign changes of $\Delta_i(x)$ in $[2, Y]$.
%\end{dfn}
The primary tools are called the ``explicit formulas'', linking the functions $\pi(x; k, l)$ to the distribution of zeros of the Dirichlet-$L$ functions $L(s, \chi)$ in the critical strip, $0< \Re(s)<1$, for characters $\chi$ modulo $k$.%\footnote{For a full introduction of Dirichlet characters, please consult Davenport's `Multiplicaive }

The asymptotic formula for $\pi(x; k, l)$ gives
\begin{equation}\label{asym}
\pi(x; k, 1)=\Pi(x; k, l)-\frac{N_k(l)}{\varphi(k)}\frac{x^{1/2}}{\log x}+o\Bigg(\frac{x^{1/2}}{\log x} \Bigg) \;\;\;\;\;(x \to \infty)
\end{equation}
where $N_k(l)$ denotes the number of incongruent solutions of the congruence $a^2 \equiv l \mod k$.

\begin{dfn}
Denote by $D_k$ and $C_k$ the sets of all characters and all non-principal characters modulo $k$, respectively. For $\chi \in D_k$, define
\[
\Psi(x; \chi):=\sum_{n\leq x}\Lambda(n)\chi(n).
\] 
\end{dfn}

Then it follows that
\begin{align*}
\varphi(k)\Pi(x; k, l) &{}= \varphi(k)\bigg(\frac{\psi(x; k, l)}{\log x}+\int_2^x\frac{\psi(t; k, l)}{t \log^2 t}\,dt \bigg)\\
&= \sum_{\chi \in D_k}\bar{\chi}(l)\bigg(\frac{\Psi(x; \chi)}{\log x}+\int_2^x\frac{\Psi(t; \chi)}{t \log^2 t}\,dt \bigg)
\end{align*}
and by equation~\eqref{asym} we have:
\begin{align*}
\varphi(k)\delta_\pi(x; k, l_1, l_2)&=\varphi(k)\bigg(\frac{\delta_\psi(x; k, l_1, l_2)}{\log x}+\int_2^x\frac{\delta_\psi(t; k, l_1, l_2)}{t \log^2 t}\, dt \bigg)\\
&\qquad-\big(N_k(l_1)-N_k(l_2) \big)\frac{x^{1/2}}{\log x}+o\bigg(\frac{x^{1/2}}{\log x}\bigg)\\
&=\sum_{\chi \in C_k}\big(\bar{\chi}(l_1)-\bar{\chi}(l_2)\big)\bigg(\frac{\Psi(x; \chi)}{\log x}+\int_2^x\frac{\Psi(t; \chi)}{t \log^2 t}\,dt\bigg)\\
&\qquad-\big(N_k(l_1)-N_k(l_2) \big)\frac{x^{1/2}}{\log x}+o\bigg(\frac{x^{1/2}}{\log x}\bigg) \;\;\;\; (x \to \infty).
\end{align*}
Furthermore, for any $\chi \in C_k$, the well-known explicit formula tells us that
\begin{equation}\label{Psi}
\Psi(x, \chi)=-\sum_{|\Im(\varrho)|\leq x}\frac{x^\varrho}{\varrho}+O(\log^2 x) \;\;\;\; (x \geq 2),
\end{equation}
where the sum runs over zeros $\varrho$ of $L(s, \chi)$ in the critical strip.
Now we see that the zeros of $L(s, \chi)$ play an important role in determining the distribution of primes to different moduli, and the zeros with the largest real part dominate the sum in equation \eqref{Psi}. Now we introduce a few conjectures that a handful subsequent results will require:

\begin{cnj}
[Generalized Riemann Hypothesis (GRH)]\label{GRH}
For any Dirichlet character $\chi$, all zeros of $L(s, \chi)$ inside the critical strip lie on the critical line $\sigma:=\Re(s)=\h$.
\end{cnj}
\noindent This of course is a generalization of the famous ``Riemann Hypothesis'', where we take $\chi$ to be the trivial character:
\begin{cnj}
[Riemann Hypothesis (RH)]\label{RH}
Inside the critical strip, the only zeros of the Riemann zeta function 
\[
\zeta(s):=\sum_{n=1}^\infty \frac{1}{n^s}
\]
satisfy $\sigma>0$ at $\sigma=\h$.
\end{cnj}

A basic tool for proving oscillation theorems is inspired by Landau's work~\cite{1905.Landau} on the location of singularities of the Mellin transforms of a non-negative function.  Suppose $f(x)$ is real-valued and non-negative for $x$ sufficiently large. Suppose also for some real numbers $\beta<\sigma$ that the Mellin transform
\[
g(s):=\int_1^\infty f(x)x^{-s-1}\,dx
\]
is analytic for $\Re(s)> \sigma$ and can be analytically continued to the real segment $(\beta, \sigma]$. Then $g(s)$ represents an analytic function in the half-plane $\Re(s)>\beta$.
\begin{xmp}
If $f(x)=\varphi(k)\delta_\psi(x; k, l_1, l_2)$ then
\[
g(s)=g(s; k, l_1, l_2)=-\frac{1}{s}\sum_{\chi \in C_k}\big(\bar{\chi}(l_1)-\bar{\chi}(l_2) \big)\frac{L'(s, \chi)}{L(s, \chi)}
\]
for $\Re(s)>1$, with the R.H.S. providing a meromorphic continuation of $g(s)$ to the whole complex plane.
\end{xmp}
\begin{rmk} Note that the poles of $g(s)$ above (except at $s=0$) are a subset of the zeros of the functions $L(s, \chi)$.  Also, $g(s)$ always has an infinite number of poles in the critical strip. Now assuming $g(s)$ with no real poles $s> \h$ and a pole $s_0$ with $\Re(s_0)>\h$, we take $\alpha$ satisfying $\h<\alpha<\Re(s_0)$ and put $f(x)=(-1)^n\delta_\psi(x; k, l_1, l_2)+c_1x^\alpha$ for some constant\footnote{Through out this paper, $c_i$ with $i\in \N$ shall always denote a calculable positive constant.} $c_1$ and $n \in \{0, 1 \}$.  The above discussion on Mellin transforms with different $n$ and $c_1$ yield that
\[
\limsup_{x \to \infty}\frac{\delta_\psi(x; k, l_1, l_2)}{x^\alpha}=+\infty, \;\;\;\;\;\;\; \liminf_{x \to \infty}\frac{\delta_\psi(x; k, l_1, l_2)}{x^\alpha}=-\infty 
\] 
\end{rmk}\section{Classical Results by Knapowski and Tur\'an, Serie I}\label{sec.Results1}
As mentioned in Section~\ref{intro section}, Knapowski and Tur\'an exhibited a  keen interest on this topic: they listed most of the problems in Section~\ref{intro section} and attempted to answer a few of them in their series of 15 papers.
Their investigation begins with the comparison of the progressions
\begin{equation*}%\label{II.1}
n\equiv 1 \mod k \;\;{\rm and } n\equiv l \mod k,\;\;\;\; {\rm where } 
\;\; l\not\equiv 1 \mod k.
\end{equation*}
First with 
\begin{equation}\label{II.2}
k=3, 4, 5, 6, 7, 8, 9, 10, 11, 12, 19, 24,
\end{equation}
which are in fact the first few numbers known to satisfy:
\begin{cnj}[Haselgrove Condition (HC) for the modulus $k$]\label{HC}
There is a function $Z(k)$ with $0< Z(k)\leq 1$ such that no $L(s, \chi)$ with $\chi \mod k$ vanishes for $0< \sigma < 1$, $|t|\leq Z(k)$, where $s=\sigma+it$, as usual.
\end{cnj}

\begin{dfn}
Define the iterated exponential and logarithmic functions by:
\begin{align*}
e_1(x):&=e^x, \;\;\;\;\;\;\;\;\, e_\nu(x):=e_{\nu-1}\big(\exp(x)\big)\\
\log_1(x):&=\log x, \;\;\;\;\; \log_\nu(x):=\log_{\nu-1}\big(\log(x)\big)
\end{align*}
\end{dfn}
%% 1.1 of 1962.Knaposwki_2 
\begin{thm}[\cite{1962.Knaposwki_2} Theorem 1.1] For any $k$ in \eqref{II.2}
\begin{align*}
\max_{T^{{1/3}}\leq x \leq T}\delta_\psi(x; k, 1, l)>& \sqrt{T}\exp\bigg{(} -41\frac{\log(T)\log_3{(T)}}{\log_2{(T)}}\bigg)\\
\min_{T^{{1/3}}\leq x \leq T}\delta_\psi(x; k, 1, l)<&- \sqrt{T}\exp\bigg{(} -41\frac{\log(T)\log_3{(T)}}{\log_2{(T)}}\bigg)
\end{align*}
\end{thm}
This essentially solves Problem~\ref{P:L-S} for $\delta_\psi$ in Section~\ref{intro section} with the $k$'s in equation \eqref{II.2}, in the case of $l_2=1$ at least.  Since C. L. Siegel proved~\cite{1945.Siegel} that for all $L(s, \chi)$ functions with primitive characters mod $k$ there is at least one zero $\varrho^*=\varrho^*(\chi)$ in the domain
%%Siegel
\begin{equation}\label{Siegel}
\sigma \geq \h,\;\;\;\;\; |t|\leq \frac{c_2}{\log_3\big(k+e_3(1) \big)}, \;\;\;\;\;(s=\sigma +it)
\end{equation}
the above theorem follows at once from:
% %1.2 of 1962.Knaposwki_2
\begin{thm}[\cite{1962.Knaposwki_2} Theorem 1.2]\label{1.2 in II}
For a $k$ in \eqref{II.2} and a $\varrho_0=\beta_0+i\gamma_0$ with 
\begin{equation}\label{II 1.4}
\beta_0\geq \h,\;\;\;\; \gamma_0 > 0,
\end{equation}
where $\varrho_0$ is a zero of an $L(s, \chi^*)$ belonging to modulo $k$ with $\chi^*(l)\neq 1$ and $T>\max\big(c_3, e_2(10|\varrho_0|)\big)$, then the inequalities 
\begin{align*}
\max_{T^{{1/3}}\leq x \leq T}\delta_\psi(x; k, 1, l)>& T^{\beta_0}\exp\bigg{(} -41\frac{\log(T)\log_3{(T)}}{\log_2{(T)}}\bigg)\\
\min_{T^{1/3}\leq x \leq T}\delta_\psi(x; k, 1, l)< &-T^{\beta_0}\exp\bigg{(} -41\frac{\log(T)\log_3{(T)}}{\log_2{(T)}}\bigg)
\end{align*}
hold.
\end{thm}
The authors juxtapose a similar result:
\begin{thm}[\cite{1962.Knaposwki_2} Theorem 2.1]
For a $k$ in \eqref{II.2} and $T>c_4$ we have:
\begin{align*}
\max_{T^{1/3}\leq x \leq T}\delta_\Pi(x; k, 1, l)>& \sqrt{T}\exp\bigg{(} -41\frac{\log(T)\log_3{(T)}}{\log_2{(T)}}\bigg)\\
\min_{T^{1/3}\leq x \leq T}\delta_\Pi(x; k, 1, l)< &- \sqrt{T}\exp\bigg{(} -41\frac{\log(T)\log_3{(T)}}{\log_2{(T)}}\bigg)
\end{align*}
\end{thm}
and the above theorem essentially solves the case $l_2=1$ for the $k$'s in~\eqref{II.2}, for Problem~\ref{P:L-S} with $\delta_\Pi$. Now by~\eqref{Siegel} this theorem is an immediate consequence of: 
%% 2.2 of 1962.Knaposwki_2
\begin{thm}[\cite{1962.Knaposwki_2} Theorem 2.2]\label{2.2 in II}
Let $k$ be a number in~\eqref{II.2}. If $\varrho_0$ with~\eqref{II 1.4} holds is a zero of an $L(s, \chi^*)$ belonging to modulo $k$, $\chi^*(l)\neq 1$ and $T>\max\big(c_5, e_2(10|\varrho_0|)\big)$, then the inequalities 
\begin{align*}
\max_{T^{1/3}\leq x \leq T}\delta_\Pi(x; k, 1, l)>& T^{\beta_0}\exp\bigg{(} -41\frac{\log(T)\log_3{(T)}}{\log_2{(T)}}\bigg)\\
\min_{T^{1/3}\leq x \leq T}\delta_\Pi(x; k, 1, l)<&-T^{\beta_0}\exp\bigg{(} -41\frac{\log(T)\log_3{(T)}}{\log_2{(T)}}\bigg)
\end{align*}
hold.
\end{thm}
Combining Theorems~\ref{1.2 in II} and~\ref{2.2 in II} (Theorems 1.2 and 2.2 in~\cite{1962.Knaposwki_2}) yields:
%% 3.1 of 1962.Knaposwki_2
\begin{thm}[\cite{1962.Knaposwki_2} Theorem 3.1]\label{3.1 in II}
For a modulus $k$ satisfying the HC (Conjecture~\ref{HC}) and for a $\varrho_0$ with \eqref{II 1.4} holds, when 
\begin{equation*}
T>\max\bigg(c_6, e_2(10|\varrho_0|), e_2(k), e_2\Big(\frac{1}{Z(k)^3}\Big)\bigg)
\end{equation*}
we have the inequalities
\begin{align*}
\max_{T^{1/3}\leq x \leq T}\delta_\psi(x; k, 1, l)>& T^{\beta_0}\exp\bigg{(} -41\frac{\log(T)\log_3{(T)}}{\log_2{(T)}}\bigg)\\
\min_{T^{1/3}\leq x \leq T}\delta_\psi(x; k, 1, l)<&-T^{\beta_0}\exp\bigg{(} -41\frac{\log(T)\log_3{(T)}}{\log_2{(T)}}\bigg)
\end{align*}
and further
\begin{align*}
\max_{T^{1/3}\leq x \leq T}\delta_\pi(x; k, 1, l)>& T^{\beta_0}\exp\bigg{(} -41\frac{\log(T)\log_3{(T)}}{\log_2{(T)}}\bigg)\\
\min_{T^{1/3}\leq x \leq T}\delta_\pi(x; k, 1, l)<&-T^{\beta_0}\exp\bigg{(} -41\frac{\log(T)\log_3{(T)}}{\log_2{(T)}}\bigg)
\end{align*}
\end{thm}
hold.\\
This concludes their answer to Problems~\ref{P:L-S} for $\delta_\pi, \delta_\Pi$ and $\delta_\psi$, at least for the case $l_2=1$.
\vskip12pt
The authors move on to other problems and their variations, where they first give, as a consequence of Theorem~\ref{3.1 in II} by taking $\varrho^*$ satisfying the condition of Siegel's Theorem~\eqref{Siegel}:
%% 4.1 of 1962.Knaposwki_2
\begin{thm}[\cite{1962.Knaposwki_2} Theorem 4.1]
In the interval 
\[
0<x<\max \bigg(c_7, e_2(k), e_2\Big(\frac{1}{Z(k)^3} \Big) \bigg)
\]
the functions $\delta_\psi(x; k, 1, l)$ and $\delta_\Pi(x; k, 1, l)$ certainly change their sign, when $k$ satisfies the HC (Conjecture~\ref{HC}). 
Here
\begin{equation*}
c_7=\max\big(c_6, e_2(10(1+c_2))\big)
\end{equation*}
\end{thm}
The above theorem gives answers to Problem~\ref{P:F-S} for $\delta_\psi$ and $\delta_\Pi$, and the authors conjecture the ``best'' interval is 
\[
0<x<\exp(c_8k)
\]
%% 4.2 of 1962.Knaposwki_2
Then they appeal to some answers for Problem~\ref{P:A-B} regarding $w_\psi$ and $w_\Pi$, giving:
\begin{thm}[\cite{1962.Knaposwki_2} Theorem 4.2]
If for a $k$ holing the HC (Conjecture~\ref{HC}) and with 
\[
T>\exp\Bigg( c_9 \bigg(\exp(k)+\exp\Big(\frac{1}{Z(k)^3} \Big) \bigg)^2\Bigg)
\]
the inequalities
\begin{align*}
w_\psi(T; k, 1, l_1) &> \frac{1}{8\log 3}\log_2{T} \\
w_\Pi(T; k, 1, l_1) &> \frac{1}{8\log 3}\log_2{T}
\end{align*}
hold.
\end{thm}
%% 4.3 of 1962.Knaposwki_2
As an obvious consequence of Theorem~\ref{3.1 in II}, they assert,
\begin{thm}[\cite{1962.Knaposwki_2} Theorem 4.3]
Let $L(s, \chi^*)$ be an arbitrary $L$-function mod $k$ ($k$ holding HC (Conjecture~\ref{HC})), and for
\begin{equation*}
T> \max\bigg(c_6, e_2(k), e_2\Big(\frac{1}{Z(k)^3}\Big)\bigg),
\end{equation*}
if $l$ is such that $\chi^*(l)\neq 1$ , then $L(s, \chi)$ does not vanish in the domain 
$$
\sigma \geq 41\frac{\log_3{T}}{\log_2{T}} +\frac{1}{\log T}\max_{T^{{1/3}} \leq x \leq T}\log \delta_\psi(x; k, 1, l)
$$
$$
|t|\leq \frac{1}{10} \log_2{T} -1.
$$
\end{thm}
\vskip12pt
Then they give partial answers to Problem~\ref{P:L-G}:
%% 5.1 of 1962.Knaposwki_2
\begin{thm}[\cite{1962.Knaposwki_2} Theorem 5.1]
If $k$ is one of the moduli \eqref{II.2} then for $T> c_{10}$, we have the inequalities
\begin{align}
\label{II.5.1.a}
\max_{\exp(\log_3^{1/130}T ) \leq x \leq T} \frac{\delta_\pi(x; k, 1, l)}{\bigg( \frac{\sqrt{x}}{\log x}\bigg)}  &> \frac{1}{100} \log_5 T\\
\label{II.5.1.b}
\min_{\exp(\log_3^{1/130}T ) \leq x \leq T} \frac{\delta_\pi(x; k, 1, l)}{\bigg( \frac{\sqrt{x}}{\log x}\bigg)}  &< - \frac{1}{100} \log_5 T
\end{align}
\end{thm}
%% 5.2 of 1962.Knaposwki_2
\begin{thm}[\cite{1962.Knaposwki_2} Theorem 5.2]
If HC (Conjecture~\ref{HC}) holds for a $k$ and
\begin{equation} 
\label{II.5.2}
T>\max\bigg( e_5(c_{11} k), e_2\Big(\frac{1}{Z(k)^3}\Big)\bigg)
\end{equation}
then the inequalities~\ref{II.5.1.a} and~\ref{II.5.1.b} hold.
\end{thm}
and further for Problem~\ref{P:F-S}, there is
%% 5.3 of 1962.Knaposwki_3
\begin{thm}[\cite{1962.Knaposwki_2} Theorem 5.3]
If HC (Conjecture~\ref{HC}) holds for a $k$ then the interval 
\begin{equation*} 
1\leq x \leq \max \bigg( e_5(c_{11} k), e_2\Big(\frac{1}{Z(k)^3}\Big)\bigg)
\end{equation*}
contains at least a zero of $\delta_\pi(x; k, 1, l)$.
\end{thm}

\vskip12pt
As for Problem~\ref{P:P-R}, they gave:
%% 5.4 of 1962.Knaposwki_2
\begin{thm}[\cite{1962.Knaposwki_2} Theorem 5.4]
If HC (Conjecture~\ref{HC}) holds for a $k$ and for $T$ with \eqref{II.5.2}, the inequalities
\begin{align*}
\max_{\exp(\log_3^{1/130}T )} \frac{\log x}{\sqrt{x}} \bigg\{ \pi(x; k, 1) - \frac{1}{\varphi(x)}\pi(x)
%%{\varphi(k)-1}\mathop{\sum\nolimits_l}_{\substack{(l, k)=1\\ l\neq 1}}\pi(x; k, l)
\bigg\} &> \frac{1}{200} \log_5{T}\\
\min_{\exp(\log_3^{1/130}T )} \frac{\log x}{\sqrt{x}} \bigg\{ \pi(x; k, 1) - \frac{1}{\varphi(x)}\pi(x)
%%{\varphi(k)-1}\mathop{\sum\nolimits_l}_{\substack{(l, k)=1\\ l\neq 1}}\pi(x; k, l)
\bigg\} &<- \frac{1}{200} \log_5{T}.
\end{align*}
hold.
\end{thm}
%%
%%Greg says: try
%%\[
%%\mathop{\sum\nolimits_l}_{\substack{(l, k)=1\\ l\neq 1}} blah
%%\]
\vskip12pt
Revisiting Problem~\ref{P:A-B}, the authors present:
%%1.1 of of 1962.Knaposwki_3
\begin{thm}[\cite{1962.Knaposwki_3} Theorem 1.1]\label{1.1 in III}
For $T>c_{12}$ we have for the moduli $k$ in \eqref{II.2} the inequality 
\begin{equation*}
w_\pi(T; k, 1, l)> c_{13} \log_4 T
\end{equation*}
holds.
\end{thm}
and more generally,
%%1.2 of of 1962.Knaposwki_3
\begin{thm}[\cite{1962.Knaposwki_3} Theorem 1.2]\label{1.2 in III}
If $k$ satisfies the HC (Conjecture~\ref{HC}) holds then for 
\begin{equation*}
T>\max \bigg(e_4(k^{c_{14}}), e_2\Big(\frac{2}{Z(k)^3} \Big)\bigg)
\end{equation*}
we have the inequality 
\begin{equation*}
w_\pi(T; k, 1, l)>k^{-c_{14}}\log_4 T.
\end{equation*}
\end{thm}
As a consequence they show
%%1.3 of of 1962.Knaposwki_3
\begin{thm}[\cite{1962.Knaposwki_3} Theorem 1.3]
If for a $k$ holding the HC (Conjecture~\ref{HC})  then in the interval
\[
0 < x \leq \max  \bigg(e_4(k^{c_{14}}), e_2\Big(\frac{2}{Z(k)^3} \Big)\bigg)
\]
there exists at least one $x$ such that 
$ \delta_\pi (T; k, 1, l)=0 $ changes its sign
for all $l\not\equiv 1 \mod k$.
\end{thm}

They also prove the analogous theorems of Theorem~\ref{1.1 in III} and Theorem~\ref{1.2 in III} with the following definition, contributing to Problem~\ref{P:P-R}:
%%dfn
\begin{dfn}
For 
\begin{equation}\label{difference}
\pi(x; k, 1)- \frac{1}{\varphi(k)-1}\sum_{\substack{(l, k) =1 \\ l \neq 1 \\ l}} \pi(x; k, l)
\end{equation}
we denote the number of sign-changes in this function for $x \in (0,  T] $ by $S_k(T)$
\end{dfn}
%%1.4 of of 1962.Knaposwki_3
\begin{thm}[\cite{1962.Knaposwki_3} Theorem 1.4]
If $k$ satisfies the HC (Conjecture~\ref{HC}) then for 
$$
T>\max \bigg(e_4(k^{c_{14}}), e_2\Big(\frac{2}{Z(k)^3} \Big)\bigg)
$$
the inequality 
$$
S_k(T)> k^{-c_{14}}\log_4{T}
$$
holds, and the same result holds if we changed formula~\eqref{difference} to 
\[\pi(x; k, 1)-\frac{1}{\varphi(x)}\pi(x)
\;\;\;\;{\rm and}\;\;\;\;\pi(x; k, 1)-\frac{1}{\varphi(x)}\Li(x)
\]
\end{thm}
\vskip12pt
The authors again revisit Problems~\ref{P:A-B} and~\ref{P:F-S}, armed with the above theorems, and assuming that equations \eqref{l_1} and \eqref{l_2} having exactly the same number of solutions, whose significance we speculated in Remark~\ref{l_1 and l_2}:
%%2.1 of  1962.Knaposwki_3
\begin{thm}[\cite{1962.Knaposwki_3} Theorem 2.1]\label{2.1 of III}
For the $k$'s in \eqref{II.2} and $l$'s satisfying the condition \eqref{l_1} and \eqref{l_2}  having the same number of solutions with $l_2=1$ then for $T> c_{14}$ the inequalities
\begin{align*}
\max_{T^{1/3}\leq x \leq T}\delta_\pi(x; k, 1, l)>& \sqrt{T}\exp\bigg{(} -41\frac{\log(T)\log_3{(T)}}{\log_2{(T)}}\bigg)\\
\min_{T^{1/3}\leq x \leq T}\delta_\pi(x; k, 1, l)<&-\sqrt{T}\exp\bigg{(} -41\frac{\log(T)\log_3{(T)}}{\log_2{(T)}}\bigg)
\end{align*}
hold
\end{thm}

which is a special case of:
\begin{thm}[\cite{1962.Knaposwki_3} Theorem 2.2]\label{2.2 of III}
For the modulus $k$ in \eqref{II.2}, for a $l$ satisfying~\eqref{l_1} and~\eqref{l_2} with $l_2=1$, of $\varrho_0=\beta_0+i\gamma_0$ with $\beta_0 \geq \h$ such that $L( \varrho_0, \chi) = 0$ with $\chi(l)\neq 1$, then we have for
\[
T> \max\big(c_{15}, e_2(10|\varrho_0|)\big)
\] 
the inequalities 
\begin{align*}
\max_{T^{1/3}\leq x \leq T}\delta_\pi(x; k, 1, l)>& T^{\beta_0}\exp\bigg{(} -41\frac{\log(T)\log_3{(T)}}{\log_2{(T)}}\bigg)\\
\min_{T^{1/3}\leq x \leq T}\delta_\pi(x; k, 1, l)<&-T^{\beta_0}\exp\bigg{(} -41\frac{\log(T)\log_3{(T)}}{\log_2{(T)}}\bigg)
\end{align*}
hold.
\end{thm}
due to Siegel's Theorem~\eqref{Siegel}, as before.
%%3.1 of of 1962.Knaposwki_3
\begin{thm}[\cite{1962.Knaposwki_3} Theorem 3.1]
If for a $k$ the HC (Conjecture~\ref{HC}) holds and $l$ satisfies \eqref{l_1} and \eqref{l_2} with $l_2=1$ then for
\[
T>\max\bigg(c_{16}, e_2(k), e_2\Big(\frac{1}{Z(k)^3} \Big)\bigg),
\]
the inequalities in Theorem~\ref{2.1 of III} hold.
\end{thm}

%%3.2 of of 1962.Knaposwki_3
\begin{thm}[\cite{1962.Knaposwki_3} Theorem 3.2]
If for a $k$ the HC (Conjecture~\ref{HC}) holds and $l$ satisfting~\eqref{l_1} and~\eqref{l_2} with $l_2=1$, and if further $\varrho = \beta_0 + i\gamma$, $\beta_0 \geq \h$ is a zero for an $L(s, \chi)$ with $\chi(l) \neq 1$, then for 
\begin{equation*}
T>\max \bigg( c_{16}, e_2(k), e_2\Big(\frac{1}{Z(k)^3} \Big), e_2(10|\varrho|) \bigg),
\end{equation*}
the inequalities in Theorem~\ref{2.2 of III} hold.
\end{thm}

Now turning in to Problem~\ref{P:A-B} again:
%%3.3 of of 1962.Knaposwki_3
\begin{thm}[\cite{1962.Knaposwki_3} Theorem 3.3]
For $T>c_1$ and $k$'s in the moduli \eqref{II.2} and $l's$ satisfying \eqref{l_1} and \eqref{l_2}, then the inequality
\[
w_\pi(T; k, 1, l)> c_{17} \log_2 T
\]
holds.
\end{thm}
and 
%%3.4 of of 1962.Knaposwki_3
\begin{thm}[\cite{1962.Knaposwki_3} Theorem 3.4]
If for a $k$ the HC (Conjecture~\ref{HC}) holds and 
\[
T>\max\bigg(c_{18}, e_2(2k), e_2\Big(\frac{2}{Z(k)^3}\Big) \bigg)
\]
$l$ satisfies \eqref{l_1} and \eqref{l_2} then 
\[
w_\pi (T; K, 1, l) > c_{17} \log_2 T
\]
\end{thm}


The authors then delve into the general case of Problems~\ref{P:L-B} and~\ref{P:A-B} with $k = 8$ and $5$.
%% IV 1.1
\begin{thm}[\cite{1963.Knaposwki_4} Theorem 1.1]
For $T>c_{18}$ and for all pairs $l_1$ and $l_2 $ with $l_1\neq l_2$ among the numbers 3, 5, 7 $\mod 8$, we have
\begin{equation*}
\max_{T^{{1/3}}\leq x \leq T}\delta_\pi(x; 8, l_1, l_2) > \sqrt{T} \Big(-23 \frac{\log T \log_3 T}{\log_2 T}\Big)
\end{equation*}
\end{thm}

%% IV 1.2
\begin{thm}[\cite{1963.Knaposwki_4} Theorem 1.2]\label{1.2 of IV}
For $T>c_{18}$, the inequality 
\begin{equation*}
w_\pi(T; 8, l_1, l_2)>c_{19}\log_2 T
\end{equation*}
holds if only $l_1 \neq l_2$ among $3, 5, 7$.
\end{thm}


Since the congruence
\[
x^2\equiv l \mod 8, \; \; \; \; \; l\not\equiv 1 \mod 8
\]
is not solvable, it implies that Theorem~\ref{1.2 of IV} is a consequence of 

%% IV 2.1
\begin{thm}[\cite{1963.Knaposwki_4} Theorem 2.1]
For $T>c_{20}$ and all pairs $l_1 \neq l_2$ among the numbers $3, 5, 7$ we have
\[
\max_{T^{{1/3}}\leq x \leq T}\delta_\Pi(x; 8, l_1, l_2) >\sqrt{T} \exp\Big(-23 \frac{\log T \log_3 T}{\log_2 T}\Big)
\]
\end{thm}
and with slight modifications we obtain:
%% IV 2.2
\begin{thm}[\cite{1963.Knaposwki_4} Theorem 2.2]
For $T>c_{20}$ and all pairs $l_1 \neq l_2$ among the numbers $3, 5, 7$ we have
\[
\max_{T^{{1/3}}\leq x \lq T}\delta_\psi(x; 8, l_1, l_2) >\sqrt{T} \exp\Big(-23 \frac{\log T \log_3 T}{\log_2 T} \Big)
\]
\end{thm}
As an corollary we have:
%%IV 2.3
\begin{thm}[\cite{1963.Knaposwki_4} Theorem 2.3]
For $T>c_{20}$ and all pairs $l_1 \neq l_2$ among the numbers $3, 5, 7$ we have
\begin{align*}
w_\psi(T; 8, l_1, l_2)> &\log_2 T \\
w_\Pi(T; 8, l_1, l_2)> &\log_2 T
\end{align*}
\end{thm}

continuing the study of the general cases, this time assuming  ``finite'' GRH (Conjecture~\ref{GRH}) Conjecture:
Problem~\ref{P:L-S} for $\delta_\psi(x; k, l_1, l_2)$ and $\delta_\Pi(x; k, l_1, l_2)$:
%%V 1.1
\begin{thm}[\cite{1963.Knaposwki_5} Theorem 1.1]\label{1.1 of V}
Supposing the truth of the ``finite'' GRH (Conjecture~\ref{GRH}), which says no $L(s, \chi)$ vanishes for a sufficiently large $c_{21} \geq 1$  
\begin{equation}\label{V 1.3}
\sigma >\h,  \; \; |t| \leq c_{21}k^{10},
\end{equation}
moreover also for 
\begin{equation}\label{V 1.4}
\sigma = \h, \;\; |t|\leq A(k)
\end{equation}
with $A(k)$ positive, $c_{22}$ sufficiently large, for
\begin{equation}\label{V 1.5}
T>\max\Bigg\{e_2(c_{22}k^{20}), \exp\bigg( 2\exp\Big(\frac{1}{A(k)^3}\Big)+c_{22}k^{20}\bigg)\Bigg\}
\end{equation}
we have for $l_1\neq l_2$ the inequalities:
\begin{align*}
\max_{T^{{1/3}}\leq x \leq T}\delta_\psi(x; k, l_1, l_2)>& \sqrt{T}\exp\bigg(-44\frac{\log T \log_3 T}{\log_2 T}\bigg) \\
\max_{T^{{1/3}}\leq x \leq T}\delta_\Pi(x; k, l_1, l_2)>& \sqrt{T}\exp\bigg(-44\frac{\log T \log_3 T}{\log_2 T}\bigg)
\end{align*}
\end{thm}


%%V 1.2
\begin{thm}[\cite{1963.Knaposwki_5} Theorem 1.2]
By the above theorem, both of $\delta_\psi(x; k, l_1, l_2)$ and  $\delta_\Pi(x; k, l_1, l_2)$ have a sign change in the interval $[T^{{1/3}},  T]$ whenever $T$ satisfies \eqref{V 1.5}, then we get at once:\\
For 
\begin{equation*}
T>\max\Bigg\{\exp\big(9\exp(2c_{22} k^{20}\big), \exp\bigg( 72\exp\Big(\frac{2}{A(k)^3}\Big)+18c_{22}^2k^{40}\bigg)\Bigg\}
\end{equation*} 
the inequalities 
\begin{align*}
w_\psi(T; k, l_1, l_2)&> \frac{\log_2 T}{2\log 3} \\
w_\Pi(T; k, l_1, l_2) &> \frac{\log_2 T}{2\log 3}
\end{align*}
hold.
\end{thm}

%%V Remark
\begin{rmk}
We note that if $l_1$ and $l_2$ are such that none of \eqref{l_1} and \eqref{l_2} are solvable, then it follows from Theorem~\ref{1.1 of V} (with $c_{22}$ being replaced by a larger constant), that for
\[
T>\max\Bigg\{e_2\big(c_{23}k^{20}\big), \exp\bigg(2\exp\bigg(\frac{1}{A(k)^3} \bigg)+c_{24}k^{40}\bigg)\Bigg\}
\] 
the inequality
\[
\max_{T^{{1/3}}\leq x \leq T}\delta_\pi(x; k, l_1, l_2)>\sqrt{T}\exp\bigg(-45\frac{\log{T}\log_3{T}}{\log_2{T}} \bigg)
\]
holds.
\end{rmk}

Returning to Problem~\ref{P:P-R} with slight variations in the question:
%%V 1.2
\begin{thm}[\cite{1963.Knaposwki_5} Theorem 3.1]
Supposing the truth of finite GRH (Conjecture~\ref{GRH}), we have for each $(l, k)=1$ and
\begin{equation*}
T>\max\Bigg\{e_2(c_{22}k^{20}), \exp\bigg(2\exp\Big(\frac{1}{A(k)^3}\Big)+c_{23}k^{20}\bigg)\Bigg\}
\end{equation*}
both the inequalities
\begin{align*}
\max_{T^{1/3}\leq x \leq T}\bigg\{\Pi(x, k, l)-\frac{1}{\varphi(k)}\Pi(x)\bigg\}>&\sqrt{T}\exp\Big(-44\frac{\log T\log_3 T}{\log_2 T}\Big) \\
\max_{T^{1/3}\leq x \leq T}\bigg\{\Pi(x, k, l)-\frac{1}{\varphi(k)}\Pi(x)\bigg\}<&-\sqrt{T}\exp\Big(-44\frac{\log T\log_3 T}{\log_2 T}\Big)
\end{align*}
hold, and the same hold if we replace $\Pi$ by $\psi$.
%%\begin{align*}
%%\max_{T^{1/3}\leq x \lq T}\bigg\{\psi(x, k, l)-\frac{1}{\varphi(k)}\psi(x)\bigg\}>&\sqrt{T}\exp\Big(-44\frac{\log T\log_3 T}{\log_2 T}\Big) \\
%%\max_{T^{1/3}\leq x \lq T}\bigg\{\psi(x, k, l)-\frac{1}{\varphi(k)}\psi(x)\bigg\}<&-\sqrt{T}\exp\Big(-44\frac{\log T\log_3 T}{\log_2 T}\Big)
%%\end{align*}
\end{thm}

\begin{rmk}
The analogous statements also hold for if we change
\[
\Pi(x, k, l)-\frac{1}{\varphi(k)}\Pi(x)
\]
to
\[
\Pi(x; k, l)-\frac{1}{\varphi(k)}\Li(x)  \;\;\;\;{\rm{and}}\;\;\;\; \psi(x; k. l)-\frac{1}{\varphi(k)}\Li(x)
\]
in the above theorem.  However main difficulties occur when trying to prove that for all $(l, k)=1$ the function
\[
\pi(x; k, l)-\frac{1}{\varphi(k)}\Li(x)
\]
changes sign infinitely often.  We speculate that this is plausible if only the congruence $x^2 \equiv l \mod k$ is not solvable, as we succeeded in proving similar results in the other cases.
\end{rmk}


\begin{thm}[\cite{1963.Knaposwki_6} Theorem 1.1]
If for a $k$ the assertions \eqref{V 1.3}  and \eqref{V 1.4} hold, then for 
\begin{equation}
T>\max\Bigg\{e_2(c_{24}k^{20}), \exp\bigg(2\exp\Big(\frac{1}{A(k)^3}\Big)+c_{24}k^{20}\bigg)\Bigg\}
\end{equation} 
and all $(l_1, l_2)$ pairs of two squares or two non-squares mod $k$, both the following inequalities hold:
\begin{align}
\max_{T^{{1/3}}\leq x \leq T}\delta_\pi(x; k, l_1, l_2)>&\sqrt{T}\exp\Big(-44\frac{\log T\log_3 T}{\log_2 T}\Big) \\
\max_{T^{{1/3}}\leq x \leq T}\delta_\pi(x; k, l_2, l_1)<-&\sqrt{T}\exp\Big(-44\frac{\log T\log_3 T}{\log_2 T}\Big) 
\end{align}
\end{thm}

now they examine how $\delta_\psi(x; k, l_1, l_2)$ changes its signs infinitely often:
\begin{thm}[\cite{1963.Knaposwki_7} Theorem 1.1]\label{VII 1.1}
 Answers Problem~\ref{P:S-C} for $\delta_\psi(x; k, l_1, l_2)$:
Under GRH (Conjecture~\ref{GRH}) for $\chi$ mod $k$, each function $\delta_\psi(x; k, l_1, l_2)$ with $l_1 \neq l_2$ changes its sign infinitely often for $1\leq x < +\infty$
\end{thm}

\begin{thm}[\cite{1963.Knaposwki_7} Theorem 1.2]\label{VII 1.2}
Regarding Problem~\ref{P:F-S} for the case of $\delta_\psi(x; k, l_1, l_2)$:
First sign change:
For all $k$'s satisfying the HC (Conjecture~\ref{HC}), all functions $\delta_\psi(x; k, l_1, l_2)$ change their sign in the interval 
\begin{equation*}
1\leq x \leq \max\bigg( e_2(k^{c_{25}}), e_2\Big( \frac{2}{Z(k)^3} \Big) \bigg)
\end{equation*}
with a sufficiently large $c_{25}$
\end{thm}

Both of Theorems~\ref{VII 1.1} and~\ref{VII 1.2} easily follows from
\begin{thm}[\cite{1963.Knaposwki_7} Theorem 1.3]
For the $k$'s satisfying the HC (Conjecture~\ref{HC}), all functions $\delta_\psi(x; k, l_1, l_2)$ change their sign in the interval 
\begin{equation*}
\omega \leq x \leq e^{2\sqrt{\omega}}
\end{equation*}
only if
\begin{equation*}
\omega \geq \max \bigg( \exp(k^{c_{26}}), \exp\Big(\frac{2}{Z(k)^3}\Big)\bigg)
\end{equation*}
for a sufficiently large $c_{26}$.
\end{thm}

Finally we have some unconditional results at the end of serie 1, for $k=8$
\begin{thm}[\cite{1963.Knaposwki_8} Theorem 1.1]\label{VIII 1.1}
If $0<\delta< c_{27}$, then for $l_1\not\equiv l_2 \not\equiv 1 \mod 8$ the inequality
\begin{equation}
\max_{\delta\leq x \leq \delta^{\frac 1 3}} 
\Delta_\vartheta(x; 8, l_1, l_2)> \frac{1} {\sqrt\delta} \exp \Bigg(-22\frac {\log\big( 1/\delta\big) \log_3\big( 1/\delta)} {\log_2\big( 1/\delta \big)} \Bigg)
\end{equation}
holds unconditionally, and since $l_1$ and $l_2$ can be interchanged,
\begin{equation*}
\max_{\delta\leq x \leq \delta^{\frac 1 3}} 
\Delta_\vartheta(x; 8, l_1, l_2)<- \frac{1} {\sqrt\delta} \exp \Bigg(-22\frac {\log\big( 1/\delta\big) \log_3\big( 1/\delta)} {\log_2\big( 1/\delta \big)} \Bigg)
\end{equation*}
also holds.
\end{thm}
For the case $l_1=1$ they show
\begin{thm}[\cite{1963.Knaposwki_8} Theorem 1.2]\label{VIII 1.2}
If for an $l \not\equiv 1 \mod 8$ 
\[
\lim_{x\to +0}\Delta_\vartheta(x; k, 1, l) = -\infty
\]
then no $L(s, \chi)$-function mod 8 with $\chi(x) \neq 1$ can vanish for $\sigma > \h$
\end{thm}
Further they prove:
\begin{thm}[\cite{1963.Knaposwki_8} Theorem 1.3]\label{VIII 1.3}
If no $L(s, \chi)$ functions $\mod 8$ with  $\chi \in C_k$ vanish for $\sigma > \h$, then for all $l\not\equiv 1\mod 8$ we have
If for an $l \not\equiv 1 \mod 8$ 
\[
\lim_{x\to  +0}\Delta_\vartheta(x; k, 1, l) = -\infty
\]
\end{thm}

\begin{rmk}
Theorem~\ref{VIII 1.2} can be shown by mimicking Landau's argument~\cite{1918.Landau.1} with slight modifications, and Theorem~\ref{VIII 1.3} by Hardy-Littlewood-Landau's argument~\cite{1918.Landau.1}~\cite{1918.Landau.2}, and \cite{1918.Littlewood}. On the other hand, of course Theorem~\ref{VIII 1.1} more difficult to show. Since the congruences
\[
x^2\equiv l \mod 8 \;\;\;\;l=3, 5, 7
\]  
and hence
\[
\max_{\delta \leq x \leq \delta^{\frac 1 3}}\sum_{\substack{p, \nu \\ \nu \geq 3}}\log p\cdot\exp(-p^\nu x)=O\bigg(\frac{1}{\delta^{{1/3}}}\log^2\frac{1}{\delta} \bigg)
\]
Theorem~\ref{VIII 1.1} is equivalent to the inequality
\begin{equation*}
\max_{\delta\leq x \leq \delta^{\frac 1 3}} 
\Delta_\vartheta(x; 8, l_1, l_2)> \frac{1} {\sqrt\delta} \exp \Bigg(-22\frac {\log\big( 1/\delta\big) \log_3\big( 1/\delta)} {\log_2\big( 1/\delta \big)} \Bigg)
\end{equation*}
\end{rmk}


\begin{thm}[\cite{1963.Knaposwki_8} Theorem 1.4]
If  $0<\delta< c_{28}\leq 1$, then for $l\not\equiv 1 \mod8 $ the following inequalities hold:
\begin{align*}
\max_{\delta\leq x \leq \delta^{\frac 1 3}}  \Delta_{\psi}(x; 8, 1, l)&>\frac{1}{\sqrt\delta}\exp\ \Bigg(-22\frac {\log\big( 1/\delta\big) \log_3\big( 1/\delta)} {\log_2\big( 1/\delta \big)} \Bigg) \\
\min_{\delta\leq x \leq \delta^{\frac 1 3}}  \Delta_{\psi}(x; 8, 1, l)&<-\frac{1}{\sqrt\delta}\exp\ \Bigg(-22\frac {\log\big( 1/\delta\big) \log_3\big( 1/\delta)} {\log_2\big( 1/\delta \big)} \Bigg)
\end{align*}
\end{thm}

%%%%%%%%
%%%%%ZWEI
%%%%%%%%

\section{Classical Results by Knapowski and Tur\'an, Serie II}\label{sec.Results2}
\begin{thm}[\cite{1964.Turan_1}]
Let $k$ fulfill the HC (Conjecture~\ref{HC}), $(l, k) =1$, and let $\varrho=\beta+i\gamma$ be an zero of an $L(s, \chi)$ with $\chi(l)\neq 1$and $\beta\geq \h$. Then with a sufficiently large $c_{29}$ for 
\begin{equation*}
T>\max\bigg(c_{29}, e_2(k), \exp\Big(\frac{1}{Z(k)}\Big), e_2(|\varrho|)\bigg)
\end{equation*}
with suitable $U_1$, $U_2$, $U_3$, $U_4$ satisfying
\begin{align*}
T\exp\big(-\log^{11/12}T\big)\leq U_1 <U_2\leq T \\
T\exp\big(-\log^{11/12}T\big)\leq U_3 <U_4\leq T
\end{align*}
the inequalities 
\begin{align*}
\sum_{\substack{n\equiv 1 \mod k\\U_1\leq n \leq U_2}}\Lambda(n)-\sum_{\substack{n\equiv l \mod k\\U_1\leq n \leq U_2}}\Lambda(n) &\geq T^{\beta}\exp\big( -\log^{11/12}T\big)\\
\sum_{\substack{n\equiv 1 \mod k\\U_1\leq n \leq U_2}}\Lambda(n)-\sum_{\substack{n\equiv l \mod k\\U_3\leq n \leq U_4}}\Lambda(n) &\leq T^{\beta}\exp\big( -\log^{11/12}T\big)
\end{align*}
hold.
\end{thm}

\begin{dfn}
To study primes in different modulo, we adapt the following notation:
\[
\varepsilon(k; p, l_1, l_2):=\begin{cases} 1&\;\;\; {\rm if}\;\;\; p\equiv l_1 \mod k\\
                                          -1&\;\;\; {\rm if}\;\;\; p\equiv l_2 \mod k\\
                                           0&\;\;\; {\rm otherwise}
                                           \end{cases}
\]
\end{dfn}

\begin{xmp}
Hardy-Littlewood-Landau's argument~\cite{1918.Landau.1}~\cite{1918.Landau.2}, and~\cite{1918.Littlewood} gave (with abundant numerical data as well) that the relation:
\[
\lim_{x \to \infty}\sum_p\varepsilon(8; p, 1, l)\log(p)\exp\bigg(-\frac{p}{x}\bigg)=-\infty \;\;\;\;(l= 3, 5, 7)
\]
holds if and only if no $L(s, \chi)$ having $\mod 8$ with $\chi \in C_k$ vanishes for $\sigma > \h$ 
\end{xmp}
A few main results when $\exp\big(-\frac{p}{x}\big)$ is replaced by $\exp\Big( -\frac{1}{r(x)}\log^2\big(\frac p x \big)\Big)$ with suitable (``small'') $r(x)$:
\begin{thm}[\cite{1964.Turan_2} Theorem I]\label{IIb 1}
For any fixed $k$ satisfies the HC (Conjecture~\ref{HC}) and for all quadratic non-residues $l \mod k$, $(l, k) =1$, the relation
\begin{equation*}
\lim_{x \to \infty} \sum_p{\varepsilon(k; p, l, 1) \log{p}\cdot  \exp\bigg( -\frac{1}{r(x)}\log^2\Big(\frac p x \Big)\bigg)} = +\infty
\end{equation*}
for every $r(x)$ satisfying $0 < r(x) \leq \log x$ is valid if and only if none of the $L$-functions $\mod k$, with $\chi \in C_k$ vanishes for $\sigma > \h$
\end{thm}
which is a special case of:
\begin{thm}[\cite{1964.Turan_2} Theorem II]\label{IIb 2} %2
For any fixed $k$ satisfies the HC (Conjecture~\ref{HC}) and for all quadratic non-residues $l \mod k$, $(l, k) =1$, the relation
\begin{equation*}
\lim_{x \to \infty} \sum_p{\varepsilon(k; p, l, 1) \log{p} \cdot  \exp\bigg( -\frac{1}{r(x)}\log^2\Big(\frac p x \Big)\bigg)} = +\infty
\end{equation*}
for every $r(x)$ satisfying $r_0< r(x) \leq \log x$ holds if and only if none of $L(s, \chi) \mod k$, with $\chi(l)\neq 1$ vanishes for $\sigma > \h$
\end{thm}
To deduce Theorem~\ref{IIb 1} from Theorem~\ref{IIb 2} we only have to note that for a character $\chi^*$, all non-residues $l$, $\chi^*(l)=1$, then $\chi^*$ is principle.  


\begin{thm}[\cite{1964.Turan_2} Theorem III]\label{IIb 3}  %IIb 3
Assume $E(k) \leq {\sqrt{\log{k}}}/{k}$, if for a $k$ satisfying the HC (Conjecture~\ref{HC}) and a prescribed quadratic non-residue $l$, no $L(s, \chi)$ with $\chi(l) \neq 1$ vanishes for $\sigma > \h$, then for suitable $c_{30}, c_{31}, c_{32}$ and
\[
r_0 = c_{30} \frac{\log k}{E(k)^2}
\] 
the inequality
\begin{equation*}
\sum_p \varepsilon(k; p, l, 1) \log{p}  \exp \bigg( -\frac{1}{r(x)}\log^2\Big(\frac{p}{x} \Big)\bigg)> c_{31} \sqrt{x}
\end{equation*}
holds whenever $r_0< r \leq \log  x$ and $x > c_{32} k^{50}.$\\
As the contribution of primes $p$ with
\[
p>x\exp\big(10\sqrt{r\log x} \big)\;\;\;\;{\rm and}\;\;\;\;p<x\exp\big(-10\sqrt{r\log x} \big)
\]
is $o\big(\sqrt{x}\big)$, this theorem asserts under the given circumstances the preponderance of primes $\equiv l \mod k$ over those $\equiv 1 \mod k$ in the interval $\Big(x\exp(-10\sqrt{r\log x}), x\exp(10\sqrt{r\log x})\Big)$.
\end{thm}

\begin{thm}[\cite{1964.Turan_2} Theorem IV]  %IIb 4
Assume $E(k) \leq {\sqrt{\log{k}}}/{k}$, and if for a $k$ satisfying the HC (Conjecture~\ref{HC}) and a quadratic non-residue $l$ there exists an $L(s, \chi)$ with $\chi(k) \neq 1$ such that 
\begin{equation}\label{IIb 5.4}
L(\varrho_0, \chi) =0, \;\;\;\;\; \varrho_0=\beta + i\gamma, \;\;\;\;\; \beta>\h, \;\;\;\;\; \gamma>0
\end{equation} 
then for all $T$ with 
\begin{equation}\label{IIb 5.5}
T>\max \bigg(c_{33}, \exp\Big( \pi^7E(k)^{-7}\Big), \exp\Big( \exp(k)\Big), \exp\Big( \big( \frac{4+\gamma^2}{\beta -\h}\big) ^{21}\Big) \bigg)
\end{equation}
then there exist integers $r_1$ and $r_2$ with
\begin{equation*}
2\log^{5/7}{T} - 4\log^{4/7}{T}\leq r_1, r_2 \leq 2\log^{5/7}{T} - 4\log^{4/7}{T}
\end{equation*}
and $x_1$, $x_2$ with 
\begin{equation*}
T \leq x_1, x_2 \leq T\exp(4 \log^{20/21}{T})
\end{equation*}
such that 
\begin{align*}
\sum_{p}\varepsilon(k; p, l, 1) \log{p}\cdot  \exp\bigg( -\frac{1}{r_1}\log^2\Big(\frac p x_1 \Big)\bigg) &\geq T^{\beta}\exp \Big(-(1+\gamma^2)\log^{5/7}{T} \Big)\\
\sum_{p}\varepsilon(k; p, l, 1) \log{p}\cdot  \exp\bigg( -\frac{1}{r_2}\log^2\Big(\frac p x_2 \Big)\bigg) &\leq -T^{\beta}\exp \Big(-(1+\gamma^2)\log^{5/7}{T} \Big)
\end{align*}
Again with the contribution of primes $p$ with $p>T \exp(\log^{41/42} T)$ and $p<T \exp(-\log^{41/42}T)$ is $o(\sqrt T)$;
\end{thm}

\begin{thm}[\cite{1964.Turan_2} Theorem V] %IIb 5
Under the conditions \eqref{IIb 5.4} and \eqref{IIb 5.5} there exist $U_1, U_2, U_3$ and $U_4$ with 
\begin{align*}
T\exp(-5\log^{20/21}T)\leq U_1< U_2\leq T\exp(5\log^{20/21}T)\\
T\exp(-5\log^{20/21}T)\leq U_3< U_4\leq T\exp(5\log^{20/21}T)
\end{align*}
such that
\begin{align*}
\sum_{\substack{U_1\leq p \leq U_2\\ p\equiv l \mod k}}{1}-\sum_{\substack{U_1\leq p \leq U_2\\ p\equiv 1 \mod k}}{1}&> T^\beta \exp\Big( (2+\gamma^2)\log^{5/7}T \Big) \\
\sum_{\substack{U_1\leq p \leq U_2\\ p\equiv l \mod k}}{1}-\sum_{\substack{U_1\leq p \leq U_2\\ p\equiv 1 \mod k}}{1}&<- T^\beta \exp\Big( (2+\gamma^2)\log^{5/7}T \Big)
\end{align*}
\end{thm}

%%6
Now Theorem~\ref{IIb 3} is a special case of:
\begin{thm}[\cite{1964.Turan_2} Theorem VI] %IIb 6
For a $k$ satisfying the HC (Conjecture~\ref{HC}) prescribe quadratic residue $l_1$ and quadratic non-residue $l_2 \mod k$ with no $L(s, \chi)$ vanishes  for $\sigma>\h$ with $\chi(l_1)\neq \chi(l_2)$, then for suitable $c_{34}, c_{35}, c_{36}$ and
\begin{equation*}
r_0=c_{34}\frac{\log k}{E(k)^2}
\end{equation*}
the inequalities
\begin{equation*}
\sum_p\varepsilon(k; p, l_2, l_1)\log{p}\exp\Big(-\frac{1}{r}\log^2\frac{p}{x} \Big) > c_{35}\sqrt{x}
\end{equation*}
holds whenever
\begin{equation*}
r_0\leq r \leq \log x
\end{equation*}
and 
\begin{equation*}
x>c_{36}k^{50}
\end{equation*}
\end{thm}

We now present the case when
\begin{equation}\label{IIIb 2.1}
l_1=1,\;\;\;\;l_2=l={\rm quadratic \; residue\; mod\;}k
\end{equation}
\begin{thm}[\cite{1965.Turan_3} Theorem I]%1
For $k$ satisfies the HC (Conjecture~\ref{HC}) and in case of equation \eqref{IIIb 2.1} and for
\begin{equation}\label{IIIb 2.2}
T>\max\bigg(c_{37}, \exp\big(4\exp(3k)\big), \exp\Big(\frac{(20\pi)^6}{E(k)^6} \Big)\bigg)
\end{equation}
there exist $x_1$, $x_2$ in the interval 
\begin{equation*}
\Big( T\exp\big(-(\log T)^{5/6}\big), T\exp\big((\log{T})^{11/15}\big) \Big)
\end{equation*}
such that for suitable 
\begin{equation*}
(2\log T)^{2/3}\leq \nu_1, \nu_2 \leq (2\log T)^{2/3}+(2\log T)^{2/5}
\end{equation*}
both the inequalities
\begin{align*}
\sum_p\varepsilon(k; p, l_2, l_1)\log{p}\exp\Big(-\frac{1}{\nu_1}\log^2\frac{p}{x_1} \Big) & > \sqrt{T}\exp\big(-c_{37}\log^{5/6}T\big)\\
\sum_p\varepsilon(k; p, l_2, l_1)\log{p}\exp\Big(-\frac{1}{\nu_2}\log^2\frac{p}{x_2} \Big) &<- \sqrt{T}\exp\big(-c_{37}\log^{5/6}T\big)
\end{align*}
hold.
\end{thm}
This is a special case of:
\begin{thm}[\cite{1965.Turan_3} Theorem II]%2
In case \eqref{IIIb 2.1} for $k$ holding the HC (Conjecture~\ref{HC}), if $\varrho = \beta+i\gamma$ is a zero of an $L(s, \chi)$ \mod k with
\begin{equation*}
\beta\geq \h, \;\; \gamma>0 \;\;, \chi(l)\neq 1
\end{equation*}
there exist for
\begin{equation*}
T>\max \bigg(c_{38}, \exp\Big( 4\exp(3k )\Big), \exp\Big( \frac{(20\pi)^6}{E(k)^6} \Big), \exp\Big(\exp(10|\varrho|)\Big) \bigg)
\end{equation*}
$x_1, x_2$ in the interval:
\begin{equation*}
T\exp\Big(-(\log T)^{5/6}\Big) < x_1,  x_2 <  T\exp\Big((\log T)^{11/15}\Big)
\end{equation*}
such that both the inequalities
\begin{align*}
\sum_p\varepsilon(k; p, l_2, l_1)\log{p}\exp\Big(-\frac{1}{r_1}\log^2\frac{p}{x_1} \Big) & > T^{\beta}\exp\big(-c_{39}\log^{5/6}T\big)\\
\sum_p\varepsilon(k; p, l_2, l_1)\log{p}\exp\Big(-\frac{1}{r_2}\log^2\frac{p}{x_2} \Big) &<- T^{\beta}\exp\big(-c_{39}\log^{5/6}T\big)
\end{align*}
hold.
\end{thm}

\begin{thm}[\cite{1965.Turan_3} Theorem III]
For a $k$ satisfies HC (Conjecture~\ref{HC}) and in the case \eqref{IIIb 2.1} for $T$'s satisfying \eqref{IIIb 2.2} there exist numbers $U_1, U_2, U_3$ and $U_4$ with
\begin{align*}
T\exp\Big(-(\log^{6/7} T) \Big) \leq U_1 &< U_2 \leq T\exp\Big( (\log^{6/7} T) \Big)\\
T\exp\Big(-(\log^{6/7} T) \Big) \leq U_3 &< U_4 \leq T\exp\Big( (\log^{6/7} T) \Big)
\end{align*}
such that
\begin{align*}
\sum_{U_1 \leq p \leq U_2} \varepsilon(k; p, 1, l)&>\sqrt{T}\exp\bigg(-c_{40}\log^{5/6}T \bigg)\\
\sum_{U_3 \leq p \leq U_4} \varepsilon(k; p, 1, l)&<-\sqrt{T}\exp\bigg(-c_{40}\log^{5/6}T \bigg)
\end{align*}
\end{thm}

Now passing to more general cases, as we showed that more primes $\equiv l_1\mod k$ than $\equiv l_2 \mod k$ if and only if $l_1$ is an quadratic non-residue and $l_2$ is  quadratic residue $\mod k$

Let $k$ satisfy the HC (Conjecture~\ref{HC}), compare the residue classes 
\[
\equiv l_1 \mod k\;\; {\rm and}\;\; \equiv l_2 \mod k
\]
when $l_1$ and $l_2$ are both quadratic non-residues, but with more conditions: we need an $\eta$ and a small positive constant $c_{41}$ with the condition
\begin{equation}\label{IVb 1.3}
0<\eta<\min\Big(c_{41}, \Big(\frac{E(k)}{6\pi} \Big)^2 \Big)
\end{equation}
the non-vanishing of all $L(s, \chi)$ functions $\mod k$ for
\begin{equation}\label{IVb 1.4}
\sigma>\h, \;\; |t|\leq \frac{2}{\sqrt \eta}
\end{equation}
And we assume without the loss of generality that 
\begin{equation}\label{IVb 1.5}
E(k)\leq \frac{1}{k^{15}}
\end{equation}

\begin{thm}[\cite{1965.Turan_4} Theorem I]\label{IVb I}
If for $k> c_{42}$ with $c_{42}$ large and satisfying the above conditions, then for
\begin{equation*}
T>\max\bigg(c_{43}, \exp\Big( \frac{2}{\eta^4}\exp\big( \frac{1}{4}k^{10}\big)\Big) \bigg)
\end{equation*}
and for quadratic non-residue $l_1$ and $l_2$ there are $x_1, x_2, \nu_1$ and $\nu_2$ with

\begin{equation*}
T^{1-\sqrt \eta} \leq x_1, x_2 \leq T \exp(\log^{3/4}T)
\end{equation*}
and
\begin{equation*}
2\eta \log T\leq \nu_1, \nu_2 \leq 2\eta \log T +\sqrt{\log  T}
\end{equation*}
so that 
\[
\sum_{p \equiv l_1 \mod k} \log{p}\exp\Big(-\frac{1}{\nu_1}\log^2\frac{p}{x_1} \Big)-
\sum_{p \equiv l_2 \mod k} \log{p}\exp\Big(-\frac{1}{\nu_1}\log^2\frac{p}{x_1} \Big) > T^{\h -4\sqrt \eta} 
\]
\end{thm}

\begin{thm}[\cite{1965.Turan_4} Theorem II]%2
Under the assumptions of the previous Theorem~\ref{IVb I} there are $\mu_1, \mu_2, \mu_3, \mu_4$ with
\begin{align*}
T^{1 -4\sqrt \eta} \leq \mu_1&< \mu_2\leq T^{1 + 4\sqrt \eta} \\
T^{1 -4\sqrt \eta} \leq \mu_3&< \mu_4\leq T^{1 + 4\sqrt \eta}
\end{align*}
so that
\begin{align*}
\sum_{\substack{p\equiv l_1 \mod k\\ \mu_1 \leq p \leq \mu_2 }}1 - \sum_{\substack{p\equiv l_2 \mod k\\ \mu_1 \leq p \leq \mu_2 }} 1&> T^{\h - 5\sqrt \eta} \\
\sum_{\substack{p\equiv l_1 \mod k\\ \mu_3 \leq p \leq \mu_4 }}1 - \sum_{\substack{p\equiv l_2 \mod k\\ \mu_3 \leq p \leq \mu_4 }} 1&<- T^{\h - 5\sqrt \eta} 
\end{align*}
\end{thm}



%%5b
\begin{thm}[\cite{1965.Turan_5} Theorem]
 If for a $\delta$ with $0<\delta < \frac{1}{10}$  and  for 
 \begin{equation*}
 k>\max\big( c_{44}, \exp(\delta^{-20})\big)
 \end{equation*}
 where no $L(s,\chi)$ with $\chi(l)\neq 1$, mod $k$, vanishes for
 \begin{equation*}
 |s-1|\leq \h +4\delta
 \end{equation*}
 then if
 \begin{equation*}
 a>\max\big( c_{45}, \exp(k\log^3k)\big)
 \end{equation*}
 and
 \begin{equation*}
 b=\exp\big(\log^2a \cdot(\log_2 a)^3  \big)
 \end{equation*}
 then we have $x_1, x_2$ where
 $$
 a\leq x_1, x_2<b $$
  such that
 \begin{align*}
\sum_{\substack{n \leq x_1\\ n \equiv 1\mod k}} \Lambda(n)-  \sum_{\substack{n\leq x_1\\ n\equiv l\mod k}} \Lambda(n) &\geq x_1^{\h-4\delta}\\
\sum_{\substack{n \leq x_2\\ n \equiv 1\mod k}} \Lambda(n)-  \sum_{\substack{n\leq x_2\\ n\equiv l\mod k}} \Lambda(n) &\leq -x_2^{\h-4\delta}
 \end{align*}
 \end{thm}


We return to ``modified Abelian means'', i.e. to compare between the number of primes belonging to progression $\equiv l_1 \mod k$ and $\equiv l_2 \mod k$, where both $l_1$ and $l_2$ are quadratic residues $\mod k$ 
\begin{thm}[\cite{1966.Turan_6} Theorem I]\label{VI 1}
For $l_1$, $l_2$ with $(l_1, k)=(l_2, k)=1$, $l_1\not\equiv l_2 \mod k$ are both quadratic residues $\mod k$, and conditions \eqref{V 1.3}, \eqref{IVb 1.3}, \eqref{IVb 1.4} and \eqref{IVb 1.5} hold, then for every
\begin{equation*}
T>e_2(\eta^{-3})
\end{equation*}
there are $x_1, x_2$ and $\nu_1, \nu_2$ with
\begin{align*}
T^{1-\sqrt\eta}\leq x_1,& x_2 \leq T\log T\\
2\eta \log T\leq \nu_1, & \nu_2\leq 2\eta \log T+\log_2 T
\end{align*}
such that:
\begin{align*}
\sum_{p \equiv l_1 \mod k} \log{p}\exp\Big(-\frac{1}{\nu_1}\log^2\frac{p}{x_1} \Big)-
\sum_{p \equiv l_2 \mod k} \log{p}\exp\Big(-\frac{1}{\nu_1}\log^2\frac{p}{x_1} \Big) &> T^{\h -2\sqrt \eta} \\
\sum_{p \equiv l_1 \mod k} \log{p}\exp\Big(-\frac{1}{\nu_2}\log^2\frac{p}{x_2} \Big)-
\sum_{p \equiv l_2 \mod k} \log{p}\exp\Big(-\frac{1}{\nu_2}\log^2\frac{p}{x_2} \Big) &< -T^{\h -2\sqrt \eta}
\end{align*}
hold.
\end{thm}

Analogously in short intervals we have:
\begin{thm}[\cite{1966.Turan_6} Theorem II]%2
Under the assumptions of the previous Theorem~\ref{VI 1} there are $\mu_1, \mu_2, \mu_3, \mu_4$ with
\begin{align*}
T^{1 -4\sqrt \eta} \leq \mu_1&< \mu_2\leq T^{1 + 4\sqrt \eta} \\
T^{1 -4\sqrt \eta} \leq \mu_3&< \mu_4\leq T^{1 + 4\sqrt \eta}
\end{align*}
so that
\begin{align*}
\sum_{\substack{p\equiv l_1 \mod k\\ \mu_1 \leq p \leq \mu_2 }}1 - \sum_{\substack{p\equiv l_2 \mod k\\ \mu_1 \leq p \leq \mu_2 }} 1&> T^{\h - 3\sqrt \eta}, \\
\sum_{\substack{p\equiv l_1 \mod k\\ \mu_3 \leq p \leq \mu_4 }}1 - \sum_{\substack{p\equiv l_2 \mod k\\ \mu_3 \leq p \leq \mu_4 }} 1&<- T^{\h - 3\sqrt \eta}. 
\end{align*}
\end{thm}


%%7b
\begin{thm}[\cite{1972.Turan_7} Theorem]
There exist numbers $U_1, U_2, U_3 , U_4$ for $T>c_{46}$ with
\begin{align*}
\log_3 T \leq U_2 \exp(-\log^{15/16}{U_2})\leq U_1 < U_2 \leq T, \\
\log_3 T \leq U_4 \exp(-\log^{15/16}{U_4})\leq U_3 < U_4 \leq T, 
\end{align*}
such that
\begin{align*}
\sum_{\substack{U_1<p<U_2\\ p\equiv 1 \mod 4}}\log p - \sum_{\substack{U_1<p<U_2\\ p\equiv 3 \mod 4}}\log p& >\sqrt{U_2},\\
\sum_{\substack{U_3<p<U_4\\ p\equiv 1 \mod 4}}\log p - \sum_{\substack{U_3<p<U_4\\ p\equiv 3 \mod 4}}\log p &<-\sqrt{U_4},
\end{align*}
\end{thm}
providing insights to Problem~\ref{P:S-L}.
%%%%%%%%%%%%
%\section{On the Oscillatory Term of the PNT}
%
%1976.Knapowski (pi-li) 2
%~\cite{1976.Knapowski_(pi-li)_2}
%\begin{thm}[\cite{1976.Knapowski_(pi-li)_2} Theorem]
%There are effectively computable constants $c_1$ and $c_2$ so that for $Y>c_1$ the inequality
%\[
%V_1(Y)>c_2\log_3 Y
%\]
%holds.
%\end{thm}
%
%
%1977.Pintz 3a
%~\cite{1977.Pintz_3a}
%Defined by E. Schmidt  
%\[
%\Pi(x):=\sum_{p^m\leq x}\frac{1}{m}= \sum_{v \geq 1}\frac{1}{v}\pi(x^{\frac{1}{v}})
%\]
%
%
%\begin{thm}[\cite{1977.Pintz_3a} Theorem 1]
%For $Y>Y_i$, $(1 \leq i \leq 4)$ the interval 
%\[
%[Y, Y\exp\{63\sqrt{\log Y}\log_2 Y\}]
%\]
%contains a sign-change of $\Delta_i(x)$ $(1 \leq i \leq 4)$, where $Y_1$ and $Y_3$ are ineffective constants, $Y_2$ and $Y_4$ are effective ones.
%\end{thm}
%
%\begin{thm}[\cite{1977.Pintz_3a} Theorem 2]
%There exists effectively computable constants $c_3$ and $c_4$ such that for $Y > c_3$, the inequality 
%\[
%V_i(Y)>c_4 \frac{\sqrt{\log Y}}{\log_2 Y}\;\;\;\; (1 \leq i \leq 4)
%\]
%\end{thm}
%
%
%1978.Pintz 4a
%~\cite{1978.Pintz_4a}
%\begin{thm}[\cite{1978.Pintz_4a} Theorem]
%There are absolute constants $Y_i$, $(1\leq i \leq 4)$ such that for $Y>Y_i$, the inequality 
%\[
%V_i(Y)> \frac{1}{10^{11}} \frac{\log Y}{(\log_2 Y)^3} \;\;\;\;(1\leq i \leq 4)
%\]
%holds, where the constants $Y_i$ are effectively computable for $i= 2, 4$ and ineffective for $i= 3, 4$.
%\end{thm}
%
%%%%%%%%%
%1979.Pintz 1a $\pi(x)-\li (x)$
%~\cite{1979.Pintz_1a}
%\begin{thm}(Tur\'{a}n 1950)
%If $\varrho_0:= \beta_0 + i \gamma_0$, $\beta_0\geq \h$, is an arbitrary non-trivial zero of $\zeta(s)$, then for
%\[
%T > \max\big(c_0, c_1(\varrho) \big)
%\]
%one has
%\[
%\max_{1\leq x \leq T}|\Delta_4(x)|> \frac{T^{\beta_0}}{|\varrho_0|^{\frac{10 \log T}{\log_2 T}}}\exp \bigg(-c_1 \frac{\log T \log_3 T}{\log_2 T} \bigg)
%\]
%where the $c_i$'s as in the following always are explicitly calculable positive constants.
%\end{thm}
%
%Applying Turan's method, Stas (1959) proved:
%\begin{thm}
%If $\varrho_0=\beta_0+i\gamma_0$, $\beta_0 \geq \h$ an arbitrary non-trivial zero of $\zeta(s)$, then for
%\[
%T>\max\big(c_2, e_2(2|\varrho_0|) \big)
%\]
%one has
%\[
%\max_{x \in I}|\Delta_4(x)|>T^{\beta_0}\exp\bigg(-8 \frac{\log T}{\log_2 T} \bigg)
%\]
%where 
%\[
%I=\Bigg[T\exp\bigg(-\frac{\log T \log_3 T}{(\log_2 T)^2} \bigg)-1, T \Bigg]
%\]
%\end{thm}
%
%\begin{thm}[\cite{1979.Pintz_1a} Theorem 1]
%Let $0< \varepsilon \leq 1/50$ and let us assume the existence of a $\varrho_0=\beta_0+i\gamma$ zero of $\zeta(s)$ with:
%\[
%\beta_0=\h +\delta_0>\h \;\;\;\; and \;\;\;\; \gamma_0>400/e^2
%\]
%Then for $1 \leq i \leq 4$ and every $H$ satisfying 
%\[
%\min(H^{\delta_0/2}H^{\varepsilon^2/4}> \max( \gamma_0, c_3)
%\]
%(where $c_3$ positive constant), we have in the interval 
%\[
%I= [H, H^{\frac{4\times 10^4}{\varepsilon^2}\log\gamma_0}]
%\] 
%an $x_i' \in I$ and $x_i'' \in I$ for which
%\begin{align*}
%\Delta_i(x_i')&>(1-\varepsilon)\frac{(x_i')^{\beta_0}}{|\varrho_0|\log x_i'}\\
%\Delta_i(x_i'')&<(1-\varepsilon)\frac{(x_i'')^{\beta_0}}{|\varrho_0|\log x_i''}
%\end{align*}
%with $i = 1, 2$ and
%\begin{align*}
%\Delta_i(x_i')&>(1-\varepsilon)\frac{(x_i')^{\beta_0}}{|\varrho_0|}\\
%\Delta_i(x_i'')&<(1-\varepsilon)\frac{(x_i'')^{\beta_0}}{|\varrho_0|}
%\end{align*}
%hold respectively for $i= 3, 4$
%\end{thm}
%
%\begin{thm}[\cite{1979.Pintz_1a} Theorem 2]
%Let $0 < \varepsilon \leq 1/50$, and assuming the existence of a $\varrho_0=\beta_0+i\gamma_0$ zero of $\zeta(s)$with $\beta_0=\h +\delta_0> \h+ \varepsilon$ and 
%\[
%\gamma_0>e_2\bigg(\frac{10^{12}}{\varepsilon^3} \bigg)
%\]
%Then for $1\leq i \leq 4$ and every $H$ with
%\[
%H^{\varepsilon^4/10^7}>\max(\gamma_0, c_4)
%\]
%we have in the interval
%\[
%I = [H, H^{1+\varepsilon}]
%\]
%and $x_i'\in I$ and $x_i'' \in I$ for which
%\begin{align*}
%\Delta_i(x_i')&>\frac{(x_i')^{\beta_0}}{\gamma_0^{1+\varepsilon}\log x_i'}\\
%\Delta_i(x_i'')&<-\frac{(x_i'')^{\beta_0}}{\gamma_0^{1+\varepsilon}\log x_i''}
%\end{align*}
%hold with $i= 1, 2$\\
%and
%\begin{align*}
%\Delta_i(x_i')&>\frac{(x_i')^{\beta_0}}{\gamma_0^{1+\varepsilon}}\\
%\Delta_i(x_i'')&<-\frac{(x_i'')^{\beta_0}}{\gamma_0^{1+\varepsilon}}
%\end{align*}
%hold with $i= 3, 4$.
%\end{thm}
%
%Appendix:
%\begin{thm}[Montgomery]
%$s$ arbitrary complex number, $(j= 1, 2, \hdots, n)$ and any $\varepsilon$, $0< \varepsilon <1$, one has
%\[
%\max_{1\leq \nu \leq n/s}\frac{\bigg|\sum_{j=1}^n z_j^\nu \bigg|}{\max_{1 \leq j \leq n}|z_j|^\nu}\geq 1-\varepsilon 
%\] 
%\end{thm}
%
%\begin{thm}(T. S\'{o}s-Tur\'{a}n)
%For $s$ arbitrary complex number $z_j$,
%\[
%\max_{m\leq \nu \leq m+n}\frac{\bigg|\sum_{j=1}^n z_j^\nu \bigg|}{\max_{1 \leq j \leq n}|z_j|^\nu}\geq \Bigg(\frac{1}{8e\Big(\frac{m}{n}+1 \Big)} \Bigg)^n
%\]
%\end{thm}
%
%\begin{thm}[Tur\'an]
%Let $\h < a <1$, $0 < \eta < \frac{1}{29}(a-\h)$ and suppose $\zeta(s)$ does NOT vanish in the parallelogram $a\leq \sigma \leq 1$, $|t-\tau|\leq \log \tau$. Denoting by $M(\tau, a, \eta)$ the number of zeros in the parallelogram $a-\eta \leq \sigma \leq a$, $|t-\tau|\leq \eta/2$, we have for
%\[
%\eta \geq 500 \frac{\log_3 \tau}{\log_2 \tau}
%\]
%the estimation 
%\[
%M(\tau, a, \eta)< \eta \log \tau
%\]
%\end{thm}
%
%
%1980.Pintz 2a
%~\cite{1980.Pintz_2a}
%\begin{thm}[Ingham]
%Suppose that $\zeta(s)$ has NO zeros in the domain
%$
%\sigma > 1-\eta(|t|)
%$ where $\eta(t)$ is, for $t\geq 0$, a decreasing function, having a continuous derivative $\eta'(t)$ and satisfying the following conditions:
%\[
%0 < \eta(t) \leq \h
%\]
%\[
%\eta'(t)\to 0 \;\;\;\;\; as \;\;\;\; t\to \infty
%\]
%\[
%\frac{1}{\eta(t)}=O(\log t) \;\;\;\; as \;\;\;\; t \to \infty
%\]
%Let $\varepsilon$ be a fixed number satisfying $0 < s < 1$, and let 
%\[
%\omega(x):=\min_{t\geq 1}\big(\eta(t)\log x + \log t \big)
%\]
%Then we have
%\[
%\frac{\Delta_i(x)}{x}=O\bigg(\frac{1}{\exp\big(\h (1-\varepsilon)\omega(x)\big)} \bigg)
%\]
%\end{thm}
%
%\begin{thm}[Tur\'an]
%If for a $\beta$ with $0<\beta<1$ we have
%\[
%\frac{\Delta_i}{x}=O\big(\exp(-c_7(\log x)^{1/(1+\beta)}) \big)
%\]
%then $\zeta(s)$ does not vanish when
%\[
%\sigma>1-\frac{c_9}{\log^\beta|t|},\;\;\;\;|t|\geq c_{10}\beta
%\]
%\end{thm}
%
%\begin{thm}[\cite{1980.Pintz_2a} Theorem 1] Suppose that $\zeta(s)$ has no zero in the domain 
%\[
%\sigma>1-\eta(|t|)\;\;\;\;(0<\eta(t)\leq 1/2)
%\]
%where $\eta(t)$ is a continuous decreasing function for $t \geq 0$\\
%Let $0<\varepsilon<1$ be fixed, and define
%\[
%\omega_1(x):=\min_{t \geq 1}\big(\eta(t)\log x +\log t \big)
%\] 
%Then we arrive at
%\[
%\bar\Delta_i(x)=O\bigg(\frac{1}{e^{(1-\varepsilon)\omega_1(x)}} \bigg)
%\]
%where
%\[
%\bar{\Delta}_i(x)=\begin{cases} \frac{\Delta_i(x)}{x} \;\;\;{\rm for}\;\;\; i=3, 4\\ \frac{\Delta_i(x)}{x/\log x}\;\;\; {\rm for}\;\;\; i=1, 2 \end{cases}
%\]
%\end{thm}
%
%\begin{thm}[\cite{1980.Pintz_2a} Theorem 2]
%Suppose that $\zeta(s)$ has an infinity number of zeros in the domain 
%\[
%\sigma \geq 1-g(\log |t|),\;\;\;\; 0<g(u)\leq 1/2
%\]
%where $g(u)$ is a continuous decreasing function, for $u \geq 0$. $g'(u)$ tends to 0 monotonically increasing for $u\geq c_{22}$ and if $\lim_{u \to \infty}g(u)=0$ then $g'(u)$ tends to $0$ strictly monotonically increasing for $u\geq c_{22}$).\\
% We define 
%\[
%\omega_2(x):=\min_{u\geq 0}\big(g(u)\log x +u \big)
%\]
%Then for $\varepsilon$ fixed, as usual,
%\[
%\bar\Delta_i =\Omega_\pm\bigg(\frac{1}{e^{(1+\varepsilon)\omega_2(x)}} \bigg)
%\]
%\end{thm}
%
%
%
%
%1980.Pintz 5a~\cite{1980.Pintz_5a}
%\begin{dfn}
%We want to give lower estimates for
%\[
%D_i(Y):= \frac{1}{Y}\int_2^{Y}\big|\Delta_i(x) \big|\, dx
%\] 
%\end{dfn}
%
%\begin{thm}[\cite{1980.Pintz_5a} Theorem I]
%If $\zeta(\beta+i \gamma)=0$ $(\beta \geq \h,\;\; \gamma>0)$ and $Y>\max(c_3, e^\gamma)$, then
%\[
%D_i(Y)\geq \frac{1}{Y}\int\limits_{Y\exp(-5\sqrt{\log y)}}^{Y}|\Delta_i(x)|\,dx > Y^\beta \exp\Big(-2\sqrt{\log Y}\log_2^2 Y \Big) \;\;\;\; {\rm for} \;\;\;\; 1 \leq i \leq 4.
%\]
%\end{thm}
%
%\begin{thm}[\cite{1980.Pintz_5a} Theorem II]
%For $Y>Y_i$ (ineffective constants)
%\begin{align*}
%D_i(Y)&>c_7\frac{\sqrt Y}{\log Y} \;\;\;\; {\rm for}\;\;\;\; i=1, 2\\
%D_i(Y)&>c_8\sqrt{Y}\;\;\;\; {\rm for}\;\;\;\; i=3, 4
%\end{align*}
%\end{thm}
%Cram\'er showed that if RH (Conjecture~\ref{RH}) is true then the inequalities are optimal, apart from the values of the constants.
%
%
%
%1980.Pintz 6a
%~\cite{1980.Pintz_6a}
%\begin{thm}[\cite{1980.Pintz_6a} Theorem 1]
%For $Y> Y_0$ ineffective constant, we have
%\begin{align*}
%D_1(Y)>\frac{1}{Y}\int\limits_{Y\exp(-5\sqrt{\log Y})}^{Y}|\Delta_1(x)|\, dx&> 0.62\frac{\sqrt{Y}}{\log Y}\\
%D_2(Y)>\frac{1}{Y}\int\limits_{Y\exp(-5\sqrt{\log Y})}^{Y}|\Delta_2(x)|\, dx&> 9 \cdot 10^{-5}\frac{\sqrt{Y}}{\log Y}\\
%D_3(Y)>\frac{1}{Y}\int\limits_{Y\exp(-5\sqrt{\log Y})}^{Y}|\Delta_3(x)|\, dx&> 0.62 \sqrt{Y}\\
%D_4(Y)>\frac{1}{Y}\int\limits_{Y\exp(-5\sqrt{\log Y})}^{Y}|\Delta_4(x)|\, dx&> 10^{-4} \sqrt{Y}
%\end{align*}
%Under RH (Conjecture~\ref{RH}), Cram\'er showed for $Y>c_2$
%\[
%\frac{1}{Y}\int_2^Y \Delta_4^2(x)\,dx < c_2 Y
%\]
%\end{thm}
%
%\begin{thm}[\cite{1980.Pintz_6a} Theorem 3]
%Under RH (Conjecture~\ref{RH}) for $Y> c_{11}$,
%\begin{align*}
%c_{12}\frac{\sqrt{Y}}{\log Y}&<D_i(Y)<c_{13}\frac{\sqrt{Y}}{\log Y} \;\;\;\;(i=1, 2)\\
%c_{14}\sqrt{Y}&<D_i(Y)<c_{15}\sqrt{Y}\;\;\;\; (i=3, 4)
%\end{align*}
%\end{thm}
%
%\begin{dfn}
%\[
%\Delta_0(x):=\int_0^x \Delta_4(t)\,dt =-\sum_{\varrho}\frac{x^{\varrho +1}}{\varrho(\varrho+1)}+O(x)
%\]
%where $\varrho_1=\beta_1+i\gamma_1$ runs through the non-trivial zeros of $\zeta(s)$.
%\end{dfn}
%
%\begin{thm}[\cite{1980.Pintz_6a} Theorem 4]
%Under RH (Conjecture~\ref{RH}) then for $Y>c_16$, there exists
%\[
%x', x''\in[Y/6, Y]
%\]
%with 
%\begin{align*}
%\Delta_0(x')&<-8\cdot 10^{-4}(x')^{3/2}<-5\cdot10^{-5}Y^{3/2}\\
%\Delta_0(x'')&>8 \cdot 10^{-4}(x'')^{3/2} > 5\cdot 10^{-5}Y^{3/2} 
%\end{align*}
%\end{thm}
%
%
%1984.Kaczorowski 1a ($\pi(x)-\li(x)$)
%~\cite{1984.Kaczorowski_1a}
%\begin{thm}[\cite{1984.Kaczorowski_1a} Theorem]
%The inequality 
%\[
%V_j(T)\geq \frac{\gamma_0}{4\pi}\log T \;\;\;\;\;\; j=2, 3
%\]
%holds for sufficiently large $T$, where $\gamma_0=14.13\hdots$ denotes the imaginary part for the ``lowest'' zero of the Riemann zeta function.
%
%\end{thm}
%%%%%%%%%%%%%%
%
%1984.Kaczorowski 2a
%~\cite{1984.Kaczorowski_2a}
%\begin{thm}[\cite{1984.Kaczorowski_2a} Theorem 1] There exists a positive but ineffective constant $c_{21}$ such that for sufficiently large $T$ we have
%\[
%V_1(T)\geq c_{21}\log T
%\]
%The same holds also for $V_4$
%\end{thm}
%
%
%Remark: P\'olya proved in 1930 that
%\[
%\limsup_{T\to \infty}\frac{V_3(T)}{\log T}\geq \frac{\gamma_1}{\pi}
%\]
%
%\begin{thm}[\cite{1984.Kaczorowski_2a} Theorem 2]
%\[
%\liminf_{T\to \infty}\frac{V_3(T)}{\log T}\geq \frac{\gamma_1}{\pi}
%\]
%\end{thm}
%
%
%1984.Pintz 1
%~\cite{1984.Pintz_1}
%\begin{thm}[\cite{1984.Pintz_1} Theorem 1]
%Assume finite GRH (Conjecture~\ref{GRH}).  Then, for every $Y$ with $Y> \exp(c_6 q^{10})$, we have, for $i= 2, 4$
%\[
%\int_{A(Y)}^Y\frac{|\Delta_i(x)|}{x}\,dx \gg \sqrt{Y} \exp\bigg(-\frac{2\log Y}{\lambda}-c_7 q \lambda \log^2 Y \bigg)
%\]
%with
%\[
%A(Y)=Y\exp\bigg(-\frac{7\log Y}{\lambda} \bigg)
%\]
%for every $\lambda$ satisfying 
%\[
%\frac{\sqrt{\log Y}}{\sqrt{q}\log_2 Y}<\lambda < \frac{c_8\log Y}{q \log_2^2 Y}
%\]
%\end{thm}
%
%
%1984.Pintz 2
%~\cite{1984.Pintz_2}
%\begin{thm}[\cite{1984.Pintz_2} Theorem]
%Assume $L(s, \chi)\neq 0$ ($\chi$ mod $q$) for $\sigma > \h$, $|t|\leq D:= c_4 q^2 \log^6 q$. Let $Y> \exp(c_5 q^{10})$,
%\[
%\frac{\sqrt{\log Y}}{\sqrt{q}\log_2 Y}< \lambda < \frac{c_6 \log Y}{q \log_2^2 Y}
%\]
%Then for $1 \leq i \leq 4$ we have
%\[
%\frac{1}{Y}\int_{Y^{1-7/\lambda}}^{Y}|\Delta_i(x)|\geq \sqrt{Y}\exp\bigg(-\frac{9\log Y}{\lambda}-c_7 q \lambda \log_2^2 Y \bigg)
%\]
%\end{thm}
%
%
%1986.Kaczorowski 1
%~\cite{1986.Kaczorowski_1}
%\begin{thm}[\cite{1986.Kaczorowski_1} Theorem 1]
%Let $f(x)$ be real for $x>0$ and suppose that the integral $\displaystyle \int_0^\infty f(x)x^{-s-1}\,dx$ converges absolutely for $\sigma \geq \sigma_1$ and represents in that half-plane a function $F(s)$ having the following properties:
%\begin{itemize}
%\item $F(s)$ is regular for $\sigma > \theta$ but NOT in any half-plane $\sigma> \theta- \varepsilon$ with $\varepsilon >0$;
%\item there exists a denumerable (finite OR infinite) set $S:=\{\varrho_\nu = \beta_\nu+i\gamma_\nu \}\;\;\; (\gamma_\nu \geq 0)$ without finite limit point satisfying $\theta-c_0 \leq \beta_\nu \leq \theta$ for some $c_0>0$ and such that $F(s)$ can be continued as a meromorphic function in the open set $D$ obtained by making the cuts $s=\sigma \pm i\gamma_n (\sigma \leq \beta_n)$ in the half-plane $\sigma> \theta - c_0$
%\item for $s \to \varrho_\nu$ $(s \in D)$.  $F(s)=P_\nu(s-s_\nu)\log(s-s_\nu)+F_\nu (s)$ where $F_\nu (s)$ is meromorphic at $s=\varrho_\nu$ and $P_\nu$ a polynomial (with the possibility $P_\nu := 0$).\\
%Let $\gamma=\min_{\beta_\nu}\gamma_\nu$ and $\gamma = \infty$ if $\beta_\nu < \theta$ for all $\nu= 1,2, \hdots$.
%\end{itemize}
%Under these conditions we have
%\[
%\liminf_{Y\to \infty}\frac{V(f, Y)}{\log Y}\geq \frac{\gamma}{\pi}
%\]
%and every interval of the form
%\[
%[Y^{1-\varepsilon}, Y],\;\;\;\; Y>Y_0(\varepsilon)
%\]
%contains at least one sign-change of $f(s)$.
%\end{thm}
%
%
%\begin{thm}[\cite{1986.Kaczorowski_1} Theorem 2]
%If $f(x)$ is real for $x>0$, $\displaystyle F(s)=\int_0^\infty f(x)x^{-s-1}\,dx$ is absolutely convergent for $\sigma > \sigma_1$ and
%\begin{itemize}
%\item $F(s)$ has a pole at $\varrho=\beta+i\gamma$, $gamma>0$ with principal part $\displaystyle \sum_{j=1}^{k+1}h_j(s-\varrho)^{-j}$,
%\item apart from the poles $\varrho$ and $\bar\varrho$, $F(s)$ is regular and to the right of the broken line $L$ defined by
%\[L=\begin{cases}
%|t|\geq \Gamma,\;\;\; \sigma=\sigma+a_0\\
%\beta_0+a_1\leq \sigma \leq \sigma_1 +a_0, |t|=\Gamma\\
%H\leq |t|\leq \Gamma, \sigma=\beta_0+a_1 \\
%\beta_0-a_2\leq \sigma \beta_0+a_1, |t|=H\\
%|t|\leq H, \sigma=\beta_0-a_2
%\end{cases}
%\]
%where $a_0>$, $0<a_2<\beta_0$, $-a_2\leq a_1 \leq \sigma_1+a_0-\beta_0$, $\Gamma\geq H \gamma_0$, further define
%\[
%d_1:=\max\Bigg(\frac{\sigma_1+a_0-\beta_0}{\log\frac{|\sigma_1+a_0+i\Gamma|}{|\varrho|}}, \frac{a_1}{\log\frac{|\beta_0+a_1+iH|}{|\varrho|}} \Bigg)<\frac{a_2}{\log\frac{|\varrho|}{\beta-a_2}}=:d_2
%\]
%\item $|F(s)|\leq M$ for $s \in L$.
%\end{itemize} 
%Then for every $\varepsilon>0$ and $Y>Y_0=Y_0(\varepsilon, a_0, a_1, a_2, \sigma_1, \beta_0, \gamma_0, H, M, \Gamma, h_1, \hdots, h_{k+1})$, effective constants, we have
%\[
%V(f, Y)>\bigg(1-\frac{d_1}{d_2}-\varepsilon\bigg)\frac{\gamma_0}{\pi}\log Y
%\]
%\end{thm}
%
%1987.Kaczoroswki 2
%~\cite{1987.Kaczorowski_2}
%Let $F(s)$ and $f(s)$ be related by the Mellin transformation
%\[
%F(s)=\int_0^\infty f(x)x^{-s-1}\,dx
%\]
%with singularities of the form
%\[
%P_\nu(s-\varrho_\nu)\log(s-\varrho_\nu)+F_\nu(s)
%\]
%where $F_\nu(s)$ meromorphic at $s=\varrho_\nu$ and $P_\nu$ a polynomial (OR $P_\nu:=0$).
%
%We define the number $V(f, Y)$ of sign-changes in the interval $(0, Y]$ as follows:
%\[
%V(f, Y):=\sup\big\{N: \exists\{x_i\}_{i=1}^N, 0<x_1<\hdots<x_N\leq Y,
%f(x_i)\neq 0, \sgn{f(x_i)}\neq \sgn{f(x_{i+1})}, 1\leq i < N \big\}
%\]
%
%Moreover, we say that $V(f, Y)>h(Y)$ with combined oscillation of size $g(x)$ if there exists a series $\{ x_i \}_{i=1}^{h(Y)}$ with $\sgn{f(x_i)}\neq \sgn{f(x_{i+1})}$ and $|f(x_i)|\geq g(x_i)$.
%
%With the above notation we have:
%\begin{thm}[\cite{1987.Kaczorowski_2} Theorem]
%Let $f(x)$ be real for $x>0$ and suppose that $\displaystyle \int_0^\infty f(x)x^{-s-1}\,dx $ converges absolutely for $\sigma \geq \sigma_1$ and represents in that half-plane a function $F(s)$ having the following properties:
%\begin{itemize}
%\item $F(s)$ regular for $\sigma > \theta$ but NOT in any half-plane $\sigma> \theta- \varepsilon$ with $\varepsilon >0$
%\item there exists a denumerable (finite OR infinite set) of singularities of $F(s)$, $S:=\{\varrho+\beta \pm i\gamma\}$, $\gamma>0$, without finite limit point satisfying $\theta-c \leq \beta \leq \theta$ for some $c>0$ and such that $F(s)$ can be continued as a meromorphic function in the open set $D$ obtained by making cuts $s=\sigma \pm i \gamma$, $\sigma \leq \beta$, in the half-plane $\sigma> \theta -c$
%\item For $|s-\varrho|\leq \eta$, $\eta >0$, $s \in D$:
%\[
%F(s)=(s-\varrho)^a  \sum_{i=0}^k g_t(s-\varrho) \log^t(s-\varrho)+F(s-\varrho)
%\]
%where $g_t$ and $F$ are regular for $|z|<\eta$, $k$ a non-negative integer, $a$ an arbitrary complex number and $g_{\nu k_\nu}(0)\neq 0$.
%\end{itemize} 
%\end{thm}
%
%1987.Kaczorowski 3a
%~\cite{1987.Kaczorowski_3a}
%\begin{thm}[\cite{1987.Kaczorowski_3a} Theorem 2.1]\label{1987.Kaczorowski_3a 2.1}
%There exists an effectively calculable numerical constant $T_3$ such that 
%\[
%V_5(T)\geq \frac{\gamma}{4\pi}\log T \;\;\;\;\; {\rm for}\;\;\; T\geq T_3
%\]
%where $\gamma=14.13\hdots$ denotes the imaginary part of the lowest zero of $\zeta(s)$, and ineffectively, 
%\[
%\liminf_{T \to \infty}\frac{V_5(T)}{\log T}\geq \frac{\gamma_0}{\pi}
%\]
%\end{thm}
%
%\begin{thm}[\cite{1987.Kaczorowski_3a} Theorem 3.1]\label{1987.Kaczorowski_3a 3.1}
%On the consequence of Ingham's condition which says that there is at least one zero on the line $\sigma = 0$,
%The following statements are equivalent:
%\begin{itemize}
%\item Ingham's condition is true;
%\item There exists an absolute constant $n_0$ such that in every interval $(T, eT)$, $T>0$, $\Delta_5$ changes sign at most $n_0$ times;
%\item $\displaystyle \limsup_{T \to \infty}V_5(T)/ \log T < \infty;$
%\item $\displaystyle \liminf_{T \to \infty}V_5(T)/ \log T < \infty.$
%\end{itemize}
%\end{thm}
%Theorems~\ref{1987.Kaczorowski_3a 2.1} and~\ref{1987.Kaczorowski_3a 3.1} immediately imply that, under Ingham's condition,
%\[
%c_5 \log T \leq V_5(T)\leq c_6 \log T
%\]
%for some positive constants $c_5$ and $c_6$.
%
%\begin{thm}[\cite{1987.Kaczorowski_3a} Theorem 4.1]\label{1987.Kaczorowski_3a 4.1}
%Suppose Ingham's condition and $\omega_1, \omega_2, \hdots$ linearly independent over $\Q$, where each $\omega_i$ denote the imaginary parts of zeros on the half-line $\sigma = \theta, t>0$. Then
%\[
%V_5(T)\sim \varkappa \log T \;\;\;{\rm as} \;\;\; T\to \infty
%\]
%where
%\[
%\varkappa:=\frac{\omega_1}{2\pi} \int_{\Omega}\nu(\eta)d\mu(\eta)
%\]
%and $d\mu$ denotes the normed Haar measure on the group $\Omega$.
%\end{thm}
%
%\begin{thm}[\cite{1987.Kaczorowski_3a} Theorem 4.2]
%Under the assumption of Theorem~\ref{1987.Kaczorowski_3a 4.1}, almost all sign-changes of $\Delta_5$ are ``big'', i.e. for every $\varepsilon >0$ there exists $\delta > 0$ such that
%\[
%V_5(T, \delta) \leq \varepsilon \log T
%\]
%\end{thm}
%
%\begin{thm}[\cite{1987.Kaczorowski_3a} Theorem 5.1]
%Under RH (Conjecture~\ref{RH}) and let $\varrho:=\h+i\gamma= \h + i\;14.13\hdots$ denote the `lowest' zero of $\zeta(s)$.
%Further, let
%\[ \varphi= \Arg \Gamma (\varrho)\]
%and
%\[
%u_k=\frac{k\pi-\varphi}{\gamma}\;\;\;\; k=1, 2, \hdots 
%\]
%Then
%\begin{itemize}
%\item All sign-changes of $\Delta_5$ are `big'; this means that for every two consecutive sign-changes $0<x_1<x_2$ we have
%\[
%\max_{x_1\leq x \leq x_2}\frac{|\Delta_5(x)|}{x^{1/2}}\geq c_7
%\]
%with absolute constant $c_7>0$.
%\item For sufficiently large $k$, every interval of the form
%\[
%(e^{u_k}, e^{u_{k+1}})
%\]
%contains exactly one sign-change of $\Delta_5$.
%\item For $T \to \infty$,
%\[
%V_5(T)=\frac{\gamma}{\pi}\log T + O(1)
%\]
%\end{itemize}
%\end{thm}
%
%\begin{thm}[\cite{1987.Kaczorowski_3a} Theorem 5.2]
%Under RH (Conjecture~\ref{RH}) we have for $T>0$,
%\[
%V(T)=\frac{\gamma}{\pi}T+\varphi(T),
%\]
%where $\varphi$ denotes an almost periodic function belonging to the Steapanov class $S^p$ for ever $p \geq 1$.  Moreover, the function $\varphi$ is bounded.
%\end{thm}
%
%
%1988.Kaczorowski 4a
%~\cite{1988.Kaczorowski_4a}
%\begin{dfn}
%\[
%\Delta_5(x):=\sum_{n=2}^\infty\big(\Lambda(n)-1 \big)e^{-n/x},\;\;\; x>0
%\]
%and define $V_5(T)$ to be the number of sign changes of $\Delta_5$ in $(0, T]$.
%\end{dfn}
%\begin{thm}[\cite{1988.Kaczorowski_4a} Theorem 1]
%For $T$ tending to infinity we have the estimate
%\[
%V_5(T)=o(\log^2 T)
%\]
%\end{thm}
%
%\begin{thm}[\cite{1988.Kaczorowski_4a} Theorem 2]
%Suppose $\theta > \h$. Then for every $\varepsilon >0$ there exists two positive constants $c_0$ and $T_0$ depending on $\varepsilon$ such that, for every $T>T_0$,
%\[
%\max_{T\leq x \leq (1+\varepsilon)T}|\Delta_5(x)|\geq c_0 T^{\theta-\varepsilon}
%\]
%\end{thm}
%
%Under RH (Conjecture~\ref{RH}), the inequality
%\[
%\max_{T\leq x \leq (1+\varepsilon)T}|\Delta_5(x)|\geq c_1(\varepsilon)\sqrt{T}
%\]
%
%\begin{thm}[\cite{1988.Kaczorowski_4a}]
%For every positive $\varepsilon$ there exist two positive constants $c_2$ and $T_1$ such that
%\[
%\max_{T\leq x \leq (1+\varepsilon)T}|\Delta_5(x)| \geq c_2 \sqrt{T} \;\;\;\; {\rm for} \;\;\; T \geq T_1
%\] 
%\end{thm}
%
%1989.Szydlo 1 [in German]
%~\cite{1989.Szydlo_1}
%\begin{thm}[\cite{1989.Szydlo_1} Theorem]
%Let $f: (0, \infty) \to \R$ be an piecewise continuous function in some interval $(a, b) \subset (0, \infty)$.  Further let $F(s)$ be its Mellin transformation: where there is such a $\varepsilon >0$ where
%\[
%F(s):=\int_0^\infty f(x)x^{-s-1}\,dx
%\]
%converges absolutely for $\sigma>\sigma_1 - \varepsilon$.
%$F$ in $D\backslash \{\varrho_1, \bar{\varrho_1},\hdots, \varrho_r, \bar{\varrho_r}\}$ holomorphic Function, where
%\begin{align*}
%D:=\{s\in \C: \sigma_0\leq \sigma \leq \sigma_1, |t|\leq h \} \cup \{s \in \C: \sigma \geq \sigma_1 \};\\
%\varrho_\nu= \beta_\nu+ i \gamma_\nu, \;\;\; \sigma_0<\beta_\nu < \sigma_1, 0< \gamma_\nu < h\;\;\;\; (\nu= 1, 2, \hdots, r);\\
%\varrho_\nu\neq \varrho_{\nu'}\;\;\;(\nu\neq \nu')
%\end{align*}
%$F$ has point $\varrho_\nu$, a pole of order $m_\nu \geq 1$ with the main term
%\[
%\sum_{l=-m_\nu}^{-1}a_{\nu, l}(s-\varrho_\nu)^l\;\;\;\; (\nu= 1, \hdots, r)
%\]
%\[
%|F(s)|\leq M < + \infty \;\;\;\; (s \in \partial{D})
%\]
%\end{thm}
%
%1989.Szydlo 2 in German
%~\cite{1989.Szydlo_2}
%\begin{thm}[\cite{1989.Szydlo_2} Theorem 1]
%For $X\geq 10^{2250}$ 
%\[
%\Delta_4(X)\geq 0.013 \log X
%\]
%\end{thm}
%
%\begin{thm}[\cite{1989.Szydlo_2} Theorem 2]
%Let $H \geq 501.5$, for each non-trivial zero $\varrho=\beta+i \gamma$ of the Riemann zeta function $\zeta$ which lies in the strip $|t|<H$, satisfies the condition $\beta= \h$. 
%
%Then for $X\geq \exp(0.09\max\{4400, H\})$ the inequality
%\[
%\Delta_4(X)\geq \bigg(1-\frac{3}{H} \bigg)\frac{\gamma_1}{\pi}\log X
%\]
%where $\gamma_1$ is the lowest zero of the Riemann zeta-function on the line $\sigma=\h$.
%\end{thm}
%
%
%1989.Szydlo 3 in German
%~\cite{1989.Szydlo_3}
%Denote $V_K(Y)$ the number of sign-changes in $[0, Y]$, $(Y>0)$ of the remainder-term $\Delta_K(x):=\psi_K(x)-x$ where 
%\[
%\psi_K(x):=\sum_{N\p^r\leq x}\log{N\p}
%\]
%for an algebraic number field $K$ with $\p$ prime ideal.
%\begin{thm}[\cite{1989.Szydlo_3} Theorem 1]
%The Dedekind zeta-function $\zeta_k$ has NO real non-trivial zero,
%\[
%H \geq \max\{20, |\varrho_K|^2 \}
%\]
%Then for $K$ ineffective positive constants $c_1$, $c_2$ and $c_3$, for
%\begin{align*}
%\log X \geq \max\Bigg\{&c_1 H^2(\log_2 H +\log_2(d+2)) \bigg(1+\frac{1}{\gamma_K g(H)} \bigg), \\
%                       &c_2\frac{H^3(n^2\log ^2 H+\log^2 d)(\log H +\log_2 (d+2))}{b(H)},\\
%                        &c_3\frac{H^2(n\log H +\log d)}{\gamma_K} \Bigg\}
%\end{align*}
%holds 
%\[
%V(\Delta_k, X) \geq \bigg(1-\frac{10}{H} \bigg)\frac{\gamma_K}{\pi}\log X
%\]
%\end{thm}
%
%\begin{thm}[\cite{1989.Szydlo_3} Theorem 2]
%For every non-trivial zero $\varrho=\beta + i\gamma$ of $\zeta_K$, in the stripe $|t|<H$ lies, where $H \geq \max\{20, |\varrho_K|^2 \}$, where $\beta=\h$ and $\zeta_K(\h)\neq 0$.
%Then for $K$ independent, there are effective constants $c_4$ and $c_5$
%\[
%\log X \geq \max \Bigg\{c_4H^2\log_2(d+2)\bigg(1+\frac{1}{\gamma_K g(H)} \bigg), c_5\frac{H}{\gamma_K} \Bigg\}
%\] 
%holds 
%\[
%V(\Delta_k, X) \geq \bigg(1-\frac{10}{H} \bigg)\frac{\gamma_K}{\pi}\log X
%\]
%\end{thm}

%2000.Puchta
%~\cite{2000.Puchta}
%Moments of the error term of $\pi(x)-\li(x)$
%\begin{thm}[\cite{2000.Puchta} Theorem 1]
%Let $\chi$ be a character $\mod k$, $k$ a natural number. 
%Then
%\[
%\frac{1}{x}\int_0^x\Bigg(\frac{\Psi(e^t, \chi)-Ee^t}{e^{t/2}} \Bigg)^k\,dt \sim (-1)^k\sum_{\gamma_1+\cdots+\gamma_k=0}(\varrho_1\cdots\varrho_k)^{-1}
%\]
%where the sum runs over all NON-trivial zeros of $L(s, \chi)$. 
%\end{thm}
%
%\begin{thm}[\cite{2000.Puchta} Theorem 5]
%Let $q$ be some natural number, $\varepsilon>0$
%Let
%\[
%\Delta(t, \chi)=\frac{\Psi(t, \chi)-Et}{\sqrt t}
%\]
%where $E=1$ or $0$, depending on whether $\chi$ is principal or not, respectively.\\
%Assume GRH (Conjecture~\ref{GRH}) for $\mod k$.  Then there are effective computable constants $X_0(q, \varepsilon)$ and $C=C(q)$ such that for all $X>X_0$ there are numbers $2<x_+, x_- < X$ such that
%\begin{itemize}
%\item $|\Delta(x_+, \chi_i)-\Delta(x_+, \chi_j)|<C$
%\item $\Delta(x_+, \chi_i)>(\h-\varepsilon)\log_3 X|$
%\item $|\Delta(x_-, \chi_i)-\Delta(x_-, \chi_j)|<C$
%\item $\Delta(x_-, \chi_j)<-(\h-\varepsilon)\log_3 X$
%\item $x_+<x_-<x_+\Big(1+\frac{1}{\log_{2}^{1-\varepsilon}X}\Big)$
%\end{itemize}
%\end{thm}

%\begin{thm}[\cite{2004.Schlage-Puchta} Theorem 2]
%Assume RH~\ref{RH}, then there exists an $x\leq e_3(16.7)$ such that $\pi(x)>\li (x)$. If $V(x)$ denote the number of sign changes of $\pi(x)-\li(x)$, then $\displaystyle V(x)>\frac{\log x}{e_2(16.7)}-1$
%\end{thm}



%%%%%%%%%%
\section{More Chebyshev Type Assertions}
Several authors, remarkably J. Besenfelder~\cite{1979.Besenfelder_1}~\cite{1980.Besenfelder_2} and H. Bentz~\cite{1982.Bentz} proved a few unconditional theorems in the flavour of Chebyshev's assertion~\eqref{eq.cheb}
\begin{thm}[\cite{1979.Besenfelder_1} Theorem]
\[
\lim_{x \to \infty}\sum_{p}(-1)^{(p-1)/2}\log p\cdot p^{-1/2}\exp\big({-{\log^2p}/{4x}}\big)=-\infty
\]
\end{thm}
which is a special care of:
\begin{thm}[\cite{1980.Besenfelder_2} Theorem]
\[
\lim_{x \to \infty}\sum_p (-1)^{{(p-1)}/{2}}\log{p}\cdot p^{-\alpha} \exp\big(- {\log^2 p}/{4x}\big)=-\infty\;\;\;\; {\rm for}\;\;\;\; 0 \leq \alpha \leq \h
\]
Where the magnitude of divergence for $\alpha = \h$ is given by $\h\sqrt{\pi y}$ and for $0\leq \alpha<\h $ is given by $\sqrt{\pi y}e^{\frac{y}{4}(1-2\alpha)}$
\end{thm}
Further, H. Bentz proves~\cite{1982.Bentz}
\begin{thm}[\cite{1982.Bentz} Theorem 1]
Unconditionally,
\[
\lim_{x \to \infty}\sum_{p}(-1)^{(p-2)/2}\log{p}\cdot{p^{-1/2}}\exp\big(-\log^2p/x \big)=-\infty 
\]
The magnitude of divergence is given by $\frac{1}{4}\sqrt{\pi x}+O(1)$.
\end{thm}
which generalizes to: 
\begin{thm}[\cite{1982.Bentz} Theorem 2]
\[
\lim_{x \to \infty}\sum_p (-1)^{(p-1)/2}\log p\cdot{p^{-\alpha}}\exp(-\log^2p/x)=-\infty\;\;\;\;{\rm for\; all}\;\;\;\; 0\leq \alpha < \h
\]
The magnitude of divergence is given by 
$
\sim\h \sqrt{\pi x} \exp\Big(\frac{x}{16}(1-2\alpha)^2\Big)
$
\end{thm}
\begin{dfn}\label{chi_3}
If we define
\[
\chi_3(m)=\begin{cases} 1 \;\;\;\; &{\rm if} \;\;\; m\equiv 1 \mod 3,\\
                       -1 \;\;\;\; &{\rm if} \;\;\; m\equiv 2 \mod 3,\\
                        0 \;\;\;\; &{\rm if} \;\;\; m\equiv 0 \mod 3
                        \end{cases}
                        \]
                        i.e. taking $k=3$ and thus $\varphi(k)=2$
\end{dfn}
We have:
%%%
\begin{thm}[\cite{1982.Bentz} Theorem 3]
Let $\chi_3$ be given as in Definition~\ref{chi_3}, then
\[
\lim_{x\to \infty}\sum_{p}\chi_3(p)\log p\cdot p^{-1/2}\exp(-\log^2p/x)=-\infty
\]
The order of magnitude of divergence is given by
$\frac{1}{4}\sqrt{\pi x}+O(1)
$\end{thm}
%%%
\begin{thm}[\cite{1982.Bentz} Theorem 4]
Let $\chi_3$ be as in Definition~\ref{chi_3}, then
\[
\lim_{x\to \infty}\sum_p\chi_3(p){\log p}\cdot{p^\alpha}\exp(-\log^2p/x)=-\infty\;\;\;\;{\rm{for}\;} 0 \leq \alpha < \h.
\]
The order of magnitude of divergence is given by
$\sim \h\sqrt{\pi x}\exp\big({\frac{x}{16}(1-2\alpha)^2}\big)
$\end{thm}
%%%
Now for a higher moduli, in the cases $k=8$ and $5$ so $\varphi(k)=4$, H. Bentz and J. Pintz prove:
\begin{thm}[\cite{1982.Bentz} Theorem 5]
\begin{align*}
\lim_{x \to \infty}\sum_{p \equiv 1 \mod 8}\log p\cdot p^{-1/2}\exp(-\log^2p/x)-\sum_{p \equiv 3 \mod 8}\log p\cdot p^{-1/2}\exp(-\log^2p/x)=- \infty\\
\lim_{x \to \infty}\sum_{p \equiv 1 \mod 8}\log p\cdot p^{-1/2}\exp(-\log^2p/x)-\sum_{p \equiv 5 \mod 8}\log p\cdot p^{-1/2}\exp(-\log^2p/x)=- \infty\\
\lim_{x \to \infty}\sum_{p \equiv 1 \mod 8}\log p\cdot p^{-1/2}\exp(-\log^2p/x)-\sum_{p \equiv 7 \mod 8}\log p\cdot p^{-1/2}\exp(-\log^2p/x)=- \infty\\
\end{align*}
with the order of magnitude of divergence being $-\frac{1}{4}\sqrt{\pi y}+O(1)$, respectively.
\end{thm}
%%%
\begin{thm}[\cite{1982.Bentz} Theorem 6]
\begin{align*}
\lim_{x \to \infty}\bigg\{\sum_{p\equiv 3 \mod 8}-\sum_{p\equiv 5 \mod 8}\bigg\}\log p\cdot p^{-1/2}  \exp(-\log^2p/x)=O(1)\\
\lim_{x \to \infty}\bigg\{\sum_{p\equiv 3 \mod 8}-\sum_{p\equiv 7 \mod 8}\bigg\}\log p\cdot p^{-1/2}  \exp(-\log^2p/ x)=O(1)\\
\lim_{x \to \infty}\bigg\{\sum_{p\equiv 5 \mod 8}-\sum_{p\equiv 7 \mod 8}\bigg\}\log p\cdot p^{-1/2} \exp(-\log^2p/x)=O(1)
\end{align*}
\end{thm}

\begin{thm}[\cite{1982.Bentz} Theorem 7]
For at least one of the two classes $2 \mod 5$, $3 \mod 5$, we have
\[
\lim_{x\to \infty}\Bigg(\bigg(\sum_{\substack{p \equiv 2 \mod 5\\{\rm or}\; p \equiv 3 \mod 5}}-\sum_{p \equiv 4 \mod 5}\bigg)\log p \cdot p^{-1/2}\exp(-\log^2 p/4x )\Bigg) =+\infty
\]
\end{thm}

and when dealing with quadratic residues and distribution of primes, H. Bentz~\cite{1980.Bentz} assumes the following two conjectures:
\begin{cnj}[$\bf{R}_2$]\label{R_2}
The domain $\sigma > \h$, $|t|\leq 1$ is zero free and there is NO zero at $s=\h$ for Dirichlet $L$-function.
\end{cnj}

\begin{cnj}[$\bf{H}_2$]\label{H_2}
All zeros $\varrho:=\beta+i \gamma$ satisfy the inequality 
\[
\beta^2-\gamma^2<\frac{1}{4}
\]
\end{cnj}
and he shows
\begin{thm}[\cite{1980.Bentz} Theorem 1]
If $l_1$ is a quadratic residue, $l_2$ a non-residue mod $k$ and ${\bf R_2}$ (Conjecture~\ref{R_2}) or even ${\bf H_2}$ (Conjecture~\ref{H_2}) valid for $L$-function mod $k$, then 
\[
\lim_{x \to \infty}\sum_p\varepsilon (k; p, l_1, l_2)\log p\cdot\exp(-{\log^2 p}/{x} ) =-\infty
\]
\end{thm}

\begin{thm}[\cite{1980.Bentz} Theorem 2]
If $l_1$ a quadratic residue, $l_2$ a non-residue mod $k$, then
\[
\lim_{x \to \infty} \sum_{p}\varepsilon(k; p, l_1, l_2)\log p\cdot\exp(-{\log^2 p}/{x}) = - \infty
\] 
holds for all $k< 25$.
\end{thm}

\begin{thm}[\cite{1980.Bentz} Theorem 3]\label{1980.Bentz Thm 3}
If ${\bf{ R_2}}$ (Conjecture~\ref{R_2}) or only ${\bf{ H_2}}$ (Conjecture~\ref{H_2}) is true for all $L$-functions $\mod k$, $l_1$ is a quadratic residue, $l_2$ a non-residue mod $k$, then for $0 \leq \alpha < \h$
\[
\lim_{x \to \infty}\sum_{p}\varepsilon(k, p, l_1, l_2)\log p \cdot p^{-\alpha}\exp(-\log^2 p/x) = -\infty
\]
\end{thm}

\begin{thm}[\cite{1980.Bentz} Theorem 4]\label{1980.Bentz Thm 4}
If $l_1$ a quadratic residue, $l_2$ a non-residue $\mod k$, $k< 25$, then for $0 \leq \alpha < \h$
\[
\lim_{x \to \infty}\sum_p \varepsilon(k; p, l_1, l_2)\log p \cdot p^{-\alpha}\exp(-\log^2 p/{x}) = -\infty
\]
\end{thm}

\begin{thm}[\cite{1980.Bentz} Theorem 5]
Under the condition of Theorem~\ref{1980.Bentz Thm 3} we have
\[
\sum_n \varepsilon(k; n, l_1, l_2)\log p\cdot{p^{-\alpha}} \exp( -\log^2 p/x)\sim \frac{N(k)}{\varphi(k)}\sqrt{\pi x}\exp\Big((x/4)(1/2 -x)^2 \Big)
\]
where $N(k)$ denotes the number of solutions of $x^2 \equiv 1 \mod k$
\end{thm}

Of course the above theorem implies:
\begin{thm}[\cite{1980.Bentz} Theorem 6]
Under the conditions of Theorem~\ref{1980.Bentz Thm 4}
\[
\sum_n \varepsilon (k, n, l_1, l_2)\log p\cdot p^{-\alpha}\exp(-\log^2 p/x) \sim \frac{N(q)}{\varphi(q)}\sqrt{\pi x}\exp\Big((x/4)(1/2-x)^2 \Big)
\]
\end{thm}





\section{A Few Other Results}
Knapowski and Tur\'an also made contributions to Problem~\ref{P:L-B} for $\Delta_\pi(r; k, l_1, l_2)$ in the cases of $k=8$ and $4$:
\begin{thm}[\cite{1965.Knapowski} Theorem I]
For any $l_1 \neq l_2$ among $3, 5, 7$ and $0<\delta<c_{47}$, we have the inequality
\begin{equation*}
\max_{\delta\leq x \leq \delta ^{{1/3}}} |\Delta_\pi(r; 8, l_1, l_2)| \geq \delta^{-1/2} \exp\bigg(\frac{23\log(1/\delta)\log_3(1/\delta)}{\log_2({1/\delta})}\bigg)
\end{equation*}
\end{thm}

\begin{thm}[\cite{1965.Knapowski} Theorem II]
For $l \neq 1, k=4$ \emph{or} $8$ and $0<\delta< c_{48}$,
\begin{equation*}
\max_{\delta\leq x \leq \delta ^{{1/3}}} |\Delta_\pi(r; k, 1, l)| \geq \delta^{-1/2} \exp\bigg(\frac{23\log(1/\delta)\log_3(1/\delta)}{\log_2({1/\delta})}\bigg)
\end{equation*}
\end{thm}
\vskip12pt
%%%%%%%%%%%%%
In his paper~\cite{1971.Stark}, H. Starks studies the asymptotic behaviours of $\varphi(k)\pi(x, k, a)- \varphi(K)\pi(x, K, A)$: If $\chi$ and $X$ are characters mod $k$ and $K$ respectively, and $\chi_0$ and $X_0$ denote the principle characters, whereas $\chi_R$ and $X_R$ denote the real characters, he defines:
\begin{dfn}
\begin{equation}\label{r}
r:=r(k, a; K, A) = \sum_{X_R}{X_R(A)}- \sum_{\chi_R}{\chi_R(A)}
\end{equation}
\begin{align*}
A_T(u)&:= A_T(u, k, a, K, A)\\
           &= \sum_{X\neq X_0}\sum_{\substack{\varrho_X \\ \beta_X > 0, |\gamma_X |< T}} \frac{\overline X(A)}{\varrho_X}\exp{(\varrho_X-\h)u} - 
                \sum_{\chi \neq \chi_0} \sum_{\substack{\varrho_\chi \\ \beta_\chi > 0, |\gamma_\chi |< T}} \frac{\bar\chi(A)}{\varrho_\chi}\exp{(\varrho_\chi-\h)u} \\
A_T^*(u)&:= A_T^*(u, k, a, K, A)\\
           &= r+ \sum_{X\neq X_0}\sum_{\substack{\varrho_X \\ \beta_X > 0, |\gamma_X |< T}} \frac{X(A)}{\varrho_X}\exp{(\varrho_X-\h)u} - 
                \sum_{\chi \neq \chi_0} \sum_{\substack{\varrho_\chi \\ \beta_\chi > 0, |\gamma_\chi |< T}} \frac{\chi(A)}{\varrho_\chi}\exp{(\varrho_\chi-\h)u}
\end{align*}
so the relation between them is simply
\begin{equation*}
A_T^*(u)= r+\frac{1}{T}\int_0^T A_t(u) \, dt
\end{equation*}
further, whenever the limit exists, define
$A_\infty(u):=A_\infty(u; k; a; K, A) = \lim_{T\to \infty}A_T(u; k, a; K, A)$
\end{dfn}
\begin{thm}[\cite{1971.Stark} Theorem 1]
Under GRH (Conjecture~\ref{GRH}), for any $T>0$ and any $u$,
\[
\limsup_{y \to \infty}\frac{\varphi(k)\pi(x, k, a)- \varphi(K)\pi(x, K, A)}{\sqrt y/\log y}\geq A_T^*(u)
\]
\end{thm}

\begin{thm}[\cite{1971.Stark} Theorem 2]
Again assuming GRH (Conjecture~\ref{GRH}) 
\begin{enumerate}
\item\label{1971.1} If $r(k, a; K, A) = 0$, then there is a constant $c>0$ such that
\begin{equation*}
\limsup_{y \to \infty}\frac{\varphi(k)\pi(x, k, a)- \varphi(K)\pi(x, K, A)}{{{\sqrt{y}}/{\log y}}}\geq c
\end{equation*}
\item If $r(k, a; K, A) > 0$, then the result of (\ref{1971.1}) is true with $c=r$, $r$ in equation~\eqref{r}.
\end{enumerate}
\end{thm}
\vskip12pt
On the sign changes of $\pi(x; q, 1)-\pi(x; q, a)$, J.-C. Schlage-Puchta engenders:
\begin{thm}[\cite{2004.Schlage-Puchta} Theorem 1]
When $q$ s a natural number, we define $q^+:=\max\big(q, \exp(1260)\big)$, and assuming GRH (Conjecture~\ref{GRH}). Let $M(q)$ be the number of solution of the congruence $x^2 \equiv 1 \mod q$. Then there exists an $x$ with$x< e_2\big((q^+)^{170}+e^{18M(q)} \big)$ such that $\pi(x; q, 1)> \pi(x; q, a)$ for all $a \not\equiv 1 \mod q$.  Moreover, let $V(x)$ demote the number of sign changes of $\pi(t; q, 1)- \max_{a \not\equiv 1 \mod q}\pi(t; q, a)$ in the range $2 \leq t \leq q$, then
\[
V(x)>\frac{\log x}{\exp \big( (q^+)^{170}+e^{18M(q)}\big)}-1
\]
\end{thm}




%%%%%%%%%%%%%


\section{Modern Developments on the Racing Problems}
%%%%%%%
Several authors had made progresses on the Shank-R\'enyi Racing Problems (Problem~\ref{P:P-R} described in Section~\ref{intro section} and their variations), notably early on by Kaczorowski~\cite{1993.Kaczorowski} as he proposed: 
\begin{cnj}[Strong Race Hypothesis]~\label{SRH}
For each permutation $a_1, a_2, \hdots, a_{\varphi(k)}$ of the reduced set of residue classes mod $k$ the set of integers $m$ with 
\[
\pi(m, k, a_1)<\pi(m, k, a_2)<\dots<\pi(m, k, a_{\varphi(k)})
\]
has positive ``lower density''.
\end{cnj}
\begin{thm}[\cite{1993.Kaczorowski} Theorem 1]
Under GRH (Conjecture~\ref{GRH}) for Dirichlet $L$-functions mod $k$, $k\geq 3$. There exists infinitely many integers $m$ with 
\[
\pi(m, k, 1)>\max_{a \not\equiv 1 \mod k}\pi(m, k, a)
\] 
Moreover, the set of $m$'s satisfying the inequality has positive density.\\
Same statement holds true for $m$ satisfying 
\[
\pi(m, k, 1)<\min_{a \not\equiv 1 \mod k}\pi(m, k, a)
\]
\end{thm}
which is an immediate consequence of:
%
\begin{thm}[\cite{1993.Kaczorowski} Theorem 2]\label{1993.Kaczorowski 2}
Under GRH (Conjecture~\ref{GRH}) for $L$-functions mod $k$, $k\geq 3$, and let $u$ denote an arbitrary non-negative real number.  Then there exist constants $c_{49}=c_{49}(u)>0$ and $c_{50}=c_{50}(u)>1$ only depending on $u$, such that for every $T\geq 1$
\begin{align*}
\#\Big\{T\leq m \leq c_{50}T: \psi(m, k, 1)\geq \max_{a\not\equiv 1 \mod k}\psi(m, k ,a)+u\sqrt{m}\Big\}\geq& c_{49}T\\
\#\Big\{T\leq m \leq c_{50}T: \pi(m, k, 1)\geq \max_{a\not\equiv 1 \mod k}\pi(m, k, a)+u\frac{\sqrt{m}}{\log m}\Big\}\geq& c_{49}T\\
\#\Big\{T\leq m \leq c_{50}T: \psi(m, k, 1)\leq \max_{a\not\equiv 1 \mod k}\psi(m, k ,a)-u\sqrt{m}\Big\}\geq& c_{49}T\\
\#\Big\{T\leq m \leq c_{50}T: \pi(m, k, 1)\leq \max_{a\not\equiv 1 \mod k}\pi(m, k ,a)-u\frac{\sqrt{m}}{\log m}\Big\}\geq& c_{49}T
\end{align*} 
\end{thm}

Kaczorowski also made some progress on the racing problem~\ref{P:P-R}, with $k=5$ for $\psi(m, 5, a_i)$
\begin{thm}[\cite{1995.Kaczorowski} Theorem 1]
Assuming GRH (Conjecture~\ref{GRH}) for modoluo 5. Then for every permutation $(a_1, a_2, a_3, a_4)$ of the sequence $(1, 2, 3, 4)$ the set of $m$'s satisfying
\[
\psi(m, 5, a_1) > \psi(m, 5, a_2) > \psi(m, 5, a_3)> \psi(m, 5, a_4)
\]
has positive density.
\end{thm}

\begin{thm}[\cite{1995.Kaczorowski} Theorem 2]
Assuming GRH (Conjecture~\ref{GRH}) with $L(s, \chi)$ mod $5$, there exist three positive constants $c_{51}, c_{52}, c_{53}$ such that for every permutation $(a_1, a_2, a_3, a_4)$ of the sequence $(1, 2, 3, 4)$ and for arbitrary $T\geq 1$ we have
\[
\#\{T\leq m \leq c_{51}T: \psi(m, 5, a_1) > \hdots> \psi(m, 5, a_4),
\min_{\substack{i \neq j \\ 1\leq i,j \leq 4}}|\delta_\psi(m; 5, i, j)|\geq c_{52}\sqrt{m}\}\geq c_{53} T
\]
\end{thm}
\vskip12pt
He employs the $k$-functions bearing his name in~\cite{1996.Kaczorowski}, with
\begin{dfn} For $q>1$ a natural number, let
\[
m(q):=\begin{cases}
\h \;\;\; {\rm if} \; 2 \parallel q\\
2 \;\;\; {\rm if} \;\;8 \,|\,q \\
1 \;\;\; {\rm otherwise}
\end{cases}
\]
\[
N_q:=\frac{1}{\varphi(q)}m(q)2^{\omega(q)}
\]
where $\omega(q)$ denotes the number of distinct prime divisors of $q$.\\
Also define
\[
p^{\nu_p(q)} \| q, \;\;\;\; q_p:=qp^{-\nu_p(q)}, \;\;\;\; g_{p, q}:=\ord \;p \mod {q_p}
\]
Now let $(a, q)=1$, and denote by $\bar{a}$ the inverse of $a \mod q$: $a\bar{a}\equiv 1 \mod q$. \\
 Moreover, he put
\begin{align*}
\varrho(q, a)&:=\begin{cases} 1 \;\;\;{\rm if}\; $a$\; {\rm is\; a\; quadratic\; residue} $\mod q$\\ 
0 \;\;\; {\rm otherwise}
\end{cases}\\
\lambda(q, a)&:= \sum_{\substack{p^\alpha \|q \\ a\equiv 1 \mod {q_p}}}\frac{\log p}{p^{\alpha-1}(p-1)}+\sum_{\substack{p^\alpha|q, \alpha< \nu_p(q) \\ a \equiv 1 \mod {qp^{-\alpha}}}}\frac{\log p}{p^\alpha}\\
\delta(q, a)&:= \begin{cases} 1 \;\;\; {\rm if}\; a\equiv -1 \mod k\\
0 \;\;\;{\rm otherwise} \end{cases}
\end{align*}
\end{dfn}

Suppose $p$ a prime and that $a \mod k_p$ belongs to the cyclic multiplicity group generated by $p \mod {q_p}$.  Then denote by $l_p(a)$ the natural number uniquely determined by:
\[
1 \leq l_p(a)\leq g_{q, p}, \;\;\;\;\;\;\;\;\;p^{l_p(a)}\equiv a \mod {q_p}
\]
then set
\[
\alpha(q, a):=\sum_{p|q}\frac{\log p}{\varphi(p^{\nu_p(q)})p^{l_p(a)}}\Bigg(1-\frac{1}{p^{g_{q, p}}} \Bigg)^{-1}
\]
if there are no such primes $p$ we put $\alpha(q, a)=0$.
\begin{rmk}
An easy consequence of Dirichlet's prime number theorem is that for every $a$ to $q$ there exists a constant $b(q, a)$ such that
\[
\sum_{\substack{n\leq x \\ n\equiv a \mod q}}\frac{\Lambda(n)}{n}=\frac{1}{\varphi(q)}\log{x}+b(q, a)+o(1)
\]
as $x$ tends to infinity, where $b(q, a)$ is called the Dirichlet-Euler constant.
\end{rmk}
Finally, Kaczorowski defines the following quantities:
\begin{align*}
r^+(q, a):&=\alpha(q, a)+b(q, a)+\h \delta(q, a)\log 2 +\lambda(q, a)\\
r^-(q, a):&=\alpha(q, a)+b(q, a)+\h \delta(q, a)\log 2\\
R^+(q, a):&=r^+(q, a)-\varrho(q, a)N_q\\
R^-(q, a):&=r^-(q, a)-\varrho(q, a)N_q
\end{align*}
so he is able to prove:
\begin{thm}[\cite{1996.Kaczorowski} Theorem]
Let $k\geq 5$, $k\neq 6$ be an integer and assume the GRH (Conjecture~\ref{GRH}) $\mod k$.\\
Define permutations:
\begin{align*}
(a_2, a_3, \hdots, a_{\varphi(k)}), \;\;\; (b_2, b_3, \hdots, b_{\varphi(k)})\\
(c_2, c_3, \hdots, c_{\varphi(k)}), \;\;\; (d_2, d_3, \hdots, d_{\varphi(k)})
\end{align*}
of the set of residue classes
\[
a \mod k, \;\;\;\;\; (a, k)=1\;\;\; a\not\equiv 1 \mod k
\]
so that the following inequalities hold:
\begin{align*}
R^+(k, \bar{a}_2)&>R^+(k, \bar{a}_3)>\hdots>R^+(k, \bar{a}_{\varphi(k)}) \\
R^-(k, \bar{b}_2)&>R^-(k, \bar{b}_3)>\hdots>R^-(k, \bar{b}_{\varphi(k)})\\
r^+(k, \bar{c}_2)&>r^+(k, \bar{c}_3)>\hdots>r^+(k, \bar{c}_{\varphi(k)})\\
r^-(k, \bar{d}_2)&>r^-(k, \bar{d}_3)>\hdots>r^-(k, \bar{d}_{\varphi(k)})
\end{align*}
Then there exists a positive constant $b_0$ such that each of the sets of natural numbers each set of natural numbers
\begin{align*}
\{m\in \N: &\pi(m; k, a_2)>\hdots > \pi(m; k, a_{\varphi(k)})>\pi(m; k, 1), \\ 
&\min_{a\not\equiv b \mod k, (ab, k)=1}|\pi(m; k, a)-\pi(m; k, b)|>b_0\sqrt{m}/\log m\}\\
\{m\in \N: &\pi(m; k, 1)> \pi(m; k, a_2)>\hdots>\pi(m; k, a_{\varphi(k)}), \\ 
&\min_{a\not\equiv b \mod k, (ab, k)=1}|\pi(m; k, a)-\pi(m; k, b)|>b_0\sqrt{m}/\log m\}\\
\{m\in \N: &\psi(m; k, c_2)>\hdots > \pi(m; k, c_{\varphi(q)})>\pi(m; k, 1), \\ 
&\min_{a\not\equiv b \mod k, (ab, k)=1}|\psi(m; k, a)-\psi(m; k, b)|>b_0\sqrt{m}/\log m\}\\
\{m\in \N: &\psi(m; k, 1)> \psi(m; k, d_2)>\hdots>\psi(m; k, d_{\varphi(q)}), \\ 
&\min_{a\not\equiv b \mod k, (ab, k)=1}|\psi(m; k, a)-\psi(m; k, b)|>b_0\sqrt{m}/\log m\}
\end{align*}
has a positive density.
\end{thm}
\vskip12pt
In their ground-breaking paper~\cite{1994.Rubinstein}, M. Rubinstein and P. Sarnak resurrected the racing-problem (Problem~\ref{P:P-R}) and fully solved a few open problems with the assumption of some unproven conditions mentioned in Section~\ref{intro section}, namely GRH (Conjecture~\ref{GRH}) and:
\begin{cnj}
[Linear Independence hypothesis (LI)]\label{LI}
The imaginary part of the zeros of all Dirichlet $L$-functions attached to primitive characters modulo $q$ are linearly independent over $\Q$.
\end{cnj}
They employed the logarithmic density:
\begin{dfn}
\begin{align*}
\bar{\delta}(P):&=\limsup_{X \to \infty}\frac{1}{X}\int_{t\in P \cap [2, X]}\,\frac{dt}{t}\\
\underline{\delta}(P):&=\liminf_{X \to \infty}\frac{1}{X}\int_{t\in P \cap [2, X]}\,\frac{dt}{t}
\end{align*}
and set $\delta(P)=\bar\delta(P)=\underline\delta(P)$ if the above two limits are equal.
\end{dfn}
By introducing the vector-valued functions,
\begin{dfn}
\[
E_{k; a_1, a_2, \hdots, a_r}(x):=\frac{\log x}{\sqrt x}\times (\varphi(k)\pi(x; k, a_1)-\pi(x), \hdots, \varphi(k)\pi(x; k, a_r)-\pi(x))
\]
for $x \geq 2$.
\end{dfn}
they studied the existence of and tried to estimate the logarithmic density of of the set $P_{k; a_1, \hdots, a_r}$, where 
\begin{dfn}
 $P_{k; a_1, \hdots, a_r}$ is the set of real numbers $x \geq 2$ such that
\[
\pi(x; k, a_1) > \pi(x; q, a_2)> \dots > \pi(x; k, a_r)
\]
with $k\geq 3$ and $2 \leq r \leq \varphi(k)$, and denote $\A_r(k)$ the set of ordered $r$-tuples of distinct residue classes $(a_1, a_2, \hdots, a_r)$ modulo $k$ which are coprime to $k$. 
\end{dfn}
so they could the following theorems:
\begin{thm}[\cite{1994.Rubinstein} Theorem 1.1]
Under GRH (Conjecture~\ref{GRH}), $E_{k; a_1, a_2, \hdots, a_r}$ has a limiting distribution $\mu_{k; a_1, \hdots, a_r}$ on $\R^r$, i.e. 
\[
\lim_{X \to \infty}\frac{1}{X}\int_2^X f(E_{k; a_1, a_2, \hdots, a_r}(x)) \frac{dx}{x}=\int_{\R^r}f(x)\,d\mu_{k; a_1, \hdots, a_r}(x)
\]
for all bounded continuous functions $f$ on $\R^r$.
\end{thm}
\begin{rmk}
If it turns out that if the measure $\mu_{k; a_1, \hdots, a_r}$ is absolutely continuous then 
\[
\delta(P_{k; a_1, \hdots, a_r})=\mu_{k; a_1, \hdots, a_r}\big(\{x \in \R^r: x_1>\cdots>x_r \}\big)
\]
the shortcoming here is that they write this assuming only GRH (Conjecture~\ref{GRH}), they do not know that $\delta(P_{k; a_1, \hdots, a_r})$ exists.
\end{rmk}

\begin{dfn}
Since the measures $\mu$ are very localized but not compactly supported:
\begin{align*}
B'_R:&= \{x\in \R^r : |x|\geq R\}\\
B^+_R:&=\{x \in B'_R : \varepsilon(a_j)x_j>0 \}\\
B^-_R:&=-B^+_R
\end{align*}
\[{\rm {where}}\;\; \varepsilon(a)= \begin{cases} 1 &\;\;\; \rm{if}\;\; a \equiv 1 \mod k\\
                                                   -1 &\;\;\; \rm{otherwise}
                                                    \end{cases}\]
\end{dfn}

\begin{thm}[\cite{1994.Rubinstein} Theorem 1.2]
With GRH (Conjecture~\ref{GRH}), there are positive constants $c_{54}, c_{55}, c_{56}, c_{57}$ depending only on $k$ such that 
\begin{align*}
\mu_{k; a_1, \hdots, a_r}(B'_R)\leq& c_{54} \exp(-c_{55}\sqrt{R})\\
\mu_{k; a_1, \hdots, a_r}(B^{\pm}_R)\geq& c_{56} \exp(-\exp c_{57}{R})
\end{align*}
\end{thm}

H. L. Montgomery~\cite{1980.Montgomery} under RH (Conjecture~\ref{RH}) and LI (Conjecture~\ref{LI}) for $\zeta(s)$, investigated the tails of the measure $\mu_{1; 1}$, where he showed
\[
\exp\big(-c_{58}\sqrt{R}\exp(\sqrt{2\pi R}\,)\big)\leq \mu_{1;1}(B^\pm_R)\leq  \exp\big(-c_{59}\sqrt{R}\exp(\sqrt{2\pi R}\,)\big)
\]
Rubinstein and Sarnak \cite{1994.Rubinstein} under GRH (Conjecture~\ref{GRH}) and LI (Conjecture~\ref{LI}) have found an explicit formula for the Fourier transform of $\mu_{k; a_1, \hdots, a_r}$: the formula says that, for $r<\varphi(k)$, $\mu_{k; a_1, \hdots, a_r}=f(x)\,dx$ with a rapidly decreasing entire function $f$.  As a consequence, under GRH (Conjecture~\ref{GRH}) and LI (Conjecture~\ref{LI}) each $\delta(P_{k; a_1, \hdots, a_r})$ does indeed exist and is non-zero (including the case $r=\varphi(k)$).  Therefore, the solution to the racing problem~\ref{P:P-R} is conditionally affirmative.

\begin{dfn}
Define $(k; a_1, \hdots, a_r)$ to be {\bf{unbiased}}, if the density function of $\mu_{k; a_1, \hdots, a_r}$ is invariant under permutations of $(x_1, \hdots, x_r)$.
Where
\[
\delta(P_{k; a_1, \hdots a_r})=\frac{1}{r!}
\]
further define
\[
c(q, a):=-1 +\sum_{\substack{b^2 \equiv a \mod q \\ 0\leq b \leq q-1}}1
\]
\end{dfn}
%
\begin{thm}[\cite{1994.Rubinstein} Theorem 1.4]
Assuming GRH (Conjecture~\ref{GRH}) and LI (Conjecture~\ref{LI}) for $\chi \mod k$, $(k; a_1, \hdots a_r)$ is unbiased if and only if either $r=2$ and $c(k, a_1)=c(k, a_2)$ or $r=3$ and there exists $\rho \neq 1$ such that $\rho^3 \equiv 1 \mod k$,\; $a_2 \equiv a_1\rho \mod k$, and \; $a_3 \equiv a_1\rho^2 \mod k$.
\end{thm}

\begin{thm}[\cite{1994.Rubinstein} Theorem 1.5] Assuming GRH (Conjecture~\ref{GRH}) and LI (Conjecture~\ref{LI}) modulo $k$, for $r$ fixed,
\[
\max_{a_1, \hdots, a_r \in \mathcal{A}_q}\bigg|\delta(P_{k; a_1, \hdots, a_r})-\frac{1}{r!} \bigg| \to 0 \;\;{\rm{as}}\;\;q\to \infty 
\]
\end{thm}
%
\begin{thm}[\cite{1994.Rubinstein} Theorem 1.6]
Assume GRH (Conjecture~\ref{GRH}) and LI (Conjecture~\ref{LI}). Let $\tilde{\mu}_{k; N, R}$ be the limiting distribution of 
\[
\frac{E_{k; N, R}(x)}{\sqrt{\log q}}
\]
then $\tilde{\mu}_{k; N, R}$ converges in measure to the Gaussian $(2\pi)^{-1/2}\exp(-x^2/2)\,dx$ as $q \to \infty$.
\end{thm}
%%%%%%%%%%
\vskip12pt
A. Feuerverger and G. Martin's paper~\cite{2000.Feuerverger} first presents some \emph{biased} examples using Rubinstein and Sanark's notation $\delta_{k; a_1, \hdots, a_r}$ with numerical values: for $k=8$ and $12$:
\begin{thm}[\cite{2000.Feuerverger} Theorem 1]
Assume GRH (Conjecture~\ref{GRH}) and LI (Conjecture~\ref{LI}). Then
\begin{align*}
\delta_{8; 3, 5, 7}=\delta_{8; 7, 5, 3}=0.1928013 \pm 0.000001\\
\delta_{8; 3, 7, 5}=\delta_{8; 5, 7, 3}=0.1664263 \pm 0.000001\\
\delta_{8; 5, 3, 7}=\delta_{8; 7, 3, 5}=0.1407724 \pm 0.000001
\end{align*}
and
\begin{align*}
\delta_{12; 5, 7, 11}=\delta_{12; 11, 7, 5}=0.1984521 \pm 0.000001\\
\delta_{12; 5, 11, 7}=\delta_{12; 7, 11, 5}=0.1215630 \pm 0.000001\\
\delta_{12; 7, 5, 11}=\delta_{12; 11, 5, 7}=0.1799849 \pm 0.000001
\end{align*}
where the indicated error bounds are rigorous.
\end{thm}
%%%
\begin{thm}[\cite{2000.Feuerverger} Theorem 2]
Assume GRH (Conjecture~\ref{GRH}) and LI (Conjecture~\ref{LI}), and let $k, r \geq 2$ be integers and let $a_1, \hdots, a_r$ be distinct reduced residue classes modulo $k$.
\begin{enumerate}
\item Letting $a_j^{-1}$ denote the multiplicative inverse of $a_j$ modulo $k$, we have $\delta_{k; a_1, \hdots, a_r}=\delta_{k; a_1^{-1}, \hdots, a_r^{-1}}$.
\item If $b$ is a reduced residue class modulo $k$ such that $c(k, a_j)=c(k, ba_j)$ for each $1\leq j \leq r$, then $\delta_{k; a_1, \hdots, a_r}=\delta_{k; ba_1, \hdots, ba_r}$. In particular, this holds if $b$ is a square modulo $k$.
\item If the $a_j$ are all squares modulo $k$ and $b$ is any reduced residue class modulo $k$, then $\delta_{k; a_1, \hdots, a_r}=\delta_{k; ba_1, \hdots, ba_r}$.
\item If the $a_j$  are either all squares modulo $k$ or all non-squares modulo $k$, then $\delta_{k; a_1, \hdots, a_r}=\delta_{k; a_r, \hdots, a_1}$.
\item If $b$ is a reduced residue class modulo $k$ such that $c(k, a_j) \neq c(k, ba_j)$ for each $1 \leq j \leq r$, then $\delta_{k; a_1, \hdots, a_r} = \delta_{k; ba_r, \hdots, ba_1}$. In particular, this holds if $k$ is an odd prime power or twice an odd prime power and $b$ is any non square modulo $k$.
\end{enumerate}
\end{thm}
%%%
\begin{thm}[\cite{2000.Feuerverger} Theorem 3]
Under GRH (Conjecture~\ref{GRH}) and LI (Conjecture~\ref{LI}) for $k \geq 2$ be an integer, let $N$ and $N'$ be distinct (invertible) non-squares modulo $k$, and let $S$ and $S'$ be distinct (invertible) squares $\mod k$. Then
\begin{enumerate}
\item $\delta_{k; N, N', S} > \delta_{k; S, N', N}$;
\item $\delta_{k; N, S, S'} > \delta_{k; S', S, N}$;
\item $\delta_{k; N, S, N'} > \delta_{k; N', S, N}$ if and only if $\delta_{k; N, S} > \delta_{k; N', S}$
\item $\delta_{k; S, N, S'} > \delta_{k; S', N, S}$ if and only if $\delta_{k; S, N} > \delta_{k; S', N}$
\end{enumerate}
\end{thm}

\begin{thm}[\cite{2000.Feuerverger} Theorem 4]
Assume GRH (Conjecture~\ref{GRH}) and LI (Conjecture~\ref{LI}) for $\chi \mod k$ with $k\geq 2$. Let $r \geq 2$ be an integer, and let $a_1, \hdots, a_r$ be distinct residue classes mod $k$. Then
\[
\delta_{k; a_1, \hdots, a_r}=2^{-(r-1)}\Bigg(1+\sum_{\substack{B\subset\{1, \hdots, r-1\}\\B \neq \emptyset}}\bigg(\frac{i}{\pi} \bigg)^{|B|}\times {\rm P.V.}\int\cdots\int{\hat{\varrho}_{k; a_1, \hdots, a_r(B)}\prod_{j\in B}\frac{d\eta_j}{\eta_j}} \Bigg)
\]
where $\hat{\varrho}_{k; a_1, \hdots, a_r}(B)$ borrows the notation
\[
f(B)=f(B)(\{x_j: j\in B\})=f(\theta_1, \hdots, \theta_n)
\] 
\[
{\rm with}\;\; \theta_j=\begin{cases} x_j\;\;\;{\rm if}\;\;\; j\in B \\ 0\;\;\;{\rm otherwise}\end{cases} \]
applied to the function
\begin{multline*}
\hat{\varrho}_{k; a_1, \hdots, a_r}(\eta_1, \hdots, \eta_{r-1})=\exp\Bigg(\sum_{j=1}^{r-1}\big(c(k, a_j)-c(k, a_{j+1})\big)\eta_j \Bigg)\\
\times\prod_{\substack{\chi \mod k\\ \chi \neq \chi_0}}F\Bigg(\Bigg|\sum_{j=1}^{r-1}\big(\chi(a_j)-\chi(a_{j-1}) \big)\eta_j \Bigg|, \chi \Bigg)
\end{multline*}
with
\[
F(z, \chi):=\prod_{\substack{\gamma>0\\L(\h+i\gamma, \chi)}}J_0\bigg(\frac{2z}{\sqrt{1/4+\gamma^2}}\bigg)
\]
and 
\[
J_0(z):=\sum_{m=0}^{\infty}\frac{(-1)^m(z/2)^{2m}}{(m!)^2},
\]
 the standard Bessel function of order zero.
\end{thm}
\vskip12pt



%%%%%%%%

%%%%%%%%%
%\subsection{2000.Bays}
%~\cite{2000.Bays}
%\begin{dfn}
%Let $b:=2^{\alpha_0}P_1^{\alpha_1}\cdots P_k^{\alpha_k}$ and let $\gamma(b):=2^{k+\beta-1}$,
%where 
%\[\beta:= \begin{cases}1 \;\;\;{\rm if}\;\;\; \alpha_0=0\;\; {\rm or}\;\;1\\ 2\;\;\;{\rm if}\;\;\; \alpha_0=3\\ 3 \;\;\; {\rm if}\;\;\; \alpha_0\geq 3 \end{cases}\]
%Then $\gamma(b)$ denotes the ratio of quadratic non-residue of $b$ to to quadratic residues.\\
%Let $\Sigma_N(x, b)$ and $\Sigma_R(x, b)$ denote respectively the number of primes $\leq x$ in all progressions $bn+c$ with $c$ a quadratic non-residue of $b$ and in all progressions $bn+c'$ with $c'$ a quadratic residue of $b$.
%\end{dfn}
%\begin{thm}[\cite{2000.Bays} Theorem 1]
%For $b=4$, $q^\alpha$, or $2q^\alpha$, $x \geq 2, T \geq1$, under GRH (Conjecture~\ref{GRH})
%\begin{align*}
%\Sigma_N(x, b)-&\Sigma_R(x, b)\\
%               &=\pi(x^\h)/2+\pi(x^\h)\sum_{0\leq \gamma T}\frac{\sin \gamma \log x}{\gamma}+O_{x, T}\Bigg(\frac{x(\log x +\log T)^2}{T(\log x)}+\frac{x^\h}{\log^2 x} \Bigg)
%\end{align*}
%\end{thm}
%

%%%%%%%%%
%2000.Ng
%~\cite{2000.Ng}
%Chebyshev's bias in  different settings: Galois Group/ Class Group
%
%2001.Bays
%~\cite{2001.Bays}
%Numerical examples of Chebyshev's Bias:

K. Ford and S. Konyagin~\cite{2002.Ford_1} also investigated the Shanks-R\'enyi prime race problem: ostensibly for the races among three competitors: Let $D:=(k, a_1, a_2, a_3)$ where $a_1, a_2, a_3$ are distinct residues modulo $k$ which are coprime to $k$. Suppose for each $\chi \in C_k$ that $B(\chi)$ is a sequence of complex numbers with positive imaginary part (possibly empty, with duplicates allowed), and denote by $\mathcal{B}\,$ the system of $B(\chi)$ for $\chi \in C_k$.  Let $n(\varrho, \chi)$ be the number of occurrences of numbers $\varrho$ in $B(\chi)$.  The system $\mathcal{B}\,$ is called a barrier for $D$ if the following hold:
\begin{enumerate}
\item all numbers in each $B(\chi)$ have real part in $[\beta_2, \beta_3]$, where $1/2<\beta_2<\beta_3 \leq 1$
\item for some $\beta_1$ satisfying $1/2\leq \beta_1<\beta_2$ if we assume that for each $\chi \in C_k$ and $\varrho \in B(\chi)$, $L(s, \chi)$ has a zero of multiplicity $n(\varrho, \chi)$ at $s= \varrho$, and all other zeros of $L(s, \chi)$ in the upper half-plane have real part $\leq \beta_1$, the one of the six ordering of the three functions $\pi_{k, a_i}(x)$ does not occur for large x.
\end{enumerate}
If each sequence $B(\chi)$ is finite, we call $\mathcal{B}$ a finite barrier for $D$ and denote by $|\mathcal{B}|$ the sum of the number of elements of each sequence $B(\chi)$, counted according to multiplicity.
\begin{thm}[\cite{2002.Ford_1} Theorem 1]
For every real number $\tau>0$ and $\sigma>\h$ and for every $D=(k; a_1, a_2, a_3)$, there is a finite barrier for $D$, where each sequence $B(\chi)$ consists of numbers with real part $\leq \sigma$ and imaginary part $> \tau$.  In fact, for most $D$, there is a barrier with $|\mathcal{B}|\leq 3$.
\end{thm}




\vskip12pt


K.Ford and J. Sneed initiated the investigation of biases for products of two primes~\cite{2010.Ford}:
\begin{dfn}
Define $\pi_2(x; k, l)$ to be the number of integers $\leq x$ that are in progression $l \mod k$ and are the product of two (not necessarily distinct) primes, and 
\[
\delta_{\pi_2}(x; k, l_1, l_2):=\pi_2(x; k, l_1)-\pi_2(x; k, l_2)
\]
\end{dfn}
\begin{thm}[\cite{2010.Ford} Theorem 1.1]
Let $a, b$ be distinct elements of $A_k$, where $A_k$ denote the set 
\[
\pi(x; k, a) \sim \frac{x}{\varphi(k)\log x},
\]
then under GRH (Conjecture~\ref{GRH}) and LI (Conjecture~\ref{LI}) for $\chi \mod k$, $\delta_2(k; a, b)$ exists.  Moreover, if $a$ and $b$ are both quadratic residues modulo $q$ or both quadratic non-residues, then $\delta_2(k; a, b)=\h$. (i.e. the race is unbiased)  Otherwise, if $a$ is a quadratic non-residue and $b$ is a quadratic residue, then
\[
1-\delta(k; a, b)<\delta_{\pi_2}(k; a, b)<\h
\]
We can accurately estimate $\delta_2(q; a, b)$ borrowing methods by methods described in~\cite{1994.Rubinstein}.  In particular, we have:
\[
\delta_2(4; 3, 1) \approx 0.10572
\]
\end{thm}

\begin{thm}[\cite{2010.Ford} Theorem 1.2]
Assume GRH (Conjecture~\ref{GRH}) for each $\chi \in C_k$, $L(\h, \chi)\neq 0$ and the zeros of $L(s, \chi)$ are simple.  Then
\[
\frac{\delta_{\pi_2}(x; k, a, b)\log{x}}{\sqrt{x}\log_2 x}=\frac{N_k(b)-N_k(a)}{2\varphi(q)}-\frac{\log x}{\sqrt x}\delta_\pi(x; x, a, b)+\Sigma(x; x, a, b)
\]
where
\[
\frac{1}{Y}\int_1^Y\left|\Sigma(e^y; q, a, b)\right|^2\,dy=o(1) \;\;\;{\rm as}\;\; Y\to \infty,
\]
and $N_k(l)$ is as defined back in~\eqref{asym}.
\end{thm}
\vskip12pt

The most recent developments on the race-problem of Shanks-R\`enyi is due to Y. Lamzouri in his two papers~\cite{201x.Lamzouri_1}~\cite{201x.Lamzouri_2}, where he defines:  
%Let $q\geq 3$ and $2 \leq r \leq \varphi(q)$ be positive integers. Denote $\A_r(q)$ to be the set of ordered $r$-tuples of distinct residue classes $(a_1, a_2, \hdots, a_r)$ modulo $q$ which are coprime to $q$. For $(a_1, a_2, \hdots, a_r) \in \A_r(q)$, let $P_{q; a_1, \hdots, a_r}$ be the set of real numbers $x \geq 2$ such that
%\[
%\pi(x; q, a_1) > \pi(x; q, a_2)> \dots > \pi(x; q, a_r)
%\]
%Under GRH(Conjecture~\ref{GRH}) and LI~(Conjecture~\ref{LI}), Rubinstein and Sarnak ~\cite{1994.Rubinstein} showed that for any $(a_1, a_2, \hdots, a_r)\in\A_r(q)$ the logarithmic density $\delta_{q; a_1, \hdots a_r}$ of $P_{q; a_1, \hdots, a_r}$ defined by
%\[
%\delta_{q; a_1, \hdots a_r}:=\lim_{x\to \infty}\frac{1}{\log x}\int_{t\in P_{q; a_1, \hdots, a_r}\cap [2, x]}\frac{dt}{t},
%\]
%exists and is positive.\\
%In fact this is corollary of a stronger result that they proved: there exists an absolute continuous measure $\mu_{q; a_1, \hdots, a_r}$ such that
%\[
%\delta_{q; a_1, \hdots a_r}=\int_{x_1>x_2>\dots>x_r}\,d\mu_{q; a_1, \hdots, a_r}(x_1, \hdots, x_r)
%\]
%In a race between two residue classes $a$ and $b$ modulo $q$, Rubinstein and Sarnak proved that $\delta_{q; a, b}=\delta_{q; b, a}=1/2$ if $a$ and $b$ are both squares OR both non-squares modulo $q$.
%Otherwise $\delta_{q; a, b}> 1/2$ if $a$ is a non-square and $b$ is a square modulo $q$, and note that $\delta_{q; b, a}+\delta_{q; a, b}=1$.
%
%They showed that $\delta_{q; a, b} \to 1/2$ as $q \to \infty$, uniformly for all distinct reduced residue classes $a, b$ modulo $q$.  In fact, they proved that in general all biases disappear when $q \to \infty$:
\begin{dfn}
In the notation of Rubinstein and Sarnak~\cite{1994.Rubinstein}, let
\[
\Delta_r(k):=\max_{(a_1, a_2, \hdots, a_r)\in \A_r(k)}\bigg|\delta_{k; a_1, \hdots a_r}-\frac{1}{r!} \bigg|.
\]
\end{dfn}
and he estimates it by:
\begin{thm}[[\cite{201x.Lamzouri_2} Theorem A]
Assume GRH(Conjecture~\ref{GRH}) and LI~(Conjecture~\ref{LI}) for modulo $k$. Let $r\geq 3$ be a fixed integer.\\
If $q$ is large, then
\[
\Delta_r(k) \asymp_r \frac{1}{\log k}.
\]
\end{thm}
%
He also \emph{redefines} {\bf{unbiased}}:
\begin{dfn}
Let $(a_1, a_2, \hdots, a_r) \in \A_r(k)$, the race $\{q; a_1, \hdots, a_r\}$ is said to be \emph{unbiased} if for every permutation $\sigma$ of the set $\{ 1, 2, \hdots, r \}$ we have
\[
\delta_{q; a_{\sigma(1)}, \hdots, a_{\sigma(r)}}=\delta_{q; a_1, \hdots, a_r}=\frac{1}{r!}.
\]
\end{dfn}

Thus, a race is said to be \emph{biased} if this condition fails to hold, and towards a conjecture made by Rubinstein and Sarnak,
%%

\begin{cnj}[Rubinstein and Sarnak~\cite{1994.Rubinstein}]
When $r\geq 3$, the race $\{q; a_1, \hdots, a_r\}$ is unbiased if and only if $r=3$ and the residue classes $a_1, a_2$ and $a_3$ satisfy the condition 
\begin{equation}
a_2 \equiv a_1\varrho \mod k, \;\;\;\; a_3\equiv a_1 \varrho^2 \mod k,
\end{equation}
for some $\varrho \neq 1$ with $\varrho^3\equiv 1 \mod k$,
\end{cnj}
Lamzouri attacks by,
\begin{thm}[\cite{201x.Lamzouri_2} Theorem B]
Assume GRH~(Conjecture~\ref{GRH}) and LI~(Conjecture~\ref{LI}) modulo $k$. Given $r\geq 3$, there exists a positive number $q_0(r)$ such that for any $k\geq q_0(r)$ there are two $r$-tuples $(a_1, \hdots, a_r), (b_1, \hdots b_r)\in \A_r(k)$, with all the $a_i$'s being squares and all of the $b_i$'s being non-squares modulo $k$, and such that both the races $\{k; a_1, \hdots, a_r\}$ and $\{k; b_1, \hdots, b_r \}$ are biased.
\end{thm}
He also generalizes the definition $\A_r$ made by Rubinstein and Sarnak:
\begin{dfn}
For distinct non-zero integers $a_1, \hdots, a_2$, we define $\mathcal{Q}_{a_1, \hdots, a_r}$ to be the set of positive integers $q$ such that $a_1, \hdots, a_r$ are distinct modulo $q$ and $(q, a_i)=1$ for all $1 \leq i \leq r$.
\end{dfn}
so he could ponders upon
\begin{cnj}[\cite{201x.Lamzouri_2} Conjecture 2]
Let $r\geq 3$ and $a_1, \hdots a_r$ be distinct non-zero integers, then for all positive integers $k\in \mathcal{Q}_{a_1, \hdots, a_2}$ such that $k>2\max(|a_i|^2)$, the race $\{k; a_1, \hdots, a_r \}$ is biased.
\end{cnj}
with
\begin{thm}[\cite{201x.Lamzouri_1} Theorem C]
Assume GRH~(Conjecture~\ref{GRH}) and LI~(Conjecture~\ref{LI}).  Let $r\geq 3$ and $a_1, \hdots, a_r$ be distinct non-zero integers such that one of the following conditions occur:
\begin{enumerate}
\item There exist $1\leq i\neq j \leq r$ such that $a_i+a_j=0$.
\item There exist $1\leq i\neq j \leq r$ such that $a_i/a_j$ is a prime power. 
\end{enumerate}
Then for all but finitely many $k\in\mathcal{Q}_{a_1, \hdots, a_r}$, the race $\{k; a_1, \hdots, a_r\}$ is biased.
\end{thm}
\vskip12pt
Finally, Lamzouri dissects the  measure $\mu_{q; a_1, \hdots, a_r}$ by:
\begin{thm}[\cite{201x.Lamzouri_2} Theorem 1]
Assume GRH (Conjecture~\ref{GRH}) and LI ~(Conjecture~\ref{LI}). For $r\geq 2$ a fixed integer, let $q$ be large and $a_1, \hdots, a_r$ be distinct reduced residues modulo $q$. Then we have 
\[
\mu_{q; a_1, \hdots, a_r}\Big(\|x\|>\lambda \sqrt{\Var(q)} \Big)=(2\pi)^{-r/2}\int_{\|x\|>\lambda}\exp\bigg( -\h \sum_{i=1}^r x_i^2\bigg)\,dx+O_r\bigg(\frac{1}{\log^2 q}\bigg)
\]
for $\lambda$ in the range of $0<\lambda\leq \sqrt{\log_2 q}$.\\
Moreover, there exists an $r$-tuple of distinct reduced classes $(a_1, \hdots,a_r)$ modulo $q$, with $\lambda$ in the range of $1/4<\lambda<3/4$ such that
\[
\bigg|
\mu_{q; a_1, \hdots, a_r}\Big(\|x\|>\lambda\sqrt{\Var(q)} \Big)-(2\pi)^{-r/2}\int_{\| x\|>\lambda}\exp\bigg(-\h\sum_{i=1}^r x_i^2 \bigg)\,dx \bigg| \gg_r \frac{1}{\log^2 q}
\]
\end{thm}
%%%%
%define V !!!!!!
\begin{thm}[\cite{201x.Lamzouri_2} Theorem 2]
Assume GRH(Conjecture~\ref{GRH}) and LI~(Conjecture~\ref{LI}). Fix an integer $r\geq 2$ and a real number $A\geq 1$, $q$ large. For all distinct reduced residues $a_1, \hdots a_r$ modulo $q$, we have
\[
\exp\bigg(-c_{60}(r, A)\frac{V^2}{\varphi(q)\log q}\bigg)\ll \mu_{q; a_1, \hdots, a_r}(\|x\|>V)\ll \exp\bigg(-c_{61}(r, A)\frac{V^2}{\varphi(q)\log q}\bigg)
\]
uniformly within the range $(\varphi(q)\log q)^{1/2}\ll V \leq A\varphi(q)\log q$, where $c_{61}(r, A)> c_{60}(r, A)$ are positive numbers that only depend on $r$ and $A$.
\end{thm}
%%%
\begin{thm}[\cite{201x.Lamzouri_2} Theorem 3]
Assume GRH (Conjecture~\ref{GRH}) and LI~(Conjecture~\ref{LI}). For integer $r\geq 2$ and $q$ large, if $V/(\varphi(q)\log q) \to \infty$ and $V/(\varphi(q)\log^2 q) \to 0$ as $q \to \infty$, then for all distinct reduced residues $a_1, \hdots  a_r$ modulo $q$, 
\[\exp\bigg(-c_{63}(r)\frac{V^2}{\varphi(q)\log q}\exp\bigg(c_{65}(r)\frac{V}{\varphi(q)\log q}\bigg)\bigg) \ll \mu_{q; a_1, \hdots, a_r}(\|x\|>V)
\]
and
\[
\mu_{q; a_1, \hdots, a_r}(\|x\|>V) \ll \exp \bigg(-c_{62}(r)\frac{V^2}{\varphi(q)\log q}\exp\bigg(c_{64}(r)\frac{V}{\varphi(q)\log q} \bigg)\bigg)
\]
where $c_{63}(r)>c_{62}(r)$, and $c_{65}(r)>c_{64}(r)$ are positive numbers only depend on $r$.
\end{thm}
%
\begin{thm}[\cite{201x.Lamzouri_2} Theorem 4]
Assume GRH (Conjecture~\ref{GRH}) and LI~(Conjecture~\ref{LI}). For $q$ large, let $r$ with $2\leq r \leq \varphi(q)-1$ be an integer.  If $V/(\varphi(q)\log^2 q)\to \infty$ as $q \to \infty$, then for all distinct reduced residue classes $a_1, \hdots a_r$ modulo $q$, the tail $\mu_{q; a_1, \hdots a_r}(|x|_\infty > V)$ equals
\[
\exp\Bigg(-e^{L(q)}\sqrt{\frac{2(\varphi(q)-1)V}{\pi}} \exp\bigg(\sqrt{L(q)^2+\frac{2\pi V}{\varphi(q)-1}}\bigg)\bigg(1+O\bigg(\Big(\frac{\varphi(q)\log^2(q)}{V} \Big)^{1/4} \bigg) \bigg)\Bigg),
\]
where 
\[
L(q)=\frac{\varphi(q)}{\varphi(q)-1}\bigg(\log q -\sum_{p|q}\frac{\log p}{p-1} \bigg) + A_0 -\log \pi,
\]
and 
\[
A_0:=\int_0 ^1\frac{\log I_0(t)}{t^2}\,dt +\int_1 ^\infty \frac{\log I_0(t)-t}{t^2}\,dt +1, 
\]
with $\displaystyle I_0(t)=\sum_{n=0}^\infty \frac{(t/2)^{2n}}{(n!)^2}$ being the modified Bessel function of order zero.
\end{thm}
giving a conditional bound of the tails to the measure $\mu_{q; a_1, \hdots, a_r}$, fully generalized the work done by Montgomery~\cite{1980.Montgomery} on $\mu_{1; 1}$.

\section{Papers I'd like to also include}

Ford and Konyagin, {\it The prime number race and zeros of $L$-functions off the critical line}, 2002 (I have this reprint)

Ford and Konyagin, {\it The prime number race and zeros of $L$-functions off the critical line, II}, 2003 (I have this preprint)

Ford and Konyagin, {\it Chebyshev�s conjecture and the prime number race}, 2004 (I have this preprint); see also Cojocaru's Math Review

Myerscough, {\it Application of an accurate remainder term in the calculation of Residue Class Distributions} (I have this preprint)

Kisilevsky and Rubinstein, {\it Chebotarev sets} (I have this preprint)

in general, look through Lamzouri's work and Fiorilli's work

%%%%%%%%%%%%%%%%
\section{References}
\begin{biblist}



%%%---------------------------------------------------------------------------------------
\bib{1853.Chebyshev}{article}{
    author={Chebyshev, P.},
     title={\href{run:bib/1853.Chebyshev.pdf}
            {Lettre de M. le professeur Tch\'{e}bychev a M. Fuss, sur un nouveau th\'{e}or\`eme r\'elatif aux nombres premiers contenus dans la formes 4n+1 et 4n+3. (French)}},
      date={1853},
   journal={Bull. de la Classe phys. math. de l'Acad. Imp. des Sciences St. Petersburg},
    volume={11},
     pages={208},
} 
%%%-------------------------------------------------------------------------------
\bib{1891.Phragmen}{article}{
    author={Phragm\'en, P.},
     title={\href{run:bib/1891.Phragmen.pdf}
             {Sur le logarithme int\'{e}gral et la fonction $f(x)$ de Riemann (French)}},
   year = {1891}
   journal={\"{O}fversigt af Kongl. Vetenskaps-Akademiens F\"{o}handlingar.},
    volume={48},
     pages={599-616},
}
%%%---------------------------------------------------------------------------------------
\bib{1905.Landau}{article}{
    author={Landau, E.},
     title={\href{run:bib/1905.Landau.pdf}
             {\"{U}ber einen Satz von Tschebyschef (German)}},
   journal = {Mathematische Annalen},
   publisher = {Springer Berlin / Heidelberg},
   issn = {0025-5831},
   keyword = {Mathematics and Statistics},
   pages = {527-550},
   volume = {61},
   issue = {4},
   year = {1905}
}
%%%---------------------------------------------------------------------------------------
\bib{1918.Landau.1}{article}{
    author={Landau, E.},
     title={\href{run:bib/1918.Landau.1.pdf}
             {\"{U}er einige \"{a}ltere Vermutungen und Behauptungen in der Primzahltheorie (German)}},
   year = {1918}
   journal={Math. Zeitschr.},
    volume={1},
     pages={1-24},
}
%%%---------------------------------------------------------------------------------------
\bib{1918.Landau.2}{article}{
    author={Landau, E.},
     title={\href{run:bib/1918.Landau.2.pdf}
             {\"{U}er einige \"{a}ltere Vermutungen und Behauptungen in der Primzahltheorie (German)}},
   year = {1918}
   journal={Zweite Abhandlung},
    volume={ibid.},
     pages={213-219},
}
%%%---------------------------------------------------------------------------------------
\bib{1918.Littlewood}{article}{
    author={Littlewood, J. E. },
     title={\href{run:bib/1918.Littlewood.pdf}
             {Sur la distribution des nombres premiers (French)}},
  date={22 June 1914},
  journal={Comptes Rendus}
}
%%%---------------------------------------------------------------------------------------
\bib{1930.Polya}{article}{
    author={P\'olya, G.},
     title={\href{run:bib/1930.Polya.pdf}
             {\"{U}ber das Vorzeichen des Restgliedes im Primzahltheorie (German)}},
  date={1930},
   journal={G\"{o}tt. Nachr.},
     pages={19-27},
}
%%%---------------------------------------------------------------------------------------
\bib{1933.Skewes.I}{article}{
    author={Skewes, S.},
     title={\href{run:bib/1933.Skewes.I.pdf}
             {On the difference $\pi(x)-{\li}(x)$ (I)}},
  date={1933},
  journal={Math. Tables and other aids to computation}
  volume={13}
  pages={272-284}
}
%%%---------------------------------------------------------------------------------------
\bib{1936.Ingham}{article}{
    author={Ingham, A.E.},
     title={\href{run:bib/1936.Ingham.pdf}
             {A note on the distribution of primes}},
  date={1936},
  journal={Acta Arith.}
  volume={1}
  pages={201-211}
}
%%%---------------------------------------------------------------------------------------
\bib{1941.Wintner}{article}{
    author={Wintner, A.},
     title={\href{run:bib/1941.Wintner.pdf}
             {On the distribution function of the remainder term of the Prime Number Theorem}},
  date={1941},
  journal={Amer. J. Math.}
  volume={63}
  pages={233-248}
}
%%%---------------------------------------------------------------------------------------
\bib{1945.Siegel}{article}{
    author={Siegel, C. L.},
     title={\href{run:bib/1945.Siegel.pdf}
             {On the zeros of the Dirichlet L-functions}},
  date={1945},
  journal={Annals of Math.}
  volume={46}
  issue={3}
  pages={409-422}
}
%%%---------------------------------------------------------------------------------------
%%\bib{1950.Rosser}{article}{
%%    author={Rosser, J. B.},
%%     title={\href{run:bib/1950.Rosser.pdf}
%%             {Real roots of real Dirichlet L-series}},
%%  date={1950},
%%}
%%%---------------------------------------------------------------------------------------
\bib{1955.Skewes.II}{article}{
     author={Skewes, S.},
     title={\href{run:bib/1955.Skewes.II.pdf}
             {On the difference $\pi(x)-{\li}(x)$ (II)}},
  date={1955},
  journal={Math. Tables and other aids to computataion}
  volume={13}
  pages={272-284}
}
%%%---------------------------------------------------------------------------------------
\bib{1957.Leech}{article}{
    author={Leech, J.},
     title={\href{run:bib/1957.Leech.pdf}
             {Note on the distribution of prime numbers}},
  date={1957},
  journal={ J. London Math. Soc.}
  volume={ 32} 
  pages={56�58}
}
%%%---------------------------------------------------------------------------------------
\bib{1959.Shanks}{article}{
    author={Shanks, D.},
     title={\href{run:bib/1959.Shanks.pdf}
             {Quadratic Residues and the Distribution of Primes}},
  date={1959},
  journal={Math. Comp. }
  volume={13} 
  pages={272�284}
}
%%%---------------------------------------------------------------------------------------
\bib{1961.Knaposwki}{article}{
    author={Knaposwki, S.},
     title={\href{run:bib/1961.Knapowski.pdf}
             {On sign-changes in the remainder-term in the prime-number formula}},
  date={1961},
  journal={Journ. Lond. Math. Soc.} 
}
%%%---------------------------------------------------------------------------------------
\bib{1962.Knaposwki_(pi-li)_1}{article}{
    author={Knaposwki, S.},
     title={\href{run:bib/1962.Knapowski_(pi-li)_1.pdf}
             {On sign changes of $\pi(x)-$\li$(x)$ }},
  date={1962},
}
%%%---------------------------------------------------------------------------------------
\bib{1962.Knaposwki_1}{article}{
    author={Knaposwki, S.},
    author={Tur\'an, P.},
     title={\href{run:bib/1962.Knapowski_1.pdf}
             {Comparative Prime-Number Theory I}},
  date={1962},
  journal={Acta Math. Acad. Sci. Hung.},
  volume={13},
   pages={299-314},
}
%%%---------------------------------------------------------------------------------------
\bib{1962.Knaposwki_2}{article}{
    author={Knaposwki, S.},
    author={Tur\'an, P.},
     title={\href{run:bib/1962.Knapowski_2.pdf}
             {Comparative Prime-Number Theory II}},
  date={1962},
    journal={Acta Math. Acad. Sci. Hung.},
  volume={13},
   pages={315-342},
}
%%%---------------------------------------------------------------------------------------
\bib{1962.Knaposwki_3}{article}{
    author={Knaposwki, S.},
    author={Tur\'an, P.},
     title={\href{run:bib/1962.Knapowski_3.pdf}
             {Comparative Prime-Number Theory III}},
  date={1962},
    journal={Acta Math. Acad. Sci. Hung.},
  volume={13},
   pages={343-364},
}
%%%---------------------------------------------------------------------------------------
\bib{1963.Knaposwki_4}{article}{
    author={Knaposwki, S.},
    author={Tur\'an, P.},
     title={\href{run:bib/1963.Knapowski_4.pdf}
             {Comparative Prime-Number Theory IV}},
  date={1963},
    journal={Acta Math. Acad. Sci. Hung.},
  volume={14},
   pages={31-42},
}
%%%---------------------------------------------------------------------------------------
\bib{1963.Knaposwki_5}{article}{
    author={Knaposwki, S.},
    author={Tur\'an, P.},
     title={\href{run:bib/1963.Knapowski_5.pdf}
             {Comparative Prime-Number Theory V}},
  date={1963},
    journal={Acta Math. Acad. Sci. Hung.},
  volume={14},
   pages={43-63},
}
%%%---------------------------------------------------------------------------------------
\bib{1963.Knaposwki_6}{article}{
    author={Knaposwki, S.},
    author={Tur\'an, P.},
     title={\href{run:bib/1963.Knapowski_6.pdf}
             {Comparative Prime-Number Theory VI}},
  date={1963},
    journal={Acta Math. Acad. Sci. Hung.},
  volume={14},
   pages={64-78},
}
%%%---------------------------------------------------------------------------------------
\bib{1963.Knaposwki_7}{article}{
    author={Knaposwki, S.},
    author={Tur\'an, P.},
     title={\href{run:bib/1963.Knapowski_7.pdf}
             {Comparative Prime-Number Theory VII}},
  date={1963},
    journal={Acta Math. Acad. Sci. Hung.},
  volume={14},
   pages={241-250},
}
%%%---------------------------------------------------------------------------------------
\bib{1963.Knaposwki_8}{article}{
    author={Knaposwki, S.},
    author={Tur\'an, P.},
     title={\href{run:bib/1963.Knapowski_8.pdf}
             {Comparative Prime-Number Theory VIII}},
  date={1963},
  journal={Acta Math. Acad. Sci. Hung.},
  volume={14},
   pages={251-268},
}
%%%---------------------------------------------------------------------------------------
\bib{1964.Turan_1}{article}{
    author={Knaposwki, S.},
    author={Tur\'an, P.},
     title={\href{run:bib/1964.Turan_1.pdf}
             {Further Developments in the Comparative Prime-Number Theory I}},
  date={1964},
  journal={Acta. Arith.}
  volume={9}
  pages={23-40}
}
%%%---------------------------------------------------------------------------------------
\bib{1964.Turan_2}{article}{
    author={Knaposwki, S.},
    author={Tur\'an, P.},
     title={\href{run:bib/1964.Turan_2.pdf}
             {Further Developments in the Comparative Prime-Number Theory II}},
  date={1964},
  journal={Acta. Arith.}
  volume={10}
  pages={293-313}
}
%%%---------------------------------------------------------------------------------------
\bib{1965.Turan_3}{article}{
    author={Knaposwki, S.},
    author={Tur\'an, P.},
     title={\href{run:bib/1965.Turan_3.pdf}
             {Further Developments in the Comparative Prime-Number Theory III}},
  date={1965},
  journal={Acta. Arith.}
  volume={11}
  pages={115-127}
}
%%%---------------------------------------------------------------------------------------
\bib{1965.Turan_4}{article}{
    author={Knaposwki, S.},
    author={Tur\'an, P.},
     title={\href{run:bib/1965.Turan_4.pdf}
             {Further Developments in the Comparative Prime-Number Theory IV}},
  date={1965},
  journal={Acta Arith.}
  volume={14}
  pages={31-42}
}
%%%---------------------------------------------------------------------------------------
\bib{1965.Turan_5}{article}{
    author={Knaposwki, S.},
    author={Tur\'an, P.},
     title={\href{run:bib/1965.Turan_5.pdf}
             {Further Developments in the Comparative Prime-Number Theory V}},
  date={1965},
  journal={Acta Arith.}
  volume={14}
  pages={43-63}
}
%%%---------------------------------------------------------------------------------------
\bib{1965.Knapowski}{article}{
    author={Knaposwki, S.},
    author={Tur\'an, P.},
     title={\href{run:bib/1965.Knapowski.pdf}
             {On an assertion of \v{C}eby\v{s}ev}},
  date={1965},
  journal={J. Analyse Math}
  volume={14}
  pages={267-274}

}
%%%---------------------------------------------------------------------------------------
\bib{1966.Turan_6}{article}{
    author={Knaposwki, S.},
    author={Tur\'an, P.},
     title={\href{run:bib/1966.Turan_6.pdf}
             {Further Developments in the Comparative Prime-Number Theory VI}},
  date={1966},
  journal={Acta Arith.}
  volume={12}
  pages={85-96}

}
%%%---------------------------------------------------------------------------------------
\bib{1967.Katai}{article}{
    author={K\'atai, I.},
     title={\href{run:bib/1967.Katai.pdf}
             {On investigations in the comparative prime number theory}},
  date={1967},
  journal={Acta Math. Acad. Sci. Hungar}
  volume={18}
  pages={379-391}

}
%%%---------------------------------------------------------------------------------------
\bib{1971.Stark}{article}{
    author={Stark, H.},
     title={\href{run:bib/1971.Stark.pdf}
             {A problem in comparative prime number theory}},
  date={1971},
  journal={Acta Arith.}
  volume={18}
  pages={311-320}
}
%%%---------------------------------------------------------------------------------------
\bib{1972.Turan_7}{article}{
    author={Knaposwki, S.},
    author={Tur\'an, P.},
     title={\href{run:bib/1972.Turan_7.pdf}
             {Further Developments in the Comparative Prime-Number Theory VII}},
  date={1972},
  journal={Acta. Arith.}
  volume={21}
  pages={193-201}
}
%%%---------------------------------------------------------------------------------------
%\bib{1976.Knapowski_(pi-li)_1}{article}{
%    author={Knaposwki, S.},
%     title={%\href{run:bib/1976.Knapowski_(pi-li)_2.pdf}
%             {On sign changes of $\pi(x)-\li(x)$. I}},
%  date={1976},
%}

%%%---------------------------------------------------------------------------------------
\bib{1976.Knapowski_(pi-li)_2}{article}{
    author={Knaposwki, S.},
     title={\href{run:bib/1976.Knapowski_(pi-li)_2.pdf}
             {On sign changes of $\pi(x)-\li(x)$. II}},
  date={1976},
  journal={Monatsh. Math.}
  volume={82}
  pages={163-175}
}
%%%---------------------------------------------------------------------------------------
\bib{1977.Bays}{article}{
 AUTHOR = {Bays, Carter and Hudson, Richard H.},
     TITLE = {\href{run:bib/1977.Bays.pdf}
     {The segmented sieve of {E}ratosthenes and primes in arithmetic
              progressions to {$10^{12}$}}},
   JOURNAL = {Nordisk Tidskr. Informationsbehandling (BIT)},
    VOLUME = {17},
      YEAR = {1977},
     PAGES = {121--127},
}
%%%---------------------------------------------------------------------------------------
\bib{1977.Pintz_3a}{article}{
    author={Pintz, J.},
     title={\href{run:bib/1977.Pintz_3a.pdf}
             {On the remainder term of the prime number formula III. Sign changes of $\pi - \li(x)$}},
  date={1977},
  journal={Studia Sci. Math. Hungar}
  volume={12}
  pages={345-369}
}
%%%---------------------------------------------------------------------------------------
\bib{1978.Pintz_4a}{article}{
    author={Pintz, J.},
     title={\href{run:bib/1978.Pintz_4a.pdf}
             {On the remainder term of the prime number formula IV. Sign changes of $\pi - \li(x)$}},
  date={1978},
  journal={Studia Sci. Math. Hungar}
  volume={13}
  pages={29-42}
  }
 %%%---------------------------------------------------------------------------------------
\bib{1979.Besenfelder_1}{article}{
    author={Besenfelder, J.},
     title={\href{run:bib/1979.Besenfelder_1.pdf}
             {\"{U}ber eine Vermutung von Tschebyschef. I. (German)}},
  date={1979},
  journal={J. Reine Angrew. Math.}
  volume={307/308}
  pages={411-417}
}
%%%---------------------------------------------------------------------------------------
\bib{1979.Pintz_1a}{article}{
    author={Pintz, J.},
     title={\href{run:bib/1979.Pintz_1a.pdf}
             {On the remainder term of the prime number formula I. On a problem of Littlewood}},
  date={1979},
  journal={Acta Arith.} 
  volume={36},
  pages={27-51}
}
%%%---------------------------------------------------------------------------------------
\bib{1980.Bentz}{article}{
    author={Bentz, H.},
    author={Pintz, J.},
     title={\href{run:bib/1980.Bentz.pdf}
             {Quadratic Residues and the Distribution of Prime Numbers}},
  date={1980},
  journal={Monatsh. Math.}
  volume={90 no.2}
  pages={91-100}

}
%%%---------------------------------------------------------------------------------------
\bib{1980.Besenfelder_2}{article}{
    author={Besenfelder, J.},
     title={\href{run:bib/1980.Besenfelder_2.pdf}
             {\"{U}ber eine Vermutung von Tschebyschef. II.}},
  date={1980},
  journal={J. Reine Angrew. Math.}
  volume={313}
  pages={52-58}
}
%%%---------------------------------------------------------------------------------------
\bib{1980.Montgomery}{book}{
AUTHOR = {Montgomery, H. L.},
     TITLE = {The zeta function and prime numbers},
 BOOKTITLE = {Proceedings of the {Q}ueen's {N}umber {T}heory {C}onference,
              1979 ({K}ingston, {O}nt., 1979)},
    SERIES = {Queen's Papers in Pure and Appl. Math.},
    VOLUME = {54},
     PAGES = {1--31},
 PUBLISHER = {Queen's Univ.},
   ADDRESS = {Kingston, Ont.},
      YEAR = {1980},}
%%%---------------------------------------------------------------------------------------
\bib{1980.Pintz_2a}{article}{
    author={Pintz, J.},
     title={\href{run:bib/1980.Pintz_2a.pdf}
             {On the remainder term of the prime number formula II. On a problem of Ingham}},
  date={1980},
  journal={Acta Arith.} 
  volume={37},
  pages={209-220}
}
%%%---------------------------------------------------------------------------------------
\bib{1980.Pintz_5a}{article}{
    author={Pintz, J.},
     title={\href{run:bib/1980.Pintz_5a.pdf}
             {On the remainder term of the prime number formula V. Effective Mean Value Theorems}},
  date={1980},
  journal={Studia Sci. Math. Hungar.} 
  volume={15},
  pages={215-223}
}
%%%---------------------------------------------------------------------------------------
\bib{1980.Pintz_6a}{article}{
    author={Pintz, J.},
     title={\href{run:bib/1980.Pintz_6a.pdf}
             {On the remainder term of the prime number formula VI. Effective Mean Value Theorems}}
  date={1980},
  journal={Studia Sci. Math. Hungar.} 
  volume={15},
  pages={225-230}
}
%%%---------------------------------------------------------------------------------------
\bib{1982.Bentz}{article}{
    author={Bentz, Has-J.},
     title={\href{run:bib/1982.Bentz.pdf}
             {Discrepancies in the Distribution of Prime Numbers}},
  date={1982},
  journal={J. Number Theory} 
  volume={15},
  pages={252-274}
}
%%%---------------------------------------------------------------------------------------
\bib{1984.Kaczorowski_1a}{article}{
    author={Kaczorowski, J.},
     title={\href{run:bib/1984.Kaczorowski_1a.pdf}
             {On sign-changes in the remainder-term of the prime-number formula, I.}},
  date={1984},
  journal={Acta Arith.} 
  volume={44},
  pages={365-377}
}
%%%---------------------------------------------------------------------------------------
\bib{1984.Kaczorowski_2a}{article}{
    author={Kaczorowski, J.},
     title={\href{run:bib/1984.Kaczorowski_2a.pdf}
             {On sign-changes in the remainder-term of the prime-number formula, II.}},
  date={1985},
  journal={Acta Arith.} 
  volume={45},
  pages={65-74}
}
%%%---------------------------------------------------------------------------------------
\bib{1984.Pintz_1}{article}{
    author={Pintz, J.},
    author={Salerno, S.},
     title={\href{run:bib/1984.Pintz_1.pdf}
             {Irregularities in the distribution of primes in arithmetic progressions, I.}},
  date={1984},
  journal={Arch. Math. (Besel)} 
  volume={42},
  pages={439-447}
}
%%%---------------------------------------------------------------------------------------
\bib{1984.Pintz_2}{article}{
    author={Pintz, J.},
    author={Salerno, S.},
     title={\href{run:bib/1984.Pintz_2.pdf}
             {Irregularities in the distribution of primes in arithmetic progressions, II.}},
  date={1984},
  journal={Arch. Math. (Besel)} 
  volume={43},
  pages={351-357}
}
%%%---------------------------------------------------------------------------------------
\bib{1986.Kaczorowski_1}{article}{
    author={Kaczorowski, J.},
    author={Pintz, J.},
     title={\href{run:bib/1986.Kaczorowski_1.pdf}
             {Oscillatory Properties of arithmetical functions. I.}},
  date={1986},
  journal={Acta Arith. Hungar.} 
  volume={48},
  pages={173-185}
}
%%%---------------------------------------------------------------------------------------
\bib{1986.Robin}{article}{
    AUTHOR = {Robin, G},
     TITLE = {\href{run:bib/1986.Robin.pdf}
                  {Irr\'egularit\'es dans la distribution des nombres premiers
              dans les progressions arithm\'etiques}},
   JOURNAL = {Ann. Fac. Sci. Toulouse Math. (5)},
  FJOURNAL = {Toulouse. Facult\'e des Sciences. Annales. Math\'ematiques.
              S\'erie 5},
    VOLUME = {8},
      YEAR = {1986/87},
    NUMBER = {2},
     PAGES = {159--173},
      ISSN = {0240-2955},
   MRCLASS = {11N13 (11Y35)},
  MRNUMBER = {928842 (89c:11143)},
MRREVIEWER = {D. R. Heath-Brown},
       URL = {http://www.numdam.org/item?id=AFST_1986-1987_5_8_2_159_0},
}
%%%---------------------------------------------------------------------------------------
\bib{1987.Kaczorowski_3a}{article}{
    author={Kaczorowski, J.},
     title={\href{run:bib/1987.Kaczorowski_3a.pdf}
             {On sign-changes in the remainder-term of the prime-number formula, III.}},
  date={1987},
  journal={Acta Arith.} 
  volume={48},
  pages={347-371}
}
%%%---------------------------------------------------------------------------------------
\bib{1987.Kaczorowski_2}{article}{
    author={Kaczorowski, J.},
    author={Pintz, J.},
     title={\href{run:bib/1987.Kaczorowski_2.pdf}
            {Oscillatory Properties of arithmetical functions. II.}},
  date={1987},
  journal={Acta Arith. Hungar} 
  volume={48},
  pages={441-453}

}
%%%---------------------------------------------------------------------------------------
\bib{1988.Kaczorowski_4a}{article}{
    author={Kaczorowski, J.},
     title={\href{run:bib/1988.Kaczorowski_4a.pdf}
             {On sign-changes in the remainder-term of the prime-number formula, IV.}},
  date={1988},
  journal={Acta Arith.} 
  volume={50},
  pages={15-21}
}
%%%---------------------------------------------------------------------------------------
\bib{1989.Szydlo_1}{article}{
    author={Szyd\l o, B.},
     title={\href{run:bib/1989.Szydlo_1.pdf}
            {\"Uber Vorzeichenwechsel einiger arithmetischer Funktionen. I (German)}},
  date={1989},
  journal={Math. Ann.} 
  volume={283},
  pages={139-149}
}
%%%---------------------------------------------------------------------------------------
\bib{1989.Szydlo_2}{article}{
    author={Szyd\l o, B.},
     title={\href{run:bib/1989.Szydlo_2.pdf}
            {\"Uber Vorzeichenwechsel einiger arithmetischer Funktionen. II (German)}},
  date={1989},
  journal={Math. Ann.} 
  volume={283},
  pages={151-163}
}

%%%---------------------------------------------------------------------------------------
\bib{1989.Szydlo_3}{article}{
    author={Szyd\l o, B.},
     title={\href{run:bib/1989.Szydlo_3.pdf}
            {\"Uber Vorzeichenwechsel einiger arithmetischer Funktionen. III (German)}},
  date={1989},
  journal={Monatsh. Math.} 
  volume={108},
  pages={325-336}
}
%%%---------------------------------------------------------------------------------------
\bib{1993.Kaczorowski}{article}{
    author={Kaczorowski, J.},
     title={\href{run:bib/1993.Kaczorowski.pdf}
            {A contribution to the Shanks-R\'enyi race problem}},
  date={1993},
  journal={Quart. J. Math. Oxford Ser (2)} 
  volume={44},
  pages={451-458}
}
%%%---------------------------------------------------------------------------------------
\bib{1994.Rubinstein}{article}{
    author={Rubinstein, M.},
    author={Sarnak, P.},
     title={\href{run:bib/1994.Rubinstein.pdf}
            {Chebyshev's Bias}},
  date={1994},
  journal={Experiment. Math.} 
  volume={3},
  pages={173-197}
}
%%%---------------------------------------------------------------------------------------
\bib{1995.Kaczorowski}{article}{
    author={Kaczorowski, J.},
     title={\href{run:bib/1995.Kaczorowski.pdf}
            {On the Shanks-R\'enyi race problem mod 5}},
  date={1995},
  journal={J. Number Theory} 
  volume={50},
  pages={106-118}
}
%%%---------------------------------------------------------------------------------------
\bib{1996.Kaczorowski}{article}{
    author={Kaczorowski, J.},
     title={\href{run:bib/1996.Kaczorowski.pdf}
            {On the Shanks-R\'enyi race problem}},
  date={1996},
  journal={Acta Arth.} 
  volume={74},
  pages={31-46}
}
%%%---------------------------------------------------------------------------------------
\bib{2000.Bays}{article}{
    author={Bays, C.}
    author={Hudson, R.}
     title={\href{run:bib/2000.Bays.pdf}
            {Zeroes of Dirichlet $L$-Functions and Irregularities in the Distribution of Primes}},
  date={2000},
  journal={Math. Comp.} 
  volume={69},
  pages={861-866}
}
%%%---------------------------------------------------------------------------------------
\bib{2000.Feuerverger}{article}{
    author={Feuerverger, A.},
    author={Martin, G.},
     title={\href{run:bib/2000.Feuerverger.pdf}
            {Biases in the Shanks�-R\'enyi Prime Number Race}},
  date={2000},
  journal={Experiment. Math.} 
  volume={9},
  pages={535-570}
}
%%%---------------------------------------------------------------------------------------
\bib{2000.Ng}{article}{
    author={Ng, N.},
     title={\href{run:bib/2000.Ng.pdf}
            {Limiting processes and Zeros of Artin L-Functions,}},
  date={2000},
  journal={Ph.D. Thesis, University of British Columbia.} 
}
%%%---------------------------------------------------------------------------------------
\bib{2000.Puchta}{article}{
    author={Puchta, J.-C.},
     title={\href{run:bib/2000.Puchta.pdf}
            {On large oscillations of the remainder of the prime number theorems}},
  date={2000},
  journal={Acta Math. Hungar.} 
  volume={87},
  pages={213-227}
}
%%%---------------------------------------------------------------------------------------
\bib{2001.Bays}{article}{
    author={Bays, C.},
    author={Ford, K.},
    author={Hudson, R. H.}, 
    author={Rubinstein, M.},
         title={\href{run:bib/2001.Bays.pdf}
            {Zeros of Dirichlet L-functions near the Real Axis and Chebyshev's Bias}},
  date={2001},
  journal={J. Number Theory} 
  volume={87},
  pages={54-76}
}
%%%---------------------------------------------------------------------------------------
\bib{2002.Ford_1}{article}{
    author={Ford, K.},
    author={Konyagin, S.},
         title={\href{run:bib/2002.Ford_1.pdf}
            {The Prime Number Race and zeros of Dirichlet $L$-Functions}},
  date={2002},
  journal={Duke Math. J.} 
  volume={113},
  pages={313-330}
}
%%%---------------------------------------------------------------------------------------
\bib{2002.Ford_2}{book}{
    author={Ford, K.},
    author={Konyagin, S.},
         title={\href{run:bib/2002.Ford_2.pdf}
            {Chebyshev's conjecture and the prime number race}},
  date={2002},
  BOOKTITLE = {I{V} {I}nternational {C}onference ``{M}odern {P}roblems of
              {N}umber {T}heory and its {A}pplications'': {C}urrent
              {P}roblems, {P}art {II} ({R}ussian) ({T}ula, 2001)},
     PAGES = {67--91},
 PUBLISHER = {Mosk. Gos. Univ. im. Lomonosova, Mekh.-Mat. Fak., Moscow},
}
%%%---------------------------------------------------------------------------------------
\bib{2002.Ford_3}{book}{
    author={Ford, K.},
    author={Konyagin, S.},
         title={\href{run:bib/2002.Ford_3.pdf}
            {The prime number race and zeros of Dirichlet L-functions off the critical line, II}},
  date={2002},
  BOOKTITLE = {Proceedings of the session in analytic number theory and {D}iophantine equations ({B}onn, {J}anuary--{J}une 2002)},
%     PAGES = {67--91},
 PUBLISHER = {Bonner Mathematische Schiften, Nr. 360 (D. R. Heath-Brown, B. Z. Moroz, editors), Bonn, Germany, 2003},
}
%%%---------------------------------------------------------------------------------------
\bib{2002.Martin}{book}{
    author={Martin, G.},
         title={\href{run:bib/2002.Martin.pdf}
            {Asymmetries in the Shanks-R\'enyi prime number race}},
  date={2002},
  BOOKTITLE = {Number theory for the millennium, {II} ({U}rbana, {IL}, 2000)},
     PAGES = {403--415},
 PUBLISHER = {A K Peters},
}
%%%---------------------------------------------------------------------------------------
\bib{2004.Schlage-Puchta}{article}{
    author={Schlage-Puchta, J.-C.},
         title={\href{run:bib/2004.Schlage-Puchta.pdf}
            {Sign changes of $\pi(x; q, 1)-\pi(x; q, a)$}},
  JOURNAL = {Acta Math. Hungar.},
    VOLUME = {102},
      YEAR = {2004},
     PAGES = {305--320},
     }
%%%---------------------------------------------------------------------------------------
\bib{2006.Granville}{article}{
    author={Granville, A.},
    author={Martin, G.},
     title={\href{run:bib/2006.Granville.pdf}
            {Prime Number Races}},
  date={2006},
  JOURNAL = {Amer. Math. Monthly},
    VOLUME = {113},
     PAGES = {1--33},
}
%%%---------------------------------------------------------------------------------------
\bib{2010.Ford}{article}{
    author={Ford, K.},
    author={Sneed, J.},
     %title={\href{http://www.ams.org/mathscinet-getitem?mr=2375776}
     title={\href{run:bib/2010.Ford.pdf}
            {Chebyshev's Bias for Products of Two Primes}},
  date={2010},
  JOURNAL = {Experiment. Math.},
  FJOURNAL = {Experimental Mathematics},
    VOLUME = {19},
     PAGES = {385--398},
}
%%%---------------------------------------------------------------------------------------
\bib{201x.Fiorilli}{article}{
    author={Fiorilli, D.},
    author={Martin, G.},
     title={\href{run:bib/2011x.Fiorilli.pdf}
            {Inequities in the Shanks-R\'enyi prime number race: an asymptotic formula for the densities}},
  date={to appear},
  journal ={J. Reine Angew. Math.},
}
%%%---------------------------------------------------------------------------------------
\bib{201x.Lamzouri_1}{article}{
    author={Lamzouri, Y.},
     title={\href{run:bib/2011x.Lamzouri_1.pdf}
            {Prime number races with three or more competitors.}}
            journal={	{\tt arXiv:1108.5342v2}}
}
%%%---------------------------------------------------------------------------------------
\bib{201x.Lamzouri_2}{article}{
    author={Lamzouri, Y.},
     title={\href{run:bib/2011x.Lamzouri_2.pdf}
            {Large deviations of the limiting distribution in the Shanks-R\'enyi prime number race. 
}}
 journal={\tt 	arXiv:1103.0060v2}  }
%%%---------------------------------------------------------------------------------------
\bib{2012x.Ford}{article}{
    author={Ford, K.},
    author={Konyagin, S.},
    author={Lamzouri, Y.},
     title={\href{run:bib/2012x.Ford.pdf}
            {The prime number race and zeros of Dirichlet $L$-functions off the critical line. III. 
}}
 journal={\tt 	arXiv:1204.6715v1}  }
\end{biblist}
\end{document}

%which gives rise to the signifigance of:
%\begin{cnj}
%[Riemann--Piltz (RP)]\label{RP}
%No Dirichlet $L$-functions $L(s, \chi)$ vanish in the half plane $\sigma > \h$.\footnote{JUSTIN: I don't get it---why isn't this just GRH?}
%\end{cnj}

%
%
%
%If for a $k$ that satisfies the HC (Conjecture~\ref{HC}) and
%\begin{equation}
%T> \max\bigg(e_5(c_3k), e_2\Big(\frac{1}{Z(k)^3}\Big)\bigg)
%\end{equation}
%with a sufficiently large $c_3$ then for all $\gcd{l, k}=1$, both the inequalities
%
%\begin{align}
%\max_{\exp(\log_3^{1/130}T)}\frac{\delta_\pi(x; k, 1, l)}{\frac{\sqrt x}{\log x}} &> \frac{1}{100}\log_5{T}\\
%\min_{\exp(\log_3^{1/130}T)}\frac{\delta_\pi(x; k, 1, l)}{\frac{\sqrt x}{\log x}} &<- \frac{1}{100}\log_5{T}
%\end{align}
%hold (i.e. {\emph{for $k|24$ or for the Davies-values unconditionally}}). This means in a little weakened form that for $T$'s with (EREF) the interval 
%\begin{equation}
%\exp(\log_3^{1/130}T)\leq x \leq T
%\end{equation}
%certainly contains values $x'$ and $x''$ with
%\begin{align}
%\delta_\pi(x', k, 1, l) &>\frac{1}{100}\frac{\sqrt{x'}}{\log{x'}}\log_5{x'}\\
%\delta_\pi(x'', k, 1, l)&<-\frac{1}{100}\frac{\sqrt{x''}}{\log{x''}}\log_5{x''}
%\end{align}
%
%3. conditions
%\section{Conditions for Some Theorems to hold}
%A list of (non-trivial) conditions is required before we are able to further delve into the meat.
%
%
%
%
%
%\begin{cnj}
%[Lindel\"{o}f Hypothesis(LH)]\label{LH}
%\end{cnj}





%%%% Problems
%\newpage
%%
%\section{Problems}\label{sec.Problems}
%%%%-----------------
%%P 11
%\begin{pbm}
%For which $(l_1, l_2)$-paris with the number of solutions of \eqref{l_1} and \eqref{l_2} being equal, does the function 
%\begin{equation}
%\sum_{p\equiv l_1 \mod k} e^{-p\Lambda(n)} - \sum_{p\equiv l_2 \mod k} e^{-p\Lambda(n)}
%\end{equation}
%changes its sign infinitely often?
%\end{pbm}
%
%%P 12
%\begin{pbm}
%Given $\epsilon >0$, and the number of solutions of \eqref{l_1} and \eqref{l_2} equal, do there exist two sequences 
%\begin{align}
%r'_1&>r'_2>r'_3\hdots \rightarrow 0\\
%r''_1&>r''_2>r''_3\hdots \rightarrow 0
%\end{align}
%such that both
%\begin{align}
%\sum_{ p\equiv l_1\mod k} e^{-p r'_\nu}  - \sum_{p\equiv l_2\mod k} e^{-p r'_\nu} &> \bigg(\frac 1{r'_v}\bigg)^{\h -\epsilon}\\
%\sum_{ p\equiv l_1\mod k} e^{-p r''_\nu}  - \sum_{p\equiv l_2\mod k} e^{-p r''_\nu} &<- \bigg(\frac 1{r''_v}\bigg)^{\h -\epsilon}
%\end{align}
%hold?
%\end{pbm}
%
%%P 13
%\begin{pbm}
%Given $\epsilon >0$ small then for what function $h_k( \frac{1}{T})>0$ can we be sure that for each $(l_1, l_2)$-pari with the number of solutions of \eqref{l_1} and \eqref{l_2} being equal both the inequalities 
%\begin{align}
%\max_{\frac{1}{T}<x<\frac{1}{T}+h_k(T)} \Bigg\{ \sum_{ p\equiv l_1\mod k} e^{-p r'_\nu}  - \sum_{p\equiv l_2\mod k} e^{-p r'_\nu} \Bigg\}&>\bigg(\frac{1}{T}\bigg)^{\h-\epsilon} \\
%\min_{\frac{1}{T}<x<\frac{1}{T}+h_k(T)} \Bigg\{ \sum_{ p\equiv l_1\mod k} e^{-p r''_\nu}  - \sum_{p\equiv l_2\mod k} e^{-p r''_\nu} \Bigg\}&<- \bigg(\frac 1 T\bigg)^{\h -\epsilon}
%\end{align}
%hold?
%\end{pbm}
%
%%P 14
%\begin{pbm}
%For which function $g_k(T)>0$ can we assert that for each $(l_1, l_2)$-pairs with the number of solutions of \eqref{l_1} and \eqref{l_2} equal, all functions
%$$
%\sum_{ p\equiv l_1\mod k} e^{-p r'_\nu}  - \sum_{p\equiv l_2\mod k} e^{-p r'_\nu}
%$$
%change sign at least once in every interval 
%$$
%T\leq x \leq T+ g_k(T)
%$$
%\end{pbm}
%
%%P 15
%\begin{pbm}
%For which function $a(k)$ can we asssert that for each $(l_1, l_2)$-pair withwith the number of solutions of \eqref{l_1} and \eqref{l_2} equal, all functions in
%$$
%\sum_{ p\equiv l_1\mod k} e^{-p r}  - \sum_{p\equiv l_2\mod k} e^{-p r}
%$$
%vanish at least some points in
%$$
%1 \leq x \leq a(k)
%$$
%\end{pbm}
%
%%P 16
%\begin{pbm}
%Let  $W_k(T; l_1, l_2)$ denote the number of sign changes of 
%$$
%\sum_{ p\equiv l_1\mod k} e^{-p r}  - \sum_{p\equiv l_2\mod k} e^{-p r}
%$$
%Then what is asymptotical behaviour of $W_k(T; l_1, l_2)$ as $T \rightarrow \infty$?
%\end{pbm}
%
%%P 17
%\begin{pbm}
%For a fixed $(l_1, l_2)$-pair with with the number of solutions of \eqref{l_1} and \eqref{l_2} equal, what is the asymptotic behaviour of $N_{l_1 l_2}(Y)$ as $Y \rightarrow \infty$, where $N_{l_1 l_2}(Y)$ denotes the number of integers $r \leq Y$ with 
%$$
%\sum_{ p\equiv l_1\mod k} e^{-p r}  \geq \sum_{p\equiv l_2\mod k} e^{-p r}$$
%\end{pbm}
%
%%P 18
%\begin{pbm}[Race-problem of Shanks-R\`enyi]
%For each permutations 
%$$
%l_1, l_2, l_3, \hdots, l_{\varphi(k)}
%$$
%of the reduced set of residue classes mod $k$, does there exist infitely many integers $r$'s with
%$$
%\sum_{ p\equiv l_1\mod k} e^{-p r}  < \sum_{p\equiv l_2\mod k} e^{-p r} < \sum_{ p\equiv l_3\mod k} e^{-p r} <\hdots < \sum_{p\equiv l_{\varphi} \mod k} e^{-p r}
%$$
%\end{pbm}
%
%%P 19
%\begin{pbm}\label{P19}
%Does there exist infinitely many integers $r_\nu$'s such that for $j=1, 2, 3, \hdots,\varphi(k)$ simultaneously 
%$$
%\sum_{ p\equiv l_j\mod k} e^{-p r_\nu} > \frac{1}{\varphi(k)} \sum_{n=2}^{\infty} \frac{e^{nr}}{\log n} 
%$$
%\end{pbm}
%
%%P 20
%\begin{pbm}
%If the answer to~\ref{P19} is positive, then what are the distribution-properties of the $r_\nu$ sequence?
%\end{pbm}
%
%
%By defining 
%$$
%\psi(x; k, l):= \sum _{\substack{n\leq x \\ n \equiv l \mod k}} \Lambda(n)
%$$
%%%%%%%
%
%%P 5.21
%\begin{pbm}
%For which $(l_1, l_2)$-paris with $l_1\neq l_2$ does the function 
%$$
%\psi(x; k, l_1)-\psi(x; k, l_2) 
%$$
%changes its sign infinitely often?
%\end{pbm}
%
%%P 5.22
%\begin{pbm}
%Given $\epsilon >0$, $l_1 \neq l_2$, do there exist two sequences 
%\begin{align}
%x_1&<x_2<x_3\hdots \rightarrow  \infty\\
%y_1&<y_2<y_3\hdots \rightarrow  \infty
%\end{align}
%such that
%\begin{align}
%\psi(x_\nu ; k, l_1)-\psi(x_\nu ; k,  l_2) &> x_\nu ^ {\frac 1 2 - \epsilon} \\
%\psi(y_\nu ; k, l_1)-\psi(y_\nu ; k,  l_2) &< -y_\nu ^ {\frac 1 2 - \epsilon}
%\end{align}
%\end{pbm}
%
%%P 5.23
%\begin{pbm}
%Given $\epsilon >0$ small then for what function $h_k(T)>0$ can we be sure that for each $(l_1, l_2)$-pairs with $l_1 \neq l_2$ and $T\geq 1$ both the inequalities 
%\begin{align}
%\max_{T \leq x \leq T+ h_k(T)}\Big\{ \psi(x ; k, l_1)-\psi(x; k,  l_2) \Big\}  &> T^ {\frac 1 2 - \epsilon} \\
%\min_{T \leq x  \leq T+ h_k(T)}\Big\{ \psi(x ; k, l_1)-\psi(x; k,  l_2) \Big\}  &< -T^ {\frac 1 2 - \epsilon} 
%\end{align}
%hold?
%\end{pbm}
%
%%P 24
%\begin{pbm}
%
%For which function $g_k(T)>0$ can we assert that for each $(l_1, l_2)$-pairs with $l_1 \neq l_2$ and $T \geq 1$, all functions
%$$
%\psi(x ; k, l_1)-\psi(x; k,  l_2) 
%$$
%change sign at least once in every interval 
%$$
%T\leq x \leq T+ g_k(T)
%$$
%\end{pbm}
%
%%P 5.25
%\begin{pbm}
%For which function $a(k)$ can we assert that for each $(l_1, l_2)$-pair with $l_1 \neq l_2$, all functions in
%$$
%\psi(x ; k, l_1)-\psi(x ; k,  l_2) 
%$$
%vanish at least some points in
%$$
%1 \leq x \leq a(k)
%$$
%\end{pbm}
%
%%P 5.26
%\begin{pbm}
%Let  $W_k(T; l_1, l_2)$ denote the number of sign changes of $\psi(x; k, l_1)-\psi(x; k, l_2)$, then what is asymptotical behaviour of $W_k(T; l_1, l_2)$ as $T \rightarrow \infty$?
%\end{pbm}
%
%%P 5.27
%\begin{pbm}
%For a fixed $(l_1, l_2)$-pair with $l_1 \neq l_2$, what is the asymptotic behaviour of $N_{l_1 l_2}(Y)$ as $Y \rightarrow \infty$, where $N_{l_1 l_2}(Y)$ denotes the number of integers $m \leq Y$ with 
%$$
%\psi(m ; k, l_1) \geq \psi(m ; k,  l_2)
%$$
%\end{pbm}
%
%%P 5.28
%\begin{pbm}[Race-problem of Shanks-R\`enyi]
%For each permutations 
%$$
%l_1, l_2, l_3, \hdots, l_{\varphi(k)}
%$$
%of the reduced set of residue classes mod $k$, does there exist infitely many integers $m$'s with
%$$
%\psi(m; k, l_1) < \psi(m; k, l_2) < \psi(m; k, l_3) < \hdots  < \psi(m; k, l_{\varphi(k)})
%$$
%\end{pbm}
%
%%P 5.29
%\begin{pbm}
%Does there exist infinitely many integers $m_\nu$'s such that for $j=1, 2, 3, \hdots \varphi(k)$ simultaneously 
%$$
%\psi(m_\nu, k, l_j)> \frac {\Li(m_\nu)}{\varphi(k)} \; \; ?
%$$
%\end{pbm}
%
%%P 5.30
%\begin{pbm}
%If the answer to~\ref{P9} is positive, then what are the distribution-properties of the $m_\nu$ sequence?
%\end{pbm}
%
%One can expect that there are "more" primes in the residue class $l_1\mod k$ than $l_2 \mod k$ if and only if the number of incongruent solutions of the congruence
%\begin{equation}\label{l_1}
%x^2 \equiv l_1 \mod k
%\end{equation} 
%is less than that of the congruence
%\begin{equation}\label{l_2}
%x^2 \equiv l_2 \mod k
%\end{equation} 
%
%
%%%%-----------------
%%P 5.31
%\begin{pbm}
%For which $(l_1, l_2)$-paris with the number of solutions of \eqref{l_1} and \eqref{l_2} being equal, does the function 
%\begin{equation}
%\sum_{p\equiv l_1 \mod k} e^{-p\Lambda(n)} - \sum_{p\equiv l_2 \mod k} e^{-p\Lambda(n)}
%\end{equation}
%changes its sign infinitely often?
%\end{pbm}
%
%%P 5.32
%\begin{pbm}
%Given $\epsilon >0$, and the number of solutions of \eqref{l_1} and \eqref{l_2} equal, do there exist two sequences 
%\begin{align}
%r'_1&>r'_2>r'_3\hdots \rightarrow 0\\
%r''_1&>r''_2<r''_3\hdots \rightarrow 0
%\end{align}
%such that both
%\begin{align}
%\sum_{ p\equiv l_1\mod k} e^{-p r'_\nu}  - \sum_{p\equiv l_2\mod k} e^{-p r'_\nu} &> \bigg(\frac 1{r'_v}\bigg)^{\h -\epsilon}\\
%\sum_{ p\equiv l_1\mod k} e^{-p r''_\nu}  - \sum_{p\equiv l_2\mod k} e^{-p r''_\nu} &<- \bigg(\frac 1{r''_v}\bigg)^{\h -\epsilon}
%\end{align}
%hold?
%\end{pbm}
%
%%P 5.33
%\begin{pbm}
%Given $\epsilon >0$ small then for what function $h_k( \frac{1}{T})>0$ can we be sure that for each $(l_1, l_2)$-pari with the number of solutions of \eqref{l_1} and \eqref{l_2} being equal both the inequalities 
%\begin{align}
%\max_{\frac{1}{T}<x<\frac{1}{T}+h_k(T)} \Bigg\{ \sum_{ p\equiv l_1\mod k} e^{-p r'_\nu}  - \sum_{p\equiv l_2\mod k} e^{-p r'_\nu} \Bigg\}&>\bigg(\frac{1}{T}\bigg)^{\h-\epsilon} \\
%\min_{\frac{1}{T}<x<\frac{1}{T}+h_k(T)} \Bigg\{ \sum_{ p\equiv l_1\mod k} e^{-p r''_\nu}  - \sum_{p\equiv l_2\mod k} e^{-p r''_\nu} \Bigg\}&<- \bigg(\frac 1 T\bigg)^{\h -\epsilon}
%\end{align}
%hold?
%\end{pbm}
%
%%P 5.34
%\begin{pbm}
%For which function $g_k(T)>0$ can we assert that for each $(l_1, l_2)$-pairs with the number of solutions of \eqref{l_1} and \eqref{l_2} equal, all functions
%$$
%\sum_{ p\equiv l_1\mod k} e^{-p r'_\nu}  - \sum_{p\equiv l_2\mod k} e^{-p r'_\nu}
%$$
%change sign at least once in every interval 
%$$
%T\leq x \leq T+ g_k(T)
%$$
%\end{pbm}
%
%%P 5.35
%\begin{pbm}
%For which function $a(k)$ can we asssert that for each $(l_1, l_2)$-pair withwith the number of solutions of \eqref{l_1} and \eqref{l_2} equal, all functions in
%$$
%\sum_{ p\equiv l_1\mod k} e^{-p r}  - \sum_{p\equiv l_2\mod k} e^{-p r}
%$$
%vanish at least some points in
%$$
%1 \leq x \leq a(k)
%$$
%\end{pbm}
%
%%P 5.36
%\begin{pbm}
%Let  $W_k(T; l_1, l_2)$ denote the number of sign changes of 
%$$
%\sum_{ p\equiv l_1\mod k} e^{-p r}  - \sum_{p\equiv l_2\mod k} e^{-p r}
%$$
%Then what is asymptotical behaviour of $W_k(T; l_1, l_2)$ as $T \rightarrow \infty$?
%\end{pbm}
%
%%P 5.37
%\begin{pbm}
%For a fixed $(l_1, l_2)$-pair with with the number of solutions of \eqref{l_1} and \eqref{l_2} equal, what is the asymptotic behaviour of $N_{l_1 l_2}(Y)$ as $Y \rightarrow \infty$, where $N_{l_1 l_2}(Y)$ denotes the number of integers $r \leq Y$ with 
%$$
%\sum_{ p\equiv l_1\mod k} e^{-p r}  \geq \sum_{p\equiv l_2\mod k} e^{-p r}$$
%\end{pbm}
%
%%P 5.38
%\begin{pbm}[Race-problem of Shanks-R\`enyi]
%For each permutations 
%$$
%l_1, l_2, l_3, \hdots, l_{\varphi(k)}
%$$
%of the reduced set of residue classes mod $k$, does there exist infitely many integers $r$'s with
%$$
%\sum_{ p\equiv l_1\mod k} e^{-p r}  < \sum_{p\equiv l_2\mod k} e^{-p r} < \sum_{ p\equiv l_3\mod k} e^{-p r} <\hdots < \sum_{p\equiv l_{\varphi} \mod k} e^{-p r}
%$$
%\end{pbm}
%
%%P 5.39
%\begin{pbm}\label{P39}
%Does there exist infinitely many integers $r_\nu$'s such that for $j=1, 2, 3, \hdots,\varphi(k)$ simultaneously 
%$$
%\sum_{ p\equiv l_j\mod k} e^{-p r_\nu} > \frac{1}{\varphi(k)} \sum_{n=2}^{\infty} \frac{e^{nr}}{\log n} 
%$$
%\end{pbm}
%
%%P 5.40
%\begin{pbm}
%If the answer to~\ref{P39} is positive, then what are the distribution-properties of the $r_\nu$ sequence?
%\end{pbm}
%% 
%
%\newpage
%\section{Problems from IIa}
%
%
%\section{Problems in IIb}
%%6.1
%\begin{pbm}[Problems of INFINITY of sign changes]
%To prove that the functions $\Delta_F(x; k, l_1, l_2)$ for $F= \psi,\; \Pi, \; \vartheta,\; \pi$  and $l_1 \equiv l_2 \mod k$ change sign infinitely often. 
%\end{pbm}
%
%%6.2
%\begin{pbm}[Problems of infinity of BIG sign changes]
%To prove that for each functions $\Delta_F(x; k, l_1, l_2)$ for $f= \psi,\; \Pi, \; \vartheta,\; \pi$  and $l_1 \equiv l_2 \mod k$ and arbitrarily small $\epsilon >0$ there is a sequence
%\begin{equation}
%r_1>r_2>r_3>\hdots \rightarrow 0
%\end{equation}
%such that, for each $\nu=1, 2, 3, \hdots$, $\Delta_F(r_\nu; k, l_1, l_2)>r_\nu^{\h-\epsilon}$, and hence owing to the symmetry of $l_1, l_2$ also a sequence
%\begin{equation}
%s_1>s_2>s_3<\hdots \rightarrow 0
%\end{equation}
%such that $\Delta_f(s_\nu; k, l_1, l_2)<-y_\nu^{\h-\epsilon}$
%\end{pbm}
%
%%5.3
%\begin{pbm}[LOCALIZED sign changes]
%To prove that for $T>T_0(k, j)$ and suitable $A(T)<T$, all functions $\delta_f(x; k, l_1, l_2)$ change sign in the interval 
%$$
%A(T)\leq x \leq T
%$$
%\end{pbm}
%
%%5.4
%\begin{pbm}[Localized BIG sign changes]
%To prove that for $T> T_0(k, j)$ and suitable $A(T)<T$ all functions $\delta_g(x; k, l_1, l_2)$ both the inequalities
%\begin{align}
%\max_{A(T)\leq x \leq T}\delta_f(x; k, l_1, l_2)> & \frac{T^\h}{\varphi(T)}\\
%\min_{A(T)\leq x \leq T}\delta_f(x; k, l_1, l_2)< &- \frac{T^\h}{\varphi(T)}
%\end{align}
%hold, with a $\varphi(x)>0$, satisfying 
%$$\lim_{x\to \infty}\frac{\log\varphi(x)}{\log x}= 0$$
%\end{pbm}
%
%%5.5
%\begin{pbm}[FIRST sign change]
%To determine for $f= \psi,\; \Pi,\; \vartheta \; \pi$ functions $A_f(k)$ such that for $1\leq x \leq A_f(k)$ all $\delta_f(x; k, l_1, l_2)$, $l_1 \not\equiv l_2 \mod k$, $k$ fixed, functions change sign at least once.
%\end{pbm}
%
%%5.6
%\begin{pbm}[ASYMPTOTIC estimation of the number of sign changes]
%\end{pbm}
%
%%5.7
%\begin{pbm}[AVERAGE preponderance problems]
%To mention a typical one, the results of Hardy-Landau-Littewood indicate that the inequality
%\begin{equation}
%\pi(n; 4, 1)-\pi(n; 4,3)<0
%\end{equation} 
%is true "much more often" than the inequality 
%\begin{equation}\label{1>3mod3}
%\pi(n; 4, 1)-\pi(n; 4,3)\geq 0
%\end{equation}
%Hence denoting $N(x)$ the number of indices $n\leq x$ with the property~\ref{1>3mod3} probably the relation
%\begin{equation}
%\lim_{x \to \infty} \frac{N(x)}x = 0
%\end{equation}
%holds
%\end{pbm}
%
%%5.8
%\begin{pbm}[STRONGLY localized accumulation problems]
%In the previous problems in various ways the number of \emph{all} primes $\leq x$ in a fixed progression occurred. One can image that one can much better localize relatively small intervals where the primes of some progression preponderate. Again instead of writing out generally the pertaining the pertaining problems we confine ourselves to indicating the character of them by mentioning just one. 
%
%Is it true for $T>c_1$, ($c_1$ numerically positive constant) that for suitable $T \leq U_1 < U_2 \leq 2T$, we have:
%\begin{equation*}
%\sum_{\substack{U_1\leq p \leq U_2 \\ p \equiv 1 \mod 4}}1 - \sum_{\substack{U_1\leq p \leq U_2 \\ p \equiv 3 \mod 4}}1 > \frac{\sqrt T}{\varphi(T)}
%\end{equation*}
%\end{pbm}
%
%%5.9
%\begin{pbm}[Littlewood-generalizations]
%A typical problem of this kind would be the existence of a sequence $x_1< x_2< x_3< \hdots \to \infty$ such that simultaneously the inequalities
%\begin{equation}
%\pi(x_\nu; 4,1)\geq \h \Li(x_\nu) = \h \int_2^{x_\nu}\frac{du}{\log u}
%\end{equation}
%and
%\begin{equation}
%\pi(x_\nu; 4,3)\geq \h \Li(x_\nu) = \h \int_2^{x_\nu}\frac{du}{\log u}
%\end{equation}
%hold. This would constitute an obvious generalization of Littlewood's classical theorem that for a suitable sequences $y_1< y_2< y_3< \hdots \to \infty$ the inequality $\pi(y_\nu) \geq \Li(y_\nu)$
%\end{pbm}
%%%
%
%%5.10
%\begin{pbm}[RACING Problem]
%Again only a sample of these problems: if $l_1, l_2, l_3, \hdots, l_\varphi(k)$ is any prescribed order of the reduced reside systems $\mod k$ then for a suitable sequence $x_1<x_2<x_3\hdots \to \infty$ the inequalities
%\begin{equation*}
%\sum_{n\equiv l_1 \mod k} e^{=n_\nu}\geq \sum_{n\equiv l_2 \mod k} e^{=n_\nu}\geq \hdots \geq \sum_{n\equiv l_\varphi{k} \mod k} e^{=n_\nu}
%\end{equation*}
%hold.
%
%G. G. Lorentz called our attention to the fact that comparison of primes of any two arithmetical progressions $\mod k_1$ and $k_2$ $(k_1 \neq k_2)$ is not trivial in the case when
%\begin{equation*}
%\varphi(k_1)= \varphi(k_2)
%\end{equation*}
%and analogous problems occur for moduli $k_1, k_2, k_3, \hdots, k_r$ with 
%\begin{equation*}
%\varphi(k_1)= \varphi(k_2)= \hdots = \varphi(k_r)
%\end{equation*}
%\end{pbm}
%
%%6.11
%\begin{pbm}[UNION-Problem]
%A typical problem of this kind is the following: For a given modulus $k$ do there exist two disjoint subsets $A$ and $B$, consisting of the same number of residue classes, such that 
%\begin{equation}
%\sum_{p\in A, p\leq x} 1 \geq \sum_{p\in B, p\leq x} 1
%\end{equation}
%For all sufficiently large $x$'s.
%\end{pbm}
%

%3.



